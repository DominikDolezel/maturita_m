\documentclass[11pt]{template/cauchy}
\usepackage[czech]{babel}

\usepackage{enumerate}
\usepackage{mlmodern}
\usepackage{template/mathsphystools}
\usepackage{amsthm,thmtools,xcolor,amsmath,amssymb}
\usepackage[scr]{rsfso}
% \usepackage{thmstyles}
\usepackage{tabularx}
\usepackage{graphicx}
\usepackage{multicol}
\usepackage{appendix}
\usepackage{tabularx}
\usepackage{subcaption}
\usepackage{hyperref}
\usepackage{comment}
\usepackage{lscape}
\usepackage{xcolor}
\graphicspath{{images/}}

\declaretheoremstyle[
  headfont=\color{red}\normalfont\bfseries,
  bodyfont=\color{black}\normalfont,
]{def}

\declaretheoremstyle[
  headfont=\color{blue}\normalfont\bfseries,
  bodyfont=\color{black}\normalfont\itshape,
]{pr}

\declaretheoremstyle[
  headfont=\color{green}\normalfont\bfseries,
  bodyfont=\color{black}\normalfont\itshape,
]{veta}

\declaretheoremstyle[
  headfont=\color{black}\normalfont\itshape,
  bodyfont=\color{black}\normalfont,
  numbered=no
]{res}

\declaretheoremstyle[
  headfont=\color{brown}\normalfont\bfseries,
  bodyfont=\color{black}\normalfont,
]{pozn}

\declaretheoremstyle[
  headfont=\color{brown}\normalfont\bfseries,
  bodyfont=\color{black}\normalfont,
]{dusl}

\declaretheoremstyle[
  headfont=\color{cyan}\normalfont\bfseries,
  bodyfont=\color{black}\normalfont,
]{axiom}

\declaretheorem[
  style=def,
  name=Definice,
]{definition}

\declaretheorem[
  style=pr,
  name=Příklad,
]{priklad}

\declaretheorem[
  style=res,
  name=Řešení,
]{reseni}

\declaretheorem[
  style=veta,
  name=Věta,
]{veta}

\declaretheorem[
  style=pozn,
  name=Poznámka,
]{pozn}

\declaretheorem[
  style=dusl,
  name=Důsledek,
]{dusledek}

\declaretheorem[
  style=axiom,
  name=Axiom,
]{axiom}

\DeclareMathOperator{\tg}{tg}
\DeclareMathOperator{\cotg}{cotg}
\DeclareMathOperator{\arctg}{arctg}
\DeclareMathOperator{\arccotg}{arccotg}

\title[Title for the Header]{Matematika}
\subtitle{}
\author{Dominik Doležel \& Honza Romanovský}
\affiliation{Gymnázium Brno, třída Kapitána Jaroše, p. o.}
\date{\today}
\begin{document}

%!TEX root = ../main.tex
\setcounter{page}{0}
\thispagestyle{fancy-blank}
\begingroup
% \vphantom{Optional note}
{\large \par}
\vspace*{35mm}
{\huge\bfseries\utitle\par}

\vspace*{4mm}
{\rule{\linewidth}{0.5mm}\par}
\vspace*{4mm}

{\large\bfseries\uauthor\par}\vspace*{1mm}

{\large\itshape{a kolektiv}\par}


\vspace{10em}
\begin{figure}[ht!]
\begin{center}
  \begin{tikzpicture}
  \begin{axis}[
      axis lines = middle,
      xlabel = \(x\),
      ylabel = {\(y\)},
      xmin=-1,
      xmax=1,
      ymin=-1,
      ymax=1.6,
      xtick = \empty,
      ytick = \empty,
  ]
  %Below the red parabola is defined
  \addplot [
      domain=-1:1,
      samples=500,
      color=red,
  ]
  {abs(x)^(2/3)+sqrt(1 - x^2)};

  \addplot [
      domain=-1:1,
      samples=500,
      color=red,
      ]
      {abs(x)^(2/3)-sqrt(1 - x^2)};

  \end{axis}
  \end{tikzpicture}
  \caption*{Graf $x^2+ \left [ y- x^\frac{2}{3} \right ]^2 = 1$}
\end{center}
\end{figure}
\vfill
{\large \par}
\endgroup
\clearpage


\frontmatter
\tableofcontents
\clearpage
% \listoffigures
% \listoftables
% \clearpage

% Každá otázka vypadá takto:
% \section*{Název otázky}
% \subsection{Základní pojmy}
% itemize se seznamem základních pojmů
% \section*{Příklady}

\mainmatter

%!TEX root = ../main.tex

\clearpage
\phantomsection
\addcontentsline{toc}{section}{Předmluva}
\begin{adjustwidth}{0.1\textwidth}{0.1\textwidth}
\begingroup
\null\vspace{0.2\textheight}
\begin{center}
{\bfseries\Large Předmluva}\par\vspace{2em}
\end{center}

Na počátku všeho řekl Bůh:
\begin{center}
\uv{Nechť je dán polynom.}
\end{center}
Byl spokojen se svým výtvorem, ale přišlo mu, že všechno vědění je vlastně
bezúčelné, nemá-li ho kdo studovat. A tak stvořil člověka \dots

\par Vážené čtenářstvo, v rukou držíte soubor všech základních definic, vět
a příkladů středoškolské matematiky tak, jak byla probírána na matematickém gymnáziu
v první polovině dvacátých let jednadvacátého století. Seznam je sestaven podle
požadavků k maturitní zkoušce.
\par Všichni, již se na psaní a úpravě podíleli, přejí příjemné čtení. A nezapomeňte,
všechno to znáte ze základní školy!


\hfill Věnováno RNDr. Pavlu Boucníkovi
\endgroup
\end{adjustwidth}
\clearpage


% \input{01_nazev_otazky}

\section{Základní pojmy z teorie množin}
\begin{definition}
  \textbf{Množina} je sourhn objektů, chápaný jako celek. Tyto objekty nazýváme prvky množiny.
\end{definition}

Množina může být konečná, nekonečná nebo prázdná. Množinu lze zadat výčtem prvků nebo pomocí charakteristické vlastnosti (např. $\left \{ 2k, k \in \mathbb{N}\right\}$).

\begin{definition}
  \textbf{Podmnožina} množiny $A$ je taková množina $B$, že všechny její prvky patří do množiny $A$.
\end{definition}

Každá neprázdná množina má dvě \textbf{nevlastní podmnožiny}: množinu prázdnou a sebe sama. Všechny ostatní její podmnožiny nazýváme vlastní.

\begin{definition}
  Množiny $A$ a $B$ se rovnají právě tehdy, když $A$ je podmnožinou $B$ a zároveň $B$ je podmnožinou $A$.
\end{definition}

\begin{definition}
  Nechť $A \subseteq B$ a $B\neq \emptyset$. Množinu všech prvků množiny $B$, které nepatří do množiny $A$, nazýváme \textbf{doplněk} (komplement) množiny $A$ v množině $B$. Značíme $A_B^\prime.$
\end{definition}

\begin{definition}
  Nechť $A, B$ jsou dvě množiny. Jejich \textbf{sjednocením} nazveme takovou množinu, která obsahuje ty prkvy, které patří alespoň do jedné z množin $A, B$. Zapisujeme $A \cup B$.
\end{definition}

\begin{definition}
  Nechť $A, B$ jsou dvě množiny. Jejich \textbf{průnikem} nazveme takovou množinu, která obsahuje ty prvky, které patří zároveň do obou těchto množin $A, B$. Zapisujeme $A \cap B$.
\end{definition}

\begin{definition}
  \textbf{Vennův diagram} je grafické schematické znázornění všech možných vztahů (sjednocení, průnik, rozdíl, doplněk) několika podmnožin univerzální množiny, jež znázorňujeme pomocí uzavřených čar.
\end{definition}

\begin{definition}
  Dvě množiny jsou \textbf{disjunktní}, pokud nemají žádný společný prvek, tedy pokud je jejich průnikem prázdná množina.
\end{definition}

\begin{definition}
  Nechť $A, B$ jsou dvě množiny. \textbf{Rozdíl} množin je množina, která obsahuje všechny prvky množiny $A$ s výjimkou těch, jež jsou zároveň prvky množiny $B$. Zapisujeme $A - B$.
\end{definition}

\begin{veta}
  \textbf{De Morganovy zákony} jsou zákony určující vztahy mezi sjednocením, průnikem a doplňkem množiny. Nechť $A, B$ jsou dvě množiny, $^\prime$ doplněk množiny. Potom platí:
  $$ (A \cup B)^\prime = A^\prime \cap B^\prime$$
  $$ (A \cap B)^\prime = A^\prime \cup B^\prime$$
\end{veta}

\begin{proof}
  Důkaz prvního vztahu:

  \begin{minipage}{0.5\textwidth}
    \centering
        \includegraphics[width=0.5\linewidth]{vennsjed.png}
        \includegraphics[width=0.5\linewidth]{venndoplsjed.png}
  \end{minipage}
  \hfill
  \noindent\begin{minipage}{0.5\textwidth}
  \centering
        \includegraphics[width=0.5\linewidth]{venndoplAaB.png}
        \includegraphics[width=0.5\linewidth]{venndoplsjed.png}
  \end{minipage}

  Důkaz druhého analogicky.
\end{proof}

\begin{pozn}[Číselné množiny]
  Rozlišujeme následující základní číselné množiny:
  \begin{itemize}
    \item $\mathbb{N}$: přirozená čísla $(1, 2, 3, \dots)$,
    \item $\mathbb{Z}$: celá čísla $(\dots, -2, -1, 0, 1, 2, \dots)$,
    \item $\mathbb{Q}$: racionální čísla $(3/5, 0,\overline{3})$,
    \item $\mathbb{R}$: reálná čísla $(e, \pi)$,
    \item $\mathbb{C}$: komplexní čísla $(3+2i)$
  \end{itemize}
\end{pozn}

Iracionální čísla ($\mathbb{I}$) jsou doplněk racionálních v $\mathbb{R}$.

\begin{definition}
  \textbf{Celá čísla} jsou čísla, která vyjadřují počty prvků množin, čísla k nim opačná a číslo 0.
\end{definition}

\begin{definition}
  \textbf{Racionálním číslem} nazveme takové číslo $a = \frac{k}{l}, k, l \in \mathbb{Z}$, a $p,q$ jsou nesoudělná.
\end{definition}

\begin{pozn}
  Přirozená čísla zapisujeme pomocí číslic 0--9 a chápeme je takto:
  $$4503=4\cdot 10^3+5\cdot 10^2 + 0 \cdot 10^1 + 3\cdot 10^0.$$
  Každé racionální číslo je v desítkové soutavě vyjádřeno buď ukončeným desetinným rozvojem nebo neukončeným periodickým rozvojem. Iracionální číslo je vyjádřeno neukončeným neperiodickým rozvojem.
\end{pozn}

\begin{definition}
  \textbf{Reálnými čísly} nazýváme všechna čísla, která jsou velikostmi úseček.
\end{definition}

\begin{definition}
  Nechť $a,b \in \mathbb{R},$ kde $a<b$. Pak množiny takových $x\in \mathbb{R},$ že $a\leq x\leq b$ (resp. $a < x < b$, resp. $a < x$, resp. $a \leq x < b$ atd.) nazýváme uzavřeným (resp. otevřeným, resp. neomezeným zleva otevřeným, resp.zprava uzavřeným, zleva otevřeným atd.) \textbf{intervalem}. Zapisujeme $\left<a,b\right>$ (resp. $\left(a,b\right)$, resp. $(a, \infty)$, resp. $\left<a, b\right)$)
\end{definition}

\begin{definition}
  \textbf{Periodický rozvoj čísla} je rozvoj, u kterého se za desetinnou čárkou donekonečna opakuje táž číslice nebo skupina číslic. Čísla s takovýmto rozvojem se nazývají ryze periodická čísla, opakující se číslice nebo skupina opakujících se číslic se nazývá perioda. Zapisují se tak, že se nad opakující se skupinou napíše pruh:
  $$0,333 … = 0,\overline{3}$$
\end{definition}

\begin{definition}
  Množina komplexních čísel $\mathbb C$ je množina uspořádaných reálných dvojic $[x, y]$, na kterých je definována rovnost, sčítání a násobení následovně:
$$[a, b] = [c, d] \Leftrightarrow a = c \land b = d,$$
$$[a, b] + [c, d] = [a + c, b + d],$$
$$[a, b][c, d] = [ac - bd, ad + bc].$$
  Dvojici $[0, 1]$ označíme $i$ a budeme ji nazývat komplexní jednotkou. Zřejmě pak platí, že $i^2 = -1$.
\end{definition}

\begin{definition}
  Pokud je uspořádaná dvojice z předchozí definice ve tvaru $[0, b], b \in \R$, nazveme toto číslo \textbf{ryze imaginárním}.
\end{definition}

\begin{example}[SÚM 169/8]
  Označme $M$ množinu všech dvojciferných přirozených čísel delitelných šesti a $N$ všechn dělitelů čísla 210, kteří jsou různí od čísla 1 a 210. Určete, která z množin má větší počet prvků, a vypište všechny prvky, které mají obě množiny stejné.
  \begin{align*}
    M & = \left\{12, 18, 24, 30, 36, 42, 48, 54, 60, 66, 72, 78, 84, 90, 96\right\}\\
    210 & = 2\cdot 3 \cdot 5  \cdot 7 \textrm{ -- hledáme násobky všech podmnožin těchto čísel} \\
    N  & = \left\{2,3,5,6,7, 10, 14, 15, 21, 30, 35, 42, 70, 105\right\} \\
    |M| & = 15, |N| = 14, M \cap N = \left\{30, 42\right\}
  \end{align*}

  \rm Množina $M$ má více prvků a společná jsou čísla 30 a 42.
\end{example}

\begin{example}[SÚM 171/26]
  $M$ je množina šech reálných čísel $x$, která splňují nerovnosti $-2<x<5$, $N$ je mn. všech reálných čísel $y$, která splňují nerovnost $|y|<4$. Určete množinu $R=M\cup N$ a $S = M\cap N.$ \hfill $R = (-4,5), S=(-2,4).$
\end{example}

\begin{example}[SÚM 172/29f]
  Znázorněte a určete výsledný interval: $(a,a+2)\cap (a-1,a+1),$ kde $a>0.$\hfill$(a,a+1)$
\end{example}

\begin{example}[SÚM (172/33)]
  Je dána kružnice $k$ se středem v bodě  $S$ a poloměrem $r$. Množinu všech bodů uvnitř kružnice označte $A$. Nakreslete rovnostranný trojúhelník $ESD$, jehož jeden vrchol je ve středu dané kružnice a délky stran jsou rovny velikosti jejího průměru. Množinu vnitřních bodů tohoto trojúhelníka ozn. $B$. Díle sestrojte osu úhlu $ESD$ a množinu bodů této přímky označte $C$. Nakreslete samostatné obrázky pro:
  \begin{itemize}
    \item $(A\cap B)\cup C,$
    \item $(A\cup C) \cap (B\cup C),$
    \item $(A\cap B) \cup (B\cap C),$
    \item $(A\cup C) \cap B$.
  \end{itemize}
\end{example}

\begin{example}[SÚM 173/34]
  Pro která $x$ je interval:
  \begin{enumerate}[a.]
    \item $\left<2x,x+3\right>$ částí intervalu $(2,7)$? \hfill $x \in (1,3)$
    \item $(x,5)$ částí intervalu $\left(-1,x+1\right)$? \hfill $x\in (4,5)$
    \item $(x,x+3)$ částí intervalu $\left<5,8\right>$? \hfill $x=5$
    \item $\left<x,2x-1\right>$ částí intervalu $\left<-2,5\right>$? \hfill $x\in\left<-2,5\right>$
    \item $\left<3x,2x+1\right>$ částí intervalu $(3,6)$? \hfill $x\in \left\{\right\}$
  \end{enumerate}
\end{example}

\begin{example}[SÚM 173/35]
  Nechť $M = (a,b), N = (1,8), Q = (1,5)$. Určete $a,b \in \mathbb{R}$ tak, aby platilo $M\cap N = Q$.\hfill $a\in \left(-\infty, 1\right>, b=5$
\end{example}

\begin{example}[SÚM 173/37*]
  Je dán trojúhelník $ABC$. Uvažujme množinu $M$ všech bodů tohoto trojúhelníka, pro které platí $|AX| \geq |BX| \geq |CX|.$ Pomocí velikosti stran a úhlů troj. $ABC$ vyjádřete podmínky pro to, aby:
  \begin{enumerate}[a.]
    \item $X$ byla pětiúhelník, \hfill $\gamma > 90^\circ, \alpha < \beta$
    \item $X$  je jeden bod, \hfill $\alpha = 90^\circ$
    \item $X$ je prázdná.\hfill $\alpha > 90^\circ$
  \end{enumerate}
\end{example}


\begin{example}[SÚM 174/42]
  Jsou dány množiny $M=\left\{1,2; 3; 4\right\},N=\left\{x;y;z\right\}.$ Uveďte alespoň jeden příklad na zobrazení množiny
  \begin{enumerate}[a.]
    \item $M$ do $N$\hfill $1,2\rightarrow x; 3\rightarrow y; 4 \rightarrow y$
    \item $N$ do $M$ \hfill $x\rightarrow 1,2; y\rightarrow 3; z\rightarrow 3$
    \item $M$ na $N.$ \hfill $1,2\rightarrow x; 3 \rightarrow y; 4 \rightarrow z$
  \end{enumerate}
\end{example}

\begin{example}[SÚM 174/46]
  Kolik je všech zobrazení (pod)množiny $\left\{a,b,c,d\right\}$ do (na) množiny $\left\{1,2\right\}$?\hfill \rm 81
\end{example}

\begin{example}[SÚM 106/20]
  Převeďte na obyčejné zlomky:
  \begin{enumerate}[a.]
    \item $0,\overline{27}$\hfill $\frac{27}{99}\frac{3}{11}$
    \item $0,\overline{6}$ \hfill $\frac{2}{3}$
    \item $2,\overline{345}$ \hfill $2+\frac{345}{999}=\frac{781}{333}$
    \item $0,\overline{1234}$\hfill $\frac{1234}{9999}$
    \item $0,7\overline{2}$\hfill $\frac{7}{10}+\frac{2}{90}=\frac{13}{18}$
    \item $0,1\overline{36}$\hfill $\frac{1}{10}+\frac{36}{990}=\frac{3}{22}$
    \item $0,7\overline{27}$\hfill $\frac{7}{10}+\frac{27}{990}=\frac{8}{11}$
    \item $3,39\overline{85}$\hfill $3+\frac{39}{100}+\frac{85}{9900}=\frac{33646}{9900}$
  \end{enumerate}
\end{example}

\begin{example}[SÚM 107/21]
  Proveďte:
  \begin{enumerate}[a.]
    \item $0,\overline{4}+0,\overline{12}$ \hfill $\frac{4}{9}+\frac{12}{9}=\frac{16}{9}$
    \item $0,\overline{7}+0,\overline{35}$  \hfill $\frac{112}{99}$
    \item $0,\overline{47}+0,\overline{023}$ \hfill $\frac{5470}{10989}$
    \item $0,\overline{47}+0,0\overline{23}$ \hfill $\frac{493}{990}$
    \item $0,5\overline{354}+0,\overline{85}$\hfill $1,394021\dots$
    \item $2,\overline{35}-1,\overline{231}$\hfill$ \frac{4111}{3663}$
    \item $1,\overline{25}-0,\overline{773}$ \hfill $\frac{5261}{10989}$
  \end{enumerate}
\end{example}

\begin{example}[SÚM 107/22*]
  Proveďte:
  \begin{enumerate}[a.]
    \item $1,\overline{2}\cdot 1,\overline{18}$\hfill $\left(1+\frac{2}{9}\right)\left(1+\frac{18}{99}\right)=\frac{11}{9}\cdot \frac{117}{99}=\frac{13}{9}$
    \item $0,\overline{32}\cdot 1,\overline{3}$\hfill $\frac{128}{297}$
  \end{enumerate}
\end{example}

\begin{example}[SÚM 107/23*]
  Řešte rovnici:
  \begin{enumerate}
    \item $0,\overline{25}x + 0,\overline{31}x = 1,\overline{13}$ \hfill $x=2$
    \item $2,\overline{64}x - 3,\overline{48} = 1,\overline{48}x$  \hfill $x = 3$
  \end{enumerate}
\end{example}

\begin{example}[SMP 140/6abc]
  Pomocí Vennových diagramů zjednodušte zápisy množin: \\
  \begin{minipage}{0.5\textwidth}
    \begin{enumerate}[a.]
      \item $(A \cap B \cap C) \cup [B \cap (A^\prime \cup C)^\prime]$
      \item $[(A \cup B)^\prime \cup (B \cup C)] \cap (C \cup A)$
      \item $[(A \cup B^\prime) \cap C] \cup [(B^\prime \cup A^\prime)^\prime \cap C]$
    \end{enumerate}
  \end{minipage}
  \hfill
  \noindent\begin{minipage}{0.5\textwidth}
      \includegraphics[width=\linewidth]{vennovy}
      \captionof{figure}{}
  \end{minipage}
\end{example}

\begin{example}[SÚM 109/36]
  Dokažte, že číslo $\sqrt{5}$ je iracionální.

  Dk. sporem: Nechť $\sqrt{5} \in \Q \Rightarrow \sqrt{5} = \frac{a}{b}, a, b \in \Z, D(a, b) = 1$.
  $$\sqrt{5} = \frac{a}{b}$$
  $$5 = \frac{a^2}{b^2}$$
  $$5b^2 = a^2 \Rightarrow 5 \mid a^2 \Rightarrow 5 \mid a \Rightarrow \exists k: a = 5k$$
  $$5b^2 = (5k)^2$$
  $$5b^2 = 25k^2$$
  $$b^2 = 5k^2 \Rightarrow 5 \mid b^2 \Rightarrow 5 \mid b$$ -- spor s předpokladem, že $D(a, b) = 1$
  $$\sqrt{5} \in \mathbb{I}$$
\end{example}

\begin{example}[SÚM 109/37]
  Dokažte, že číslo $\sqrt{2} - 1$ je iracionální.

  Dk. sporem: Nechť $(\sqrt{2} - 1) \in \Q \Rightarrow \sqrt{2} - 1 = \frac{a}{b}, a, b \in \Z, D(a, b) = 1$.
  $$\sqrt{2} - 1 = \frac{a}{b}$$
  $$1 - 2\sqrt{2} = \frac{a^2}{b^2}$$
  $$\frac{b^2-a^2}{2b^2} = \sqrt{2} \Rightarrow \sqrt{2} = \frac{p}{q}, p,q \in \Z \Rightarrow \sqrt(2) \in \Q$$ -- spor
  $$(\sqrt{2} - 1) \in \mathbb{I}$$
\end{example}

\begin{example}[SÚM 109/38]
  Dokažte, že číslo $2\sqrt{5}$ je iracionální.

  Dk. sporem: Nechť $2\sqrt{5} \in \Q \Rightarrow 2\sqrt{5} = \frac{a}{b}, a, b \in \Z, D(a, b) = 1$.
  $$2\sqrt{5} = \frac{a}{b}$$
  $$10 = \frac{a^2}{b^2}$$
  $$10b^2 = a^2 \Rightarrow 10 \mid a^2 \Rightarrow 10 \mid a \Rightarrow \exists k: a = 10k$$
  $$10b^2 = (10k)^2$$
  $$10b^2 = 100k^2$$
  $$b^2 = 10k^2 \Rightarrow 10 \mid b^2 \Rightarrow 10 \mid b$$ -- spor s předpokladem, že $D(a, b) = 1$
  $$2\sqrt{5} \in \mathbb{I}$$
\end{example}

\begin{example}[SÚM 109/39]
  Dokažte, že jestliže přirozené číslo $m$ není druhou mocninou žádného přirozeného čísla, potom $\sqrt{m}$ je číslo iracionální.

  Dk. sporem: Nechť $m \in \N, m != n^2 \forall n \in \N, \sqrt{m} \in \Q \Rightarrow \sqrt{m} = \frac{a}{b}, a, b \in \N, D(a, b) = 1$.
  $$\sqrt{m} = \frac{a}{b}$$
  $$m = \frac{a^2}{b^2}, m \in \N \Rightarrow b^2 = 1$$
  $$m = a^2, a \in \N$$ -- spor s předpokladem, že $m != n^2 \forall n \in \N$
  QED
\end{example}

\begin{example}[SÚM 144/301]
  Dokažte, že:
  \begin{enumerate}[a.]
    \item součet dvou dvojciferných čísel přirozených, která se liší jen pořadím cifer, je dělitelný jedenácti: $$S = \overline{ab} + \overline{ba} = 10a + b + 10b + a = 11(a + b) \Rightarrow 11 \mid S$$
    \item rozdíl dvou dvojciferných čísel přirozených, která se liší jen pořadím cifer, je dělitelný devítí: $$S = \overline{ab} - \overline{ba} = 10a + b - 10b - a = 9(a - b) \Rightarrow 9 \mid S$$
    \item rozdíl přirozeného čísla trojciferného  a čísla, které vznikne z tohoto záměnou krajních cifer, je dělitelný 99: $$S = \overline{abc} - \overline{cba} = 100a + 10b + c - 100c - 10b - a = 99(a - c) \Rightarrow 99 \mid S$$
  \end{enumerate}
\end{example}

\begin{example}[SÚM 145/303]
  Dokažte, že tři mocniny čísla 2, jejichž exponenty jsou tři po sobě jdoucí přirozená čísla, mají součet dělitelný sedmi: $$S = 2^a + 2^{a+1} + 2^{a+2} = 2^a + 2^a*2^1 + 2^a*2^2 = 7*2^a \Rightarrow 7 \mid S$$
\end{example}

\begin{example}[SÚM 145/305]
  Dokažte, že součet třetích mocnin tří po sobě jdoucích přirozených čísel je dělitelný třemi: $$S = a^3 + (a+1)^3 + (a+2)^3 = a^3 + a^3 + 3a^2 + 3a + 1 + a^3 + 6a^2 + 12a + 8 = 3(a^3 + 3a^2 + 5a + 3) \Rightarrow 3 \mid S $$
\end{example}

\begin{example}[SÚM 145/306]
  Dokažte, že:
  \begin{enumerate}[a]
    \item číslo utvořené z rozdílu třetí mociny přirozeného čísla $n$ a tohoto čísla je dělitelné šesti: $$S = n^3 - n = n(n^2-1) = (n-1)n(n+1)$$ Jsou to tři po sobě jdoucí čísla $\Rightarrow$ právě 1 z nich je dělitelné třemi $\Rightarrow 3 \mid S$
    \item je-li číslo $n$ liché, je uvažovaný rozdíl dělitelný čísel 24: $$S = (n-1)n(n+1)$$ Jsou to tři po sobě jdoucí čísla a to prostřední je liché $\Rightarrow$ dělitelné třemi, z dalších čísel je jedno dělitelné 2 a jedno dělitelné čtyřmi: 2*4*3 = 24 $\Rightarrow 24 \mid S$
  \end{enumerate}
\end{example}

\begin{example}[SÚM 145/307]
  Dokažte, že je—li přirozené číslo $x$ liché, je výraz $V = x^3 + 3x^2 — x — 3$ dělitelný číslem 48: $$V = x^3 + 3x^2 — x — 3 = x^2(x+3) -(x+3) = (x^2 - 1)(x+3) = (x-1)(x+1)(x+3)$$ $\Rightarrow$ tři po sobě jdoucí sudá čísla $\Rightarrow$ jedno dělitelné 2, jedno 4 a jedno 6 $\Rightarrow 2*4*6 = 48 \Rightarrow 48 \mid V$
\end{example}

\begin{example}[SÚM 145/308]
  Dokažte, že výraz $V = 5x^3 + 15x^2 + 10x$ je dělitelný číslem 30 prokaždé přirozené číslo $x$: $$V = 5x^3 + 15x^2 + 10x = 5x(x^2 + 3x + 2) = 5x(x+2)(x+1)$$ $\Rightarrow$ $x$, $x+1$, $x+2$ tři po sobě jdoucí čísla $\Rightarrow$ jedno dělitelné 3, alespoň jedno dělitelné 2, $5x$ dělitelné 5 $\Rightarrow 2*3*5=30 \Rightarrow 30 \mid V$
\end{example}

\begin{example}[SÚM 145/312]
  Dokažte, že je-li $n$ číslo přirozené, je číslo $N = n^3 + 11n$ dělitelné šesti:
  mod 6:
  \begin{enumerate}
    \item $n = 6k$: $n^3 + 11n \equiv 0^3 + 11*0 \equiv 0 \Rightarrow 6 \mid N$
    \item $n = 6k + 1$: $n^3 + 11n \equiv 1^3 + 11*1 \equiv 12 \equiv 0 \Rightarrow 6 \mid N$
    \item $n = 6k + 2$: $n^3 + 11n \equiv 2^3 + 11*2 \equiv 30 \equiv 0 \Rightarrow 6 \mid N$
    \item $n = 6k + 3$: $n^3 + 11n \equiv 3^3 + 11*3 \equiv 60 \equiv 0 \Rightarrow 6 \mid N$
    \item $n = 6k + 4$: $n^3 + 11n \equiv 4^3 + 11*4 \equiv 108 \equiv 0 \Rightarrow 6 \mid N$
    \item $n = 6k + 5$: $n^3 + 11n \equiv 5^3 + 11*5 \equiv 180 \equiv 0 \Rightarrow 6 \mid N$
  \end{enumerate}
  $\Rightarrow 6 \mid N \forall n \in \N$
\end{example}

\begin{example}[SÚM 145/315]

\end{example}

\section{Výroková logika}
\begin{definition}
  \textbf{Výrokem} nazýváme každou oznamvací větu, která je buď pravdivá, nebo nepravdivá. \textbf{Pravdivostní hodnotou} výroku rozumíme jeho pravdivost / nepravdivost.
\end{definition}

\begin{definition}
  \textbf{Negací výroku} $V$ nazýváme výrok $V^\prime$, který má opačnou pravdivostní hodnotu než výrok $V$.
\end{definition}

\begin{pozn}
  \textbf{Kvantifikované výroky} jsou výroky, které uvádějí počet objektů. Pro to lze použít
  \begin{itemize}
    \item obecný kvantifikátor $\forall$ (pro všechno platí),
    \item existenční kvantifikátor $\exists$ (existuje alespoň jeden, že pro něj platí) a
    \item zesílený existenční kvantifikátor $\exists !$ (existuje právě jeden, že pro něj platí) a
  \end{itemize}
\end{pozn}

\begin{definition}
  \textbf{Složeným výrokem} rozumíme více výroků spojených logickými spojkami:
  \begin{center}
    \begin{tabular}{l | c c}
      název & zápis & význam \\
      \hline
      negace & $X^\prime$ & není pravda, že \\
      konjunkce & $X\land Y$ & $X$ a $Y$ platí současně \\
      alternativa & $X\lor Y$ & platí alespoň jedno z $X,Y$\\
      implikace & $X\implies Y$ & jestliže $X$, pak $Y$\\
      ekvivalence & $X\iff Y$ & $X$ platí právě tehdy, když platí $Y$
    \end{tabular}
  \end{center}
\end{definition}


\begin{example}[SMP 143/4]
  Jsou následující výroky tautologie?
  \rm
  \begin{enumerate}[a.]
    \item $\left[(A\implies B)\land A\right]\implies B$
    \begin{center}
      \begin{tabular}{c c | c c c}
        $A$ & $B$ & $A \implies B$ & $(A\implies B)\land A$ & $(A\implies B)\land A]\implies B$ \\
        \hline
        1 & 1 & 1 & 1 & 1 \\
        1 & 0 & 0 & 0 & 1 \\
        0 & 1 & 1 & 0 & 1 \\
        0 & 0 & 1 & 0 & 1
      \end{tabular}
    \end{center}
    Výrok je tautologií.
    \item $\left[(A\implies B)\land B^\prime\right]\implies A^\prime$
    \begin{center}
      \begin{tabular}{c c c c | c c c}
        $A$ & $B$ & $A^\prime$ & $B^\prime$ &  $A \implies B$ & $(A\implies B)\land B^\prime$ & $\left[(A\implies B)\land B^\prime\right]\implies A^\prime$ \\
        \hline
        1 & 1 & 0 & 0 & 1 & 0 & 1 \\
        1 & 0 & 0 & 1 & 0 & 0 & 1 \\
        0 & 1 & 1 & 0 & 1 & 0 & 1 \\
        0 & 0 & 1 & 1 & 1 & 1 & 1
      \end{tabular}
    \end{center}
    Výrok je tautologií.
  \end{enumerate}
\end{example}

\begin{example}[SMP 144/8]
  Na modelu kolejiště je možno uvést do pohybu tři vlakové soupravy A,B,C. V daném okamžiku je jejich
situace charakterizována formulí $$\left[(A^\prime \lor B^\prime) \implies C\right]\land\left[(A \lor C) \implies B^\prime\right].$$ Které soupravy jsou v pohybu?

\rm Napišme tabulku pravdivostních hodnot.
\begin{center}
  \begin{tabular}{c c c c c | c c c c c}
    $A$ & $B$ & $C$ & $A^\prime$ & $B^\prime$ & $A^\prime \lor B^\prime$ & $(A^\prime \lor B^\prime) \implies C$ & $A\lor C$ & $(A \lor C) \implies B^\prime $ & celkem\\
    \hline
    1 & 1 & 1 & 0 & 0 & 0 & 1 & 1 & 0 & 0 \\
    1 & 1 & 0 & 0 & 0 & 0 & 1 & 1 & 0 & 0 \\
    1 & 1 & 0 & 0 & 0 & 1 & 1 & 1 & 1 & 1 \\
    1 & 0 & 1 & 0 & 1 & 1 & 0 & 1 & 1 & 0 \\
    0 & 1 & 1 & 1 & 0 & 1 & 1 & 1 & 0 & 0 \\
    0 & 1 & 0 & 1 & 0 & 1 & 0 & 0 & 1 & 0 \\
    0 & 0 & 1 & 1 & 1 & 1 & 1 & 1 & 1 & 1 \\
    0 & 0 & 0 & 1 & 1 & 1 & 0 & 0 & 0 & 0
  \end{tabular}
\end{center}
Buď jsou v provozu soupravy $A$ a $C$ nebo jen souprava $C$.
\end{example}

\begin{example}[SMP 144/10]
  Květa si pozvala na oslavu svých osmnáctin přátele. Uvažuje takto:
  \begin{enumerate}[a.]
    \item Alena a Boris chodí vždycky spolu. Přijdou oba, nebo ani jeden.
    \item Přijde Boris nebo Dan, ale určitě ne oba.
    \item Když přijde Alena, pak přijde i Eva.
    \item Když Eva nepřijde, nepřijde Dan.
  \end{enumerate}
  S jakým největším počtem přátel může Květa počítat? Vyplývá z Květiny úvahy, že může nastat situace, kdy nepřijde ani jeden z pozvaných?

  \rm Přeložme tato tvrzení symbolicky.
  \begin{enumerate}[a.]
    \item $A\iff B$
    \item $B^\prime \lor D^\prime$
    \item $A\implies E$
    \item $E^\prime \implies D^\prime$
  \end{enumerate}
  Protože alespoň jeden z dvojice Boris, Dan nemůže přijít a zbytek podmínek si neodporují, na oslavu můžou přijít nejvýše tři lidé.

  Situace, že nepřijde nikdo nastat může.
\end{example}

\begin{example}[SMP 144/12]
  Trenér se věnuje trojici gymnastů -- Adamovi, Břéťovi a Čeňkovi. Rozhodněte, koho vyšle na kontrolní
závod, jestliže splní tyto tři podmínky:
\begin{itemize}
  \item Tělovýchovnou jednotu budou reprezentovat nejvýše dva závodníci, přitom pojede aspoň jeden.
  \item Pojede Adam nebo Čeňek, ale určitě ne oba součastně.
  \item Nepojede-li Čeňek, pak nepojede ani Břéťa.
\end{itemize}

\rm Rozdělme příklad na dva případy.
\begin{enumerate}[$i.$]
  \item pojede Adam: pak nepojede Čeněk a tedy ani Břéťa,
  \item pojede Čeněk: pak může jet i Břéťa.
\end{enumerate}

Buď pojede Adam sám nebo Čeněk sám nebo Čeněk s Břéťou.

\end{example}

\section{Dělitelnost přirozených čísel}
\begin{definition}
  Nechť $a,b\in\mathbb Z.$ Číslo $a$ dělí číslo $b$, jestliže $\exists c \in \mathbb Z: b=ac$. Zapisujeme $a\, | \, b$.
\end{definition}

\begin{definition}
  Nechť $a\in \mathbb R$. Číslo $|a|$ takové, že
  \begin{enumerate}[$i.$]
    \item $a\geq 0 \implies |a| = a$,
    \item $a<0 \implies |a| = - a$
  \end{enumerate}
  nazýváme \textbf{absolutní hodnotou} čísla $a$.
\end{definition}

\begin{veta}[O dělení se zbytkem]
  Nechť $a\in \mathbb Z, b\in \mathbb N.$ Pak $\exists ! q \in \mathbb Z, r\in \mathbb N_0:$
  $$a=bq+r, 0 \leq r < b.$$
\end{veta}

\begin{definition}
  Nechť $a,b\in \mathbb N$. Pak $c$ je \textbf{společným dělitelem} čísel $a,b$, jestliže $c \, | \, a \land c\, | \, b.$
\end{definition}

\begin{definition}
  $d\in \mathbb N$ je \textbf{největší společný dělitel} čísel $a,b \in \mathbb N,$ jestliže jsou splněny zároveň obě podmínky:
  \begin{enumerate}[$i.$]
    \item $d\, | \, a \land d \, | \, b$ a
    \item $\forall c \in \mathbb N: c \, | \, a \land c \, | \, b \implies c \, | \, d.$
  \end{enumerate}
  Takové číslo značíme $d=D(a,b)=(a,b).$
\end{definition}

\begin{definition}
  -- INSERT EUKLIDŮV ALGORITMUS --
\end{definition}

\begin{definition}
  Nechť $a,b\in \mathbb N.$ Tato čísla jsou \textbf{nesoudělná}, jestliže $D(a,b)=1$. V opačném případě jsou \textbf{soudělná}.
\end{definition}

\begin{veta}[Fundamentální věta aritmetiky]
  Nechť $a_1,a_2,b\in \mathbb N, b>1.$ Pak $b \, | \, a_1a_2 \land D(a_1,b)=1\implies b\, | \, a_2.$
\end{veta}

\begin{definition}
  Nechť $a,b\in \mathbb N.$ Pak $c$ je \textbf{společným násobek} čísel $a,b$, jestliže $a \, | \, c \land b\, | \, c.$
\end{definition}

\begin{definition}
  $n\in \mathbb N$ je \textbf{nejmenší společný násobek} čísel $a,b \in \mathbb N,$ jestliže jsou splněny zároveň obě podmínky:
  \begin{enumerate}[$i.$]
    \item $a\, | \, n \land b \, | \, n$ a
    \item $\forall m \in \mathbb N: a \, | \, m \land b \, | \, m \implies m \, | \, n.$
  \end{enumerate}
  Takové číslo značíme $n=n(a,b)=\left [ a,b\right ] .$
\end{definition}

\begin{veta}
  $\forall a,b \in \mathbb N: ab=D(a,b)\cdot n(a,b).$
\end{veta}

\begin{definition}
  Nechť $n\in \mathbb N, n>1.$ Má-li číslo $n$ pouze triviální dělitele ($1 \, | \, n, n \, | \, n$), nazýváme jej \textbf{prvočíslem}. V opačném případě hovoříme o \textbf{čísle složeném}.
\end{definition}

\begin{veta}
  Každé přirozené složené číslo $n$ má alespoň jednoho prvočíselného dělitele $p\leq \sqrt{n}$.
\end{veta}

\begin{veta}
  Prvočísel je nekonečně mnoho.
\end{veta}

\begin{veta}[Základní věta aritmetiky]
  Každé přirozené číslo $n>1$ lze zapsat ve tvaru:
  $$n=p_1^{m_1}\cdot p_2^{m_2} \cdot p_3^{m_3}\cdot \hdots \cdot p_r^{m_r},$$
  kde $p_i,i\in\{ 1, 2, \dots, r \}$ jsou navzájem různá prvočísla, $m_i\in \mathbb N_0$. Toto vyjádření je jednoznačné až na pořadí činitelů a říkáme mu \textbf{rozklad čísla} $n$ \textbf{na součin prvočinitelů}.
\end{veta}

\begin{veta}[Věta o iraciálnosti odmocnin]
  Nechť $n\in \mathbb N.$ Pak platí: Pokud $n$ není druhou mocninou přirozeného čísla, pak odmocnina z $n$ je iracionální.
\end{veta}

\subsection*{Kritéria dělitelnosti}
\begin{věta}
  Nechť $n\in \mathbb N, n=a_k\cdot 10^k+a_{k-1}\cdot 10^{k-1}+\dots + a\cdot 10 + a_0.$ Pak platí:
  \begin{enuemrate}[$i.$]
    \item $2 \, | \, n \iff 2 \, | \, a_0$,
    \item $4 \, | \, n \iff 4 \, | \, (10a_1 + a_0)$,
    \item $5 \, | \, n \iff 5 \, | \, a_0$,
    \item $8 \, | \, n \iff 8 \, | \, (10^2a_2 + 10a_1 + a_0)$ a
    \item $10 \, | \, n \iff a_0 = 0$.
  \end{enuemrate}
\end{věta}

\begin{proof}
  $$n = 10(a_k\cdot 10^{k-1}+a_{k-1}\cdot 10 ^{k-2}+\dots+a_1)+a_0 = 10l+a_0, l\in \mathbb N$$
\end{proof}

\begin{definition}
  Nechť $n\in \mathbb N, n=a_k\cdot 10^k+a_{k-1}\cdot 10^{k-1}+\dots + a\cdot 10 + a_0.$ Pak číslo
  $$S(n) = \sum_{i=0}^k a_i$$
  nazveme \textbf{ciferným součtem} čísla $n$.
\end{definition}

\begin{veta}
  Nechť $n\in \mathbb N, n=a_k\cdot 10^k+a_{k-1}\cdot 10^{k-1}+\dots + a\cdot 10 + a_0,$ $S(n)$ je ciferný součet čísla $n$. Pak platí:
  \begin{enumerate}[$i.$]
    \item $3\, | \, n \iff 3 \, | \, S(n)$ a
    \item $9\, | \, n \iff 9 \, | \, S(n)$
  \end{enumerate}
\end{veta}

\begin{proof}
  už se mi nechce
\end{proof}

\section{Základní pojmy z planimetrie, rovinné útvary, úhly v nich}
\begin{definition}
  Nechť $A,B,C\in \mathbb E_2$ jsou tři body. Jestliže všechny tři leží (resp. neleží) na jedné přímce, řekneme, že jsou \textbf{kolineární} (resp. \textbf{nekolineární}).
\end{definition}

\begin{definition}
  Nechť $p,q\in \mathscr P$. Jestliže platí:
  \begin{enumerate}[$i.$]
    \item $p=q$, pak přímky $p,q$ se nazývají \textbf{splývající rovnoběžky},
    \item $p \ne q \land p\cap q = \emptyset$, pak přímky $p,q$ se nazývají \textbf{různé rovnoběžky},
    \item $p \ne q \land p\cap q \ne \emptyset$, pak přímky $p,q$ se nazývají \textbf{různoběžky}.
  \end{enumerate}
\end{definition}

\begin{definition}
  Nechť $p\in \mathscr P,A \in p, B \in p, A \ne B.$ Množinu
  \[
    P(A)=\left \{ X\in \mathbb E_2; X=B \lor X\,\mu\, AB \lor B\, \mu\, AX \right \}
  \]
  (resp. množinu $P(A)\cup \{A\}$) nazýváme \textbf{otevřenou} (resp. \textbf{uzavřenou}) \textbf{polopřímkou} $AB$ s počátkem v bodě $A$. Množinu
  \[
    Q(A)=\left \{ X\in \mathbb E_2; A\,\mu\, BX \right \}
  \]
  (resp. množinu $Q(A)\cup \{A\}$) nazýváme \textbf{otevřenou} (resp. \textbf{uzavřenou}) \textbf{polopřímku opačnou} k polopřímce $AB$ s počátkem $A$.
\end{definition}

\begin{definition}
  Nechť body $A,B\in \mathbb E_2, A\ne B$. Průnik uzanřených polopřímek $AB$ a $BA$ nazveme \textbf{úsečkou} $AB$. Body $A,B$ se nazývají \textbf{krajní body úsečky} $AB$, bod $X\,\mu \, AB$ se nazývá \textbf{vnitřní bod} úsečky $AB$.
\end{definition}

\begin{definition}
Nechť $a\in \mathscr P, A\notin a, B\notin a, A\ne B.$ Pak \textbf{přímka} $a$ \textbf{odděluje body} $A,B$ a zapisujeme $a\, \nu\, AB.$ V opčaném případě \textbf{přímka} $a$ \textbf{neodděluje body} $A,B$. Zapisujeme $a\, \overline \nu \,AB.$
\end{definition}

\begin{definition}
  Nechť $a \in \mathscr P$. Pak všechny $X\in \mathbb E_2 - a$ lze rozdělit do dvou podmnožin $P(a), Q(a)$ tak, že:
  \begin{enumerate}[$i.$]
    \item přímka $a$ odděluje každé dva body z různých podmnožin:
      \[
        \forall x \in P(a), \forall Y \in Q(a): a \, \nu \, XY
      \]
    \item přímka $a$ neodděluje žádné dva body z jedné podmnožiny
      \[
        \forall X, Y \in P(a) \land \forall X,Y \in Q(a): a \,\overline \nu\, XY.
      \]
  \end{enumerate}
  Pak množinu $P(a)$ (resp. množinu $Q(a)$) nazveme \textbf{otevřenou polorovinou s hraniční přímkou} $a$ (resp. \textbf{otevřenou polorovinou s hraniční přímkou} $a$ \textbf{opačnou k} $P(a)$). Množinu $P(a)\cup a$ (resp. $Q(a) \cup a$) nazveme \textbf{uzavřenou polorovinou s hraniční přímkou} $a$ (resp. \textbf{uzavřenou polorovinou s hraniční přímkou} $a$ \textbf{opačnou k} $P(a)$).
\end{definition}

\begin{definition}
  Nechť $A,B,V \in \mathbb E_2$ jsou tři různé nekolineární body. Průnik polorovin $VBA \cap VAB$ nazveme \textbf{konvexním úhlem} (zapisujeme $\sphericalangle BVA$), $V$ jeho \textbf{vrcholem}, polopřímky $VA, VB$ jeho \textbf{rameny}. \textbf{Nekonvexním úhlem} $BVA$ nazveme sjednocení polorovin opačných k polorovinám $VBA,VAB.$
\end{definition}

\begin{definition}
  Nechť $A,B,C\in \mathbb E_2$ jsou tři různé nekolineární body. \textbf{Trojúhelníkem} $ABC$ (značíme $\triangle ABC$) nazýváme \textbf{vrcholy} tohoto trojúhelníka, úsečky $AB, BC, CA$ jeho \textbf{stranami}.
\end{definition}

\begin{definition}
  Nechť $A,B\in \mathbb E_2.$ Přiřaďme uspořádané dvojici bodů $(A,B)$ reálné číslo označené $|AB|,$ pro něž platí:
  \begin{enumerate}[$i.$]
    \item $|AB|\geq 0,$ přičemž $|AB|=0 \iff A=B,$
    \item $|AB|=|BA|,$
    \item $C\, \mu \, AB \implies |AB|=|CB|+|AC|,$
    \item nechť $AB$ je polopřímka, $m\in \mathbb R^+_0.$ Pak $\exists ! C\in AB$ tak, že $|AC|=m.$
  \end{enumerate}
\end{definition}

\begin{definition}
  Každému konvexnímu úhlu $\sphericalangle AVC$ přiřaďme \textbf{velikost úhlu} (ozn. $|\sphericalangle AVC|$) ve stupních tak, že
  \begin{enumerate}[$i.$]
    \item nulový úhel má velikost $0^\circ$, přímý $180^\circ$,
    \item každý jiný konvexní úhel má velikost $n^\circ, $ kde $0^\circ< n ^\circ< 180^\circ, u \in \mathbb R,$
    \item jestliže polopřímka $VB$ prochází mezi rameny konvexního úhlu $\sphericalangle AVC$, pak $|\sphericalangle AVC|=|\sphericalangle AVB|+ |\sphericalangle BVC|$ a
    \item nechť $VA$ je polopřímka, $u \in \mathbb R, u \in \left < 0 ^\circ,180^\circ \right >$. Pak existuje polopřímka $VB$ taková, že $|\sphericalangle AVB|=u^\circ.$
  \end{enumerate}
\end{definition}

\begin{pozn}
  \textbf{Radián} je středový úhel příslušný v jednotkové kružnici kruhovému oblouku délky 1. Z definice plyne:
  \[
    360^\circ = 2\pi.
  \]
\end{pozn}

\begin{definition}
  Nechť $p,q\in \mathscr P$ jsou dvě různoběžky. Řekneme, že $p,q$ jsou na sebe \textbf{kolmé} (a zapisujeme $p\perp q$), jestliže všechny čtyři úhly, které spolu svírají, jsou shodné (a tedy pravé).
\end{definition}

\begin{pozn}[Klasifikace úhlů podle velikosti]
  Nechť $\alpha$ je velikost úhlu $|\sphericalangle ABC|$. Potom je úhel $|\sphericalangle ABC|$
  \begin{itemize}
    \item nulový, pokud $\alpha=0^\circ$,
  \item ostrý, pokud $0^\circ <\alpha < 90^\circ,$
  \item pravý, pokud $\alpha = 90^\circ,$
  \item tupý, pokud $90^\circ < \alpha < 180^\circ,$
  \item přímý, pokud $\alpha = 180^\circ,$
  \item konvexní, pokud $0^\circ \leq \alpha \leq 180^\circ,$
  \item nekonvexní, pokud $180^\circ < \alpha < 360^\circ$,
  \item plný, pokud $\alpha = 360^\circ,$
  \item kosý, pokud je ostrý nebo tupý.
  \end{itemize}
\end{pozn}

\begin{definition}
  Nechť $A,B\in \mathbb E_2$. \textbf{Vzdáleností bodů} $A,B$ nazveme délku úsečky $AB.$
\end{definition}


\begin{definition}
  Nechť $P\in \mathbb E_2, p \in \mathscr P.$ \textbf{Vzdáleností bodu} $P$ \textbf{od přímky} $p$ nazveme reálné číslo označené $\rho(P,p)$ takové, že $\rho(P,p)=|PP_0|,$ kde $P_0$ je kolmý průmět bodu $p$ na přímku $p$.
\end{definition}

\begin{definition}
  Nechť $a,b \in \mathscr P, a \parallel b.$ \textbf{Vzdáleností dvou přímek} $a,b$ nazveme reálné číslo označené $\rho(a,b)$ takové, že $\rho(a,b)=\rho(A,b),$ kde $A\in a$ je libovolný bod.
\end{definition}

\begin{definition}
  Nechť $a,b\in \mathscr P.$ \textbf{Odchylkou přímek} $a,b$ nazveme reálné číslo $\varphi^\circ\in \left <0, 180\right>$, kde $\varphi$ je velikost úhlu, který spolu přímky $a,b$ svírají. U rovnoběžek klademe $\varphi = 0^\circ.$
\end{definition}

\begin{definition}
  Dvojice úhlů:
  \begin{itemize}
    \item \textbf{vrcholové úhly} -- dvojice úhlů, jejichž ramena jsou opačné polopřímky
    \item \textbf{vedlejší úhly} --	dvojice úhlů, jejichž jedno rameno je společné a druhá ramena jsou opačné polopřímky
    \item \textbf{souhlasné úhly} -- dvojice úhlů, jejichž první ramena leží na jedné přímce a druhá ramena jsou rovnoběžná, přitom směr příslušných ramen je stejný
    \item \textbf{střídavé úhly}	-- dvojice úhlů, jejichž první ramena leží na jedné přímce a druhá ramena jsou rovnoběžná, přitom směr příslušných ramen je opačný
    \item \textbf{přilehlé úhly} -- dvojice úhlů, jejichž první ramena leží na jedné přímce a jdou do opačných směrů a druhá ramena jsou rovnoběžná
  \end{itemize}
\end{definition}

\begin{definition}
  Dvě polopřímky, ležící na téže přímce nazývámě \textbf{souhlasnými}, jestliže jedna z nich je podmnožinou druhé. V opačném případě je nazveme \textbf{nesouhlasnými}.
\end{definition}

\begin{definition}
  Nechť $\triangle ABC$. \textbf{Vnitřním úhlem} $\triangle ABC$ při vrcholu A (B, C) nazýváme $\sphericalangle CAB (ABC, BCA)$, \textbf{vnějším úhlem} $\triangle ABC$ při vrcholu A (B, C) pak vedlejší úhel k úhlu $\sphericalangle CAB (ABC, BCA)$.
\end{definition}

\begin{veta}
  Součet všech vnitřních úhlů v trojúhelníku je $180^\circ$.
\end{veta}

\begin{proof}
  \begin{figure}[h]
      \includegraphics[width=\linewidth]{trojuhelnik_1.png}
  \end{figure}
\end{proof}

\begin{veta}
  Součet libovolných dvou vnitřních úhlů v trojúhelníku je menší něž $180^\circ$.
\end{veta}

\begin{proof}
  Plyne z předchozí věty a tvrzení, že vnitřní úhel v trojúhelníku je nenulový.
\end{proof}

\begin{veta}
  V každém trojúhelníku je velikost vnějšího úhlu při jednom vrcholu rovna součtu velikostí dvou zbylých vnitřních úhlů.
\end{veta}

\begin{proof}
  $$\alpha + \alpha^\prime = 180^\circ$$ (úhly vedlejší)
  $$\alpha + \beta + \gamma = 180^\circ$$
  $$\Rightarrow \alpha^\prime = \beta + \gamma$$
\end{proof}

\begin{definition}
  Trojúhelník se nazývá \textbf{rovnoramenný}, jestliže alespoň dvě jeho strany jsou shodné úsečky. Strany stejné délky nazveme \textbf{rameny}, tu třetí \textbf{základnou} a vrchol proti základně \textbf{vrcholem}.
  Trojúhelník se nazývá \textbf{rovnostranný}, jestliže všechny tři jeho strany jsou shodné úsečky.
  Trojúhelník se nazývá \textbf{pravoúhlý}, jestliže je jeden jeho vnitřní úhel pravý. Dvě strany, které jsou rameny pravého úhlu nazveme \textbf{odvěsnami}, tu třetí \textbf{přeponou}.
\end{definition}

\begin{definition}
  \textbf{Střední příčkou} trojúhelníku nazýváme úsečku spojující středy dvou stran trojúhelníka.
  \textbf{Těžnicí} trojúhelníku nazýváme úsečku spojující vrchol trojúhelníku se středem protější strany. Průsečík těžnic nazýváme \textbf{těžíště}.
  \textbf{Výškou} trojúhelníku nazýváme úsečku procházející vrcholem trojúhelníka, která je kolmá na přímku, na které leží protější strana trojúhelníku. Průsečík výšek nazýváme \textbf{ortocentrum}.
\end{definition}

\begin{veta}
  Každá střední příčka je rovnoběžná s protilehlou stranou a je dvakrát menší než protilehlá strana.
\end{veta}

\begin{proof}
  Trojúhelníky vrchol - střední příčka a vrchol - protilehlá strana jsou zjevně podobné s koeficientem 2.
\end{proof}

\begin{veta}
  Těžnice trojúhelníku se všechny protínají v jednom bodě, který je dělí v poměru $2:1$.
\end{veta}

\begin{proof}
  Doslova nevim, ve skriptách to není.
\end{proof}

\begin{veta}

\end{veta}

\section{Kartézský součin, binární relace, zobrazení}
\begin{definition}
  \textbf{Kartézským součinem množin} $A,B$ nazýváme množinu $A\times B$ všech uspořádaných dvojic $(a,b)$ takových, že $a\in A,b\in B$.
  \[
    A \times B = \left \{ (a,b); a\in A,b\in B \right \}.
  \]
\end{definition}

\begin{pozn}
  Kartézský součin $A\times A$ nazveme \textbf{kartézským čtvercem} množiny $A$.
\end{pozn}

\begin{pozn}
  Kartézský graf je graf, kde prvky na ose $x$ (resp. $y$) jsou prvky z množiny $A$ (resp. $B$) a uspořádanou dvojici $(a,b)$ zaneseme jako příslušný bod se souřadnicemi $[a,b].$
\end{pozn}

\begin{definition}
  Nechť $A,B$ jsou dvě množiny. Pak každou podmnožinu kartézského součinu $A\times B$ nazýváme \textbf{binární relací} mezi množinami $A,B$
  (v tomto pořadí). Je-li speciálně $A=B$, pak hovoříme o \textbf{binární relaci v množině} $A$.
\end{definition}

\begin{definition}
Nechť $\alpha\subseteq A\times B$ je binární relace. Je-li uspořádaná dvojice $(a,b), a \in A, b \in B$ prvkem množiny $\alpha$,
říkáme, že $a$ \textbf{je v relaci s} $b$ a píšeme $a \sim b$.
\end{definition}

\begin{priklad}
Znázorněte graf relace $U=\left \{ \left [ x,y \right ]\in \mathbb R^2: x-y+1=0  \right \}. $
\end{priklad}

\begin{reseni}
Řešením je přímka $y=x+1.$
\end{reseni}

\begin{definition}
  Relaci na množině $A$ nazveme
  \begin{enumerate}[$i.$]
    \item \textbf{reflexivní}, pokud $\forall a\in A: a\sim a$,
    \item \textbf{symetrickou}, pokud $\forall a,b \in A: a\sim b \implies b\sim a,$
    \item \textbf{tranzitivní}, pokud $\forall a,b,c\in A: a\sim b \land b\sim c \implies a\sim c.$
  \end{enumerate}
\end{definition}

\begin{definition}
  Relaci na množině $A$, která je zároveň reflexivní, symetrická a tranzitivní, nazveme \textbf{relací ekvivalence}.
\end{definition}

\begin{pozn}
  Každé relaci ekvivalence na množině $A$ přísluší \textbf{rozklad příslušný ekvivalenci} tak, že množinu $A$ rozdělíme na po dvou disjunktní podmnožiny,
  jejichž sjednocení dává množinu $A$ a navíc platí, že všechny prvky v jedné podmnožině jsou navzájem ekvivalentní a~žádné dva prvky z jiných podmnožin ekvivalentní nejsou. Tyto podmnožiny potom nazveme \textbf{třídami rozkladu}.
\end{pozn}

\begin{definition}
  Nechť $A,B$ jsou dvě množiny. \textbf{Zobrazením} $f$ \textbf{z množiny $A$ do množiny $B$} nazýváme relaci $f\subseteq A \times B,$ pro níž platí: $\forall x \in A: \exists \text{ max. 1 } y \in B: (x,y) \in f$. Prvek $x$ nazveme \textbf{vzorem} a $y$ \textbf{obrazem}.
\end{definition}

\begin{definition}
  Nechť $f\subseteq A\times B$ je zobrazení. \textbf{Definičním oborem} (resp. \textbf{oborem hodnot}) zobrazení $f$ nazveme množinu $D(f)\subseteq A$
  (resp. $H(f)\subseteq B$) všech prvků $a\in A$ (resp. $b\in B$) takových, že k nim existuje právě jedno $b\in B$ (resp. alespoň jedno $a\in A$) tak, že $b=f(a)$.
\end{definition}

\begin{definition}
  Pokud $D(f) = A$ (resp. $D(f)\ne A$), hovoříme o \textbf{zobrazení množiny} (resp. \textbf{zobrazení z množiny}) $A$. Pokud $H(f)=B$ (resp. $H(f)\ne B$), hovoříme o \textbf{zobrazení na množinu} (resp. \textbf{zobrazení do množiny}) $B$.
  Zobrazení množiny na množinu nazýváme \textbf{surjekcí}. Je-li $A=B$ (resp. $A=B \land ( A = D(f) \lor B = H(f))$), hovoříme o \textbf{zobrazení v množině} (resp. \textbf{zobrazení na množině}) $A$.
\end{definition}

\begin{definition}
  Nechť $f$ je zobrazení z $A$ do $B$. Zobrazení $f$ je \textbf{prosté} neboli \textbf{injektivní}, jestliže ke každému $b\in B$ existuje nejvýše jedno $a \in A$ takové, že $b=f(a).$
\end{definition}

\begin{definition}
  Zobrazení, které je současně surjekcí a injekcí, nazýváme \textbf{bijekce}.
\end{definition}

\begin{definition}
  Nechť $\alpha \subseteq A\times B$ je binární relace. \textbf{Inverzní relací} k relaci $\alpha$ nazveme relaci $\alpha^{-1} \subseteq B\times A$ takovou, že
  \[
    \alpha^{-1}=\left\{ (b,a)\in B\times A;  (a,b)\in \alpha\right\}.
  \]
  Je-li $\alpha^{-1}$ zobrazení, nazveme relaci $\alpha ^{-1}$ \textbf{inverzním zobrazením} k zobrazení $\alpha$.
\end{definition}

\begin{definition}
  Nechť $f:B\to C, g:A\to B$ jsou zobrazení. \textbf{Složeným zobrazením} ze zobrazení $f$ a $g$ nazveme zobrazení $h: A\to C$ takové, že
  \[
    h=\left\{ (a,c)\in A\times C;\exists b\in B: f(b)=c \land g(a)=b \right\}.
  \]
  Značíme $c=f(g(a))$, $h=f\, \circ \, g $ (čteme \uv{$f$ po $g$}).
\end{definition}

\section{Funkce a jejich základní vlastnosti}
\begin{definition}
  \textbf{Funkcí} $f$ nazýváme každé zobrazení z $\mathbb R$ do $\mathbb R$.
\end{definition}

\begin{definition}
  Nechť $f\subseteq \mathbb R \times \mathbb R$ je funkce. Množinu
  \[
    D(f) = \left  \{ x \in \mathbb R:\exists ! y \in \mathbb R:y=f(x) \right \}
  \]
  nazveme \textbf{definičním oborem} funkce $f$. Množinu
  \[
    H(f) = \left  \{ y \in \mathbb R:\exists ! x \in \mathbb R:y=f(x) \right \}
  \]
  nazveme \textbf{oborem hodnot} funkce $f$.
\end{definition}

\begin{pozn}
  \textbf{Grafem funkce} rozumíme množinu všech bodů $[x,f(x)]$, kde $x\in D(f).$
\end{pozn}


\begin{definition}
  Funkce $f$ se nazývá \textbf{sudá} (resp. \textbf{lichá}), jestliže platí
\begin{itemize}
  \item
  $\forall x \in D(f):(-x) \in D(f)$
\item
 $\forall x \in D(f): f(-x)=f(x)$, resp. $f(-x)=f(x)$.
\end{itemize}

\end{definition}

\begin{definition}
  Funkce $f$ se nazývá \textbf{prostá}, právě tehdy když platí
  \[
    \forall x_1,x_2\in D(f): x_1\ne x_2 \implies f(x_1)\ne f(x_2)
  \]
\end{definition}

\begin{definition}
  Nechť $f$ je funkce a $M$ alespoň dvouprvková množina z $D(f)$. Řekneme, že funkce $f$ je v množině $M$
  \begin{enumerate}[$i.$]
    \item \textbf{rostoucí} $\iff \forall x_1, x_2 \in M: x_1 < x_2 \implies f(x_1) < f(x_2),$
    \item \textbf{klesající} $\iff \forall x_1, x_2 \in M: x_1 < x_2 \implies f(x_1) > f(x_2),$
    \item \textbf{neklesající} $\iff \forall x_1, x_2 \in M: x_1 < x_2 \implies f(x_1) \leq f(x_2),$
    \item \textbf{nerostoucí} $\iff \forall x_1, x_2 \in M: x_1 < x_2 \implies f(x_1) \geq f(x_2).$
  \end{enumerate}
  Je-li $f$ neklesající nebo nerostoucí, je \textbf{monotónní}. POkud je klesající nebo rostoucí, je \textbf{ryze monotónní}.
\end{definition}

\begin{definition}
  Nechť $f$ je funkce, $M\subseteq D(f)$. Řekneme, že funkce $f$ je v množině $M$
  \begin{enumerate}[$i.$]
    \item \textbf{shora omezená} $\iff \exists k \in \mathbb R: \forall x \in M: f(x)\leq k,$
    \item \textbf{zdola omezená} $\iff \exists k \in \mathbb R: \forall x \in M: f(x)\geq k,$
    \item \textbf{omezená} $\iff $ je zdola i shora omezená.
  \end{enumerate}
\end{definition}


\begin{definition}
  Nechť $f$ je funkce, $M \subseteq D(f)$, v ní prvek $a \in M$.
  Řekneme, že funkce f má v bodě a:
  \begin{enumerate}[i.]
    \item \textbf{ostré maximum} na množině $M$ právě tehdy, když $\forall x \in M; x \not = a: f(x) < f(a)$
    \item \textbf{maximum} (neostré) na množině $M$ právě tehdy, když $\forall x \in M : f(x) \leq f(a)$
    \item \textbf{ostré minimum} na množině $M$ právě tehdy, když $\forall x \in M; x > a: f(x) > f(a)$
    \item \textbf{minimum} (neostré) na množině $M$ právě tehdy, když $\forall x \in M : f(x) \geq f(a)$
  \end{enumerate}
\end{definition}

\begin{definition}
  Nechť $f$ je funkce. Funkce $f$ se nazývá \textbf{periodická}, pokud $\forall p \in \mathbb R^{+}: \forall x \in D(f):$
  \begin{enumerate}
    \item $x \in D(f) \implies x \pm p \in D(f)$
    \item $f(x) = f(x \pm p)$
  \end{enumerate}
  Číslo $p$ se nazývá \textbf{periodou} této funkce. Periodu $p_0$ nazveme \textbf{nejmenší periodou} funkce, pokud pro všechny ostatní periody $p$ platí $p > p_0$. V opačném případě se funkce nazývá \textbf{neperiodická}.
\end{definition}

\begin{definition}
  Máme funkci $f: y = f(u)$ s definičním oborem $D(f)$ a funkci $g: u=g(x)$ s oborem hodnot $H(g)$. Jestliže je $H(g) \subseteq D(f)$, pak funkci $h: y = f(g(x))$ nazveme \textbf{složenou funkcí} (někdy píšeme též $h=f \circ g$).
\end{definition}

\begin{definition}
  \textbf{Dirichletova funkce} je definována vzorcem
  $$\mathbf{D} (x) = \begin{cases}
1 \text{ pokud } x \in \mathbb Q \\
0 \text{ jinak }
  \end{cases}  $$
\end{definition}

\begin{definition}
  Funkce \textbf{signum} je definováno následujícím způsobem $$\operatorname {sgn} x={\begin{cases}-1,&x<0\\0,&x=0\\1,&x>0\end{cases}}$$
\end{definition}

\begin{definition}
  Nechť $x \in \mathbb{R}$ je libovolné číslo. Pak existuje právě jedna dvojice $z \in \mathbb{Z}, a \in \left \langle 0;1 \right) \text{ tak, že } x = z + a$.
Číslo z nazýváme \textbf{celou částí} čísla x a zapisujeme $[x] = z (\lfloor x \rfloor = z)$.
\end{definition}

\begin{definition}
  O funkci $f:\mathbb {R} \rightarrow \mathbb {R}$ řekneme, že je \textbf{spojitá} v bodě $a$, pokud ke každému libovolně malému číslu $\varepsilon >0$ existuje takové číslo $\delta >0$, že pro všechna $x$, pro něž platí $|x-a|<\delta$, platí také $|f(x)-f(a)|<\varepsilon$.
\end{definition}

\begin{definition}
  Číslo $A\in \mathbb {R}$ je limitou funkce $f:\mathbb {R} \rightarrow \mathbb {R}$ v bodě $ a\in \mathbb {R}$, jestliže k libovolnému $ \varepsilon >0$ existuje takové $ \delta >0$, že pro všechna $x\in D(f)$ taková, že $ \left|x-a\right|<\delta$ ($x$ leží v prstencovém okolí bodu $a$) platí $\left|f(x)-A\right|<\varepsilon $.

  Limitu má smysl zkoumat jen v definičním oboru funkce neobsahujícím bod $a$, tj. libovolně blízko k bodu $a$ musí být funkce definována.
\end{definition}

\begin{definition}
  Nejběžnější moderní definice \textbf{derivace} funkce $f$ v bodě $a$, zapisujeme $f'(a)$ je
  $$f'(a)=\lim _{h\to 0}{\frac {f(a+h)-f(a)}{h}}=\lim _{x\to a}{\frac {f(x)-f(a)}{x-a}}.$$

  Co to znamená se mě neptejte (klidně se zeptejte). Vložte intuitivní definici derivace.
\end{definition}

\begin{definition}
  Nechť $f$ je funkce spojitá na intervalu $(a,b)$. Pak říkáme, že funkce $f$ je na intervalu $(a,b)$ \textbf{konvexní} (resp. \textbf{ryze konvexní}) právě tehdy, když pro libovolné číslo $\lambda \in (0,1)$ s vlastností $\forall x,y\in (a,b),x<y:f(\lambda x+(1-\lambda )y) < (\text{resp. } \leq) \lambda f(x)+(1-\lambda )f(y)$

  Pokud spojitá funkce není na intervalu konvexní (resp. ryze konvexní), je na něm ryze konkávní (resp. konkávní).
\end{definition}

\begin{definition}
  \textbf{Primitivní funkce} k funkci $f$ na intervalu $(a,b)$ je taková funkce $F$, že pro každé $x\in (a,b)$ je $F'(x)=f(x)$.

  Procesu hledání primitivní funkce se často říká \textbf{integrování} nebo \textbf{integrace} (od slova integrál).
\end{definition}

\section{Polynomy, kořeny polynomů}
\begin{definition}
  \textbf{Polynomem} nazýváme každý výraz $P(x)$ tvaru
  \[
    P(x)=a_nx^n + a_{n-1}x^{n-1}+\dots + a_2x^2+a_1x+a_0,
  \]
  kde $a_i\in \mathbb R, i\in \{ 0,1,\dots , n \},n\in \mathbb N$. Čísla $a_i$ se nazývají \textbf{koeficienty polynomu}, sčítance $a_ix^i$ \textbf{členy polynomu}. Je-li $a_n\ne 0$, číslo $n$ se nazývá \textbf{stupeň polynomu} a označuje se $n= \text{st}(P(x))$. Je-li $a_i=0$ pro všechna $i \in \{ 0,1,\dots ,n\}$, pak klademe $\text{st}(P(x))=-\infty$ a $P(x)$ se nazývá \textbf{nulovým polynomem}. Označujeme jej $O(x).$ Je-li $a_n=1$, nazývá se $P(x)$ \textbf{normovaným polynomem}.
\end{definition}

\begin{veta}[O dělení polynomů se zbytkem]
  Nechť $A(x), B(x) \in \mathbb R[x],B(x) \ne O(x).$ Pak existuje právě jedna dvojice polynomů $Q(x),R(x)\in \mathbb R[x]$ tak, že
  \[
    A(x)=B(x)\cdot Q(x)+R(x),
  \]
  kde $R(x)=O(x)$ nebo $\text{st}(R(x))<\text{st}(B(x)).$
\end{veta}

\begin{definition}
  Nechť $A(x), B(x) \in \mathbb R[x]$. Řekneme, že polynom $B(x)$ \textbf{dělí} polynom $A(x)$ právě tehdy, když existuje takový polynom $C(x) \in \mathbb R[x]$ tak, že
  \[
    A(x) = B(x)\cdot C(x).
  \]
  Zapisujeme $B(x) \, | \, A(x).$
\end{definition}

\begin{definition}
  Nechť $P(x)\in \mathbb R[x], P(x)=a_nx^n+a_{n-1}x^{n-1}+\dots +a_1x+a_0.$ Nechť $c\in \mathbb R$ je libovolné číslo. \textbf{Hodnotou polynomu} $P(x)$ \textbf{v čísle} (v bodě) $c$ nazýváme reálné číslo $P(c)$
  \[
    P(c) = a_nc^n+a_{n-1}c^{n-1}+\dots+a_1c+a_0.
  \]
  Číslo $c\in \mathbb R$ nazveme \textbf{kořenem polynomu} $P(x) \iff P(c) = 0$.
\end{definition}


\begin{definition}
  Nechť $P(x) \in \mathbb R [x]$ a $c\in \mathbb R$ je jeho kořen. Lineární polynom $x-c$ nazveme \textbf{kořenovým činitelem}.
\end{definition}

\begin{definition}
  Nechť $P(x) \in \mathbb R [x], c \in \mathbb R$ je jeho kořen a $k \in \mathbb N$. Číslo $C \in \mathbb R$ nazýváme \textbf{$k$-násobným kořenem} polynomu $P(x)$ právě tehdy, když platí
  \[
    (x-c)^k \, | \, P(x) \land (x-c)^{k+1} \nmid  P(x).
  \]
\end{definition}

\begin{veta}[Vi\`{e}tovy vztahy]
  Nechť $P(x)=a_nx^n+a_{n-1}x^{n-1}+\dots + a_1x+a_0$ je polynom stupně $n$, který má v množině $\mathbb R$ právě $n$ kořenů $x_1,x_2\dots,x_n$ (každý počítáme tolikrát, jaká je jeho násobnost). Pak platí:
  \begin{align*}
    \sum_{i=1}^n x_i = & -\frac{a_{n-1}}{a_n} \\
    \sum_{i,j=1; i<j}^{n}x_ix_j= & \frac{a_{n-2}}{a_n} \\
    \vdots & \\
    \prod_{i=1}^nx_i=&(-1)^n\frac{a_0}{a_n}
  \end{align*}
\end{veta}

\begin{pozn}
  \textbf{Hornerovo schéma} je numerická metoda pro vyhodnocení funkční hodnoty polynomu $P(x) \in \mathbb R [x]$ v bodě $ x_0 \in \mathbb R$. Příklad:
  Vyhodnoťte $f_{1}(x)=2x^{3}-6x^{2}+2x-1\,$ v bodě $x=3\;$.
  Opakovaným vytknutím $x$, může být $f_{1}$ zapsáno jako $x(x(2x-6)+2)-1\;$. Pro větší přehlednost užijeme k zápisu průběhu výpočtu tzv. syntetický diagram.
  \begin{tabular}{ c|c c c c }
    $x_{0}$ & $x^{3}$ & $x^{2}$ & $x^{1}$ & $x^{0}$\\
    \hline
    $3$ & $2$ & $0$ & $2$ & $5$
  \end{tabular}

  Do prvního místa opíšeme $a_n$. Čísla v řádku jsou součty koeficientu $a_k$ součinu hodnoty $x$, v níž polynom vyhodnocujeme (v tomto příkladě tedy $3$) s číslem v řádku o jeden sloupec vlevo (tedy pod $a_{k+1}$). Výsledek vyhodnocování je vpravo dole – v našem případě tedy $5$.

  Důsledkem věty o dělení polynomu polynomem je, že zbytek po vydělení f1 polynomem (x-3) je 5 a výsledkem tohoto dělení je polynom stupně 2 s koeficienty danými zbylými třemi čísly ve třetím řádku. Díky tomuto pozorování lze Hornerovo schéma použít i jako efektivní algoritmus k dělení polynomů.
\end{pozn}

\begin{veta}[Hledání racionálních kořenů polynomu s racionálními koeficienty]
  Mějme polynom $P(x) \in \mathbb Q [x]$. Potom najdeme jeho kořeny $\frac{r}{s} \in \mathbb Q$ takto:
  \begin{enumerate}[1.]
    \item Nalezneme všechny celočíselné dělitele $r$ absolutního členu polynomu $a_0$.
    \item Nalezneme všechny přirozené dělitele $s$ vedoucího členu $a_n$.
    \item Utvoříme všechny zlomky tvaru $\frac{r}{s}, (r,s) = 1$.
    \item Hornerovým schématem určíme $P(1)$, případně $P(-1)$ ($1$ a $-1$ také mohou být kořeny).
    \item Vyškrtáme ty zlomky $\frac{r}{s}$, které nesplňují podmínky $(r-s) \mid P(1) \land (r+s) \mid P(-1)$.
    \item U ostatních zlomků vyzkoušíme Hornerovým schématem, zda jsou kořeny daného polynomu.
  \end{enumerate}
\end{veta}

\begin{veta}[Rozklad polynomu v reálném a komplexním oboru]
  Nechť $P(x) \in \mathbb R [x], P(x) = a_n x^n + a_{n-1} x^{n-1} + ... + a_1 x + a_0$, kde $st(P(x)) \geq 1$. Pak $P(x)$ lze v $\mathbb R$ vyjádřit jako součin polynomů 1. a 2. stupně a koeficientu $a_n$:
  \[
    P(x) = a_n(x-c_1)^{k_1}(x-c_2)^{k_2} ... (x-c_k)^{k_k}(x^2+p_1 x + q_1)^{r_1}(x^2+p_2 x + q_2)^{r_2} ... (x^2+p_r x + q_r)^{r_r}
  \]
  kde $c_1,c_2, ..., c_k$ jsou všechny jeho reálné různé kořeny s násobnostmi $l_1, k_2, ..., k_k \in \mathbb N$; $p_1, p_2, ..., p_r$ a $q_1, q_2, ..., q_r$ jsou reálná čísla, $r_1, r_2, ..., r_n \in \mathbb N$.
  Polynomy $(x^2+p_1 x + q_1)^{r_1}(x^2+p_2 x + q_2)^{r_2} ... (x^2+p_r x + q_r)^{r_r}$ jsou kvadratické polynomy se záporným diskriminantem. Uvedený rozklad je až na pořadí činitelů jednoznačný a platí $st(P(x)) = k_1 + k_2 + ... + k_k + 2(r_1 + r_2 + ... + r_r)$.

  Jestliže $a_1 \pm ib_1, a_2 \pm ib_2, .., a_s \pm ib_s$ jsou všechny navzájem různé dvojice komplexně sdružených kořenů s násobností $r_1, r_2, ..., r_s$, $P(x)$ můžeme v $\mathbb C$ psát ve tvaru:
  \[
    P(x) = a_n(x-c_1)^{k_1}(x-c_2)^{k_2} ... (x-c_k)^{k_k}[(x-a_1)^2+b_1^2]^{r_1}[(x-a_2)^2+b_2^2]^{r_2} ... [(x-a_s)^2+b_s^2]^{r_s}
  \]
\end{veta}

\begin{definition}
  Nechť $A(x), B(x) \in \mathbb R [x]$. Polynom $C(x) \in \mathbb R [x]$ se nazývá \textbf{společný dělitel polynomů} $A(x), B(x)$, právě tehdy, když platí: $C(x) \mid A(x) \land C(x) \mid B(x)$.
  Polynom $D(x) \in \mathbb R [x]$ se nazývá \textbf{největší společný dělitel polynomů} $A(x), B(x)$, který označujeme $D(A(x), B(x))$ nebo $NSD(A(x), B(x))$, právě tehdy, když platí:
  \begin{enumerate}[1.]
    \item $D(x) \mid A(x) \land D(x) \mid B(x)$
    \item $\forall C(x) \in \mathbb R [x]: C(x) \mid A(x) \land C(x) \mid B(x) \implies C(x) \mid D(x)$
  \end{enumerate}
\end{definition}

\begin{pozn}
  Největší společný dělitel hledáme \textbf{Euklidovým algoritmem}: Nechť $A(x), B(x) \in \mathbb R [x]$.
  \begin{enumerate}[a.]
    \item $A(x) = 0(x) \implies D(A(x), B(x)) = B(x)$
    \item $st(A(x)) \geq st(B(x)) \geq 0$
  \end{enumerate}
  Proveďme následující posloupnost dělení se zbytkem. Toto dělení ukončíme, až dostaneme zbytek -- nulový polynom. Vzhledem k nerovnosti na pravé straně, tento nulový zbytek existuje.
  \[
    A(x) = B(x) \cdot Q_1(x) + R_1(x), \hfill st(R_1(x)) < st(B(x))\\
    B(x) = R_1(x) \cdot Q_2(x) + R_2(x), \hfill st(R_2(x)) < st(R_1(x))\\
    R_1(x) = R_2(x) \cdot Q_3(x) + R_3(x), \hfill st(R_3(x)) < st(R_2(x))\\
    ... ...\\
    R_{n-2}(x) = R_{n-1}(x) \cdot Q_n(x) + R_n(x), \hfill st(R_n(x)) < st(R_{n-1}(x))\\
    R_{n-1}(x) = R_{n}(x) \cdot Q_{n+1}(x), \hfill st(R_{n-1}(x)) < 0(x))
  \]
  Potom $(A(x), B(x)) = R_n(x)$, či libovolný násobek $R_n(x)$. \textbf{Normovaný největší společný dělitel polynomů} ale existuje právě jeden, ten se označuje $(A(x), B(x))$.
\end{pozn}

\begin{definition}
  Nehcť polynomy $A(x), B(x)$. Pokud $(A(x), B(x)) = 1$, řekneme, že polynomy $A(x), B(x)$ jsou \textbf{nesoudělné}. V opačném případě jsou \textbf{soudělné}.
\end{definition}

\begin{definition}
    Nechť $P(x) \in \mathbb R[x], P(x)=a_nx^n+a_{n-1}x^{n-1}+\dots+a_1x+a_0.$ \textbf{Derivací polynomu} $P(x)$ rozumíme polynom $P^\prime(x)$ definovaný takto:
    $$
        P^\prime(x)=\begin{cases}
        0, &\text{ je-li st} (P(x)) \leq 0,\\
        a_n n x^{n-1} + a_{n-1}(n-1)x^{n-2} + \dots + a_1, & \text{ je-li st} (P(x)) \geq 1.
        \end{cases}
    $$
\end{definition}

\begin{veta}
    Nechť $P(x) \in R[x], c \in \mathbb R$ je jeho $k$-násobný kořen, $k\in \mathbb N.$ Pak platí: $c$ je
    $k$-násobný kořen $P(x) \iff P(c)=P^\prime(c)=P^{\prime \prime}(c)=\dots P^{k-1}(c)=0 \land P^{k}(c)\ne 0$.
\end{veta}

\section{Polynomická funkce (především lineární a kvadratické)}

\begin{definition}\,
\begin{enumerate}
  \item Nechť $b \in \mathbb R$. Funkci $f:y = b$ nazveme \textbf{konstantní funkcí}.
  \item Nechť $a, b \in \mathbb R, a \neq 0$. Pak funkci $f:y= ax + b$ nazveme \textbf{lineární funkcí}.
\end{enumerate}
\end{definition}

\begin{pozn}\,
  \begin{itemize}
    \item Definičním oborem konstantní i lineární funkce je $\mathbb R$. \\
          Oborem hodnot konstantní funkce je ${b}$.\\
          Oborem hodnot lineární funkce je $\mathbb R$.
    \item Grafem konstantní funkce je přímka rovnoběžná s osou $x$. \\
          Grafem lineární funkce je přímka, která není rovnoběžná s osou $x$ ani s osou $y$.
  \end{itemize}
\end{pozn}

\begin{veta}
  Nechť $f: y = ax + b, a \neq 0$ je lineární funkce. Pak platí:
  \begin{enumerate}[1.]
    \item $b=0 \implies f$ je lichá\\
          $b \neq 0 \implies f$ není ani sudá, ani lichá
    \item $a > 0 \implies f$ je rostoucí\\
          $a < 0 \implies f$ je klesající
    \item $f$ není ani shora, ani zdola omezená
    \item $f$ nemá extrémy
    \item $f$ není periodická
  \end{enumerate}
\end{veta}

\begin{proof}
  \begin{enumerate}[1.]
    \item $b=0 \implies f:y= ax = f(x) \land -ax = -f(x) \implies f$ je lichá\\
          $b\neq 0 \implies f:y = ax + b = f(x) = -ax + b \implies f$ není sudá ani lichá
    \item $a>0 \implies x_1 < x_2 \implies ax_1 < ax_2 \implies ax_1 + b < ax_2 + b \implies f$ je rostoucí\\
          $a<0 \implies x_1 < x_2 \implies ax_1 > ax_2 \implies ax_1 + b > ax_2 + b \implies f$ je klesající
    \item Obor hodnot je $\mathbb R \implies f$ není omezená.
    \item Plyne z grafu.
    \item Plyne z z grafu nebo z toho, že $f$ je ryze monotónní
  \end{enumerate}
\end{proof}

\begin{definition}
  Nechť $a,b,c \in \mathbb R, a \neq 0$. Pak $f:y = ax^2 + bx+ x$ nazýváme \textbf{kvadratickou funkcí}.
\end{definition}

\begin{pozn}\,
  \begin{itemize}
    \item Definičním oborem kvadratické funkce je $\mathbb R$.
    \item Grafem kvadratické funkce je parabola.
  \end{itemize}
\end{pozn}

\begin{veta}
  Nechť $f:y = ax^2, a \neq 0$ je kvadratická funkce. Pak platí:\\
  \begin{tabularx}{\textwidth}{ >{\raggedright\arraybackslash}X >{\raggedright\arraybackslash}X >{\raggedright\arraybackslash}X }
    \, & $a>0$ & $a<0$ \\
    obor hodnot & $H(f) = \mathbb R^{+}_0$ & $H(f) = \mathbb R^{-}_0$ \\
    parita & sudá & sudá \\
    monotónnost & $x \in (-\infty, 0\rangle$ -- klesající & $x \in (-\infty, 0\rangle$ -- rostoucí \\
    \, & $x \in \langle 0, \infty)$ -- rostoucí & $x \in \langle 0, \infty)$ -- klesající \\
    omezenost & zdola omezená & zdola neomezená \\
    \, & shora neomezená & shora omezená \\
    extrémy &  ostré minimum v bodě $x_0=0$ & ostré maximum v bodě $x_0=0$ \\
    periodicita & neperiodická & neperiodická
  \end{tabularx}
\end{veta}

\section{Racionální lomená funkce}
\begin{definition}
    Nechť $k \in \mathbb R, k \ne 0.$ Pak funkcí $y=\frac{k}{x}$
    nazýváme \textbf{nepřímou úměrností} s~koeficientem nepřímé úměrnosti $k$.
\end{definition}

\begin{definition}
    Nechť $a,b,c,d \in \mathbb R, c \ne 0, ad-bc \ne 0.$ Pak funkci
    $f:y=\frac{ax+b}{cx+d}$ nazýváme \textbf{lineární lomenou
    funkcí}.
\end{definition}

\begin{priklad}
Nakreslete graf funkce $f: y=\frac{2x+3}{x+1}.$
\end{priklad}

\begin{reseni}
Vydělením získáme přehled o posunech počátku.
\end{reseni}

\begin{pozn}
  Pro lineární lomenou funkci platí
  \begin{itemize}
      \item $D(f)=\mathbb R \smallsetminus \left \{ -\frac{d}{c} \right \},$
    	\item $H(f)=\mathbb R \smallsetminus \left \{ \frac{a}{c} \right \},$
     	\item grafem je rovnoosá hyperbola s asymptotami
      $x=-\frac{d}{c}, y=\frac{a}{c}.$
  \end{itemize}
\end{pozn}

\begin{veta}
    Lineární lomená funkce:
    \begin{enumerate}[$i.$]
        \item není ani sudá, ani lichá;
       	\item $bc-ad > 0 \implies$ klesající v
        $\left ( -\infty, -\frac{d}{c} \right ),
        \left ( -\frac{d}{c}, \infty \right ) $ \\
        $bc-ad < 0 \implies$ rostoucí v
        $\left ( -\infty, -\frac{d}{c} \right ),
        \left ( -\frac{d}{c}, \infty \right ) $;
       	\item není omezená ani shora, ani zdola;
       	\item nemá extrémy;
       	\item není periodická;
       	\item je prostá.
    \end{enumerate}
\end{veta}

\begin{definition}
    Nechť $P(x), Q(x) \mathbb R[x]$ jsou polynomy, $Q(x) \ne O(x).$
    Pak funkci $F:y=\frac{P(x)}{Q(x)}$ nazýváme \textbf{racionální
    lomenou funkcí}.
\end{definition}

\begin{definition}
    Racionální lomená funkce $f: y=\frac{P(x)}{Q(x)}, Q(x) \ne O(x)$
    se nazývá
    \begin{enumerate}[$i.$]
        \item \textbf{ryze lomená} racionální funkce
        $\iff \text{st}\left (P(x) \right ) <
        \text{st}\left (Q(x) \right )$;
       	\item \textbf{neryze lomená} racionální funkce
        $\iff \text{st}\left (P(x) \right ) \geq
        \text{st}\left (Q(x) \right )$.
    \end{enumerate}
\end{definition}

\begin{priklad}
  Určete, pro která $x \in \mathbb R$ platí
  $$P(x) = \frac{(x-1)^2(x+5)^3x^4(x^2-5x+6)(x^2+1)(x-5)^5}{(x-3)^3(x^2+7)} \geq 0.$$
\end{priklad}

\begin{reseni}
  Rozložíme čitatele i jmenovatele v reálném oboru. Znaménko se obecně mění v kořenech čitatele nebo kořenech jmenovatele. Členy se záporným diskriminantem nemají reálný kořen,
  neuvažujeme je tedy. V sudých mocninách se nemění znaménko. Zároveň musíme vyloučit samotné kořeny jmenovatele, protože je v nich výraz nedefinován. Toto vše vyneseme na číselnou osu.
\end{reseni}

\begin{veta}[Rozklad na parciální zlomky]
    Nechť $f: y=\frac{P(x)}{Q(x)}, Q(x) \ne O(x)$ je ryze lomená
    racionální funkce, kde polynomy $P(x)$ a $Q(x)$ nemají společné
    kořeny. Nechť
    \begin{align*}
        Q(x) = & \, a_n(x-c_1)^{k_1}(x-c_2)^{k_2} \dots
        (x-c_l)^{k_l} \\
        \cdot & \, (x^2+p_1x + q_1)^{k_1}(x^2+p_2x+q_2)
        ^{k_2} \dots (x^2+p_sx+q_s)^{r_s}
    \end{align*}
    je rozklad polynomu $Q(x)$ v reálném oboru. Pak existují čísla
    \begin{center}
        \begin{tabular}{l l l l}
            $C_{11}$, & $C_{12}$, & $\dots$, & $C_{1k_1}$, \\
            $C_{21}$, & $C_{22}$, & $\dots$, & $C_{2k_2}$, \\
            \,        & \,        & $\dots$  & \,          \\
            $C_{l1}$, & $C_{l2}$, & $\dots$, & $C_{lk_l}$; \\
            $P_{11}$, & $P_{12}$, & $\dots$, & $P_{1r_1}$, \\
            $P_{21}$, & $P_{22}$, & $\dots$, & $P_{2r_2}$, \\
            \,        & \,        & $\dots$  & \,          \\
            $P_{s1}$, & $P_{s2}$, & $\dots$, & $P_{sr_s}$; \\
            $Q_{11}$, & $Q_{12}$, & $\dots$, & $Q_{1r_1}$, \\
            $Q_{21}$, & $Q_{22}$, & $\dots$, & $Q_{2r_2}$, \\
            \,        & \,        & $\dots$  & \,          \\
            $Q_{s1}$, & $Q_{s2}$, & $\dots$, & $Q_{sr_s}$
        \end{tabular}
    \end{center}
    tak, že $\forall x \in \mathbb R$ platí
    \begin{align*}
        y = & \left [
              \frac{C_{11}}{x-c_1} + \frac{C_{12}}{(x-c_1)^2} +
              \dots + \frac{C_{1k_1}}{(x-c_1)^{k_1}} \right . \\
          & + \frac{C_{21}}{x-c_2} + \frac{C_{22}}{(x-c_2)^2} +
              \dots + \frac{C_{2k_2}}{(x-c_1)^{k_2}} \\
          & + \dots \\
          & + \frac{C_{l1}}{x-c_l} + \frac{C_{l2}}{(x-c_l)^2} +
              \dots + \frac{C_{lk_l}}{(x-c_l)^{k_l}} \\
          & + \frac{P_{11}x + Q_{11}}{x^2 + p_1x + q_1}
              + \frac{P_{12}x + Q_{12}}{(x^2 + p_1x + q_1)^2} + \dots
              + \frac{P_{1r_1}x + Q_{1r_1}}{(x^2 + p_1x + q_1)^{r_1}} \\
          & + \frac{P_{21}x + Q_{21}}{x^2 + p_2x + q_2}
              + \frac{P_{22}x + Q_{22}}{(x^2 + p_2x + q_2)^2} + \dots
              + \frac{P_{2r_2}x + Q_{2r_2}}{(x^2 + p_2x + q_2)^{r_2}} \\
          & + \dots \\
          & + \left . \frac{P_{s1}x + Q_{s1}}{x^2 + p_sx + q_s}
              + \frac{P_{s2}x + Q_{s2}}{(x^2 + p_sx + q_s)^2} + \dots
              + \frac{P_{sr_s}x + Q_{sr_s}}{(x^2 + p_sx + q_s)^{r_s}} \right ]
              \cdot \frac{1}{a_n}
    \end{align*}
\end{veta}

\begin{priklad}
Rozložte na parciální zlomky: $y=\frac{x-4}{x^2-5x+6}.$
\end{priklad}

\begin{reseni}
Platí
$$\frac{x-4}{x^-5x+6}=\frac{A}{x-2}+\frac{B}{x-3}.$$
Metodou porovnání koeficientů tedy řešíme rovnici $x-4=A(x-3)+B(x-2)$.
\end{reseni}

\section{Mocniny, mocninná funkce}
\begin{definition}
    Nechť $n\in \mathbb N.$ Pak funkci $f: y=x^n$ nazýváme \textbf{mocninnou funkcí}
    s přirozeným exponenetem.
\end{definition}

\begin{definition}
    Nechť $n\in \mathbb N.$ Pak funkci $f: y=x^{-n}=\frac{1}{x^n}$ nazýváme
    \textbf{mocninnou funkcí} s celým záporným exponenetem.
\end{definition}

\begin{definition}
    Nechť $\frac{p}{q}\in \mathbb Q, x \in \mathbb R^+.$ Pak funkci $f: y=
    x^{\frac{p}{q}}=\sqrt[q]{x^p}$ nazýváme \textbf{mocninnou funkcí}
    s racionálním exponentem.
\end{definition}

\section{Exponenciální funkce}
\begin{definition}
    Nechť $a\in \mathbb R^+ \smallsetminus \left \{ 1 \right \}. $ Pak funkci $f:y=a^x$
    nazýváme \textbf{exponenciální funkcí} o základu $a$. Funkci $f:y=e^x$ pak
    nazveme \textbf{přirozenou exponenciální funkcí}.
\end{definition}

\begin{figure}[ht!]
  \centering
  \begin{tikzpicture}
    \begin{axis}[
        axis lines = middle,
        xlabel = \(x\),
        ylabel = {\(y\)},
        line width=1.5pt,
        width=.8\textwidth,
        height=.6\textwidth,
        ymin=-2.5,
        ymax=4.4,
        xmin=-4.5,
        xmax=4.5,
        grid
    ]
    %Below the red parabola is defined
    \addplot [
        domain=-5:5,
        samples=100,
        color=red
    ]
    {2^x};
    \addlegendentry{\(2^x\)}

    \addplot [
        domain=-5:5,
        samples=100,
        color=blue
        ]
        {0.5^x};
    \addlegendentry{\((1/2)^x\)}

    \addplot [
        domain=-5:5,
        samples=100,
        color=green
        ]
        {e^x};
    \addlegendentry{\(e^x\)}

    \end{axis}
    \end{tikzpicture}
  \caption{Grafy některých exponenciálních funkcí}
\end{figure}

\begin{veta}
    Vlastnosti exponenciálních funkcí $y= a^x$, kde $a$ je přirozené číslo:
    \begin{enumerate}[$i.$]
        \item $D(f)= \mathbb R$, $H(f)= \mathbb R^+$.
       	\item Graf každé exponenciální funkce prochází bodem $[0,1].$
        \item Není ani sudá, ani lichá.
        \item Je zdola omezená.
        \item Pro $a >1$ je rostoucí, pro $a<1$ je klesající.
        \item Nemá extrémy a není periodická.
    \end{enumerate}
\end{veta}

\begin{definition}
    (Ne)rovnice s neznámou v exponentu se nazývá \textbf{exponenciální (ne)rovnice}.
\end{definition}

\begin{veta}
    $\forall a \in \mathbb R^+ \smallsetminus \left \{ 1 \right \}, \forall x_1, x_2
    \in \mathbb R: a^{x_1}=a^{x_2}\iff x_1=x_2.$
\end{veta}

\begin{proof}
    Plyne z toho, že exponenciální funkce je prostá.
\end{proof}

\begin{priklad}
Určete, pro která $b\in \mathbb R\smallsetminus\left \{ 1 \right \} $ je $f:y=\left ( \frac{b}{b-1} \right )^x $
rostoucí a pro která klesající.
\end{priklad}

\begin{reseni}
Je rostoucí, pokud základ je větší než 1 a klesající, pokud základ je menší než 1.
Řešíme tedy
\begin{align*}
\frac{b}{b-1}>1, & & \textrm{resp.} & & \frac{b}{b-1}<1.
\end{align*}
\end{reseni}

\section{Logaritmus, logaritmická funkce}
\begin{definition}
    Nechť $a \in \mathbb R^+ - \left \{ 1 \right \} $. Pak inverzní
    funkci k exponenciální funkci $f:y=a^x$ nazýváme \textbf{logaritmickou
    funkcí} o základu $a$. Zapisujeme $f^{-1}: y=\log_a x.$
\end{definition}

\begin{veta}
    Nechť je dána logaritmická funkce $f:y=\log_a x$, $a \in \mathbb R^+ -
    \left \{ 1 \right \} $. Pak
    \begin{enumerate}[$i.$]
        \item $a>1: f$ je rostoucí,
       	\item $a < 1: f$ je klesající.
    \end{enumerate}
\end{veta}

\begin{definition}
    Nechť $f: y=\log_a x, a \in \mathbb R^+ - \left \{ 1 \right \}$ je
    logaritmická funkce a $x_0\in D(f).$ Pak číslo $f(x_0)$ nazýváme
    \textbf{logaritmem} čísla $x_0$ o základu $a.$
\end{definition}

\begin{pozn}
    Platí $\log_a b=c \iff a^c = b.$
\end{pozn}

\subsection*{Vlastnosti logaritmů}
\begin{veta}
    $\forall a \in \mathbb R^+ - \left \{ 1 \right \}:$
    \begin{enumerate}[$i.$]
        \item $\forall x \in \mathbb R^+: a^{\log_a x}=x$
       	\item $\forall x \in \mathbb R: \log_a a^x=x$
    \end{enumerate}
\end{veta}

\begin{proof}
    \begin{enumerate}[$i.$]
        \item $log_a x = r \iff a^r = x \iff a^{\log_a x}=x$
       	\item $a^x = r \iff \log_a r=c \iff \log_a a^x = x$ \qedhere
    \end{enumerate}
\end{proof}

\begin{veta}\label{zakl_reseni}
    $\forall a \in \mathbb R^+ - \left \{ 1 \right \},
    \forall x_1, x_2\in \mathbb R^+: \log_a x_1 = \log_a x_2 \iff x_1 = x_2$
\end{veta}

\begin{veta}
    $\forall a,b,c\in \mathbb R^+, a\ne 1, b\ne 1:$
    \begin{enumerate}[$i.$]
        \item $\log_a b \cdot \log_b c = \log_a c$
       	\item $\log_a b \cdot \log_b a = 1$
    \end{enumerate}
\end{veta}

\begin{proof}
    \begin{enumerate}[$i.$]
        \item \begin{align*}
            \log_a b = r &\iff a^r = b \\
            \log_b c = s &\iff b^s = c \\
            \log_a c = t &\iff a^t = c
        \end{align*}
        \begin{align*}
            c & =b^s=\left ( a^r \right )^s\\
            c & = a^t\\
            a^{rs} & = a^t \implies rs = t \implies \log_a b \cdot \log_b c = \log_a c
        \end{align*}
        \item $\log_a b\cdot \log_b a = \log_a a = 1$\qedhere
    \end{enumerate}
\end{proof}

\begin{veta}
    $\forall a,b,c\in \mathbb R^+, a\ne 1, b\ne 1:$
    \begin{enumerate}[$i.$]
        \item $\log_b c = \frac{\log_a c}{\log_a b}$
       	\item $\log_a b = \frac{1}{\log_b a}$
    \end{enumerate}
\end{veta}

\begin{veta}
    $\forall a \in \mathbb R^+ - \left \{ 1 \right \}  , \forall x \in \mathbb R^+:$
    \begin{enumerate}[$i.$]
        \item $\log_{\frac{1}{a}} x = -\log_a x$
       	\item $\log_a \frac{1}{x} = -\log_a x$
    \end{enumerate}
\end{veta}

\begin{proof}
    \begin{enumerate}[$i.$]
        \item Nechť $\log_{\frac{1}{a}} x = r, \log_a x = s.$ Pak
        $\left ( \frac{1}{a} \right )^r = a^s,$ tedy $a^{-r}=s,$ takže $r=-s$.
        Odtud $\log_{\frac{1}{a}} x = -\log_a x.$
       	\item Nechť $\log_{\frac{1}{a}} x = r, \log_a x = s.$ Pak
        $x = \frac{1}{a^r}=a^s,$ obdobně $r=-s,$ takže $\log_a \frac{1}{x}=
        -\log_a x.$\qedhere
    \end{enumerate}
\end{proof}

\begin{veta}
    $\forall a \in \mathbb R^+ - \left \{ 1 \right \}, \forall x,y \in \mathbb R^+
    ,\forall r \in \mathbb R:$
    \begin{enumerate}[$i.$]
        \item $\log_a (xy)=\log_a x + \log_a y$
       	\item $\log_a \left ( \frac{x}{y} \right ) = \log_a x - \log_a y $
       	\item $\log_a x^r = r\cdot \log_a x$
    \end{enumerate}
\end{veta}

\begin{proof}
    Ve všech případech předpokládejme $\log_a x = s \iff x = a^s,
    \log_a y = t \iff y = a^t.$
    \begin{enumerate}[$i.$]
        \item $xy = a^sa^t=a^{s+t}$\\
        $\log_a (xy) = \log_a (a^{s+t}) = s+t=\log_a x + \log_a y$
       	\item $\log_a \left ( \frac{x}{y} \right )  = \log_a
        \left ( \frac{a^s}{a^t} \right ) =\log_a \left ( a^{s-t} \right ) =
        s-t $
       	\item $\log_a x^r = \log_a \left [ \left ( x^s \right )^r  \right ] =
        \log_a a^{rs}=rs=r\cdot \log_a x$
    \end{enumerate}
\end{proof}

\begin{definition}
    Nechť $x\in \mathbb R^+.$ \textbf{Dekadickým logaritmem} čísla $x$ rozumíme
    jeho logaritmus o základu 10. Zapisujeme $\log x.$
\end{definition}

\begin{veta}\label{mantisa}
    Nechť $m\in \mathbb R^+.$ Pak platí
    $$\log m = \log m_1 + c,$$
    kde $\log m_1 \in \left < 0,1 \right >, c \in \mathbb Z,$ a toto vyjádření
    je jednozančné.
\end{veta}

\begin{proof}
    $\log m = \log \left ( m_1\cdot 10^c \right ) = \log m_1 + \log 10^c
    = \log m_1 + c$
\end{proof}

\begin{definition}
    Číslo $\log m_1$ z věty \ref{mantisa} nazýváme \textbf{mantisou} a číslo
    $c$ \textbf{charakteristikou} čísla $\log m.$
\end{definition}

\begin{definition}
    Nechť $x \in \mathbb R-\left \{ 1 \right \}. $ \textbf{Přirozeným logaritmem} čísla $x$
    rozumíme logaritmus o základu $e$. Zapisujeme $\ln x.$
\end{definition}

\begin{pozn}
    Logaritmická (ne)rovnice je taková (ne)rovnice, ve které se vyskytuje logaritmus.
    Řešíme je na základě věty \ref{zakl_reseni}.
\end{pozn}

\begin{veta}
    $\forall a \in \mathbb R^+ - \left \{ 1 \right \} , \forall x_1, x_2
    \in \mathbb R:$
    \begin{enumerate}[$i.$]
        \item pro $a>1:a^{x_1}<a^{x_2}\iff x_1 < x_2,$
       	\item pro $a \in (0,1): \log_a x_1 < \log_a x_2 \iff x_1 > x_2.$
    \end{enumerate}
\end{veta}

\begin{veta}
    $\forall a \in \mathbb R^+ - \left \{ 1 \right \} , \forall x_1, x_2
    \in \mathbb R^+:$
    \begin{enumerate}[$i.$]
        \item pro $a>1:\log_a x_1 < \log_a x_2 \iff x_1 < x_2$,
       	\item pro $a\in (0,1): \log_a x_1 < \log_a x_2 \iff x_1 > x_2.$
    \end{enumerate}
\end{veta}

\section{Funkce sinus, kosinus, arkussinus, arkuskosinus}
\begin{definition}
    Nechť $\sphericalangle AVB$ je úhel. Pak mu přiřazujeme číslo $a\in \mathbb R$,
    které nazýváme \textbf{velikostí úhlu} $\sphericalangle AVB$ takto:\\
    Nechť $k(V,r), X\in k\cap \overrightarrow{VA}, Y\in k\cap \overrightarrow{VB}.$
    Označme $s$ velikost oblouku $XY.$ Velikost úhlu pak definujeme jako
    $$\alpha = \frac{s}{r}.$$
\end{definition}

\begin{pozn}
    Je-li $s=r,$ má úhel $\alpha$ velikost 1 \textbf{radián}. Velikost úhlu lze vyjadřovat
    též ve stupních, přičemž platí $2\pi=360^\circ$.
\end{pozn}

\begin{veta}
    Nechť $\sphericalangle AVB$ je úhel, $\alpha$ jeho velikost v radiánech a $\beta$
    jeho velikost ve stupních. Pak platí:
    \begin{align*}
        \alpha = \beta\cdot \frac{\pi}{180^\circ}, & & \beta = \alpha\cdot \frac{180^\circ}{\pi}.
    \end{align*}
\end{veta}

\begin{definition}
    \textbf{Orientovaným úhlem} $\sphericalangle AVB$ nazýváme úhel s uspořádanou dvojicí
    polopřímek $\overrightarrow{VA}, \overrightarrow{VB}$ se společným počátkem v bodě $V.$
    Polopřímku $\overrightarrow{VA}$, resp. $\overrightarrow{VB}$ nazýváme \textbf{počátečním},
    resp. \textbf{koncovým ramenem}, bod $V$ \textbf{vrchol}.
\end{definition}

\begin{definition}
    Nechť $\sphericalangle AVB$ je orientovaný úhel o základní velikosti $\alpha.$ Pak
    \textbf{velikostí orientovaného úhlu} $\sphericalangle AVB$ rozumíme každé z
    čísel $\alpha +2k\pi, k \in \mathbb Z.$
\end{definition}

\begin{definition}[Sinus a kosinus]
  Nechť je dána jednotková kružnice. Hodnotou funkce \textbf{sinus} (resp. \textbf{kosinus})
  v bodě $x$ nazveme $y$-ovou (resp. $x$-ovou) souřadnici průsečíku této kružnice s ramenem
  úhlu svírajícím s $x$-ovou osou úhel $\theta = x$.
\end{definition}

\begin{veta}
    Vlastnosti funkce sinus:\\
    Nechť $k\in \mathbb Z.$
    \begin{enumerate}[$i.$]
        \item $D(f)= \mathbb R$, $H(f)= \left < -1,1 \right > $.
       	\item Je lichá.
        \item Je rostoucí v intervalu $\left < -\frac{\pi}{2}+2k\pi, \frac{\pi}{2}+2k\pi \right > $
        a klesající v $\left < \frac{\pi}{2}+2k\pi, \frac{3\pi}{2}+2k\pi \right > $
        \item Je omezená.
        \item Maximum má v bodech $\frac{\pi}{2}+2k\pi$, minimum v bodech $\frac{3\pi}{2}+2k\pi.$
        \item Je periodická s nejmenší periodou $2\pi.$
    \end{enumerate}
    Vlastnoti funkce kosinus:
    \begin{enumerate}[$i.$]
        \item $D(f)= \mathbb R$, $H(f)= \left < -1,1 \right > $.
       	\item Je sudá.
        \item Je rostoucí v intervalu $\left < \pi+2k\pi, 2\pi+2k\pi \right > $
        a klesající v $\left < 0+2k\pi, \pi+2k\pi \right > $
        \item Je omezená.
        \item Maximum má v bodech $2k\pi$, minimum v bodech $\pi+2k\pi.$
        \item Je periodická s nejmenší periodou $2\pi.$
    \end{enumerate}
\end{veta}


\begin{definition}[Akrussinus]
Funkce \textbf{arkussinus}, označená $f^{-1}: y=\arcsin x$, se nazývá funkce inverzní
k funkci $f: y=\sin x$, kde $D(f)=\left < -\frac{\pi}{2},\frac{\pi}{2} \right >$.
\end{definition}
\begin{pozn}
    Funkce arkussinus má následující vlastnosti:
    \begin{enumerate}
    \item $D(f^{-1}) = \left < -1, 1 \right >$, $H(f^{-1}) = \left < -\frac{\pi}{2},\frac{\pi}{2} \right >,$
    \item $f^{-1}$ je rostoucí v celém $D(f^{-1}).$
    \end{enumerate}
\end{pozn}


\begin{definition}[Arkuskosinus]
Funkce \textbf{arkuskosinus}, označená $g^{-1}: y=\arccos x$, se nazývá funkce inverzní
k funkci $g: y=\cos x$, kde $D(g)=\left < 0,\pi \right >$.
\end{definition}
\begin{pozn}
Funkce arkuskosinus má následující vlastnosti:
\begin{enumerate}
    \item $D(g^{-1}) = \left < -1, 1 \right >$, $H(g^{-1}) = \left < 0, \pi \right >,$
    \item $g^{-1}$ je klesající v celém $D(g^{-1}).$
\end{enumerate}
\end{pozn}

\begin{veta}[Trigonometrická jednička]
  $\forall x \in \mathbb{R}:\sin^2 x + \cos^2 x = 1$
\end{veta}

\begin{veta}
    V pravoúhlém trojúhelníku $ABC$ s přeponou $AB$ platí:
    \begin{align*}
        \sin \alpha = \frac{ |BC| }{ |AB| }, & & \cos \alpha = \frac{|AC|}{|AB|}.
    \end{align*}
\end{veta}

\begin{pozn}
    \textbf{Goniometrická (ne)rovnice} je (ne)rovnice, v níž se vyskytují goniometrické funkce.
\end{pozn}

-- insert aplikace diferenciálního a integrálního počtu --

\begin{pozn}
    Grafy jednotlivých funkcí a tabulka se základními hodnotami jsou k nalezení
    v příloze \ref{appa}.
\end{pozn}

\section{Funkce tangens, kotangens, arkustangens, arkuskotangens}
\begin{definition}
  Funkcí \textbf{tangens} (resp. \textbf{kotangens}) nazýváme funkci danou vztahem:
  \begin{align*}
    \tg x = \frac{\sin x}{\cos x}, & & \cotg x = \frac{\cos x}{\sin x}.
  \end{align*}
\end{definition}

\begin{veta}
    Vlastnosti funkce tangens:\\
    Nechť $k\in \mathbb Z.$
    \begin{enumerate}[$i.$]
        \item $D(f)= \mathbb R-\left \{ (2k+1)\frac{\pi}{2}, k\in \mathbb Z \right \} $, $H(f)= \mathbb R $.
       	\item Je lichá.
        \item Je rostoucí v každém z intervalů $\left ( -\frac{\pi}{2}+k\pi, \frac{\pi}{2}+k\pi \right ) $.
        \item Není omezená.
        \item Nemá extrémy.
        \item Je periodická s nejmenší periodou $\pi.$
    \end{enumerate}
    Vlastnoti funkce kotangens:
    \begin{enumerate}[$i.$]
        \item $D(f)= \mathbb R-\left \{ k\pi, k\in \mathbb Z \right \} $, $H(f)= \mathbb R $.
       	\item Je lichá.
        \item Je klesající v každém z intervalů $\left ( k\pi, (k+1)\pi \right ) $.
        \item Není omezená.
        \item Nemá extrémy.
        \item Je periodická s nejmenší periodou $\pi.$
    \end{enumerate}
\end{veta}

\begin{priklad}
Vypočtěte $\tg \left ( -\frac{19}{6}\pi \right ) $.
\end{priklad}

\begin{definition}
  Funkce \textbf{arkustangens}, označená $f^{-1}: y=\arctg x$, se nazývá funkce inverzní k funkci $f: y=\tg x$, kde $D(f)=\left ( -\frac{\pi}{2}, \frac{\pi}{2} \right )$.
\end{definition}

\begin{pozn}
Funkce arkustangens má následující vlastnosti:
\begin{enumerate}
\item $D(f)^{-1} = \mathbb{R}$, $H(f)^{-1} = \left ( -\frac{\pi}{2}, \frac{\pi}{2} \right )$,
\item $f^{-1}$ je rostoucí v celém $D(f^{-1})$.
\end{enumerate}
\end{pozn}

\begin{definition}
  Funkce \textbf{arkuskotangens}, označená $g^{-1}: y=\arccotg x$, se nazývá funkce inverzní k funkci $g: y=\cotg x$, kde $D(g)=\left ( 0, \pi \right )$.
\end{definition}

\begin{pozn}
Funkce arkuskotangens má následující vlastnosti:
\begin{enumerate}
\item $D(g)^{-1} = \mathbb{R}$, $H(g)^{-1} = \left ( 0, \pi \right )$,
\item $g^{-1}$ je klesající v celém $D(g^{-1})$.
\end{enumerate}
\end{pozn}

\begin{veta}
    V pravoúhlém trojúhelníku $ABC$ s přeponou $AB$ platí:
    \begin{align*}
        \tg \alpha = \frac{ |BC| }{ |AC| }, & & \cotg \alpha = \frac{|AB|}{|BC|}.
    \end{align*}
\end{veta}

\begin{pozn}
    Grafy jednotlivých funkcí a tabulka se základními hodnotami jsou k nalezení
    v příloze \ref{appa}.
\end{pozn}

\section{Vztahy mezi goniometrickými funkcemi, součtové vzorce}
\begin{pozn}
    Věty \ref{soucsc} a \ref{dvojnas}
    musíme umět nazpaměť.
\end{pozn}

\begin{veta}\label{soucsc}
    $\forall x, y \in \mathbb{R}: $
    \begin{align*}
        \sin \left(x\pm y\right) & = \sin x \cdot \cos y \pm \cos x \cdot \sin y \\
        \cos \left(x\pm y\right) & = \cos x \cdot \cos y \mp \sin x \cdot \sin y
    \end{align*}
\end{veta}

\begin{proof}
    Důkaz máme umět, ale nedělali jsme jej.
\end{proof}



\begin{veta}\label{dvojnas}
    $ \forall x \in \mathbb{R}:$
    \begin{align*}
        \sin 2x = 2\sin x \cdot \cos x, & & \cos 2x = \cos^2 x - \sin^2 x.
    \end{align*}
\end{veta}

\begin{proof}
    Plyne jednoduše z věty \ref{soucsc}.
\end{proof}

\begin{pozn}
    Následující vztahy stačí \uv{rychle odvodit}.
\end{pozn}

\begin{veta}[Vyjádření gon. funkce pomocí jiné gon. funkce]\label{vyjadrenigonf}
  \,\\
  \begin{tabular}{| c || c | c | c | c |}
    \hline
    & $\sin x$ & $\cos x$ & $\tg x$ & $\cotg x$ \\
    \hline\hline
    $\sin x$ & -- & $\sqrt{1-\cos^2 x}$ & $\frac{\tg x}{\sqrt{1+\tg^2 x}}$ & $\frac{1}{\sqrt{1+\cotg^2 x}}$\\
    \hline
    $\cos x$ & $\sqrt{1-\sin^2 x}$ & -- & $\frac{1}{\sqrt{1+\tg^2 x}}$ & $\frac{\cotg x}{\sqrt{1+\cotg^2 x}}$\\
    \hline
    $\tg x$ & $\frac{\sin x}{\sqrt{1-\sin^2 x}}$ &  $\frac{\sqrt{1-\cos^2 x}}{\cos x}$ & -- & $\frac{1}{\cotg x}$\\
    \hline
    $\cotg x$ & $\frac{\sqrt{1-\sin^2 x}}{\sin x}$ & $\frac{\cos x}{\sqrt{1-\cos^2 x}}$ & $\frac{1}{\tg x}$ & --\\
    \hline
  \end{tabular}

\end{veta}



\begin{proof}
    Pro druhý až čtvrtý kvadrant v tabulce plyne z definice. \\
    Vyjádření sinu nebo kosinu pomocí funkce tangens nebo kotangens nalezneme
    jednoduše ze vztahů v pravoúhlém trojúhelníku. Nechť je dán pravoúhlý trojúhelník
    s přeponou $b.$ Pak $\tg \alpha = \frac{a}{c},$ položme tedy $a = \tg \alpha$ a $c=1. $
    Potom z Pythagorovy věty plyne $b=\sqrt{1+\tg^2 \alpha}. $ Pak $\sin\alpha = \frac{\tg x}{\sqrt{\tg^2 x+1} }$ a
    $\cos \alpha = \frac{1}{\sqrt{\tg^2 \alpha + 1} }.$ Obdobně i pro kotangens.
\end{proof}

\begin{priklad}
Určete hodnoty všech goniometrických funkcí, je-li dáno $\sin x=\frac{1}{3}, x \in \left ( \frac{\pi}{2},\pi \right ). $
\end{priklad}

\begin{reseni}
Podle vztahů ve větě \ref{vyjadrenigonf}.
\end{reseni}

\begin{priklad}
Vypočtěte hodnoty všech goniometrických funkcí v bodě $\alpha = 105^\circ$.
\end{priklad}

\begin{reseni}
Využijeme toho, že $105=60+45,$ což jsou tabulkové hodnoty, a součtových vzorců.
\end{reseni}

\begin{veta}
  $\forall x,y\in\mathbb{R}-\bigcup\limits_{k\in\mathbb{Z}} \left\{\frac{(2k+1)\pi}{2}\right\}, \forall (x+y) \in \mathbb{R}-\bigcup\limits_{k\in\mathbb{Z}} \left\{\frac{(2k+1)\pi}{2}\right\}, \forall (x-y) \in \mathbb{R}-\bigcup\limits_{k\in\mathbb{Z}} \left\{\frac{(2k+1)\pi}{2}\right\}:$
  $$\tg\left(x\pm y\right) = \frac{\tg x \pm \tg y}{1\mp\tg x \cdot \tg y}.$$
\end{veta}
\begin{proof}
    Počítejme:
    \begin{align*}
        \tg \left(x+y\right) & = \frac{\sin \left(x+y\right)}{\cos \left(x+y\right)}=\frac{\sin x\cdot  \cos y + \cos x\cdot  \sin y}{\cos x \cdot \cos y - \sin x\cdot  \sin y} \\
        & = \frac{\frac{\sin x \cdot \cos y + \cos x\cdot  \sin y}{\cos x \cdot \cos y}}{\frac{\cos x\cdot  \cos y - \sin x \cdot \sin y}{\cos x\cdot  \cos y}}=\frac{\tg x + \tg y}{1-\tg x \cdot \tg y}\qedhere
    \end{align*}
\end{proof}


\begin{veta}
  $\forall x,y\in\mathbb{R}-\bigcup\limits_{k\in\mathbb{Z}} \left\{\frac{k\pi}{2}\right\}, \forall (x+y) \in \mathbb{R}-\bigcup\limits_{k\in\mathbb{Z}} \left\{\frac{k\pi}{2}\right\}, \forall (x-y) \in \mathbb{R}-\bigcup\limits_{k\in\mathbb{Z}} \left\{\frac{k\pi}{2}\right\}:$
  $$\cotg\left(x\pm y\right) = \frac{\cotg x \cdot \cotg y \mp 1}{\cotg x \pm \cotg y}.$$
\end{veta}

\begin{proof}
    Počítejme:
    \begin{align*}
        \cotg \left(x+y\right) & = \frac{\cos \left(x+y\right)}{\sin \left(x+y\right)}=\frac{\cos x \cdot \cos y - \sin x\cdot  \sin y}{\sin x\cdot  \cos y + \cos x\cdot  \sin y}\\
       &  = \frac{\frac{\cos x\cdot  \cos y - \sin x \cdot \sin y}{\sin x \cdot \sin y}}{\frac{\sin x \cdot \cos y + \cos x\cdot  \sin y}{\sin x\cdot  \sin y}}=\frac{\cotg x \cdot \cotg y-1}{\cotg x + \cotg y}\qedhere
    \end{align*}
\end{proof}

\begin{veta}
    $\forall x \in \mathbb{R}:$
    \begin{align*}
        \left| \sin \frac{x}{2} \right| = \sqrt{\frac{1-\cos x}{2}}, & & \left| \cos \frac{x}{2}\right| = \sqrt{\frac{1+\cos x}{2}}.
    \end{align*}
\end{veta}

\begin{proof}
  $\cos^2 \frac{x}{2} - \sin^2 \frac{x}{2} = \cos x$ a
  $\cos^2 \frac{x}{2} + \sin^2 \frac{x}{2} =1 $
  \begin{align*}
    2\cos^2 \frac{x}{2}&=\cos x +1 & 2\sin^2 \frac{x}{2} & =1-\cos x \\
    \cos^2 \frac{x}{2}&=\frac{\cos x +1}{2} & \sin^2 \frac{x}{2} & =\frac{1-\cos x }{2} \\
    \left| \cos \frac{x}{2} \right| &= \sqrt{\frac{\cos x +1}{2}} & \left| \sin \frac{x}{2} \right | & = \sqrt{\frac{1-\cos x}{2}}\qedhere
  \end{align*}
\end{proof}

\begin{veta}
  $\forall x,y \in \mathbb{R}:$
  \begin{align*}
    \sin x + \sin y &= 2\sin \frac{x+y}{2}\cdot \cos \frac{x-y}{2}& & \cos x + \cos y = 2 \cos \frac{x + y}{2}\cdot \cos \frac{x - y}{2}\\
    \sin x - \sin y &= 2\cos \frac{x + y}{2}\cdot \sin \frac{x - y}{2}& & \cos x - \cos y =-2\sin \frac{x + y}{2}\cdot \sin \frac{x -y}{2}
  \end{align*}
\end{veta}

\begin{proof}
    Počítejme:
    \begin{align*}
        \sin x + \sin y & = \sin \left(\frac{x+y}{2}+\frac{x-y}{2}\right)+ \sin \left(\frac{x+y}{2}-\frac{x-y}{2}\right ) \\
        & = \sin \frac{x+y}{2} \cos\frac{x-y}{2} + \cos\frac{x+y}{2} \sin \frac{x-y}{2} \\
        & + \sin \frac{x+y}{2} \cos\frac{x-y}{2} -  \cos\frac{x+y}{2} \sin \frac{x-y}{2}\\
        & = 2\sin \frac{x+y}{2} \cos\frac{x-y}{2}
    \end{align*}
  v ostatních případech analogicky
\end{proof}

\begin{priklad}
Vyjádřete jako součin: $\sin 3y+\sin y.$
\end{priklad}

\begin{reseni}
Platí
$$\sin 3y+\sin y=2\sin \frac{3y+y}{2}\cdot \cos \frac{3y-y}{2}=2\sin 2y \cdot \cos y.$$
\end{reseni}

\begin{veta}[Sinová věta]
    V každém trojúhelníku $ABC$ platí
    $$\frac{a}{\sin \alpha} = \frac{b}{\sin \beta}=\frac{c}{\sin \gamma}=2R,$$
    kde $R$ je poloměr kružnice opsané.
\end{veta}

\begin{veta}[Kosinová věta]
 V každém trojúhelníku $ABC$ platí
 \begin{align*}
    a^2 &= b^2+c^2-2bc\cos \alpha, \\
    b^2 &= a^2+c^2-2ac\cos \beta ,\\
    c^2 &= a^2+b^2-2ab\cos \gamma.
 \end{align*}
\end{veta}

\begin{veta}\label{obsahtroj}
     V každém trojúhelníku $ABC$ platí
     $$\frac{abc}{4R}=S=\rho s,$$
     kde $R$ je poloměr kružnice opsané a $\rho$ poloměr kružnice vepsané.
\end{veta}

\begin{veta}[Heronův vzorec]
    V každém trojúhelníku $ABC$ platí
    $$S=\sqrt{s(s-a)(s-b)(s-c)}, $$
    kde $s$ je polovina obvodu.
\end{veta}

\begin{priklad}
Nalezněte vztah mezi poloměrem kružnice vepsané a opsané trojúhelníku $ABC$.
\end{priklad}

\begin{reseni}
Zjistíme jednoduše z věty \ref{obsahtroj}.
\end{reseni}

\begin{priklad}
Vrchol věžě stojící na rovině vidíme z místa $A$ pod výškovým úhlem $\alpha= 39^\circ 258^\prime$.
Přijdeme-li o 50 metrů blíž, je vidět pod úhlem $\beta=58^\circ 42^\prime.$ Jak vysoká je věž?
\end{priklad}

\begin{reseni}
Použitím sinové a kosinové věty.
\end{reseni}

\section{Kombinatorika}
\begin{veta}[Pravidlo součtu]
    Nechť $M$ je konečná množina, $M_1, M_2, \dots M_k, k
    \in \mathbb N$ její podmnožiny takové, že
    \begin{enumerate}[$i.$]
    \item $M_1\cup M_2 \cup \dots \cup M_k = M,$
   	\item $M_i \cap M_j = \emptyset$ pro libovolná
    $i,j \in \left \{ 1,2,\dots,k \right \},i\ne j $.
    \end{enumerate}
    Pak platí $|M|=|M_1|+|M_2|+\dots +|M_k|,$ kde
    symbolem $|A|$ značíme počet prvků množiny $A$.
\end{veta}

\begin{veta}[Pravidlo součinu]
    Nechť $M_1, M_2, \dots, M_k, k\in \mathbb N$ jsou konečné množiny takové,
    že $|M_1| = m_1, |M_2|=m_2, \dots, |M_k| = m_k.$ Pak platí
    $$|M_1\times M_2 \times \dots \times M_k| = m_1m_2\dots m_k,$$
    kde  $M_1\times M_2\times \dots \times M_k= \left \{ \left [ a_1, a_2, \dots, a_k \right ]  \right \}
    , a_1\in M_1, a_2 \in M_2, \dots, a_k \in M_k.$
\end{veta}

\begin{veta}[Dirichletův princip]
    Má-li být alespoň $nk+1$ předmětů rozděleno do $k$ přihrádek, pak
    alespoň v jedné přihrádce je alespoň $n+1$ předmětů.
\end{veta}

\begin{definition}
    Nechť $M$ je neprázdná množina. \textbf{Rozklad množiny} $M$ značíme
    $\mathscr R(M)$ a definujeme jako neprázdný systém neprázdných podmnožiny
    $M_1, M_2, \dots, M_k, k\in \mathbb N,$ pro které platí
    \begin{enumerate}[$i.$]
    \item $M_1\cup M_2 \cup \dots \cup M_k = M,$
   	\item $M_i \cap M_j = \emptyset$ pro libovolná
    $i,j \in \left \{ 1,2,\dots,k \right \},i\ne j $.
    \end{enumerate}
    Množiny $M_1, M_2, \dots, M_k$ nazýváme \textbf{třídami rozkladu} $\mathscr R(M).$
\end{definition}

\section{Pravděpodobnost}
\begin{pozn}
    Rozlišujeme dva typy pokusů:
   	\begin{itemize}
    \item náhodný (předem neznáme výsledek),
   	\item determinovaný (předem známe výsledek).
    \end{itemize}
\end{pozn}

\begin{definition}
    \textbf{Náhodným jevem} rozumíme jakékoliv tvrzení o výsledku náhodného pokusu,
    o kterém můžeme po provedení říci, zda je nebo není pravdivé.
\end{definition}

\begin{definition}
    Nechť $A,B$ jsou jevy.
    \begin{enumerate}[$i.$]
    \item Řekneme, že \textbf{jev $A$ má za důsledek jev $B$} a zapisujeme
    $A\subseteq B$ nebo $A\implies B$ právě tehdy, když jev $B$ nastane vždy, když
    nastane jev $A.$
   	\item Řekneme, že jevy $A$ a $B$ \textbf{jsou si rovny} a zapisujeme $A=B$ právě
    tehdy, když $A\subseteq B \land B\subseteq A,$ tedy $B$ nastane právě tehdy,
    když nastane $A.$
   	\item Jev, který nastane při každé realizaci pokusu nazveme jevem \textbf{jistým}
    a označíme $\Omega.$ Jev, který nemůže nikdy nastat nazveme jevem \textbf{nemožným}
    a označíme $\emptyset.$
    \end{enumerate}
\end{definition}

\begin{definition}
    Nechť $A_1, A_2, \dots, A_n$ jsou jevy. Pak \textbf{sjednocením} (resp. \textbf{
    průnikem}) jevů $A_1, A_2, \dots, A_n$ rozumíme takový jev $A$, který nastane
    právě tehdy, když nastane alespoň jeden (resp. všechny z) jevů $A_1, A_2,
    \dots, A_n$.
\end{definition}

\begin{definition}
    Nechť $A,B$ jsou jevy.
    \begin{enumerate}[$i.$]
    \item \textbf{Opačným jevem} k jevu $A$ nazveme takový jev $\overline A,$ který
    nastane právě tehdy, kdyý nenastane jev $A.$
   	\item \textbf{Rozdílem jevů} $A,B$ nazýváme jev $A-B,$ který nastane právě tehdy,
    když nastane jev $A$ a nenastane jev $B$.
    \end{enumerate}
\end{definition}

\begin{definition}
    Jevy nazveme \textbf{neslučitelné} (disjunktní), jestliže $A\cap B = \emptyset.$
\end{definition}

\begin{definition}
    Nechť je dán náhodný pokus. Jev $A$ nazveme \textbf{elementárním jevem} právě tehdy,
    když neexistují žádné dva jevy $B,C; A\ne B\ne C$ takové, že $A=B\cup C,$ tj.
    $A$ nelze vyjádřit jako sjednocení dvou jevů různých od $A.$ Množinu všech
    elementárních jevů nazveme \textbf{jevovým polem}. Je to množina všech možných
    výsledků daného náhodného pokusu. Tuto množinu označíme $\Omega.$
\end{definition}

\begin{definition}
    Nechť $\Omega$ je konečná neprázdná množina stejně možných výsledků daného
    náhodného pokusu, tzn. $\Omega$ je jevové pole. Nechť $A\subseteq \Omega$ je jev.
    Označme $|A|,$ resp. $|\Omega|$ počet prvků množiny $A$, resp. $\Omega.$ Pak
    \textbf{pravděpodobností} jevu $A$ nazýváme číslo
    $$P(A)=\frac{|A|}{|\Omega|}.$$
\end{definition}

\begin{veta}
    Nechť $A,B\subseteq \Omega$ jsou jevy. Pak
    $$P(A\cup B)=P(A)+P(B)-P(A\cap B).$$
\end{veta}

\begin{definition}
    Nechť $A,B\subseteq \Omega$ jsou jevy. Nazveme je \textbf{nezávislé} právě tehdy,
    když $P(A\cap B)=P(A)\cdot P(B).$
\end{definition}

\begin{definition}
    Nechť $A,B\subseteq \Omega$ jsou jevy takové, že $P(B)>0.$ \textbf{Podmíněnou
    pravděpodobností} jevu $A$ za předpokladu nastoupení jevu $B$ nazýváme reálné
    číslo dané vzorcem
    $$P(A\, | \, B) = \frac{P(A\cap B)}{P(B)}.$$
\end{definition}

\begin{veta}[Formule úplné pravděpodobnosti]
    Nechť $A,B_i \subseteq \Omega, i \in \left \{ 1, 2, \dots, n \right \} $ jsou jevy
    takové, že $A\subseteq \bigcup_{i=1}^n B_i$ a $\forall i\ne j: B_i\cap B_j\ne
    \emptyset.$ Pak $P(A)=P(B_1)\cdot P(A\, |\, B_1) + P(B_2)\cdot P(A \, |\, B_2)+
    \dots + P(B_n)\cdot P(A \,|\, B_n)=\sum_{i=1}^n P(B_i)\cdot P(A\, |\, B_i).$
\end{veta}

\begin{veta}[Bayesův vzorec]
    Nechť $A,B_i \subseteq \Omega, i \in \left \{ 1, 2, \dots, n \right \} $ jsou
    takové jevy, že $A\subseteq \bigcup_{i=1} B_i, \forall i,j, i\ne j: B_i\cap B_j=
    \emptyset.$ Nechť $\exists i \in \left \{ 1, 2, \dots, n \right \} : P(B_i) >0.$
    Pak platí: $\forall k\in \left \{ 1, 2, \dots, n \right \}:$
    $$P(B_k \, |\, A)=\frac{P(B_k)\cdot P(A\, |\, B_k)}{\sum_{i=1}^n P(B_i)\cdot
    P(A\, |\, B_i)}. $$
\end{veta}

\begin{veta}[Bernoulliho věta]
    Provádíme-li sérii $n$ nezávislých pokusů, kdy pravděpodobnost příslušného
    pokusu je $p,$ pak pravděpodobnost toho, že právě $k$ pokusů bude úspěšných, platí
    $$P(k,n)=\binom{n}{k}\cdot p^k \cdot (1-p)^{n-k}.$$
\end{veta}

\section{Stereometrie -- polohové vlastnosti}

\begin{definition}
    Body ležící na jedné přímce (resp. v jedné rovině) se nazývají \textbf{kolineární}
    (resp. \textbf{komplanární}).
\end{definition}

\begin{definition}
    Přímky $p,q \in \mathscr P$ nazveme:
    \begin{enumerate}[$i.$]
        \item \textbf{různé rovnoběžky}, jestliže $p\cap q = \emptyset \land
            p,q$ jsou komplanární;
        \item \textbf{mimoběžky}, jestliže $p\cap q = \emptyset \land p,q$ nejsou
            komplanární;
        \item \textbf{různoběžky} s \textbf{průsečíkem} $P$, jestliže $p\cap q =
            \left \{ P \right \} $;
       	\item \textbf{splývající rovnoběžky}, jestliže $p\cap q = p$.
    \end{enumerate}
\end{definition}

\begin{veta}[Axiom rovnoběžnosti]
    Každým bodem v $\mathbb E_2$ lze vést ke každé přímce právě jednu rovnoběžku.
\end{veta}

\begin{definition}
    Roviny $\alpha,\beta \subseteq \mathbb E_3$ nazveme:
    \begin{enumerate}[$i.$]
        \item \textbf{rovnoběžné splývající}, jestliže $\alpha = \beta$;
        \item \textbf{rovnoběžné různé}, jestliže $\alpha \ne \beta \land \alpha \cap
        \beta = \emptyset$;
        \item \textbf{různé roviny} s \textbf{průsečnicí} $p$, jestliže $\alpha
        \ne \beta \land \alpha \cap \beta = p$.
    \end{enumerate}
\end{definition}

\begin{veta}[Kritérium rovnoběžnosti dvou rovin]
    Nechť $\alpha, \beta$ jsou roviny. Jestliže rovina $\alpha$ obsahuje dvě různoběžky
    $a,b$ takové, že $\alpha \cap \beta = \emptyset \land b \cap \beta = \emptyset,$
    pak $\alpha \parallel \beta.$
\end{veta}

\begin{proof}
    Je-li $\alpha = \beta,$ předpoklad tvrzení neplatí, takže implikace platí
    triviálně. \\
    Dále sporem: Nechť $\alpha \nparallel \beta \implies \alpha \cap \beta \ne
    \emptyset \implies \exists c = \alpha \cap \beta.$ Protože $a,b$ jsou různoběžky,
    alespoň jedna z těchto přímek protíná přímku $c.$ Nechť je to např. $a.$ Pak
    $\alpha \cap c = \left \{ B \right \} .$ Protože $c\subseteq \beta,$ průsečík
    $B\in \beta \implies \alpha \cap \beta \ne \emptyset,$ což je spor s předpokladem
    $\alpha \cap \beta = \emptyset.$
\end{proof}

\begin{definition}
    Přímku $a \subseteq \mathbb E_3$ nazveme s rovinou $\alpha \subseteq \mathbb E_3:$
    \begin{enumerate}[$i.$]
        \item \textbf{rovnoběžnou}, jestliže $a\cap \alpha = \emptyset$;
        \item \textbf{různoběžnou}, jestliže $a\cap \alpha = \left \{ P \right \} $;
        \item $a$ \textbf{leží v rovině} $\alpha$, jestliže $a \cap \alpha = a$.
    \end{enumerate}
\end{definition}

\begin{veta}[Kritérium rovnoběžnosti přímky a roviny]
    $\forall p \in \mathscr P, \forall \rho \subseteq \mathbb E_3: p \parallel \rho
    \iff \exists q\subseteq \rho: p\parallel q.$
\end{veta}

\begin{proof}
    Pokud $p \subseteq \rho \implies q=p$ a tvrzení platí.\\
    Nechť $p\not \subseteq \rho:$
    \begin{enumerate}[$i.$]
        \item \uv{$\implies$}: Nechť $p\parallel\rho\implies p\land \rho=\emptyset.$
        Nechť $A\in\rho$ je lib. bod. Potom bodem $A$ a přímkou $p$ je jednoznačně
        určena rovina $\sigma.$ $A\in \rho\cap\sigma \implies \rho\cap\sigma=q,$
        $p,q$ jsou komplanární a mají prázdný průnik $\implies p\parallel q.$
        \item \uv{$\impliedby$}: Sporem: Předpokládejme, že $p\parallel q \land
        p\nparallel \rho.$ $p\parallel q \implies p\cap q=\emptyset\land q\subseteq \rho,
        p,q$ leží v téže rovině a $p\ne q\implies p\cap q = \emptyset$, což je spor.
    \end{enumerate}
\end{proof}


\begin{priklad}
Je dána krychle $ABCDEFGH$, na jejích hranách body $R,S,T$ podle obrázku. Určete
průsečík přímky $DF$ a $RST$.
\end{priklad}

\begin{definition}\label{kolmeprimky}
    Přímky $p,q \subseteq \mathbb E_3$ se nazývají navzájem \textbf{kolmé}, právě když
    existují $p^\prime, q^\prime \subseteq \mathbb E_3$ takové, že
    \begin{enumerate}[$i.$]
    \item   $p^\prime \parallel p \land q^\prime \parallel q,$
   	\item $p^\prime, q^\prime$ jsou komplanární,
   	\item $p^\prime \perp q^\prime.$
    \end{enumerate}
\end{definition}

\begin{pozn}
    Definice \ref{kolmeprimky} je i kritériem.
\end{pozn}

\begin{veta}\label{kolmostprimekposun}
    Nechť $p,q \subseteq \mathbb E_3, p\perp q.$ Pak $\forall p^\prime, q^\prime
    \subseteq \mathbb E_3: p^\prime \parallel p, q^\prime \parallel q \implies
    p^\prime \perp q^\prime.$
\end{veta}

\begin{priklad}
Je dána krychle $ABCDEFGH$. Bod $K$ je středem $EA$, $C$ je středem $FG$. Rozhodněte,
zda následující dvojice přímek jsou kolmé:
\begin{enumerate}[$a.$]
\item $DH$ a $BC$,
\item $CL$ a $KH$.
\end{enumerate}
\end{priklad}

\begin{reseni}
Využijeme věty \ref{kolmostprimekposun}.
\begin{enumerate}[$a.$]
\item Triviálně.
\item Jednu z přímek posuneme do roviny té druhé. Pak plyne triviálně.
\end{enumerate}
\end{reseni}

\begin{definition}
    Nechť $p\subseteq \mathbb E_3$ je přímka a $\alpha \subseteq \mathbb E_3$ je rovina.
    Řekneme, že \textbf{přímka} $p$ je \textbf{kolmá k rovině} $\alpha$ právě tehdy,
    když $\forall q \subseteq \alpha: q \perp p.$
\end{definition}

\begin{veta}[Kritérium kolmosti přímky a roviny]
    Nechť $p\subseteq \mathbb E_3$ je přímka a $\alpha \subseteq \mathbb E_3$ je rovina.
    Pak  $p\perp \alpha \iff \exists q,r \subseteq \alpha: q \nparallel r: q\perp p\land
    r\perp p.$
\end{veta}

\begin{priklad}
Nechť je dán pravidelný čtyřstěn $ABCD$. Dokažte, že $AB$ je kolmá k $CD$.
\end{priklad}

\begin{reseni}
Jednou přímkou proložíme rovinu a dokážeme, že tato rovina je kolmá k druhé přímce.
Konkrétně: přímkou $CD$ proložíme rovinu $\overleftrightarrow{SCD}$ ($S$ je střed
$AB$). Pak plyne triviálně.
\end{reseni}

\begin{definition}\label{kolmeroviny}
    Řekneme, že \textbf{rovina} $\alpha$ \textbf{je kolmá k rovině} $\beta$, jestliže
    $\exists a\subseteq\alpha: \alpha \perp\beta.$
\end{definition}

\begin{pozn}
    Definice \ref{kolmeroviny} je i kritériem.
\end{pozn}

\begin{pozn}
    Musíme umět sestrojit průsečík přímky a roviny, průsečnici dvou rovin a řez
    tělesa rovinou.
\end{pozn}

\begin{definition}
\textbf{Shodným} (resp. \textbf{podobným}) \textbf{zobrazením} v prostoru (shodností,
resp. podobností) nazýváme zobrazení $\mathscr Z:
\mathbb E_3 \to \mathbb E_3$, jestliže platí
\begin{align*}
    \forall X, Y \in \mathbb E_3: |\mathscr Z(X)\mathscr Z(Y)| =|XY|, & & \textrm{resp. } |\mathscr Z(X)\mathscr Z(Y)|=k|XY|, \,\,\, k \in \mathbb R^+
\end{align*}
a číslo $k$ \textbf{koeficientem podobnosti}.
\end{definition}

\begin{pozn}
    Shodné zobrazení je podobné zobrazení s koeficientem podobnosti 1.
\end{pozn}

\begin{definition}
    Nechť $S\in \mathbb E_3$ je bod. Pak zobrazení $\mathscr S_S: \mathbb E_3 \to
    \mathbb E_3$ nazýváme \textbf{středovou souměrností} se středem $S$, jestliže
    \begin{enumerate}[$i.$]
    \item $X=S\implies X=\mathscr S_S(X),$
   	\item $X\ne S \implies S$ je střed $X\mathscr S_S(X).$
    \end{enumerate}
\end{definition}

\begin{definition}
    $\forall \mathscr U \subseteq \mathbb E_3: \mathscr U\ne \emptyset \,\,\, \mathscr
    U$ je \textbf{útvar}.
\end{definition}

\begin{definition}
    $S\in \mathbb E_3$ je \textbf{střed souměrnosti} útvaru $\mathscr U,$ jestliže
    $\exists \mathscr U = \mathscr S_S(\mathscr U).$ Útvar je středově souměrný,
    pokud $\exists S:\exists\mathscr U = \mathscr S_S(\mathscr U).$
\end{definition}

\begin{definition}
    Nechť $\alpha$ je rovina. Pak zobrazení $\mathscr S_\alpha:\mathbb E_3 \to
    \mathbb E_3$ je \textbf{rovinová}  souměrnost, jestliže
    \begin{enumerate}[$i.$]
    \item $X\in\alpha\implies \mathscr S_\alpha(X)=X,$
   	\item $X\in\alpha\implies X\mathscr S_\alpha(X) \perp\alpha\,\land$ střed úsečky
    $X\mathscr S_\alpha(X)\in \alpha.$
    \end{enumerate}
\end{definition}

\begin{definition}
    Nechť $A,B$ jsou dva různé body, $\alpha \subseteq \mathbb E_3$ rovina, která
    prochází středem $AB$ a $AB\perp \alpha$. Pak rovinu $\alpha$ nazýváme
    \textbf{rovinou souměrnosti} bodů $A,B.$
\end{definition}

\begin{definition}
\begin{enumerate}[$i.$]
\item $\alpha$ je \textbf{rovina souměrnosti} útvaru $\mathscr U\subseteq \mathbb E_3,$
pokud $\mathscr U = \mathscr S_\alpha(\mathscr U).$
\item \textbf{Útvar} $\mathscr U \subseteq \mathbb E_3$ je \textbf{rovinově souměrný},
jestliže existuje jeho rovina souměrnosti.
\end{enumerate}
\end{definition}

\begin{veta}
    Nechť $\mathscr S_\beta \circ \mathscr S_\alpha: \mathbb E_3 \to \mathbb E_3$ je
    složené zobrazení dvou rovin souměrných s rovinami $\alpha, \beta\subseteq \mathbb
    E_3$. Nechť $\alpha \parallel \beta$, pak $\forall X\in \mathbb E_3: \mathscr
    S_\beta \cap \gamma,$ kde $\gamma$ je taková rovina, že $\alpha \perp \gamma \land
    \beta \perp \gamma, X \in \gamma.$
\end{veta}

\begin{definition}
    Složením dvou rovinových souměrností s rovnoběžnými rovinami souměrnosti vznikne
    zobrazení, které nazýváme \textbf{posunutím} v $\mathbb E_3.$ Směr kolmý k těmto
    rovinám nazýváme \textbf{směr postunutí}.
\end{definition}

\begin{definition}
    Složením dvou rovinových souměrností s různoběžnými rovinami souměrnosti vznikne
    zobrazení, které nazýváme \textbf{otočením} v $\mathbb E_3.$ Průsečnici těchto
    rovin nazýváme \textbf{osu otočení}.
\end{definition}

\begin{definition}
    Zobrazením $\mathscr S_\beta \circ \mathscr S_\alpha = \mathscr O_p,$ kde
    $\alpha \perp \beta, p=\alpha \cap \beta$, nazýváme \textbf{osovou souměrností}
    a $p$ \textbf{osou} osové souměrnosti.
\end{definition}

\begin{definition}
\begin{enumerate}[$i.$]
\item $p$ je \textbf{osa souměrnosti} útvaru $\mathscr U\subseteq \mathbb E_3,$
pokud $\mathscr U = \mathscr O_p(\mathscr U).$
\item \textbf{Útvar} $\mathscr U \subseteq \mathbb E_3$ je \textbf{osově souměrný},
jestliže existuje jeho osa souměrnosti.
\end{enumerate}
\end{definition}

\begin{definition}
    Nechť je dán bod $S\in \mathbb E_3$ a číslo $\lambda \in \mathbb R - \left \{ 0
    \right \}. $ Pak zobrazení $\mathscr H_{S,\lambda}: \mathbb E_3 \to \mathbb E_3$
   nazýváme \textbf{stejnolehlost}, jestliže $\forall X \in \mathbb E_3$ platí:
   \begin{enumerate}[$i.$]
   \item $X=S\implies X^\prime =X=S$,
  	\item $X\ne S\implies |SX^\prime| = |\lambda|\cdot |SX|,$ přičemž
   \begin{enumerate}[$a.$]
   \item $\lambda > 0: X\in \overrightarrow{SX},$
  	\item $\lambda < 0: X$ leží na opačné polopřímce k $\overrightarrow{SX}.$
   \end{enumerate}
   \end{enumerate}
\end{definition}

\begin{priklad}
Sestrojte řez krychle rovinou $\rho=\overleftrightarrow{VWU}$, kde $V$ je střed
úsečky $AE$, $W$ je střed úsečky $AB$ a $V$ je bod hrany $CG$ takový, že $|CU|:|UG|=2:1$.
\end{priklad}

\section{Stereometrie -- metrické vlastnosti}
\begin{definition}
    Nechť $A, B \in \mathbb E_3.$ \textbf{Vzdálenost bodů} $A,B$ nazýváme délku
    úsečky $AB$ a~označujeme $\rho(A,B).$
\end{definition}

\begin{definition}
    Nechť $A\in \mathbb E_3$ je bod a $\alpha \subseteq \mathbb E_3$ je rovina.
    \textbf{Kolmým průmětem} bodu $A$ do roviny $\alpha$ nazýváme bod $A_0$
    splňující
    \begin{enumerate}[$i.$]
    \item $A\in \alpha \implies A_0=A,$
   	\item $A\notin \alpha \implies A_0 \in p\cap \alpha, p \perp\alpha, A \in p.$
    \end{enumerate}
    \textbf{Vzdáleností bodu $A$ od roviny $\alpha$} nazýváme reálné číslo označené
    $\rho(A,\alpha) $ a definované
    $$\rho(A,\alpha) = \rho(A, A_0) = |AA_0|,$$
    kde $A_0$ je kolmý průmět bodu $A$ do roviny $\alpha.$
\end{definition}

\begin{priklad}
Určete vzdálenost bodu $F$ od roviny $\overleftrightarrow{BEG}$ v pravidelném
čtyřbokém hranolu $ABCDEFGH$, kde $|AB|=|BC|=a, |AE|=b$.
\end{priklad}

\begin{definition}
    Nechť $A\in \mathbb E_3$ je bod a $p \subseteq \mathbb E_3$ je přímka.
    \textbf{Kolmým průmětem} bodu $A$ na přímku $p$ nazýváme bod $A_0$
    splňující
    \begin{enumerate}[$i.$]
    \item $A\in p \implies A_0=A,$
    \item $A\notin p \implies A_0 \in p \cap q, q \perp p, A \in q.$
    \end{enumerate}
    \textbf{Vzdáleností bodu $A$ od přímky $p$} nazýváme reálné číslo označené
    $\rho(A,p) $ a definované
    $$\rho(A,p) = \rho(A, A_0) = |AA_0|,$$
    kde $A_0$ je kolmý průmět bodu $A$ na přímku $p.$
\end{definition}

\begin{definition}
    Nechť $\alpha, \beta \subseteq \mathbb E_3$ jsou dvě rovnoběžné roviny. Pak
    \textbf{vzdáleností dvou rovnoběžných rovin} $\alpha, \beta$ nazýváme reálné
    číslo označené $\rho(\alpha, \beta)$ a definované
    $$\rho(\alpha, \beta)=\rho(A,\beta),$$
    kde $A\in\alpha$ je libovolný bod.
\end{definition}

\begin{definition}
    Nechť $p\subseteq \mathbb E_3$ je přímka, $\alpha \subseteq \mathbb E_3$ je rovina.
    Nechť $p\parallel \alpha.$ \textbf{Vzdáleností přímky $p$ od roviny $\alpha$
    s ní rovnoběžné} nazýváme reálné číslo označené $\rho(p,\alpha)$ a definované
    $$\rho(p,\alpha)=\rho(A,\alpha),$$
    kde $A\in p$ je libovolný bod.
\end{definition}

\begin{definition}
    Nechť $p,q\subseteq \mathbb E_3$ jsou rovnoběžné přímky. \textbf{Vzdáleností
    dvou rovnoběžných přímek $p,q$} nazýváme reálné číslo označené $\rho(p,q)$ a
    definované
    $$\rho(p,q) = \rho(A,q),$$
    kde $A\in p$ je libovolný bod.
\end{definition}

\begin{definition}
    Nechť $p,q\subseteq \mathbb E_3$ jsou mimoběžné přímky. \textbf{Vzdáleností dvou
    mimoběžných přímek} $p,q$ nazýváme reálné číslo označené $\rho(p,q)$ a definované
    $$\rho(p,q)=\rho(\alpha, \beta),$$
    kde $\alpha \parallel \beta, p\subseteq\alpha, q\subseteq \beta.$
\end{definition}

\begin{definition}
    Nechť $p,q\subseteq \mathbb E_3$ jsou dvě komplanární přímky. \textbf{Odchylkou
    dvou komplanárních přímek} $p,q$ nazýváme reálné číslo označené $|\sphericalangle
    p,q|$ a definované
    \begin{enumerate}[$i.$]
    \item je-li $p\parallel q, |\sphericalangle p,q|=0^\circ,$
   	\item je-li $p\nparallel q$, odchylka $p,q$ je velikost ostrého nebo pravého úhlu,
        který $p,q$ svírají.
    \end{enumerate}
\end{definition}

\begin{veta}
    Nechť $p^\prime, q^\prime; p,q\subseteq \mathbb E_3: p\parallel p^\prime\land
    q\parallel q^\prime.$ Pak platí: $|\sphericalangle p^\prime q^\prime|=|\sphericalangle p q|.$
\end{veta}

\begin{priklad}
Je dán pravidelný čtyřboký jehlan $ABCDV$ s podstavnou hranou $a$ a výškou $v$.
Určete odchylku přímek $p,q$, kde $p=\overleftrightarrow{AV}, q=\overleftrightarrow{CD}.$
\end{priklad}

\begin{definition}
    Nechť $p,q\subseteq \mathbb E_3$ jsou dvě mimoběžné přímky. \textbf{Odchylkou
    dvou mimoběžných přímek} $p,q$ nazýváme reálné číslo označené $|\sphericalangle
    p,q|$ a definované
    $$|\sphericalangle p,q| = |\sphericalangle p^\prime, q^\prime|,$$
    kde $p^\prime \parallel p \land q^\prime \parallel q,$ kde $p^\prime, q^\prime$
    jsou komplanární a různoběžné.
\end{definition}

\begin{definition}
    Nechť $p\subseteq \mathbb E_3$ je přímka a $\alpha\subseteq \mathbb E_3$ je
    rovina. \textbf{Odchylkou přímky $p$ od roviny $\alpha$} nazýváme
    reálné číslo označené $|\sphericalangle
    p,\alpha|$ a definované
    \begin{enumerate}[$i.$]
    \item je-li $p\parallel \alpha, |\sphericalangle p,\alpha|=0^\circ,$
    \item je-li $p\nparallel \alpha$, odchylka $p,\alpha$ je velikost ostrého nebo
        pravého úhlu,
        který svírají přímky $p,q$, kde $q$ je průsečnice rovin $\alpha, \beta$,
        přičemž $\beta \perp \alpha, p\in\beta.$
    \end{enumerate}
\end{definition}

\begin{definition}
    Nechť $\alpha,\beta\subseteq \mathbb E_3$ jsou dvě roviny. \textbf{Odchylkou
    rovin} $\alpha, \beta$ nazýváme reálné číslo označené $|\sphericalangle
    \alpha, \beta|$ a definované
    \begin{enumerate}[$i.$]
    \item je-li $\alpha\parallel \beta, |\sphericalangle \alpha,\beta|=0^\circ,$
   	\item je-li $\alpha\nparallel \beta$, odchylka $\alpha,\beta$ je velikost
        ostrého nebo pravého úhlu,
        který svírají přímky $p,q,$ kde $p\subseteq \alpha, q\subseteq \beta$ a obě
        přímky jsou kolmé k průsečnici rovin $\alpha, \beta.$
    \end{enumerate}
\end{definition}

\begin{definition}
    Nechť jsou dány nekomplanární body $A,B,C,D.$ Pak
  \textbf{čtyřstěn} je množina bodů ohraničená trojúhelníky $\triangle ABC, \triangle
  ABD, \triangle ACD, \triangle BCD.$
  \textbf{Pravidelný čtyřstěn} je tvořen čtyřmi stejnými rovnostrannými trojúhelníky.
\end{definition}

\begin{definition}
    Mějme v prostoru rovinu $\rho,$ v ní konvexní mnohoúhelník $A_1\dots A_n$ a nechť
    $A_1^\prime$ je bod, který v rovině $\rho$ neleží. Nechť $T: \mathbb E_3 \to
    \mathbb E_3$ je takové posunutí, že $A_1^\prime=T(A_1).$ Při tomto zobrazení se rovina
    $\rho$ zobrazí na rovinu $\rho^\prime,$ tyto dvě roviny jsou rovnoběžné. Množinu všech
    bodů $X$, všech úseček $BB^\prime$ takových, že $B\in A_1\dots A_n$ a $B^\prime$ je
    obraz bodu $B$ v posunutí $T$, nazýváme \textbf{hranolem}. Mnohoúhelníky
    $A_1A_2\dots A_n$ a $A^\prime_1A^\prime_2\dots A^\prime_n$ nazýváme \textbf{podstavami},
    rovnoběžníky $A_iA_{i+1}A^\prime_{i+1}A_i^\prime, i=1,\dots,n$, nazýváme
    \textbf{bočními stěnami hranolu}. Všechny boční stěny tvoří \textbf{plášť hranolu}.
    Podstavy spolu s bočními stěnami tvoří \textbf{stěny hranolu}. Úsečky $A_iA^\prime_i$,
    resp. $A_iA_{i+1}, A^\prime_iA^\prime_{i+1}$ se nazývají \textbf{boční}, resp. \textbf{podstavné hrany}
    hranolu. Body $A_1,A_2,\dots,A_n$ a $A^\prime_1,A^\prime_2,\dots,A^\prime_n$ se
    nazývají \textbf{vrcholy}. Je-li směr posunutí kolmý k~rovině podstavy, mluvíme
    o \textbf{hranolu kolmém}, v opačném případě jde o \textbf{hranol kosý}.
\end{definition}

\begin{definition}
    Hranol, jehož podstavy jsou rovnoběžníky, nazýváme \textbf{rovnoběžnostěn}.
\end{definition}

\begin{definition}
    Rovnoběžnostěn, jehož všechny stěny jsou pravoúhelníky (resp. čtverce) nazýváme
    \textbf{kvádr} (resp. \textbf{krychle}).
\end{definition}

\begin{definition}
    Mějme v prostoru rovinu $\rho,$ v ní kruh $K$ ohraničený kružnicí $k$, na níž
    leží bod $A$ a nechť
    $A^\prime$ je bod, který v rovině $\rho$ neleží. Nechť $T: \mathbb E_3 \to
    \mathbb E_3$ je takové posunutí, že $A^\prime=T(A).$ Označme $T(\rho)=\rho^\prime,
    T(k)=k^\prime, T(K)=K^\prime.$ Všechny body všech úseček $XX^\prime$, kde $X\in K$
    a $X^\prime$ je obraz bodu $X$, vytvoří \textbf{válec.} Omezíme-li se pouze na
    body $X$ ležící na kružnici $k$, dostaneme \textbf{plášť válce}. Kruhy $K,K^\prime$
    tvoří \textbf{podstavy válce}. Je-li směr posunutí $T$ kolmý k rovině $\rho$,
    mluvíme o \textbf{kolmém válci}, v opačném případě o \textbf{kosém válci}.
\end{definition}

\begin{definition}
Mějme v prostoru rovinu $\rho,$ v ní konvexní mnohoúhelník $A_1\dots A_n$ a nechť
$V$ je bod, který v rovině $\rho$ neleží. Úsečky $VX$, kde $X$ jsou všechny body
mnohoúhelníka $A_1\dots A_n$, nazýváme \textbf{jehlanem}. Bod $V$ se nazývá
\textbf{hlavním vrcholem} jehlanu, jeho další \textbf{vrcholy} jsou $A_1,\dots,A_n$.
Mnohoúhelník $A_1\dots A_n$ je \textbf{podstava} jehlanu. Trojúhelníky
$A_iA_{i+1}V, i=1,\dots,n-1$ jsou \textbf{bočními hranami} jehlanu.
Úsečky $A_iA_{i+1}$ jsou \textbf{podstavnými hranami}. Boční stěny tvoří \textbf{plášť}
jehlanu. \textbf{Jehlan} se nazývá \textbf{pravidelný}, jestliže je jeho podstavou
pravidelný mnohoúhelník a~jeho hlavní vrchol má stejně velké vzdálenosti od
všech vrcholů podstavy.
\end{definition}

\begin{definition}
    Nechť je dán jehlan s hlavním vrcholem $V$ a podstavou $A_1\dots A_n$ v rovině
    $\rho.$ Nechť $k \in \mathbb R, k \ne 1, k >0.$ Zaveďme $\mathscr H_{V,k}(A_1\dots A_n)
    =A_1^\prime\dots A_n^\prime.$ Těleso ohraničené podstavami $A_1\dots A_n$,
    $A_1^\prime\dots A_n^\prime$ a stěnami $A_iA_{i+1}A_i^\prime A_{i+1}^\prime$ se
    nazývá \textbf{komolý jehlan}.
\end{definition}

\begin{definition}
    Nechť je dán vrchol $V$ a kruhová podstava $K$ v rovině
    $\rho.$ Množina všech bodů úseček $VX,$ kde $X\in K,$ se nazývá \textbf{kužel}.
    Jestliže $Y\in k,$ tvoří body úseček $VY$ \textbf{plášť} kužele, kruh $K$ je
    \textbf{podstavou kužele}, bod $V$ \textbf{vrcholem} kužele. Je-li
    $\overleftrightarrow{SV}\perp \rho,$ nazývá se \textbf{kužel kolmý} (rotační).
\end{definition}

\begin{definition}
Nechť je dán kužel s hlavním vrcholem $V$ a podstavou $K$ v rovině
$\rho.$ Nechť $k \in \mathbb R, k \ne 1, k >0.$ Zaveďme $\mathscr H_{V,k}(K)
=K^\prime.$ Množina všech bodů úseček $X\mathscr H(X),$ kde $X\in K$, je
\textbf{komolý kužel}.
\end{definition}

\begin{veta}
  Označme $V$ objem tělesa, $S$ jeho povrch, dále obsahy $S_{\rm pláště}$, $S_{\rm podstavy}$, $\mathbf{a},\mathbf{b},\mathbf{c}$ vektory stran, $a$, $b$, $c$ jejich délky. Pak se rovnají:

  \begin{center}
   \footnotesize
    \begin{tabularx}{\textwidth}{ l | l  l  l }

      \, & $V$ & $S$ & $S_{\text{pláště}}$ \\
      \hline
      prav. čtyřstěn & $\frac{\sqrt{2}}{12}a^3$ & $\sqrt{3}a^2$ & \\
      hranol & $S_{\rm podstavy} \cdot v$ & $2S_{\rm podstavy}+S_{\rm pláště}$ & $o_{\rm podstavy} \cdot v$ \\
      rovnoběžnostěn & $| ( \mathbf{a} \times \mathbf{b} ) \cdot \mathbf{c} | $ & {\rm z vektorového součinu} & \, \\
      kvádr & $abc$ & $2(ab+bc+ca)$ & \, \\
      krychle & $a^3$ & $6a^2$ & \, \\
      válec & $S_{podstavy}\cdot v$ & $2S_{\rm podstavy} + S_{\rm pláště}$ & $o_{\rm podstavy} \cdot v$ \\
      jehlan & $\frac{1}{3}\cdot S_{\rm podstavy}\cdot v$ & $S_{\rm podstavy} + S_{\rm pláště}$ & \, \\
      komolý jehlan & $\frac{1}{3}(S_{\rm p1} + \sqrt{S_{\rm p1} S_{\rm p2}} + S_{\rm p2})\cdot v$ & \, & \,\\
      kužel & $\frac{1}{3}\cdot S_{\rm podstavy}\cdot v$ & \, & \, \\
      komolý kužel & $\frac{1}{3}\pi(r_1^2 + r_1r_2 + r_2^2)\cdot v$ & \, & \,\\
      koule & $\frac{4}{3}\pi r^3$ & $4\pi r^2$ & \, \\
      kulová úseč & $\frac{\pi v^2}{3}(3r-v)$ & $\pi v (4r-v)$ & \, \\
      kulový vrchlík & \, & $2\pi r v$ & \,
    \end{tabularx}
  \end{center}
  \normalsize
\end{veta}

\begin{veta}[Cavalieriho princip]
    Nechť tělesa $T_1, T_2$ leží mezi dvěma rovinami $\rho_1, \rho_2$ a každá rovina
    $\rho \parallel \rho_1\parallel \rho_2$ protne tělesa $T_1, T_2$ v konvexních
    rovinných útvarech s obsahy $P_1, P_2.$ Jestliže pro každou rovinu $\rho$ platí
    $P_1=P_2$, mají $T_1$ a $T_2$ stejný objem.
\end{veta}

\begin{definition}
\textbf{Mnohostěn} je konvexní část prostoru, hranice je tvořena konečným počtem
mnohoúhelníků.
\end{definition}

\begin{veta}[Eulerova věta]
    Nechť $s$ je počet stěn, $h$ počet hran a $v$ počet vrcholů daného tělesa. Pak
    platí
    \begin{align*}
        s-h+v &=2,\\
        s+v &=h+2.
    \end{align*}
\end{veta}

\begin{pozn}[Pravidelné mnohostěny]\,
\begin{center}
\begin{tabular}{l|l|c|c|c|r}
    prav. mnohostěn & tvar stěny & $v$ & $s$ & $h$ & \, \\
    čtyřstěn        & rovnostranný trojúhelník & 4 & 4&6 & tetraedr \\
    šestistěn       & čtverec & 8 & 6&12 & hexaedr \\
    osmistěn        & rovnostranný trojúhelník & 6 & 8&12 & oktaedr \\
    dvanáctistěn    & pravidelný pětiúhelník & 20 & 12&30 & dodekaedr \\
    dvacetistěn     & rovnostranný trojúhelník & 12 & 20&30 & ikosaedr \\
\end{tabular}
\end{center}

\end{pozn}

\begin{priklad}
Odvoďte vztah pro povrch pláště rotačního kužele o výšce $v$ a poloměru
podstavy $r$.
\end{priklad}

\begin{definition}
    O tělesu $M$ v prostoru říkáme, že má $p$ za osu rotace a že je \textbf{rotačním
    tělesem}, jestliže se zobrazí samo na sebe při každém otočení kolem přímky $p.$
\end{definition}

\begin{pozn}\,
\begin{itemize}
\item \textbf{rotační válec} -- pravoúhelník otáčený kolem své strany,
\item \textbf{rotační kužel} -- pravidelný trojúhelník otáčený kolem odvěsny,
\item \textbf{koule} -- půlkruh nad průměrem, hranice se nazývá \textbf{kulová plocha},
\item \textbf{torus} -- kruh kolem osy, která leží mimo něj
\end{itemize}
\end{pozn}

\section{Vektorové prostory}
\begin{definition}
    Nechť $P, X, Q, Y\in \mathbb E_3$ jsou čtyři libovolné body. Řekneme, že body
    $P, X, Q, Y$ tvoří
    vrcholy zobecněného rovnoběžníku, jestliže střed úsečky $PQ$ splývá se středem úsečky
    $XY$.
\end{definition}

\begin{definition}
    Na množině $\mathscr U$ všech orientovaných úseček s pevným počátečním bodem
    $P\in \mathbb E_3$
    definujeme operaci sčítání takto:
    $$\forall \overrightarrow{PX}, \overrightarrow{PY} \in \mathscr U:
        \overrightarrow{PX} +  \overrightarrow{PY} = \overrightarrow{PQ},$$
    je-li bod $Q\in \mathbb E_3$
    vrcholem zobecněného rovnoběžníku $PXQY$.
\end{definition}

\begin{definition}
    Na množině $\mathscr U$ všech orientovaných úseček s pevným počátečním bodem
    $P\in \mathbb E_3$
    definujeme operaci \textbf{násobení orientovaných úseček reálným číslem}
    (tzv. vnější násobení) takto: je-li $p\in \mathbb R$ a $\overrightarrow{PX}\in \mathscr U$,
    pak $p$-násobkem orientované
    úsečky $\overrightarrow{PX}$ nazveme orientovanou úsečku $\overrightarrow{PY}$
    (a zapisujeme $\overrightarrow{PY}=p \cdot \overrightarrow{PX}$), přičemž
    platí:
    \begin{enumerate}[$i.$]
    \item $p=0 \implies Y=P,$
   	\item $p\ne 0 \implies Y = \mathscr H_{P,p}(X).$
    \end{enumerate}
\end{definition}

\begin{definition}
    Nechť je dána množina $G$ s operací $*:G\times G \to G$ (je na $G$ uzavřená).
    Pak dvojice $(G,*)$ je \textbf{grupa}, jestliže platí:
    \begin{enumerate}[$i.$]
        \item $\forall a,b,c, \in G: a*(b*c) = (a*b)*c$ (asociativita),
       	\item $\exists e \in G$ takové, že $\forall a\in G a*e=e*a=a$ (existence neutrálního prvku),
       	\item $\forall a\in G \exists a^{-1}$ takové, že $a*a^{-1}=a^{-1}a=1$ (existence inverzního prvku).
    \end{enumerate}
    Pokud navíc
    \begin{enumerate}[$iv.$]
        \item $\forall a,b \in G: a*b=b*a$ (komutativita),
    \end{enumerate}
    je $(G,*)$ \textbf{komutativní} (též \textbf{Abelovská}) \textbf{grupa}.
\end{definition}

\begin{definition}\label{vekt_prost}
    Nechť je dána množina $V$, těleso $T$ a dvě operace $\bigoplus: V\times V \to V$ a
    $\bigotimes: T\times V \to V$ (jsou na $V$ uzavřené). Pak čtveřici $(V,T,\bigoplus,
    \bigotimes)$ je \textbf{vektorový prostor} nad tělesem $T$, jestliže $\forall p,q \in T, \vec u,
    \vec v, \vec w \in V$ platí:
    \begin{enumerate}[$i.$]
    \item $\vec u \bigoplus \vec v = \vec v \bigoplus \vec u$ (komutativita sčítání),
   	\item $(\vec u \bigoplus \vec v)\bigoplus \vec w = \vec u \bigoplus (\vec v \bigoplus \vec w)$ (asociativita sčítání),
   	\item $\exists \vec o\in V$ takové, že $\forall \vec u\in V:\vec u \bigoplus \vec o = \vec o \bigoplus \vec u = \vec u$ (existence nulového prvku),
   	\item $\forall \vec u \in V: \exists -\vec u \in V: \vec \bigoplus (-\vec u) = (-\vec u) \bigoplus \vec u = \vec o$ (existence opačného prvku),
   	\item $p\bigotimes (\vec u \bigoplus \vec v)=p\bigotimes \vec u \bigoplus p\bigotimes \vec v$ (distributivita),
   	\item $(p+q)\bigotimes \vec u=p\bigotimes \vec u \bigoplus p\bigotimes \vec v$ (distributivita),
   	\item $(p\cdot q)\bigotimes \vec u = p\bigotimes (q\bigotimes \vec u)$ (asociativita vnějšího násobení),
   	\item $\exists 1 \in T$ taková, že $\forall \vec u \in V: 1\bigotimes \vec u = \vec u$ (existence neutrálního prvku vzhledem k násobení).
    \end{enumerate}
\end{definition}

\begin{pozn}
    Výčet prvních čtyř podmínek z definice \ref{vekt_prost} lze zjednodušit jako:
    $(V,+)$ je komutativní grupa.
\end{pozn}

\begin{pozn}
    Místo znaků $\bigoplus$ (resp. $\bigotimes$) používáme znaky $+$ (resp. $\cdot$).
    Byly použity, aby bylo jednoznačně odlišeno sčítání vektorů a čísel (resp. násobení
    vektorů skalárem a násobení čísel). Z kontextu je však jasně zřejmé, kterou operaci
    použít. V dalším textu budeme již používat znaky $+$ (resp $\cdot$).
\end{pozn}

\begin{pozn}
    Množina $\mathscr U_n$ všech orientovaných úseček s počátečním bodem $P\in \mathbb E_n$
    je vektorovým prostorem
    \textbf{vázaných vektorů}.
\end{pozn}

\begin{pozn}
    Množina $\mathbb R^{(n)}$ všech uspořádaných $n$-tic tvoří \textbf{aritmetický}
    vektorový prostor.
\end{pozn}

\begin{definition}
    Nechť $A,B,C,D \in \mathbb E_n$ jsou body. Řekneme, že orientované úsečky
    $\overrightarrow{AB}$ a $\overrightarrow{CD}$ jsou \textbf{ekvipolentní}, jestliže
    střed úsečky $AD$ je i středem úsečky $BC$. Zapisujeme $\overrightarrow{AB}\varepsilon\overrightarrow{AB}.$
\end{definition}

\begin{definition}
    Nechť je dána relace $R$ na množině $M$. Pokud
    \begin{enumerate}[$i.$]
    \item $\forall a,\in M:[a,a] \in R$ (reflexivita),
   	\item $\forall a,b\in M: [a,b]\in R \implies [b,a] \in R$ (symetričnost),
   	\item $\forall a,b,c\in M: ([a,b]\in R \land [b,c] \in R) \implies [a,c] \in R$ (tranzitivita),
    \end{enumerate}
    je $R$ \textbf{relací ekvivalence}.
\end{definition}

\begin{pozn}
    Ke každé relaci ekvivalence na množině $M$ existuje rozklad na
    \textbf{třídy ekvivalence} tak, že každé dva prvky v rámci každé třídy jsou navzájem ekvivalentní,
    třídy jsou po dvou disjunktní a sjednocením všech tříd dostaneme množinu $M$.
\end{pozn}

\begin{definition}
    Nechť je dána množina $M$ všech orientovaných úsešek v $\mathbb E_n$ a
    relace ekvipolence $\varepsilon \subseteq M\times M.$ Pak třída rozkladu množiny
    $M$, který přísluší relaci $\varepsilon$, je \textbf{volný vektor}.
\end{definition}

\begin{definition}
    Nechť $V$ je množina všech volných vektorů v $\mathbb E_n$. Pak na množině $V$ definujeme
    operace sčítání a vnější násobení takto:
    \begin{enumerate}[$i.$]
    \item $\forall \vec u, \vec v \in V: \vec u + \vec v = \vec w,$ kde $\vec w = \left \{ \overrightarrow{XY}; \overrightarrow{XY} \, \varepsilon \, \overrightarrow{PC} \right \} $, kde $\overrightarrow{PA}\in\vec u, \overrightarrow{PB}\in V$ a $\overrightarrow{PC}=\overrightarrow{PA}+\overrightarrow{PB}$ je součet orientovaných úseček $\overrightarrow{PA},\overrightarrow{PB}.$
   	\item $\forall p \in \mathbb R, \forall \vec u \in V: p\cdot \vec u = \vec z, \vec z = \left \{ \overrightarrow{XY}; \overrightarrow{XY} \, \varepsilon \, \overrightarrow{PC} \right \} $, přitom $\overrightarrow{PE}\in \vec u$ a $\overrightarrow{PF} = p\cdot \overrightarrow{PE}$ je vnější součin orientované úsečky $\overrightarrow{PE}$ a čísla $p$ (tj. pomocí stejnolehlosti).
    \end{enumerate}
\end{definition}

\begin{definition}
Nechť $\vec u \subseteq M$ je libovolný volný vektor a $\overrightarrow{AB}\subseteq \vec u$ orientovaná úsečka, tedy
$u =\left  \{ \overrightarrow{XY} : \overrightarrow{XY} \, \varepsilon \, \overrightarrow{AB} \right \}$. Pak orientovanou úsečku $\overrightarrow{AB}$ nazveme \textbf{umístěním} vektoru $\vec u$.
\end{definition}

\begin{definition}
    Pokud $\overrightarrow{PA}\in \vec u$, nazýváme orientovanou úsečku $\overrightarrow{PA}$ též reprezentantem vektoru $\vec u$.
\end{definition}

\begin{definition}
    Nechť $V$ je vektorový prostor, $\vec u_1,\dots, \vec u_k\in V$ vektory, $p_1,\dots,
    p_k\in \mathbb R.$ Vektor
    $$\vec x = p_1\vec u_1 + p_2\vec u_2 + \dots + p_k\vec u_k = \sum_{i=1}^{k} p_i\vec u_i$$
    nazýváme \textbf{lineární kombinací} vektorů $\vec u_1,\dots, \vec u_k.$
\end{definition}

\begin{definition}
    Podmožina $W$ vektorového prostoru $V$ se nazývá \textbf{podprostor} vektorového
    prostoru $V$, jestliže $W$ je vektorový prostor.
\end{definition}

\begin{definition}
    Množina všech lineárních kombinací vektorů množiny $S$ označená $\left < S \right >$ se nazývá \textbf{lineární
    obal} množiny $S$ a její prvky \textbf{generátory} $\left < S \right >$.
\end{definition}

\begin{definition}
    Nechť $S= \left \{ \vec u_1, \dots, u_k \right \} $ je množina vektorů vektorového prostoru $V$. Množina $S$ je
   	\begin{enumerate}[$i.$]
    \item \textbf{lineární nezávislá}, jestliže
    $$p_1\vec u_1 + p_2\vec u_2 + \dots + p_k\vec u_k = \vec o \iff p_1 = p_2 = \dots = p_k = 0;$$
   	\item \textbf{lineárně nezávislá}, jestliže existuje $p_i\ne 0$ takové, že
    $$ p_1\vec u_1 + p_2\vec u_2 + \dots + p_k\vec u_k = \vec o.$$
    \end{enumerate}
\end{definition}

\begin{veta}[Kriterium lineární závislosti]
    Vektory jsou závislé, jestliže alespoň jeden z nich lze vyjádřit jako lineární
    kombinaci ostatních.
\end{veta}

\begin{definition}
    Nechť je dán vektorový prostor $V$. Pak množina $\left < W \right >,$ kde $
    W\subseteq V$, se nazývá \textbf{podprostor} prostoru $V$ \textbf{generovaný}
    množinou $W$ a prvky množiny $W$ \textbf{generátory} tohoto podprostoru.
\end{definition}

\begin{definition}
    Nechť $(\vec u_1, \dots \vec u_k)$ je konečná posloupnost vektorů vektorového
    prostoru $V$. Tato posloupnost tvoří \textbf{bázi} vektorového prostoru $V$, jestliže
    \begin{enumerate}[$i.$]
    \item $\vec u_1, \dots \vec u_k$ jsou generátory $V$ a
   	\item $\vec u_1, \dots \vec u_k$ jsou lineárně nezávislé.
    \end{enumerate}
\end{definition}

\begin{definition}
Nechť $(\vec e_1,\dots, \vec e_k)$ je báse vektorového prostoru $V$. Pak číslo $k \in
\mathbb N_0$ je \textbf{dimenzí} vektorového prostoru $V$ a píšeme $\dim V=k$.
\end{definition}

\begin{definition}\label{izom}
    Nechť $(\vec e_1,\dots, \vec e_k)$ je báze vektorového prostoru $V$. Zobrazení
    $\varphi: A\to B$ se nazývá \textbf{homomorfismus} vzhledem k operaci $*$, jestliže
    $$\forall x,y \in A:\varphi(x*y)=\varphi(x)*\varphi(y).$$
    Pokud je $\varphi$ navíc bijektivní, nazývá se \textbf{izomorfismus}.
\end{definition}

\begin{pozn}
    Dva vektorové prostory jsou izomorfní, jestliže vztah z \ref{izom} platí pro
    operace sčítání i vnější násobení.
\end{pozn}

\begin{definition}
    Nechť $(\vec e_1, \dots, \vec e_n)$ je báze vektorového prostoru $V$. Nechť
    $\vec x \in V$ je libovolná vektor takový, že
    $$\vec x = x_1\vec e:1 + x_2\vec e_2 + \dots + x_n\vec e_n,$$
    kde $x_i \in \mathbb R, i\in \left \{ 1,\dots,n \right \} .$ Pak uspořádanou
    $n$-tici $(x_1, x_2,\dots,x_n)$ nazýváme \textbf{souřadnice vektoru} $\vec x$ v
    bázi $(\vec e_1, \dots, \vec e_n)$.
\end{definition}

\begin{definition}
    Nechť je dán bod $P\in \mathbb E_3$ a báze $(\vec e_1, \vec e_2, \vec e_3)$ vektorového
    prostoru volných vektorů $\mathscr V_3$. Pak uspořádaná čtveřice
    $(P, \vec e_1, \vec e_2, \vec e_3)$ je
    \textbf{afinní soustava souřadnic} v $\mathbb E_3$ s počátkem $P$.
\end{definition}

\begin{definition}
    Nechť je dán bod $P\in \mathbb E_3$ a báze $(\vec e_1, \vec e_2, \vec e_3)$ vektorového
    prostoru volných vektorů $\mathscr V_n$ taková, že platí:\\
    Jsou-li $X,Y,Z\in \mathbb E_3$ takové body, že
    $\overrightarrow{PX}\in\vec e_1,\overrightarrow{PY}\in\vec e_2,
    \overrightarrow{PZ}\in\vec e_3$, souřadné osy $\overrightarrow{PX},\overrightarrow{PY},
    \overrightarrow{PZ}$ jsou navzájem kolmé (tedy báze je ortogonální) a jejich velikost je jedna (tedy báze je navíc ortonormální),
    pak uspořádaná čtveřice $(P, \vec e_1, \vec e_2, \vec e_3)$ je \textbf{kartézská
    soutava souřadnic} v $\mathbb E_3$ s počátkem $P$.
\end{definition}

\begin{definition}
\textbf{Velikostí vektoru} $\overrightarrow{AB}$ rozumíme délku úsečky $AB$ a zapisujeme
$|\overrightarrow{AB}| = |AB|.$ Vektor o velikosti jedna nazýváme \textbf{jednotkovým
vektorem}.
\end{definition}

\begin{definition}
\textbf{Skalární součin} je zobrazení $V\times V\to \mathbb R$ (označené $\cdot$) s následujícími vlastnostmi:
\begin{enumerate}[$i.$]
\item $\vec u \cdot \vec v = \vec v \cdot \vec u,$
\item $(\vec u+\vec v)\cdot \vec w = \vec u\cdot \vec w + \vec v\cdot \vec w,$
\item $(p\cdot \vec u)\cdot v = p\cdot(\vec u \cdot \vec v),$
\item $\vec u\cdot \vec v \geq 0, \vec u \cdot \vec u = 0 \iff \vec u = \vec o$
\end{enumerate}
pro všechna $\vec u, \vec v, \vec w \in V.$
\end{definition}

\begin{veta}
    Pro nenulové vektory $\vec u, \vec v \in V$ platí
    $$\vec u \cdot \vec v = |\vec u|\cdot |\vec v|\cdot \cos \varphi,$$
    kde $\varphi$ je konvexní úhel, který svírají vektory $\vec u, \vec v.$
\end{veta}

\begin{definition}
    Vektory jsou \textbf{ortogonální} (kolmé), jestliže jejich skalární součin
    je roven nule.
\end{definition}

\begin{definition}
    Množina vektorů je \textbf{ortonormální}, jestliže je ortogonální a všechny vektory mají
    velikost 1.
\end{definition}

\begin{veta}[Gramm-Schmidtův ortogonalizační proces]
    Je dán následující algoritmus.
    Je dána množina vektorů $M=\left \{ \vec u_1,\dots, \vec u_n \right \}$.\\
    Hledáme ortogonální bázi $\left \{ \vec e_1, \dots, \vec e_1 \right \}$
    generující $\left < M \right >.$
    \begin{enumerate}[1.]
    \item Položme $\vec e_1 = \vec u_1.$
   	\item $k$-tý vektor volíme tak, aby
    \begin{equation}\label{gs}
    \vec e_k = p_1\vec e_1 + p_2\vec e_2 + \dots +p_{k-1} \vec e_{k-1} + \vec u_k.
    \end{equation}
    Dosazením do (\ref{gs}) dostaneme
    $$p_i = -\frac{\vec u_k\cdot \vec e_i}{\vec e_i\cdot \vec e_i}.$$
    (Skalární součiny všech bázových vektorů jsou nula.)
    \end{enumerate}
\end{veta}

\begin{definition}
\textbf{Vektorový součin} je zobrazení $V\times V\to  V$ (označené $\times$) s následujícími vlastnostmi:
\begin{enumerate}[$i.$]
\item $\vec u\times \vec v = -(\vec v \times \vec u),$
\item $(\vec u + \vec v)\times \vec w = \vec u \times \vec w + \vec v \times \vec w = \vec u\times (\vec v + \vec w),$
\item $(p\cdot \vec u)\times \vec v = p\cdot(\vec u \times \vec v) = \vec u \times (p\cdot \vec v)$
\end{enumerate}
pro všechna $\vec u, \vec v, \vec w \in V.$
\end{definition}

\begin{veta}
    Pro každé tři lineárně nezávislé vektory $\vec u, \vec v, \vec w \in V$ takové,
   že $\vec u\times \vec v = \vec w,$ platí:
   \begin{enumerate}[$i.$]
   \item $|\vec w| = |\vec u|\cdot |\vec v|\cdot \sin \varphi,$ kde $\varphi$ je úhel, který svárají,
  	\item $\vec w\perp u, \vec w \perp \vec v,$
  	\item $\vec u, \vec v, \vec w$ jsou kladně orientované (v pravotočivé bázi).
   \end{enumerate}
\end{veta}

\begin{pozn}
    Velikost vektorového součinu je obsah trojúhelníka, jehož dvě strany jsou
    násobené vektory.
\end{pozn}

\begin{definition}
    Pro všechny vektory $\vec u, \vec v, \vec w \in V$ je
    $$(\vec u \times \vec v) \cdot \vec w$$
    \textbf{smíšený součin} vektorů $\vec u, \vec v, \vec w.$
\end{definition}

\section{Analytická geometrie lineárních útvarů -- polohové vlastnosti}
\begin{definition}
    Nechť je dána přímka $p$ a vektor $\vec u\ne \vec o$
    takový, že existuje orientovaná úsečka $\overrightarrow{AB}$ taková, že
     $A\in p, B\in p.$ Pak $\vec u$
    je \textbf{směrový vektor} přímky $p.$
\end{definition}

\begin{definition}
    Rovnici $X = A+t\vec u, t\in \mathbb R, \vec u \ne  \vec o,$ nazveme
    \textbf{parametrickou rovnicí přímky} v $\mathbb E_2,$ resp. $\mathbb E_3.$
\end{definition}

\begin{priklad}
Napište rovnici přímky $p(A,\vec u),$ je-li $A[1,-1,2], \vec u = (0,1,2).$
\end{priklad}

\begin{veta}[Určení vzájemné polohy dvou přímek v $\mathbb E_2$]
    Jsou dány přímky $p(A,\vec u), q(B,\vec v).$ Pak
    \begin{enumerate}[$i.$]
    \item Jestliže má soustava rovnic
    \begin{align*}
        a_1+tu_1 &= b_1 + rv_1 \\
        a_2+tu_2 &= b_2 + rv_2
    \end{align*}
    \begin{enumerate}[$a.$]
    \item 0 řešení, jsou přímky $p,q$ rovnoběžné různé.
   	\item 1 řešení, jsou přímky $p,q$ různoběžné.
   	\item nekonečně mnoho řešení, přímky $p,q$ splývají.
    \end{enumerate}
   	\item Jestliže
    \begin{enumerate}[$a.$]
    \item jsou vektory $\vec u, \vec v$ lineárně nezávislé, jsou přímky $p,q$ různoběžné.
   	\item jsou vektory $\vec u, \vec v$ lineárně závislé a
   	\begin{itemize}
    \item $B\in p,$ přímky $p,q$ splývají.
   	\item $B\notin p,$ přímky $p,q$ jsou rovnoběžné různé.
    \end{itemize}
    \end{enumerate}
    \end{enumerate}
\end{veta}
\begin{priklad}
Určete vzájemnou polohu dvou přímek $AB, CD$, je-li $A[3,2], B[4,-1],$\linebreak $C[-4,5],D[-1,-2]$.
\end{priklad}

\begin{reseni}
Zjišťujeme, zda jsou směrové vektory přímek lineárně závislé.
\end{reseni}

\begin{veta}[Určení vzájemné polohy dvou přímek v $\mathbb E_3$]
    Jsou dány přímky $p(A,\vec u), q(B,\vec v).$ Pak
    \begin{enumerate}[$i.$]
    \item Jestliže má soustava rovnic
    \begin{align*}
        a_1+tu_1 &= b_1 + rv_1 \\
        a_2+tu_2 &= b_2 + rv_2 \\
        a_3+tu_3 &= b_3 + rv_3
    \end{align*}
    \begin{enumerate}[$a.$]
    \item 0 řešení a vektory $\vec u, \vec v$ jsou
    \begin{itemize}
    \item lineárně nezávislé, jsou přímky $p,q$ mimoběžné.
   	\item lineárně závislé, jsou přímky $p,q$ rovnoběžné různé.
    \end{itemize}
   	\item 1 řešení, jsou přímky $p,q$ různoběžné.
   	\item nekonečně mnoho řešení, přímky $p,q$ splývají.
    \end{enumerate}
   	\item Jestliže
    \begin{enumerate}[$a.$]
    \item jsou vektory $\vec u, \vec v$ lineárně nezávislé a
    \begin{itemize}
    \item $p\cap q \ne \emptyset,$ jsou přímky $p,q$ různoběžné.
   	\item $p\cap q = \emptyset,$ jsou přímky $p,q$ mimoběžné.
    \end{itemize}
   	\item jsou vektory $\vec u, \vec v$ lineárně závislé a
   	\begin{itemize}
    \item $B\in p,$ přímky $p,q$ splývají.
   	\item $B\notin p,$ přímky $p,q$ jsou rovnoběžné různé.
    \end{itemize}
    \end{enumerate}
    \end{enumerate}
\end{veta}

\begin{priklad}
Rozhodněte o vzájemné poloze přímek $p=\left \{ [1-4t,2+4t,3+t], t \in \mathbb R \right \},q=\left \{ [-4,8r,-5-8r,-2r],r \in \mathbb R \right \} . $
\end{priklad}

\begin{reseni}
Zjišťujeme, zda jsou směrové vektory přímek lineárně závislé.
\end{reseni}

\begin{veta}[Vzájemná poloha dvou přímek]
    Nechť $p(A,\vec u), q(B, \vec v)$ jsou dvě přímky. Pak platí
    \begin{enumerate}[$i.$]
    \item $p\parallel q \land p=q \iff \dim(\vec u, \vec v, \overrightarrow{AB})=1,$
   	\item $p\parallel q \land p\ne q \iff \dim(\vec u, \vec v, \overrightarrow{AB})=2 \land \dim ( \vec u, \vec v) = 1$,
    \item $p, q$ jsou různoběžné $ \iff \dim(\vec u, \vec v, \overrightarrow{AB})=2 \land \dim ( \vec u, \vec v) = 2$,
    \item $p, q$ jsou mimoběžné $\iff \dim(\vec u, \vec v, \overrightarrow{AB})=3.$
    \end{enumerate}
\end{veta}

\begin{priklad}
Rozhodněte o vzájemné poloze dvou přímek $p=\left \{ [t,-1+2t,-2+2t], t \in \mathbb R \right \}, q=\left \{ [1+t,1,r], r \in \mathbb R \right \}.  $
\end{priklad}

\begin{reseni}
Je $\vec u = (1,2,2), \vec v = (1,0,1), \overrightarrow{AB}=(1,2,2).$ Hledáme tedy
dimenzi prostorů $\left < \vec u, \vec v \right >, \left < \vec u, \vec v, \overrightarrow{AB} \right >.  $
\end{reseni}

\begin{definition}
    $ax+by+c=0, (a,b)\ne(0,0)$ je \textbf{obecná rovnice přímky} v $\mathbb E_2$.
\end{definition}

\begin{definition}
    Vektor kolmý ke směrovému vektoru přímky se nazývá \textbf{normálový}.
\end{definition}

\begin{veta}
    Normálový vektor přímky o rovnici $ax+by+c=0$ je $\vec n=(a,b)$.
\end{veta}

\begin{priklad}
Napište obecnou rovnici přímky $p=\left \{ [1+t,2-t], t \in \mathbb R \right \} .$
\end{priklad}

\begin{reseni}
Buď vyloučením parametru: je-li $x=1+t, y= 2-t,$ pak $x+y=3,$ takže $x+y-3=0$, nebo
přes normálový vektor: $\vec n(1,1),$ takže dosadíme do rovnice a dopočteme $c$ tak,
aby jí lib. bod přímky $p$ vyhovoval.
\end{reseni}

\begin{priklad}
Napište parametrickou rovnici přímky $x-2y+1=0.$
\end{priklad}

\begin{reseni}
Buď substitucí, např. $y=t$, pak $x-2t+1=0,$ takže $x=-1+2t$, tedy $p=\left \{ [-1+2t,t], t \in \mathbb R \right \}$,
nebo přes normálový vektor.
\end{reseni}

\begin{definition}
    Rovnice $y=kx+q$ se nazývá \textbf{směrnicový tvar rovnice přímky} v $\mathbb E_2$,
    $k$~je \textbf{směrnice} přímky.
\end{definition}

\begin{definition}
Rovnice $\frac{x}{q}+\frac{y}{p}=1$ se nazývá \textbf{úsekový tvar rovnice přímky} v $\mathbb E_2$.
\end{definition}

\begin{definition}
Nechť $\rho$ je rovina, $\vec u, \vec v$ lineárně nezávislé vektory a $\overrightarrow{AB},
\overrightarrow{AC}, A,B,C\in \rho$ jejich umístění. Pak $\vec u, \vec v$ je
\textbf{zaměření} roviny $\rho.$
\end{definition}

\begin{definition}
    Rovnice $X=A+r\vec u+s\vec v$ se nazývá \textbf{parametrická rovnice roviny}.
\end{definition}

\begin{definition}
    $ax+by+cz+d=0, (a,b,c)\ne(0,0,0)$ se nazývá \textbf{obecná rovnice roviny}.
\end{definition}

\begin{definition}
    Vektor kolmý k oběma směrovým vektorům roviny se nazývá \textbf{normálový}.
\end{definition}

\begin{priklad}
Napište obecnou rovnici roviny $\rho = \overleftrightarrow{ABC}, A[1,1,1], B[2,0,-1], C[1,0,0].$
\end{priklad}

\begin{reseni}
Opět buď vyloučením parametrů, nebo přes normálový vektor: spočítáme dva libovolné vektory
mezi body $A,B,C$ a $\vec n = \vec u \times \vec v$, posun dopočteme.
\end{reseni}

\begin{priklad}
Napište parametrickou rovnici roviny $\rho: x-y+z-1=0.$
\end{priklad}

\begin{reseni}
Pokud $y=r, z=s,$ pak $x-r+s-1=0,$ takže $x=1+r-s$, tedy $\rho = \left \{ [1+r-s,r,s],r,s\in \mathbb R \right \}. $
\end{reseni}

\begin{veta}[Vzájemná poloha dvou rovin daných obecnými rovnicemi]
    Nechť $\rho: ax+by+cz\linebreak+ d=0, \sigma: ex+fy+gz+h=0$ jsou roviny. Pak
    \begin{enumerate}[$i.$]
    \item $\rho = \sigma \iff (a,b,c,d) = k(e,f,g,h), k\in \mathbb R,$
   	\item $\rho \parallel \sigma \land \rho \ne \sigma \iff (a,b,c) = k(e,f,g), k\in \mathbb R
    \land d\ne kh,$
   	\item $\rho \nparallel \sigma \iff \forall k \in \mathbb R: (a,b,c) \ne k(e,f,g)$.
    \end{enumerate}
\end{veta}

\begin{priklad}
Určete vzájemnou polohu roviny $\rho:2x+3y+4z+5=0$, $\sigma: x-y-z+1=0$.
\end{priklad}

\begin{reseni}
Normálové vektory jsou lineárně nezávislé. Průsečnice je řešení soustavy rovnic
\begin{align*}
    2x+3y+4z+5& =0,\\
    x-y-z+1&=0.
\end{align*}
\end{reseni}

\begin{veta}[Vzájemná poloha dvou rovin daných parametrickými rovnicemi]
    Nechť $\rho(A,\vec u, \vec v), \linebreak\sigma(B,\vec k, \vec l)$ jsou roviny. Pak
    \begin{enumerate}[$i.$]
    \item $\rho = \sigma \iff \dim (\vec u, \vec v, \vec k, \vec l) = 2 \land \dim (\vec u, \vec v, \vec k, \vec l, \overrightarrow{AB})=2,$
   	\item $\rho \parallel \sigma \land \rho \ne \sigma \iff \dim (\vec u, \vec v, \vec k, \vec l) = 2 \land \dim (\vec u, \vec v, \vec k, \vec l, \overrightarrow{AB})=3,$
   	\item $\rho \nparallel \sigma \iff \dim (\vec u, \vec v, \vec k, \vec l) = 3$.
    \end{enumerate}
\end{veta}

\begin{priklad}
Určete vzájemnou polohu rovin $\rho = \{ [4+t_1+2t_2,5+2t_1,3+2t_1+2t_2],$ $t_1,t_2\in \mathbb R \}$,
$\sigma = \left \{ [1+2r_1+r_2,-2-2r_1-2r_2,1+r_1],r_1,r_2\in \mathbb R \right \}. $
\end{priklad}

\begin{reseni}
Hledáme dimenzi prostorů $\left < \vec u, \vec v, \vec k, \vec l \right >, \left < \vec u, \vec v, \vec k, \vec l, \overrightarrow{AB} \right > .$
\end{reseni}

\begin{veta}[Vzájemná poloha přímky a roviny dané parametrickou rovnicí]
    Nechť $p(A,\vec u),\linebreak \rho(B,\vec v, \vec w)$ jsou přímka a rovina. Pak
    \begin{enumerate}[$i.$]
    \item $p\subseteq \rho \iff \dim (\vec v, \vec w, \vec u) = 2 \land \dim (\vec v, \vec w, \vec u, \vec l, \overrightarrow{AB})=2,$
   	\item $p \parallel \rho \land p \not\subset \rho \iff \dim (\vec v, \vec w, \vec u) = 2 \land \dim (\vec v, \vec w, \vec u, \vec l, \overrightarrow{AB})=3,$
   	\item $p \nparallel \rho \iff \dim (\vec v, \vec w, \vec u) = 3$.
    \end{enumerate}
\end{veta}

\begin{priklad}
Určete vzájemnou polohu přímky $p=\left \{ [3+t,1+2t,2-t],t \in \mathbb R \right \} $
a roviny $\rho:\left \{ [1-3r+s,2r-s,1+4r-s],r,s\in \mathbb R \right \} .$
\end{priklad}

\begin{reseni}
Hledáme dimenzi prostorů $\left < \vec u, \vec v, \vec w \right >, \left < \vec u, \vec v, \vec w,  \overrightarrow{AB} \right > .$
\end{reseni}

\begin{veta}[Vzájemná poloha přímky a roviny dané obecnou rovnicí]
    Nechť $p(A,\vec u), \rho: ax+by+cz+d=0$ jsou přímka a rovina. Pak
    \begin{enumerate}[$i.$]
    \item $p\subseteq \rho \iff \vec u \cdot \vec n = 0\land A\in p,$
   	\item $p \parallel \rho \land p \not\subset \rho \iff \vec u\cdot \vec n = 0\land A\notin p,$
   	\item $p \nparallel \rho \iff \vec u\cdot \vec n\ne 0$.
    \end{enumerate}
\end{veta}

\begin{priklad}
Určete vzájemnou polohu přímky $p=\left \{ [1-t,1+3t,-3],t \in \mathbb R \right \} $ a
roviny $\rho:3x+y+5z+7=0.$
\end{priklad}

\begin{reseni}
Počítáme skalární součin směrového vektoru přímky a normálového vektoru roviny.
\end{reseni}

\begin{definition}
Nechť $p,q$ jsou dvě mimoběžné přímky. Přímka $r$, která je různoběžná s~přímkami
$p,q$ se nazývá \textbf{příčkou mimoběžek} $p,q$. Pokud je navíc $r$ kolmá na $p,q$,
nazývá se \textbf{osou mimoběžek} $p,q$.
\end{definition}

\begin{priklad}\label{prmimob}
Jsou dány mimoběžky  $p(A,\vec u)$ $q(B,\vec v)$ a dále vektor $\vec w$, přičemž
$A[1,-2,5],$ $B[-1,1,-5],$ $\vec u = (1,3,-1),$ $\vec v=(1,1,2),$ $\vec w = (1,1,4).$
Nalezněte příčku $p,q$, která je rovnoběžná s vektorem $\vec w.$
\end{priklad}

\begin{reseni}
Nechť $P=A+k\vec u, Q=B+l\vec v.$ Pak $\overrightarrow{PQ}=Q-P=B+l\vec v - A - k\vec u=x\vec w,$
úpravou dostaneme $\overrightarrow{AB}=-l\vec v+k\vec u+x\vec w.$ Máme tedy
sosutavu tří rovnic o třech neznámých.
\end{reseni}

\begin{priklad}
Jsou dány mimoběžky $p(A,\vec u), q(B,\vec v)$ a bod $M$, přičemž
$A[1,5,2],B[0,-1,1],$ $\vec u=(1,2,1),$ $\vec v = (3,1,0),$ $M[0,1,-5].$  Nalezněte
příčku $p,q$, která prochází bodem $M$.
\end{priklad}

\begin{reseni}
Obdobně jako v příkladu \ref{prmimob} dostaneme $\overrightarrow{MA}=x\overrightarrow{MB}+m\vec v-k\vec u$,
tedy $A-M=x(B-M)+m\vec v - k\vec u$. Máme tedy
sosutavu tří rovnic o třech neznámých.
\end{reseni}

\section{Analytická geometrie lineárních útvarů -- metrické vlastnosti}
\begin{definition}
    Nechť $A,B\subseteq \mathbb E_3$ jsou dva podprostory $\mathbb E_3$. \textbf{Vzdáleností
    podprostorů} $A,B$ nazveme nezáporné reálné číslo $\rho(A,B)$ definovené takto:
    $$\rho(A,B) = \min \{ |XY|: X\in A, Y\in B \},$$
    kde $|XY|$ je délka úsečky $XY.$
\end{definition}

\begin{veta}
    Vzdálenost dvou bodů $A,B$ je délka úsečky $AB.$
\end{veta}

\begin{veta}
    Nechť $A[a_1,a_2]\in \mathbb E_2$ je bod, $p:ax+by+c=0$ je přímka. Pak
    $$\rho(A,p)=\frac{|aa_1+ba_2+c|}{\sqrt{a^2+b^2} }.$$
\end{veta}

\begin{pozn}
    Vzdálenost bodu $A$ od přímky $p$ v $\mathbb E_3$ řešíme třeba vyjádřením délky úsečky
    $AX,$ kde $X$ je nějaký bod na přímce $p$ a nalezením jejího minima.
\end{pozn}

\begin{priklad}\label{primkaabod}
Určete $\rho(A,p), A[1,0,1], p=\left \{ [2-t,t,0],t \in \mathbb R \right \} $.
\end{priklad}

\begin{reseni}
Přímka $p=p(P,\vec u), $ kde $P[2,0,0],\vec u(-1,1,0).$
\begin{enumerate}[1.]
\item způsob: Nechť $A_0=[2-t^*,t^*,0]$. Pak $\vec n = \overrightarrow{AA_0}=A_0-A=(1-t^*,t^*,-1).$
Platí $\vec u\cdot\vec n=0,$ tedy $t^*=-\frac{1}{2},$ odtud $A_0[1,5;0,5;0].$ Vzdálenost
$\rho(A,p)$ je pak $\left |\overrightarrow{AA_0}\right |.$
\item způsob: Najdeme takovou rovinu, která je kolmá na přímku $p$ a leží na ní bod $A$
(normálový vektor roviny je tedy směrový vektor přímky). Dále najdeme průsečík přímky
$p$ s hledanou rovinou -- vzdálenost dvou bodů už počítat umíme.
\item způsob: Libovolný bod $X$ na přímce $p$ má souřadnice $[2-t,t,0]$. Hledáme
minimum velikosti vektoru $\overrightarrow{AX}.$
\end{enumerate}
\end{reseni}

\begin{veta}\label{vzdbodrov}
    Nechť $A[a_1,a_2,a_3]\in \mathbb E_3$ je bod, $\alpha:ax+by+cz+d=0$ je rovina. Pak
    $$\rho(A,\alpha)=\frac{|aa_1+ba_2+ca_2+d|}{\sqrt{a^2+b^2+c^2} }.$$
\end{veta}

\begin{veta}\label{vzdpr}
    Nechť $p:ax+by+c=0,q:ax+by+d=0$ jsou dvě rovnoběžné přímky. Pak
    $$\rho(p,q)=\frac{|d-c|}{\sqrt{a^2+b^2} }.$$
\end{veta}

\begin{pozn}
    Při hledání vzdálenosti dvou rovnoběžných přímek v $\mathbb E_3$ na jedné z nich zvolíme
    libovolný bod a dále pokračujeme podle příkladu \ref{primkaabod}.
\end{pozn}

\begin{pozn}
    Při hledání vzdálenosti přímky od roviny s ní rovnoběžné na přímce zvolíme
    libovolný bod a dále pokračujeme podle věty \ref{vzdbodrov}.
\end{pozn}

\begin{veta}\label{dverov}
    Nechť $\alpha:ax+by+cz+d=0,\beta:ax+by+cz+e=0$ jsou dvě rovnoběžné roviny. Pak
    $$\rho(\alpha,\beta)=\frac{|e-d|}{\sqrt{a^2+b^2+c^2} }.$$
\end{veta}

\begin{pozn}
    Vzdálenost mimoběžných přímek určujeme buď pomocí normálového vektoru rovin,
    ve kterých přímky leží a následně podle věty \ref{dverov}, nebo určením
    osy mimoběžek.
\end{pozn}

\begin{priklad}
Určete vzdálenost mimoběžných přímek $p=\left \{ [9+4t,-2-3t,t],t \in \mathbb R \right \},$ $q=\left \{ [-2r,-7+9r,2+2r],r \in \mathbb R \right \} $.
\end{priklad}

\begin{reseni}
\begin{enumerate}[1.]
\item způsob: Směrové vektory mimoběžek určují rovinu. Jedna obsahuje přímku $p$ a~druhá,
která je s ní rovnoběžná, obsahuje přímku $q$. Dále počítáme podle vzorce.
\item způsob: Hledáme příčku mimoběžek, která je rovnoběžná s vektorem, který
získáme jako vektorový součin směrových vektorů přímek $p,q$.
\end{enumerate}
\end{reseni}

\begin{priklad}
Určete rovnici přímky, která prochází bodem $A[-2,1]$ a od bodu $[3,1]$ má vzdálenost 4.
\end{priklad}

\begin{reseni}
Hledáme přímku tvaru $ax+by+c=0$. Jistě platí $-2a+b+c=0$ a~vzorec pro vzdálenost přímky a roviny.
Máme tři neznámé, ale jen dvě rovnice. Proto jednu z nich zvolíme (musíme uvažovat
nulové a nenulové řešení) a zbylé dvě dopočítáme.
\end{reseni}

\begin{priklad}
Určete rovnici přímky, která prochází bodem $A[1,2]$ a má stejnou vzdálenost od bodů
$B[3,3], C[5,2].$
\end{priklad}

\begin{reseni}
    Hledáme přímku $p: ax+by+c=0$. Protože $A\in p$, platí $a+2b+c=0$. Protože má stejnou vzdálenost od bodů $B$ a $C$, platí navíc
    $$\frac{|3a+3b+c|}{\sqrt{a^2+b^2} }=\frac{|5a+2b+c|}{\sqrt{a^2+b^2} }.$$
    S výhodou druhou rovnici upravíme za použití té první na
    $|2a+b|=|4a|$. Dále už řešíme soustavu rovnic, přičemž jsou buď obě absolutní hodnoty kladné, nebo jedna z nich záporná (ostatní případy
    nám dají stejný vektor o opačné orientaci). Jednu z proměnných zvolíme.
\end{reseni}

\begin{definition}
    \textbf{Odchylka} vektorů je úhel, který dané vektory svírají. Značíme $|\sphericalangle \vec u, \vec v|.$
\end{definition}

\begin{veta}
    Nechť jsou dány vektory $\vec u, \vec v.$ Pak
    $$|\sphericalangle \vec u, \vec v|=\arccos \frac{|\vec u \cdot \vec v|}{|\vec u|\cdot |\vec v|}.$$
\end{veta}

\begin{pozn}
    Platí obdobně i pro přímky dané obecnou rovnicí (vektory $\vec u, \vec v$ jsou
    jejich normálové vektory.)
\end{pozn}

\begin{veta}
    Nechť jsou dány přímky $p(A,\vec u), q(B, \vec v).$ Pak
    $$|\sphericalangle p, q|=\arccos \frac{|\vec u \cdot \vec v|}{|\vec u|\cdot |\vec v|}.$$
\end{veta}

\begin{priklad}
Určete odchylku přímek $\overleftrightarrow{AB}$ a $\overleftrightarrow{BC^\prime}$
krychle $ABCDA^\prime B^\prime C^\prime D^\prime$.
\end{priklad}

\begin{reseni}
Vyjádříme jako vektory a dosadíme do vzorce.
\end{reseni}

\begin{veta}
Nechť jsou dány přímky $p(A,\vec u), q:ax+by+c=0, \vec n(a,b).$ Pak
$$|\sphericalangle p, q|=\arcsin \frac{|\vec u \cdot \vec n|}{|\vec u|\cdot |\vec n|}.$$
\end{veta}

\begin{veta}
    Nechť $p(A,\vec u)$ je přímka, $\alpha:ax+by+cz+d=0$ rovina, $\vec n(a,b,c)$. Pak
    $$|\sphericalangle p, \alpha|=\arcsin \frac{|\vec u \cdot \vec n|}{|\vec u|\cdot |\vec n|}.$$
\end{veta}

\begin{veta}
    Nechť $\alpha:ax+by+cz+d=0, ex+fy+gz+h=0$ jsou dvě roviny a~$\vec m(a,b,c), \vec n(e,f,g).$
    Pak
    $$|\sphericalangle \alpha, \beta|=\arccos \frac{|\vec m \cdot \vec n|}{|\vec m|\cdot |\vec n|}.$$
\end{veta}

\begin{priklad}
Určete rovnici přímky, která má od přímky $p:x-2y+3=0$ odchylku $30^\circ$ a~prochází
jejím průsečíkem s osou $y$.
\end{priklad}

\begin{reseni}
Dosadíme do vzorce a jednu proměnnou zvolíme.
\end{reseni}

\begin{priklad}
Najděte parametrické vyjádření přímky, která prochází počátkem a protíná přímku $p=\left \{ [4+t,3+4t,1-3t],t\in \mathbb R \right \} $
a jejich odchylka je $30^\circ$.
\end{priklad}

\begin{reseni}
Dosadíme do vzorce a jednu proměnnou zvolíme.
\end{reseni}

\section{Komplexní čísla}
\begin{definition}
Množinu označenou $\mathbb C = \left \{ (a,b)\ a,b, \in \mathbb R \right \} $
spolu s operacemi $+$ a $\cdot$ definovanými následovně:
\begin{enumerate}[$i.$]
\item $(a,b) + (c,d) = (a+c, b+d),$
\item $(a,b) \cdot (c,d) = (ac-bd, ad+bc)$,
\end{enumerate}
nazýváme \textbf{komplexními čísly}.
\end{definition}

\begin{veta}
    Pro všechna $x=(a,b),y=(c,d) \in \mathbb C$ platí
    \begin{enumerate}[$i.$]
    \item $x-y=(a-c, b-d)$,
   	\item $y\ne (0,0): \frac{x}{y}=\left ( \frac{ac+bd}{b^2+d^2}, \frac{bc-ad}{b^2+d^2} \right ) .$
    \end{enumerate}
\end{veta}

\begin{veta}
    Množina $\mathbb C$ spolu s operacemi $+,\cdot$ tvoří pole.
\end{veta}

\begin{pozn}
    Množina $M$ spolu s operacemi $+,\cdot$ tvoří pole, pokud
    \begin{enumerate}[$i.$]
    \item $(M,+)$ je komutativní grupa,
   	\item $(M-\left \{ 0 \right \} ,\cdot)$ je komutativní grupa,
   	\item platí distributivní zákony.
    \end{enumerate}
    Polem je třeba $\mathbb R, \mathbb Q;$ $\mathbb Z$ však polem není (neexistují
    inverzní prvky vzhledem k násobení -- 5 krát co je jedna?).
\end{pozn}

\begin{veta}
    Množiny $\mathbb R$ a $(x,0)\in \mathbb C$ jsou izomorfní.
\end{veta}

\begin{proof}
    Pro všechna $x,y \in \mathbb R$ platí
    \begin{enumerate}[$i.$]
    \item $\varphi(x)+\varphi(y) = \varphi(x+y),$
   	\item $\varphi(x)\cdot\varphi(y) = \varphi(xy),$
    \end{enumerate}
    kde $\varphi$ je zobrazení z $\mathbb R$ do $\mathbb C$: $\varphi(a)=(a,0).$
\end{proof}

\begin{proof}
    Dále budeme zapisovat komplexní číslo $(a,b)$ jako $a+bi.$
\end{proof}

\begin{definition}
    Tvar komplexního čísla $a+bi$ nazýváme \textbf{algebraický}. Číslo $a$ je jeho
   \textbf{reálná část}, $b$ \textbf{imaginární část}.\\
  Číslo $(0,1)$ označme $i$ a nazývejme \textbf{imaginární jednotka}. \\
  Čísla $a+bi,$ kde $b\ne 0$ nazývejme \textbf{imaginární}. Pokud taky $a=0$,
  nazýváme je \textbf{ryze imaginární}.
\end{definition}

\begin{pozn}
    Komplexní číslo $x=(a,b)$ lze chápat jako bod v rovině se souřadnicemi $[a,b].$
\end{pozn}

\begin{definition}
    Nechť je dáno komplexní číslo $x= a+bi.$ Číslo $\overline{x}=a-bi$ nazýváme
   \textbf{komplexně sdruženým} k číslu $x$. Číslo $|x|=\sqrt{a^2+b^2}$ je \textbf{
   absolutní hodnota} čísla $x$. Jestliže $|x|=1,$ číslu $x$ říkáme \textbf{komplexní jednotka}.
\end{definition}

\begin{definition}
\textbf{Argumentem} komplexního čísla nazýváme orientovaný úhel $\varphi$, který svírá kladná
poloosa reálné osy s polohovým vektorem daného bodu představujícím dané komplexní číslo.
\end{definition}

\begin{veta}
    Každé komplexní číslo $x=a+bi, x\ne 0,$ lze zapsat ve tvaru
    $$x=|x|(\cos \varphi + i \sin \varphi),$$
    kde $\varphi$ je argumentem čísla $x$. Dále platí:
    \begin{align*}
        \cos \varphi = \frac{a}{|x|}, && \sin \varphi = \frac{b}{|x|}.
    \end{align*}
\end{veta}

\begin{definition}
    Tvar komplexního čísla $x= |x|(\cos \varphi + i\sin\varphi)$ nazýváme \textbf{goniometrický}.
\end{definition}

\begin{pozn}
    Každé komplexní číslo lze zapsat uspořádanou dvojicí $x=(|x|, \varphi).$
    Tato čísla jsou \textbf{polárními souřadnicemi} komplexního čísla $x$.
\end{pozn}

\begin{veta}[Moivreova]
    Nechť je dáno komplexní číslo $x=|x|(\cos \varphi + i \sin \varphi)$ a $n\in \mathbb N$
    je nenulové číslo. Pak platí
    $$x^n = |x|^n (\cos n\varphi + i \sin n \varphi).$$
\end{veta}

\begin{pozn}
    Gaussova rovina je množina bodů $[a,b]; a,b \in \mathbb R$.
    Každému prvku této množiny odpovídá komplexní číslo $(a,b).$ Osa $x$ odpovídá
    reálné části tohoto komplexního čísla, osa $y$ imaginární části.
\end{pozn}

\begin{definition}
\textbf{Binomickou rovnicí} s neznámou $x\in \mathbb C$ nazýváme každou rovnici
tvaru $x^n=a$, kde $a\in \mathbb C,n\in \mathbb N, n\geq 2.$ Každý komplexní
kořen binomické rovnice nazýváme \textbf{komplexní $n$-tou odmocninou} z čísla $a.$
\end{definition}

\begin{definition}
\textbf{Kvadratickou rovnicí s} (komplexní) \textbf{neznámou} $x\in \mathbb C$ a
\textbf{reálnými} (resp. \textbf{komplexními}) \textbf{koeficienty} $a,b,c$
nazýváme každou rovnici tvaru
$ax^2+bx+c=0,a,b,c\in \mathbb R,a\ne0$ (resp. $a,b,c \in \mathbb C,a\ne 0$).
\end{definition}

\begin{definition}
\textbf{Algebraická rovnice} $n$-tého stupně s jednou neznámou $x\in \mathbb C$ je každá
rovnice tvaru $P(x)=0,$ kde $P(x)$ je polynom $n$-tého stupně s reálnými (resp.
komplexními) koeficienty.
\end{definition}

\begin{veta}[Základní věta algebry]
    Každý polynom $n$-tého stupně s komplexními koeficienty má v množině $\mathbb C$
    právě $n$ kořenů, počítáme-li každý kořen tolikrát, kolik je jeho násobnost.
\end{veta}

\begin{veta}
    Má-li algebraická rovnice s reálnými koeficienty imaginární kořen
    $a+bi$, má taky kořen $a-bi.$
\end{veta}

\begin{definition}
\textbf{Trinomická rovnice} s neznámou $x\in \mathbb C$ a reálnými koeficienty $a,b,c \in \mathbb R$
nazýváme rovnici tvaru $ax^p + bx^q + c = 0,$ kde $p,q\in \mathbb N, p>q, a,b\ne 0.$
\end{definition}

\section{Posloupnosti}
\begin{definition}
Nechť $A$ je množina. Zobrazení $a:\mathbb N\to A$ (nebo $a:\left \{ 1,2,...,k \right \} \to A$)
se nazývá \textbf{posloupnost} prvků množiny $A$, $k\in \mathbb N.$
\end{definition}

\begin{definition}
Nechť $\left \{ a_n \right \}_{n=1}^\infty $ je posloupnost reálných čísel. Pak je
\begin{enumerate}[$i.$]
\item \textbf{rostoucí} právě tehdy, když $\forall n \in \mathbb N: a_n< a_{n+1},$
\item \textbf{klesající} právě tehdy, když $\forall n \in \mathbb N: a_n> a_{n+1},$
\item \textbf{nerostoucí} právě tehdy, když $\forall n \in \mathbb N: a_n\geq a_{n+1},$
\item \textbf{neklesající} právě tehdy, když $\forall n \in \mathbb N: a_n\leq a_{n+1}.$
\end{enumerate}
Posloupnost, která má jednu z těchto vlastností, se nazývá \textbf{monotónní}.
Posloupnost rostoucí nebo klesající nazýváme \textbf{ryze monotónní}.
\end{definition}

\begin{definition}
    Nechť $\left \{ a_n \right \}_{n=1}^\infty $ je posloupnost reálných čísel. Pak je
    \begin{enumerate}[$i.$]
    \item \textbf{prostá} právě tehdy, když $\forall m,n\in \mathbb N: m\ne n \implies a_m\ne a_n,$
   	\item \textbf{stacionární} (konstantní) právě tehdy, když $\forall n \in \mathbb N:a_n = a_{n+1}.$
    \end{enumerate}
\end{definition}

\begin{definition}
    Nechť $\left \{ a_n \right \}_{n=1}^\infty $ je posloupnost reálných čísel. Pak je
    \begin{enumerate}[$i.$]
    \item \textbf{zdola omezená}, pokud existuje $k\in \mathbb R$ takové, že
    $\forall n \in \mathbb N: a_n \geq k,$
   	\item \textbf{shora omezená}, pokud existuje $k\in \mathbb R$ takové, že
    $\forall n \in \mathbb N: a_n \leq k.$
    \end{enumerate}
    Posloupnost je \textbf{omezená}, pokud je omezení shora i zdola.
    Pokud není omezení ani shora, ani zdola, je \textbf{neomezená}.
\end{definition}

\begin{definition}
Nechť $\left \{ a_n \right \}_{n=1}^\infty $ je posloupnost reálných čísel a
$\left \{ k_1,k_2,\dots,k_n,\dots \right \} $ je rostoucí posloupnost přirozených čísel.
Pak posloupnost $\left \{{a_k}_n \right \}_{n=1}^\infty = \left \{ {a_k}_1, {a_k}_2, \dots, {a_k}_n,\dots \right \}  $
je \textbf{posloupnost vybraná z posloupnosti} $\left \{ a_n \right \}_{n=1}^\infty $.
\end{definition}

\begin{definition}
Posloupnost $\left \{ a_n \right \}_{n=1}^\infty $ je \textbf{aritmetická} právě
tehdy, když
$$\exists d \in \mathbb R: \forall n \in \mathbb N: a_n=a_{n-1}+d$$
a číslo $d$ se nazývá \textbf{diference}.
\end{definition}

\begin{veta}
    Nechť $\left \{ a_n \right \}_{n=1}^\infty $ je aritmetická posloupnost s diferencí
    $d$. Pak platí:
    \begin{enumerate}[$i.$]
    \item $\forall n\in \mathbb N: a_n = a_1 + (n-1)d,$
   	\item $\forall r,s \in \mathbb N: a_s = a_r + (s-r)d.$
    \end{enumerate}
\end{veta}

\begin{proof}
\begin{enumerate}[$i.$]
\item Matematickou indukcí:
\begin{enumerate}[1.]
\item $n=1: a_1 = a_1+(1-1)d$ platí
\item $a_n = a_1+(n-1)d \implies a_{n-1}=a_1+(n)d,$ takže $a_{n+1}=a_n+d=a_1+(n-1)d+d=a_1(n)d$
\end{enumerate}
\item Užitím již dokázeného vztahu:
\begin{align*}
    a_s &= a_1 + (s-1)d,\\
    a_r &= a_1 + (r-1)d.
\end{align*}
Sečtením dostáváme
$$a_s-a_r = (s-1)d-(r-1)d,$$
takže
$$a_s=a_r+(s-r)d,$$
což jsme chtěli dokázat.\qedhere
\end{enumerate}
\end{proof}

\begin{veta}
Nechť $\left \{ a_n \right \}_{n=1}^\infty $ je aritmetická posloupnost s diferencí
$d$. Nechť
$$S_n=a_1+a_2+\dots+a_n$$
je součet prvních $n$ členů nekonečné aritmetické posloupnosti nebo součet
$n$-členné aritmetické poslouppnosti. Pak platí:
$$\forall n \in \mathbb N:S_n = \frac{1}{2}n(a_1+a_n).$$
\end{veta}

\begin{proof}
\begin{align*}
    S_n &= a_1+a_2+a_3+\dots+a_{n-1}+a_n \\
    &= a_1 + (a_1+d)+ (a_1+2d)+ \dots + [a_1+(n-2)d]+[a_1+(n-1)d]\\
    &= [a_1+(n-1)d]+[a_1+(n-2)d]+\dots+(a_1+2d)+(a_1+d)+a_1\\
    2S_n &= 2a_1 + (n-1)d + 2a_1 + (n-1)d + \dots + 2a_1 + (n-1)d + 2a_1 + (n-1)d\\
    2S_n &= n [a_1 + a_1 + (n-1)d], \textrm{ takže }\\
    S_n &= \frac{n(a_1+a_n)}{2},
\end{align*}
což jsme chtěli dokázat. \qedhere
\end{proof}

\begin{definition}
Posloupnost $\left \{ a_n \right \}_{n=1}^\infty $ je \textbf{geometrická} právě
tehdy, když
$$\exists q \in \mathbb R: \forall n \in \mathbb N: a_n=a_{n-1}\cdot q$$
a číslo $q$ se nazývá \textbf{kvocient}.
\end{definition}

\begin{veta}
    Nechť $\left \{ a_n \right \}_{n=1}^\infty $ je geometrická posloupnost s kvocientem
    $q$. Pak platí:
    \begin{enumerate}[$i.$]
    \item $\forall n\in \mathbb N: a_n = a_1 \cdot q^{n-1},$
   	\item $\forall r,s \in \mathbb N: a_s = a_r \cdot q^{s-r}.$
    \end{enumerate}
\end{veta}

\begin{proof}
\begin{enumerate}[$i.$]
\item Matematickou indukcí:
\begin{enumerate}[1.]
\item $n=1: a_1 = a_1\cdot q^0$ platí
\item $a_n = a_1\cdot q^{n-1}\implies a_{n+1}=a_1\cdot q^n,$ takže $a_{n+1}=a_na_n\cdot q=a_1\cdot q^{n-1}\cdot q=a_1\cdot q^n$
\end{enumerate}
\item Užitím již dokázeného vztahu:
\begin{align*}
    a_s &= a_1\cdot q^{s-1},\\
    a_r &= a_1\cdot q^{r-1}.
\end{align*}
Vydělením dostáváme
$$\frac{a_s}{a_r} = \frac{a_1\cdot q^{x-1}}{a_1\cdot q^{r-1}}=q^{s-r},$$
takže
$$a_s=a_r\cdot q^{s-r},$$
což jsme chtěli dokázat.\qedhere
\end{enumerate}
\end{proof}

\begin{veta}
Nechť $\left \{ a_n \right \}_{n=1}^\infty $ je geometrická posloupnost s kvocientem
$q$. Nechť
$$S_n=a_1+a_2+\dots+a_n$$
je součet prvních $n$ členů nekonečné geometrické posloupnosti nebo součet
$n$-členné geometrické poslouppnosti. Pak platí $\forall n \in \mathbb N$:
\begin{enumerate}[$i.$]
\item $q=1: S_n=na_1,$
\item $q\ne 1: S_n = a_1\cdot \frac{q^n-1}{q-1}.$
\end{enumerate}
\end{veta}

\begin{proof}
\begin{align*}
    S_n &= a_1+a_2+a_3+\dots+a_{n-1}+a_n \\
    &= a_1 + a_1q+ a_1q^2+ \dots + a_1q^{n-1}\\
    qS_n &= a_1q + a_1q^2 +  \dots + a_1q^{n-1} + a_1q\\
    qS_n-S_n &= -a_1+a_1q^n\\
    S_n(q-1) &= a_1(q^n-1), \textrm{ jestli } q\ne 0, \textrm{ pak}\\
    S_n &= \frac{a_1(q^n-1)}{q-1}
\end{align*}
což jsme chtěli dokázat. \qedhere
\end{proof}

\begin{pozn}
    Rozlišujeme dva typy zadání posloupnosti, a to:
    \begin{enumerate}[$i.$]
    \item \textbf{rekurentní}: pomocí jednoho nebo několika předchozích členů
    \begin{align*}
        a_n=a_{n-1}+d, & & a_{n-1}\cdot q;
    \end{align*}
   	\item \textbf{explicitní}: $n$-tý člen je vyjádřen pomocí $n$
    \begin{align*}
        a_n=a_1+(n-1)d, & & a_n = a_1\cdot q^{n-1}.
    \end{align*}
    \end{enumerate}
\end{pozn}

\section{Limita funkce}
\begin{definition}[Vlastní limita ve vlastním bodě]\label{vlimv}
    Funkce $f$ má v bodě $x_0\in \mathbb R$ \textbf{limitu} $A\in \mathbb R,$
    jestliže ke každému $\varepsilon \in \mathbb R^+$ existuje
    $\delta \in \mathbb R^+$ takové, že $\forall x \in
    \left ( x_0-\delta, x_0+\delta \right ), x\ne x_0 $ platí $f(x)\in
    \left ( A-\varepsilon,A+\varepsilon \right ). $ Píšeme
    $$\lim_{x\to x_0}f(x)=A.$$
\end{definition}

\begin{pozn}
    Definici \ref{vlimv} lze taky zapsat symbolicky. Řekneme, že
    $\lim_{x\to x_0} f(x) =A$ právě tehdy, když
    $$
    \forall \varepsilon \in \mathbb R^+:
        \exists \delta \in \mathbb R^+: \forall x \in
        \left ( x_0-\delta,x_0+\delta \right )-\left \{ x_0 \right \}:
        f(x) \in \left ( A-\varepsilon, A+\varepsilon \right ).
    $$
\end{pozn}

\begin{definition}[Nevlastní limita ve vlastním bodě]\label{nlimv}
Funkce $f$ má v bodě $x_0\in \mathbb R$ \textbf{nevlastní limitu} $+\infty$ (resp.
$-\infty$),
jestliže ke každému $M \in \mathbb R$ existuje
$\delta \in \mathbb R^+$ takové, že $\forall x \in
\left ( x_0-\delta, x_0+\delta \right ), x\ne x_0 $ platí $f(x)>M $
(resp. $f(x)<M$). Píšeme
$$\lim_{x\to x_0}f(x)=\infty \,\,\,(\textrm{resp. } -\infty).$$
\end{definition}

\begin{pozn}
    Definici \ref{nlimv} lze taky zapsat symbolicky. Řekneme, že
    $\lim_{x\to x_0} f(x) =\infty$ (resp. $-\infty$) právě tehdy, když
    $$
    \forall M \in \mathbb R:
        \exists \delta \in \mathbb R^+: \forall x \in
        \left ( x_0-\delta,x_0+\delta \right )-\left \{ x_0 \right \}:
        f(x) > M\,\,\, (\textrm{resp. } f(x) < M) .
    $$
\end{pozn}

\begin{definition}[Vlastní limita v nevlastním bodě]\label{vlimn}
Funkce $f$ má v bodě $\infty$ (resp. $-\infty$) \textbf{limitu} $A\in \mathbb R,$
jestliže ke každému $\varepsilon \in \mathbb R^+$ existuje
$K \in \mathbb R$ takové, že $\forall x \in \mathbb R, x > K$ (resp. $x<K$)
platí $f(x)\in \left ( A-\varepsilon,A+\varepsilon \right ). $ Píšeme
\begin{align*}
\lim_{x\to \infty}f(x)=A, & & \textrm{resp. } \lim_{x\to -\infty}f(x)=A.
\end{align*}
\end{definition}

\begin{pozn}
Definici \ref{vlimn} lze taky zapsat symbolicky. Řekneme, že
$\lim_{x\to \infty} f(x) =A$ (resp. $\lim_{x\to -\infty} f(x) =A$) právě tehdy, když
$$
\forall \varepsilon \in \mathbb R^+:
    \exists K \in \mathbb R: \forall x \in \mathbb R, x > K \,\,\, (\textrm{resp. } x< K):
    f(x) \in \left ( A-\varepsilon, A+\varepsilon \right ).
$$
\end{pozn}

\begin{pozn}
    Definice nevlastní limity v nevlastním bodě jsou celkem čtyři (dvakrát
    dvě možnosti pro plus / minus nekonečno).
    Pro ušetření místa budou oba případy zaznačeny jako $\pm \infty$.
\end{pozn}

\begin{definition}[Nevlastní limita v nevlastním bodě]\label{nlimn}
    Funkce $f$ má v bodě $\pm \infty$ \textbf{limitu} $\pm \infty,$
    jestliže ke každému $M \in \mathbb R$ existuje
    $K \in \mathbb R$ takové, že $\forall x \in
    \mathbb R, x> K $ (resp. $x<K$) platí $f(x)>M$ (resp. $f(x) <M$). Píšeme
    $$\lim_{x\to \pm\infty}f(x)=\pm \infty.$$
\end{definition}

\begin{pozn}
    Definici \ref{nlimn} lze taky zapsat symbolicky. Řekneme, že
    $\lim_{x\to \pm\lim} f(x) =\pm\infty$ právě tehdy, když
    $$
    \forall M \in \mathbb R:
        \exists K \in \mathbb R: \forall x \in \mathbb R, x>K\,\,\, (\textrm{resp. }x<K):
        f(x) > M \,\,\, (\textrm{resp. } f(x)<M).
    $$
\end{pozn}

\begin{definition}
\textbf{Okolím bodu}
\begin{enumerate}[$i.$]
\item $x_0\in \mathbb R$ rozumíme množinu $(x_0-\delta,x_0+\delta), \delta \in \mathbb R^+,$
\item $\infty$ rozumíme množinu $(h,\infty), k\in \mathbb R,$
\item $-\infty$ rozumíme množinu $(-\infty,k), k\in \mathbb R$
\end{enumerate}
a značíme $\mathscr O(x_0)$ (resp. $\mathscr O(\infty)$). \textbf{Prstencovým okolím
bodu} $x$ je množina $\mathscr O(x)-\left \{ x_0 \right \} $ a značíme $\mathscr P(x).$
Množinu $\mathbb R \cup \left \{ \pm \infty \right \} $ nazvěme \textbf{rozšířenou množinou
reálných čísel} a označme $\mathbb R^*.$
\end{definition}

\begin{definition}[Souhrnná definice limity]
Řekneme, že $f$ má v bodě $x_0\in \mathbb R^*$ limitu $A \in \mathbb R^*$,
jestliže každému okolí $\mathscr O(A)$ bodu $A$ existuje prstencové
okolí $\mathscr P(x_0)$ bodu $x_0$ takové, že pro všechna $x\in \mathscr P(x_)$
platí $f(x)\in \mathscr O(A).$ Píšeme
$$
\lim_{x\to x_0}f(x)=A.
$$
\end{definition}

\begin{veta}
    Funkce $f$ má v bodě $x_0\in \mathbb R^*$ nejvýše jednu limitu.
\end{veta}

\begin{definition}
Funkce $f$ je \textbf{spojitá v bodě} $x_0\in \mathbb R,$ jestliže
$$\lim_{x\to x_0}f(x)=f(x_0).$$
\end{definition}

\begin{definition}
Funkce $f$ je \textbf{spojitá na intervalu} $J\subseteq \mathbb R,$ jestliže
\begin{enumerate}[$i.$]
\item $f$ je spojitá v každém vnitřním bodě intervalu $J,$
\item patří-li počáteční (resp. koncový) bod $j$ k tomuto intervalu,
je v něm funkce $f$ spojitá zprava (resp. zleva).
\end{enumerate}
\end{definition}

\begin{pozn}[Neurčité výrazy]
Neurčité výrazy jsou
$$\frac{0}{0},  \,\,\, \frac{\infty}{\infty}, \,\,\, 0\cdot \infty, \,\,\, \infty - \infty, \,\,\,0^0,\,\,\, \infty^0,\,\,\, 1^\infty.$$
\end{pozn}

\begin{veta}
Nechť $x_0\in \mathbb R^*$ a nechť existuje $\lim_{x\to x_0}f(x)$ a $\lim_{x\to x_0}g(x)$.
Pak platí:
\begin{enumerate}[$i.$]
\item $\lim_{x\to x_0} \left \{ f(x)\pm g(x) \right \}= \lim_{x\to x_0}f(x) \pm \lim_{x\to x_0}g(x),$
\item $\lim_{x\to x_0}f(x)g(x)=\lim_{x\to x_0}f(x)\cdot \lim_{x\to x_0}g(x),$
\item $\lim_{x\to x_0}\frac{f(x)}{g(x)}=\frac{\lim_{x\to x_0}f(x)}{\lim_{x\to x_0}g(x)},$
\item $\lim_{x\to x_0}|f(x)| = |\lim_{x\to x_0}f(x)|.$
\end{enumerate}
\end{veta}

\begin{veta}
Nechť jsou dány funkce $f$ a $g$ a nechť existuje prstencové okolí
$\mathscr P(x_0)$ takové, že $\forall x \in \mathscr P(x_0):f(x)=g(x).$
Nechť $\lim_{x\to x_0}g(x)=A, A\in \mathbb R^*.$ pak existuje $\lim_{x\to x_0}f(x)$
a platí $\lim_{x\to x_0}f(x)=A.$
\end{veta}

\begin{veta}[Věta o sevření]
Nechť $f,g,h$ jsou funkce a nechť rxistuje prstencové okolí $\mathscr P(x_0)$
bodu $x_0\in \mathbb R^*$ takové, že $\forall x \in \mathscr P(x_0):q(x)\leq f(x)\leq h(x).$
Nechť $\lim_{x\to x_0}g(x)=\lim_{x\to x_0}h(x)=A, A \in \mathbb R^*.$ Pak existuje
$\lim_{x\to x_0}f(x)$ a  platí $ \lim_{x\to x_0}f(x) = A.$
\end{veta}

\begin{veta}[Věta o součinu nulové a ohraničené funkce]
Nechť $f,g$ jsou funkce a $\lim_{x\to x_0}f(x)=0.$ Nechť existuje prstencové
okolí $\mathscr P(x_0), x_0\in \mathbb R^*$ bodu $x_0\in \mathbb R^*$ takové, že
funkce $g$ je na tomto okolí ohraničená. Pak $\lim_{x\to x_0}f(x)g(x)=0.$
\end{veta}

\begin{veta}[Věta o limitě složené funkce]\label{slozf}
Nechť $f,g$ jsou funkce a $x_0\in \mathbb R^*, A \in \mathbb R$ a nechť platí
\begin{enumerate}[$i.$]
\item $\lim_{x\to x_0}g(x)=A,$
\item funkce $f$ je spojitá v bodě $x_0.$
\end{enumerate}
Pak platí
$$\lim_{x\to x_0}f \left ( g(x) \right ) =f \left ( \lim_{x\to x_0}g(x) \right ) =f(A).$$
\end{veta}

\begin{pozn}
    Důsledkem věty \ref{slozf} je fakt, že
    $$\lim_{x\to x_0}g(x)=A \implies \lim_{x\to x_0}e^{g(x)}=a^A.$$
    To využíváme při výpočetu limit exponenciálních výrazů. Platí totiž
    $$f(x)^{g(x)}=e^{g(x)\ln f(x)}.$$
\end{pozn}

\begin{veta}[Věta o limitě typu $1 / 0$]
Nechť $f$ je funkce a nechť existuje pravé prstencové okolí $\mathscr P^+(x_0)$ bodu
$x_0\in \mathbb R^*$ takové, že pro každé $x\in \mathscr P^+(x_0)$ platí
$f(x)>0$ (resp. $f(x)<0$). Nechť $\lim_{x\to x_0^+}=0.$ Pak platí
$$\lim_{x\to x_0^+} \frac{1}{f(x)}=\infty \,\,\, (\textrm{resp. } -\infty).$$
Analogicky pro levé okolí bodu.
\end{veta}

\begin{veta}[l`H\^ospitalovo pravidlo]
Nechť $x_0\in \mathbb R^*$. Nechť je splněna jedna z podmínek
\begin{enumerate}[$i.$]
\item $\lim_{x\to x_0}f(x)=\lim_{x\to x_0}g(x)=0,$
\item $\lim_{x\to x_0}|g(x)| = +\infty.$
\end{enumerate}
Existuje-li $\lim_{x\to x_0}\frac{f^\prime (x)}{g^\prime (x)},$ pak existuje také
$\lim_{x\to x_0}\frac{f(x)}{g(x)}$ a platí
$$\lim_{x\to x_0}\frac{f(x)}{g(x)}=\lim_{x\to x_0}\frac{f^\prime (x)}{g^\prime(x)}.$$
\end{veta}

\section{Derivace funkce}
\begin{definition}\label{derivace}
Nechť $a \in \mathbb R$ a $f$ je funkce. Jestliže existuje limita
$$\lim_{h\to 0} \frac{f(a+h)-f(a)}{h},$$
pak tuto limitu nazýváme \textbf{derivací funkce} $f$ \textbf{v bodě} $a$ a značíme ji $f^\prime (a).$
Obdobně definujeme \textbf{derivaci zprava} a \textbf{derivaci zleva funkce} $f$\textbf{v bodě}
$a$ předpisy
\begin{align*}
    f_+^\prime(a) = \lim_{h\to 0+} \frac{f(a+h)-f(a)}{h} & & a & & f_-^\prime(a) = \lim_{h\to 0-} \frac{f(a+h)-f(a)}{h}.
\end{align*}
Derivaci zleva a derivaci zprava souhrnně nazýváme \textbf{jednostrannými derivacemi}.
\end{definition}

\begin{pozn}
    Derivaci lze obdobně zavést jako limitu
    $$\lim_{x\to a} \frac{f(x)-f(a)}{x-a}.$$
\end{pozn}

\begin{pozn}
     Při počítání derivace funkce $f$ v bodě $a\in \mathbb R$ mohou nastat
     tyto případy:
     $$
     \textrm{derivace v bodě } a \begin{cases}
        \textrm{neexistuje,} \\
        \textrm{existuje a je} \begin{cases}
            \textbf{vlastní}\textrm{, tj. je rovna reálnému číslu}, \\
            \textbf{nevlastní}\textrm{, tj. je rovna } +\infty \textrm{ nebo } - \infty.
        \end{cases}
     \end{cases}
     $$
\end{pozn}

\begin{pozn}
    Hodnoda derivace v daném bodě je směrnice tečny funkce v tomto bodě.
\end{pozn}

\begin{veta}
    Jestliže existuje derivace funkce $f$ v bodě $a$ (vlastní či nevlastní),
    pak je určena jednoznačně.
\end{veta}

\begin{definition}
Nechť existuje vlastní derivace funkce $f(x)$ pro všechna $x\in M, M\subset D(f).$
Pak funkci $f^\prime(x)$, která každému bodu $x\in M$ přiřadí derivaci funkce $f$
v tomto bodě, nazýváme \textbf{derivaci funkce} $f$ \textbf{na množině} $M.$
\end{definition}

\begin{veta}[Bolzanova]
Má-li funkce $f$ v bodě $a$ vlastní derivaci, je v tomto bodě spojitá.
\end{veta}

\begin{pozn}
    Má-li funkce $f:M\to \mathbb R$ na množině $M\subset \mathbb R$ vlastní derivaci
    v každém bodě $x\in M,$ pak zobrazení $f^\prime:M\to \mathbb R,$ které přiřadí
    bodu $x\in M$ hodnotu $f^\prime(x),$ je reálnou funkcí definovanou na množině $M$.
\end{pozn}

\begin{pozn}[Derivace základních funkcí]
Platí:
\begin{enumerate}[$i.$]
\item $(c)^\prime = 0, c \in \mathbb R,$
\item $(x^r)^\prime = r\cdot x^{r-1}, r\in \mathbb R, x \in \mathbb R^+,$
\item $(\sin x)^\prime = \cos x, x \in \mathbb R,$
\item $(\cos x) ^\prime = -\sin x, x \in \mathbb R,$
\item $(e^x)^\prime = e^x,x \in \mathbb R,$
\item $(\tg x)^\prime = \frac{1}{\cos^2 x}, x \in \mathbb R-\left \{ \frac{\pi}{2}+k\pi, k\in \mathbb Z \right \} $
\item $(\cotg x)^\prime = -\frac{1}{\sin^2 x}, x \in \mathbb R-\left \{ k\pi, k \in \mathbb Z \right \} $
\item $(\ln x)^\prime = \frac{1}{x}, x \in \mathbb R^+,$
\item $(\arcsin x)^\prime = \frac{1}{\sqrt{1-x^2}}, x \in(-1,1),$
\item $(\arccos x)^\prime=-\frac{1}{\sqrt{1-x^2} }, x \in (-1,1),$
\item $(\arctg x)^\prime = \frac{1}{x^2+1}, x \in \mathbb R,$
\item $(\arccotg x)^\prime = -\frac{1}{x^2+1}, x \in \mathbb R,$
\item $(a^x)^\prime = a^x \ln a, a>0, a\ne 1, x\in \mathbb R,$
\item $(\log_a x)^\prime=\frac{1}{x\ln a}, a>0, a\ne 1, x \in \mathbb R^+.$
\end{enumerate}
\end{pozn}

\begin{veta}[Věta o derivaci součtu, rozdílu, součinu a podílu funkcí]
Nechť existují derivace funkcí $f$ a $g$ v bodě $a\in \mathbb R.$ Pak
také funkce $f\pm g, fg, f/g$ a $cf,$ kde $c\in \mathbb R$, mají v bodě $a$ derivaci a platí:
\begin{enumerate}[$i.$]
\item $(f\pm g)^\prime (a) = f^\prime(a) \pm g^\prime(a),$
\item $(fg)^\prime (a) = f^\prime (a)g(a)+ f(a)g^\prime(a),$
\item $\left ( \frac{f}{g} \right )^\prime(a)=\frac{f^\prime (a)g(a)-f(a)g^\prime(a)}{g^2(a)} $ pro $g(a)\ne 0,$
\item $(cf)^\prime(a)=cf^\prime (a).$
\end{enumerate}
\end{veta}

\begin{veta}[Derivace inverzní funkce]
Nechť funcke $f:x=f(y)$ je spojitá a ryze monotónní na intervalu $I.$ Nechť $y_0$
je vnitřní bod intervalu $I$ a nechť má $f$ v $y_0$ derivaci $f^\prime(y_0).$
Pak inverzní funkce $f^{-1}:y=f^{-1}(x)$ má v bodě $x_0=f(y_0)$ derivaci a platí
$$
\left ( f^{-1} \right )^\prime(x_0)= \begin{cases}
\frac{1}{f^\prime(y_0)}, & \textrm{je-li } f^\prime(y_0)\ne 0,\\
+\infty & \textrm{je-li } f^\prime (y_0)=0 \textrm{ a funkce } f \textrm{ je na } I \textrm{ rostoucí},\\
-\infty & \textrm{je-li } f^\prime (y_0)=0 \textrm{ a funkce } f \textrm{ je na } I \textrm{ klesající}.
\end{cases}
$$
\end{veta}

\begin{veta}[Derivace složené funkce]
Nechť $f, g$ jsou funkce. Nechť existuje derivace funkce $g$ v bodě $a$ a
derivace funkce $f$ v bodě $b=g(a).$ Pak i složená funkce $F=f\circ g$ má derivaci
v bodě $a$ a platí
$$F^\prime(a)=(f\circ g)^\prime (a) = f^\prime(b)\cdot g^\prime(a)=f^\prime(g(a))\cdot g^\prime(a).$$
\end{veta}

\begin{pozn}
    Funkce tvaru $f(x)^{g(x)}$ derivujeme jako složenou funkci $e^{g(x)\ln f(x)},$
    protože $f(x)^{g(x)}=e^{g(x)\ln f(x)}$ a funkce $e^r, r \in \mathbb R$ je prostá
    a spojitá.
\end{pozn}

\begin{definition}
Nechť $n \in \mathbb N$. Potom \textbf{$n$-tou derivací} funkce $f$ rozumíme funkci,
kterou označujeme $f^{(n)}$ a definujeme
$$f^{(n)}=\left ( f^{(n-1)} \right )^\prime, $$
přičemž $f^{(0)}=f.$
\end{definition}

\begin{definition}
Přímka $t$ o rovnici
$$y-f(x_0)=f^\prime(x_0)(x-x_0)$$
se nazývá \textbf{tečna ke grafu funkce} $f$ v dotykovém bodě $T=(x_0,f(x_0))$.
Přímka $n$, která prochází bodem $T$ a je kolmá k přímce $t$, se nazývá \textbf{normála
ke grafu funkce} $f$ v bodě $T.$
\end{definition}

\begin{veta}
    Má-li tečna grafu funkce $t$ v dotykovém bodě $T=(x_0, f(x_0))$ rovnici
    $$y-y_0=f^\prime(x_0)(x-x_0), \textrm{ kde } y_0=f(x_0),$$
    pak normála $n$ má rovnici
    $$y-y_0=-\frac{1}{f^\prime(x_0)}(x-x_0), \textrm{ pokud } f^\prime(x_0)\ne 0.$$
\end{veta}

\begin{pozn}
    Funkci lze vyjádřit buď jako $y=f(x)$ (explicitně -- tedy vždy máme přímo vyjádřeno
    $y$), nebo ve tvaru $f(x,y)=0$ (implicitně -- z této rovnice obecně nelze vyjádřit
    $y$). Například $x^2+y^2-r^2=0$ je implicitní vyjádření kružnice,
    zatímco $y=\pm \sqrt{r^2-x^2} $ je explicitní vyjádření.
\end{pozn}

\begin{veta}[Derivace funkce dané implicitně]\label{implfce}
Nechť máme funkci danou implicitně rovnicí $F(x,y)=0$. Pak
$$F^\prime (x,y)=\frac{\partial F(x,y)}{\partial x} + \frac{\partial F(x,y)}{\partial y}\cdot y^\prime.$$
\end{veta}

\begin{pozn}
Věta \ref{implfce} je složitá. Lze ji shrnout asi takhle:\\
Funkci zderivujeme podle $x$ ($y$ považujeme za konstantu) tak, jak jsme zvyklí. Pak
funkci zderivujeme ještě podle $y$ a předtím, než to přičteme k tomu,
co jsme předtím derivovali podle $x$, to vynásobíme $y^\prime$. \\
To funguje,
protože se vlastně díváme na $y$ jako na funkci $x$ ($y=f(x)$), takže
derivace $y=f^\prime(x)\cdot (f(x))^\prime$ (např. $\left ( y^2 \right )^\prime =2yy^\prime$).
\end{pozn}


\begin{veta}[Cauchy-Bolzanova věta]
Nechť je funkce $f$ spojitá na uzavřeném intervalu $\left < a,b \right > $ a platí
$f(a)\cdot f(b)<0.$ Pak existuje alespoň jedno číslo $x_0\in (a,b)$ takové, že $f(x_0)=0.$
\end{veta}

\begin{veta}[Rolleho věta]
Nechť funkce $f$ má následující vlastnosti:
\begin{enumerate}[$i.$]
\item je spojitá na uzavřeném ohraničeném intervalu $\left < a,b \right > $,
\item má derivaci na otevřeném intervalu $(a,b)$,
\item platí $f(a)=f(b)$.
\end{enumerate}
Pak existuje alespoň jedno číslo $x_0\in (a,b)$ takové, že $f^\prime(x_0)=0.$
\end{veta}

\begin{veta}[Lagrangeova věta]
Nechť funkce $f$ má následující vlastnosti:
\begin{enumerate}[$i.$]
\item je spojitá na uzavřeném ohraničeném intervalu $\left < a,b \right > $,
\item má derivaci na otevřeném intervalu $(a,b)$.
\end{enumerate}
Pak existuje alespoň jedno číslo $x_0\in (a,b)$ takové, že
$$f^\prime(x_0)=\frac{f(b)-f(a)}{b-a}.$$
\end{veta}

\begin{veta}[Cauchyho věta]
Nechť funkce $f$ a $g$ mají následující vlastnosti:
\begin{enumerate}[$i.$]
\item jsou spojité na uzavřeném ohraničeném intervalu $\left < a,b \right > $,
\item mají derivaci na otevřeném intervalu $(a,b)$, přičemž $g^\prime(x)\ne 0$ na $\left < a,b \right > $.
\end{enumerate}
Pak existuje alespoň jedno číslo $x_0\in (a,b)$ takové, že
$$f^\prime(x_0)=\frac{f(b)-f(a)}{g(b)-g(a)}.$$
\end{veta}

\section{Průběh funkce}
\begin{pozn}
    Musíme umět nakreslit graf a popsat základní vlastnosti lineární a
    kvadratické funkce, nepřímé úměrnosti, lineární lomené, mocninné, exponenciální,
    logaritmické, goniometrické a cyklometrické funkce.
\end{pozn}

\begin{veta}
Nechť funkce $f$ má na intervalu $(a,b),a,b\in \mathbb R^*$ derivaci. Je-li pro
všechna $x$ z intervalu $(a,b)$:
\begin{enumerate}[$i.$]
\item $f^\prime(x)>0,$ pak je $f$ rostoucí,
\item $f^\prime(x)\geq 0,$ pak je $f$ neklesající,
\item $f^\prime(x)<0,$ pak je $f$ klesající,
\item $f^\prime(x) \leq 0,$ pak je $f$ nerostoucí,
\item $f^\prime(x)=0,$ pak je $f$ konstantní
\end{enumerate}
ma intervau $(a,b).$
\end{veta}

\begin{definition}
Funkce $f$ má v bodě $x_0$ \textbf{lokální minimum} (resp. \textbf{maximum}),
jestliže existuje okolí $\mathscr O(x_0)$ takové, že $\forall x \in \mathscr O(x_0):$
\begin{align*}
    f(x)\geq f(x_0) & & \textrm{resp. } f(x)\leq f(x_0).
\end{align*}
Lokální minimum (resp. maximum) je \textbf{ostré}, jestliže
\begin{align*}
    f(x)> f(x_0) & & \textrm{resp. } f(x)<f(x_0).
\end{align*}
\end{definition}

\begin{definition}
Bod $x_0\in D(f)$ takový, že $f^\prime(x_0)=0$ je \textbf{stacionární bod}.
\end{definition}

\begin{veta}
Nechť $f$ má v bodě $x_0$ lokální extrém. Pak buď $f^\prime(x_0)=0,$ nebo
$f^\prime(x_0)$ neexistuje.
\end{veta}

\begin{definition}\label{konvkonk}
Funkce $f$ je \textbf{konvexní} (resp. \textbf{konkávní}) v bodě $x_0\in D(f),$
jestliže existuje okolí $\mathscr O(x_0)$ takové, že pro všechna $x\in\mathscr O(x_0)$
platí
\begin{align*}
    f(x)\geq g(x), & & \textrm{resp. } f(x)\leq g(x),
\end{align*}
kde $g(x)$ jsou funkční hodnoty na tečně v bodě $(x_0,f(x_0)).$

Funkce je konvexní (resp. konkávní) na intervalu, jestliže je konvexní (resp.
konkávní) v každém jeho bodě.
\end{definition}

\begin{definition}
Funkce $f$ má v bodě $x_0$ \textbf{inflexi}, jestliže existuje $f^\prime(x_0) \in
\mathbb R$ a $f$ je v nějakém levém okolí $x_0$ konvexní a v nějakém pravém okolí
tohoto bodu konkávní, resp. naopak. Má-li funkce $f$ v bodě $x_0$ inflexi, pak bod
$(x_0, f(x_0))$ nazýváme \textbf{inflexním bodem} funkce $f$.
\end{definition}

\begin{veta}
Nechť má funkce $f$ v intervalu $(a,b)$ druhou derivaci. Je-li
\begin{enumerate}[$i.$]
\item $f'' (x)>0$ pro všechna $x \in (a,b)$, pak je $f$
konvexní na $(a,b)$.
\item $f''(x)<0$ pro všechna $x \in (a,b)$, pak je $f$
konkávní na $(a,b)$.
\item $f''(x)=0$ v nějakém bodě $x_0\in(a,b)$ a dále je $f''$
kladná v nějakém levém okolí bodu $x_0$ a záporná v nějakém pravém okolí bodu $x_0$,
resp. naopak, pak má $f$ v $x_0$ inflexi.
\end{enumerate}
\end{veta}

\begin{definition}
Přímka $x=x_0,x_0\in \mathbb R$ se nazývá \textbf{asymptota bez směrnice}
grafu funkce $f$, jestliže alespoň jedna jednostranná limita funkce $f$
v bodě $x_0$ je nevlastní, tj.
\begin{align*}
    \lim_{x\to x_0^+} f(x) = \pm\infty, & & \lim_{x\to x_0^-} f(x)=\pm\infty.
\end{align*}
\end{definition}

\begin{definition}
Přímka $y=ax+b, a,b\in \mathbb R$ se nazývá \textbf{asymptota se směrnicí} grafu
funkce $f$ v $+\infty$ (resp. $-\infty$), jestliže
\begin{align*}
    \lim_{x\to\infty}(f(x)-(ax+b))=0, & & \textrm{resp. }\lim_{x\to-\infty}(f(x)-(ax+b))=0.
\end{align*}
\end{definition}

\begin{veta}
    Přímka $y=ax+b$ je asymptota se směrnicí v $\pm\infty$ právě tehdy, když
    \begin{align*}
        \lim_{x\to\pm\infty}\frac{f(x)}{x}=a, a\in \mathbb R, & & \lim_{x\to\pm\infty}(f(x)-ax)=b,b \in \mathbb R.
    \end{align*}
\end{veta}

\begin{pozn}
    Při určování průběhu funkce musíme:
    \begin{enumerate}[$i.$]
    \item vyšetřit $D(f)$, body nespojitosti, nulové body, znaménka funkce, popř.
    sudost / lichost, periodičnost,
   	\item určit intervaly monotonie, lokální extrémy,
   	\item určit intervaly konvexnosti / konkávnosti, inflexní body,
   	\item určit asymptoty bez směrnice a se směrnicí a
   	\item načrtnout graf funkce.
    \end{enumerate}
\end{pozn}

\section{Neurčitý integrál}
\begin{definition}
Nechť $f(x)$ je definována na intervalu $I$. Funkce $F(x)$ se nazývá \textbf{primitivní}
k funkci $f(x)$ na $I$, jestliže platí $F^\prime(x)=f(x)$ pro každé $x\in I.$
Množina všech primitivních funkcí k funkci $f(x)$ na $I$ se nazývá \textbf{neurčitý
integrál} z funkce $f(x)$ a značí se $\int f(x)\, dx.$ Tedy
$$\int f(x)\, dx= \left \{ F(x):F(x) \textrm{ je primitivní k } f(x) \textrm{ na } I \right \}. $$
\end{definition}

\begin{veta}
    Nechť $F(x)$ je primitivní funkce k $f(x)$ na intervalu $I$, je taky
    $F(x)+c$ primitivní funkce k $f(x)$ na $I$. Má-li funkce $f(x)$ aspoň
    jednu primitivní funkci, má jich nekonečně mnoho.
\end{veta}

\begin{veta}
Je-li funkce $f$ spojitá na intervalu $I$, pak na tomto intervalu existuje
primitivní funkce k funkci $f$.
\end{veta}

\begin{veta}[Pravidla pro integrování]
Nechť na intervalu $I$ existují integrály $\int f(x) \, dx$ a $\int f(x)\, dx.$
Pak na $I$ existují také integrály $\int(f(x)\pm g(x))\, dx$ a $\int \alpha f(x), \alpha \in \mathbb R.$
Platí:
\begin{align*}
\int (f(x)\pm g(x)) \, dx &= \int f(x)\, dx \pm \int g(x) \, dx,\\
\int \alpha f(x)\, dx &= \alpha\int f(x)\, dx.
\end{align*}
\end{veta}

\begin{pozn}[Tabulkové integrály]
\begin{align*}
    &\int 0\, dx = c & & \, \\
    &\int dx = x+c & & \, \\
    &\int x^n \, dx = \frac{x^{n+1}}{n+1}+x, n\ne -1 & & \, \\
    &\int \frac{1}{x} \, dx = \ln |x| + c & & \int \frac{1}{x+a}\, dx = \ln |x+a|+c \\
    &\int e^x\, dx = e^x+c & & \int a^{ax}\, dx = \frac{1}{a}e^{ax}+c \\
    &\int a^x \, dx = \frac{a^x}{\ln a} + c, a >0 & & \int a^{bx} \, dx = \frac{1}{b}\cdot \frac{a^{bx}}{\ln a}+c, a>0\\
    &\int \sin x \, dx = -\cos x + c & & \int \sin ax \, dx = -\frac{1}{a}\cos ax+c \\
    &\int \cos x \, dx = \sin x +c & & \int \cos ax\, dx = \frac{1}{a}\sin ax + c \\
    &\int \frac{1}{\cos^2 x}\, dx = \tg x + c & & \int \frac{1}{\cos^2 ax}\, dx=\frac{1}{a}\tg ax+c\\
    &\int \frac{1}{\sin^2 x} \, dx = -\cotg x+c & & \int \frac{1}{\sin^2 ax}\, dx = \frac{1}{a}\cotg ax+c \\
    &\int \frac{1}{\sqrt{1-x^2} }\, dx = \arcsin x + c & & \int \frac{1}{\sqrt{1-a^2x^2} }\, dx = \frac{1}{a}\arcsin ax+c \\
    &\int \frac{1}{x^2+1}\, dx = \arctg x + c & & \int \frac{1}{a^2x^2 + 1}\, dx = \frac{1}{a}\arctg ax + c
\end{align*}
\end{pozn}

\begin{veta}[Integrační metoda \textit{per partes}]
Nechť funkce $u(x) $ a $v(x)$ mají derivaci na intervalu $I$. Pak platí
$$\int u(x)v^\prime (x) \, dx = u(x) v(x) - \int u^\prime (x) v(x) \, dx,$$
pokud alespoň jeden z integrálů existuje.
\end{veta}

\section{Určitý integrál}
\begin{pozn}
    Supremum a infimum množiny jsme již zavedli v definici \ref{supinf}.
\end{pozn}

\begin{definition}
\textbf{Dělením} $D$ uzavřeného \textbf{intervalu} $\left < a,b \right > $
rozumíme každou konečnou množinu čísel $x_0,\dots,x_n \in \mathbb R$ takových, že
$a=x_0<x_1<\dots<x_n=b.$ Intervaly $\left < x_0, x_1 \right > ,
\left < x_1, x_2 \right >, \dots,$ $\left < x_{n-1}, x_n \right >  $ nazýváme
\textbf{dělící intervaly}, které označíme $D=\left \{ x_0,x_1,\dots,x_n \right \} .$
Body $x_0,\dots,x_n$ nazýváme \textbf{dělící body} dělení $D$. Číslo
$\max \left ( x_i-x_{i-1} \right ), i=1,\dots,n $ nazýváme \textbf{normou dělení}
$D$ a označujeme $\nu(D).$ Množinu všech dělení na intervalu $\left < a,b \right > $
označme $\mathscr D(a,b).$
\end{definition}

\begin{definition}
Nechť $D_1, D_2 \in \mathscr D(a,b)$. Řekneme, že dělení $D_2$ je \textbf{zjemněním
dělení} $D_1,$ jestliže $D_1 \subset D_2.$
\end{definition}

\begin{definition}
Nechť $f(x)$ je omezená na $\left < a,b \right > , D\in \mathscr D(a,b),
D=\left \{ x_0,\dots,x_n \right \}. $ Pak množina $\left \{ f(x),
x \in \left < x_{i-1},x_i \right >  \right \} $ je pro všechna $i \in \left \{
1,\dots,n\right \} $ neprázdná a omezená a tedy má supremum a infimum.
Označme
\begin{align*}
    m_i &= \inf \left \{ f(x), x \in \left < x_{i-1},x_i \right >  \right \} \textrm{ pro } D=\left \{ x_0,\dots,x_n \right \},\\
    M_i &=   \sup \left \{ f(x), x \in \left < x_{i-1},x_i \right >  \right \} \textrm{ pro } D=\left \{ x_0,\dots,x_n \right \}.
\end{align*}
Číslo $S(D,f)=\sum_{i=1}^n M_i\cdot (x_i-x_{i-1})$
(resp. $s(D,f)=\sum_{i=1}^n m_i\cdot (x_i-x_{i-1})$) je \textbf{horní} (resp. \textbf{dolní}) \textbf{součet funkce}
$f(x)$ příslušný dělení $D$.
\end{definition}

\begin{pozn}
    Platí $s(D,f)\leq S(D,f).$
\end{pozn}

\begin{veta}
Nechť $f(x)$ je omezená na intervalu $\left < a,b \right >. $ Pak
$\left \{ s(D,f); D\in \mathscr D(a,b) \right \} $ je shora omezená a množina
$\left \{ S(D,f); D\in \mathscr D(a,b) \right \} $ je zdola omezená.
\end{veta}

\begin{pozn}
     Množina dolních součtů má supremum a množina horních součtů má infimum.
\end{pozn}

\begin{definition}
Nechť $f(x)$ je omezená na intervalu $\left < a,b \right > .$ Označme
\begin{align*}
    \int_{\underline{a}} ^b f(x) \, dx & = \sup \left \{ s(D,f); D \in \mathscr D(a,b) \right \}, \\
    \int_{a} ^{\underline{b}} f(x) \, dx & = \inf \left \{ S(D,f); D \in \mathscr D(a,b) \right \}.
\end{align*}
Číslo $\int_{\underline{a}} ^b f(x)\, dx$ (resp. $\int_{a} ^{\underline{b}} f(x) \, dx$)
nazýváme \textbf{dolní} (resp. \textbf{horní}) \textbf{integrál} funkce $f(x)$ od
$a$ do $b$.
\end{definition}

\begin{veta}
Nechť $f(x)$ je omezená na intervalu $\left < a,b \right >. $ Pak
$$\int_{\underline{a}} ^b f(x)\, dx \leq \int_{a} ^{\underline{b}} f(x) \, dx.$$
\end{veta}

\begin{definition}
Nechť $f(x)$ je omezená na intervalu $\left < a,b \right > .$ Pak funkce
$f(x)$ je na intervalu $\left < a,b \right > $ (Riemmanovsky) \textbf{integrovatelná}
(integrace schopna), jestliže
$$\int_{\underline{a}} ^b f(x)\, dx = \int_{a} ^{\underline{b}} f(x) \, dx.$$
Je-li $f(x)$ na intervalu $\left < a,b \right > $ integrovatelná, klademe
$$\int_{\underline{a}} ^b f(x)\, dx = \int_{a} ^{\underline{b}} f(x) \, dx = \int_{a} ^b f(x) \, dx.$$
Číslo $\int_{a} ^b f(x)\, dx$ nazýváme \textbf{Riemannův integrál} funkce $f(x)$ od
$a$ do $b$. Pokud $\int_{\underline{a}} ^b f(x)\, dx < \int_{a} ^{\underline{b}} f(x) \, dx$,
pak $f(x)$ není na intervalu $\left < a,b \right > $ integrovatelná a
$\int_{a} ^b f(x)\, dx$ nedefinujeme.
\end{definition}

\begin{veta}[Newton-Leibnitzova věta]
Nechť $f(x)$ je integrovatelná na intervalu $\left < a,b \right > , F(x)$ je spojitá
na intervalu $\left < a,b \right > $. Buď $F(x)$ primitivní funkce k funkci $f(x)$
na intervalu $\left ( a,b \right ) $. Pak platí:
$$\int_{a} ^b f(x)\, dx=F(b)-F(a).$$
\end{veta}

\begin{veta}
Nechť $f(x)$ je spojitá na intervalu $\left < a,b \right > $. Pak $f(x)$ je na
intervalu $\left < a,b \right > $ integrovatelná.
\end{veta}

\begin{priklad}
Vypočtěte integrál $\int_0^{\frac{\pi}{4}}\cos x \, dx.$
\end{priklad}

\begin{reseni}
Platí
$$\int_0^{\frac{\pi}{4}}\cos x \, dx=\left [ \sin x  \right ]_0^{\frac{\pi}{4}}=\sin \frac{\pi}{4}-\sin 0=\frac{\sqrt{2} }{2}. $$
\end{reseni}

\begin{veta}
Nechť $f(x), f_1(x),\dots,f_n(x), g(x)$ jsou integrovatelné funkce na intervalu $\left < a,b \right > $
a $c; c_1,\dots,c\in \mathbb R.$ Pak
\begin{enumerate}[$i.$]
\item funkce $f(x)+g(x)$ je integrovatelná na int. $\left < a,b \right > $ a platí
$$\int _a^b \left [ f(x)+g(x) \right ] \, dx = \int_a ^b f(x)\, dx + \int_a ^b g(x) \, dx,$$
\item funkce $c\cdot f(x)$ je integrovatelná na int. $\left < a,b \right > $ a platí
$$\int _a ^b c\cdot f(x) \, dx = c \int _a ^b f(x)\, dx,$$
\item funkce $c_1\cdot f_1(x) + \dots + c_n \cdot f_n(x)$ je integrovatelná na int. $\left < a,b \right > $ a platí
$$\int_a ^b \left [ c_1\cdot f_1(x) + \dots + c_n \cdot f_n (x) \right ]\, dx = c_1 \int _a ^b f_1(x)\, dx + \dots + c_n \int _a ^b f_n(x)\, dx. $$
\end{enumerate}
\end{veta}

\begin{veta}
Nechť $f(x)$ je integrovatelná a nezáporná na intervalu $\left < a,b \right > $. Pak
$$\int_a ^b f(x) \, dx \geq 0.$$
\end{veta}

\begin{veta}
Nechť funkce $f(x),g(x)$ jsou integrovatelné na intervalu $\left < a,b \right > .$
Nechť pro každé $x\in \left < a,b \right > :f(x)\leq g(x).$ Pak
$$\int_a^b f(x) \, dx \leq \int_a^b g(x) \, dx.$$
\end{veta}

\begin{veta}
Nechť $a<b<c$ jsou reálná čísla a $f(x)$ je integrovatelná na intervalech
$\left < a,b \right >, \left < b,c \right > . $ Pak
$$\int_a^c f(x)\, dx = \int_a^b f(x)\, dx + \int_b^c f(x)\, dx.$$
\end{veta}

\begin{definition}\label{pocastechspoj}
Funkce $f(x)$ se nazývá \textbf{po částech spojitá} na intervalu $\left < a,b \right > $,
jestliže na intervalu $\left < a,b \right > $ spojitá s výjimkou konečného počtu
bodů $c_1,\dots,c_n; a<c_1<\dots<c_n<b.$
\end{definition}

\begin{veta}
Nechť funkce $f(x)$ je po částech spojitá na intervalu $\left < a,b \right > $, jako
v definici \ref{pocastechspoj}. Pak
$$\int_a^b f(x)\, dx = \int_a^{c_1}f(x)\, dx + \int_{c_1}^{c_2}f(x)\, dx + \dots + \int_{c_n}^b f(x)\, dx.$$
\end{veta}

\begin{definition}
Je-li $f(x)$ definována v čísle $a \in \mathbb R,$ klademe
$$\int _a ^a f(x)\, dx = 0.$$
Je-li $f(x)$ integrovatelná na intervalu $\left < a,b \right > ,$ definujeme
$$\int_b^a f(x)\, dx = -\int_a^b f(x)\, dx.$$
\end{definition}


\begin{veta}[O substituci pro určité integrály]\label{ppui}
Nechť $\varphi (t)$ má na intervalu $\left < \alpha, \beta  \right > $ spojitou derivaci
$\varphi^\prime (t)$, $f(x)$ je spojitá na intervalu $\left < a,b \right >;
\varphi: \left < \alpha, \beta \right >\to \left < a,b \right >  $. Označme
$\varphi(\alpha)=a, \varphi(\beta)=b.$ Pak platí
$$\int _a ^b f(x)\, dx = \int _\alpha ^\beta f \left [ \varphi(t) \right ]\cdot \varphi^\prime (t)\, dt. $$
\end{veta}

\begin{pozn}
    Ve větě \ref{ppui} využíváme tzv \textbf{transformace mezí}.
\end{pozn}

\begin{priklad}
Vypočtěte integrál $\int _1^2 \frac{x}{(1+x^2)}\, dx.$
\end{priklad}

\begin{reseni}
Musíme provést transformaci mezí:
\begin{align*}
\int _1^2 \frac{x}{(1+x^2)}\, dx &= \left |\begin{array}{l l}
    t=x^2+1 & x=2\implies t=5 \\
    dt = 2x \, dx & x=1\implies t=2
\end{array}\right | = \frac{1}{2}\int _2^5 \frac{1}{t^{\frac{3}{2}}}\, dt \\
&= \frac{1}{2}\left [ \frac{t^{-\frac{1}{2}}}{-\frac{1}{2}} \right ] _2^5 = \frac{1}{2}\left ( \frac{5^{-\frac{1}{2}}}{-\frac{1}{2}}-\frac{2^{-\frac{1}{2}}}{-\frac{1}{2}} \right )=-\frac{1}{\sqrt{5} }+\frac{1}{\sqrt{2} }
\end{align*}
\end{reseni}

\begin{veta}[O integraci \textit{per partes} pro určité integrály]
Nechť funkce $u(x)$ a $v(x)$ mají na intervalu $\left < a,b \right > $ spojité derivace
$u^\prime (x)$ a $v^\prime (x)$. Pak platí
$$\int_a ^b u^\prime (x)v(x)\, dx = \left [ u(x)v(x) \right ]_a^b -\int _a^b u(x)v^\prime(x)\,dx.$$
\end{veta}

\begin{priklad}
Vypočtěte integrál $\int _0^{\frac{\pi}{2}}x \cos \frac{x}{2}\, dx$.
\end{priklad}

\begin{reseni}
Platí
\begin{align*}
\int _0^{\frac{\pi}{2}}x \cos \frac{x}{2}\, dx &= \left | \begin{array}{ll}
    u=x & u^\prime = 1 \\
    v^\prime = \cos \frac{x}{2} & v = 2\sin \frac{x}{2}
\end{array}   \right | \\
& =\left [ x\cdot2\sin \frac{x}{2} \right ]_0^{\frac{\pi}{2}}-\int_0^{\frac{\pi}{2}}2\sin \frac{x}{2}\, dx =\frac{\pi}{2}\cdot \frac{\sqrt{2} }{2}-0-\left [ -4\cos \frac{x}{2} \right ]_0^{\frac{\pi}{2}}\\
&= \pi \cdot \frac{\sqrt{2} }{2}+4\cdot \frac{\sqrt{2} }{2}-4
\end{align*}
\end{reseni}

\begin{pozn}
    Je-li $f(x)$ spojitá a nezáporná na intervalu $\left < a,b \right > ,$ je
    $\int_a ^b f(x)\, dx$ obsah plochy omezené osou $x$, přímkami $x=a, x=b$ a
    grafem funkce $y=f(x)$. Platí tedy
    $$S=\int_a^b f(x)\, dx.$$
\end{pozn}

\begin{priklad}
Vypočtěte obsah plochy $M=\left \{ [x,y], y \geq  x^2 , y \leq 2x\right \} $.
\end{priklad}

\begin{reseni}
Nejprve určíme průsečíky funkcí $x^2$ a $2x$: $x_1=0,x_2=2$. Plochu spočítáme
jako
$$\int_0^2 2x\, dx - \int_0^2 x^2 \, dx.$$
\end{reseni}

\begin{priklad}
Vypočtěte obsah kruhu o poloměru $r$.
\end{priklad}

\begin{reseni}
Počítáme $2\int_{-r}^r \sqrt{r^2-x^2}\, dx  $.
\end{reseni}

\begin{pozn}
    Nechť je $f(x)$ spojitá a nezáporná na intervalu $\left < a,b \right > $ a
    $M=\left \{ \left [ x,y \right ], x \in \left < a,b \right >,\linebreak y \in \left < y, f(x) \right >    \right \} .$
    Nechť $M$ rotuje kolem osy $x$. Dostaneme rotační těleso, jehož objem je
    $$V=\pi \int _a^b \left [ f(x) \right ]^2 \, dx. $$
\end{pozn}

\begin{priklad}
Vypočtěte objem koule o poloměru $r$.
\end{priklad}

\begin{reseni}
Počítáme $\pi\int _{-r}^r(\sqrt{r^2-x^2} )^2 \, dx=\pi\int_{-r}^r(r^2-x^2)\, dx$.
\end{reseni}

\begin{pozn}
Nechť je $f(x)$ funkce se spojitou derivací na intervalu $\left < a,b \right > $. Délka
grafu funkce $f(x)$ je
$$L=\int_a^b \sqrt{1+(f^\prime(x))^2}\, dx. $$
\end{pozn}

\begin{priklad}
Spočtěte délku kružnice o poloměru $r$.
\end{priklad}

\begin{reseni}
Počítáme $2\int _{-r}^r \sqrt{1+\left ( \frac{-x}{\sqrt{r^2-x^2}} \right )^2 }\, dx.  $
\end{reseni}

\begin{pozn}
    Nechť $f(x)$ je funkce se spojitou derivací na intervalu $\left < a,b \right > .$
    Pak plášť rotačního kužele, který vznikne rotací grafu $f(x)$ kolem osy $x$, má
   povrch
  $$S=2\pi\int_a^b f(x)\cdot \sqrt{1+(f^\prime(x))^2}\, dx. $$
\end{pozn}

\begin{priklad}
Určete povrch koule o poloměru $r$.
\end{priklad}

\begin{reseni}
Počítáme $S=2\pi\int_{-r}^r \sqrt{r^2-x^2}\cdot \sqrt{1+\left ( \frac{-x}{\sqrt{r^2-x^2} } \right )^2 } \, dx. $
\end{reseni}

\section{Kuželosečky}
\begin{definition}
    Nechť je v $\mathbb E_2$ dána afinní soustava souřadnic $\left < P,\vec e_1, \vec e_2 \right > $.
    Nechť
    \begin{equation}\label{kuzelosec}
        a_{11}x^2+2a_{12}xy+a_{22}y^2+2a_{13}x+2a_{23}y+a_{33}=0,
    \end{equation}
    kde $a_{ij}\in \mathbb R, a_{11}^2 + a_{22}^2 + a_{33}^2\ne 0.$ Pak množinu všech
    bodů $[x,y] \in \mathbb E_2$, jejichž souřadnice vyhovují rovnici \ref{kuzelosec},
    nazýváme \textbf{kuželosečkou}. Rovnici \ref{kuzelosec} nazýváme \textbf{rovnicí kuželosečky}.
\end{definition}

\begin{definition}
    Nechť $k$ je kuželosečka s rovnicí \ref{kuzelosec}. Pak matici
    $$
        A = \begin{pmatrix}
            a_{11} & a_{12} & a_{13} \\
            a_{12} & a_{22} & a_{23} \\
            a_{13} & a_{23} & a_{33}
        \end{pmatrix}
    $$
    nazýváme \textbf{maticí kuželosečky} $k$.
\end{definition}

\begin{priklad}
Napište matici kuželosečky $x^2-2xy+3y+1=0.$
\end{priklad}
\begin{reseni}
Z definice je
$$
A = \begin{pmatrix}
    1 &-1 &0\\
    -1 &0 &\frac{3}{2}\\
    0 &\frac{3}{2} &1
\end{pmatrix}.
$$
\end{reseni}

\begin{pozn}
    Rovnici kuželosečky lze zapsat pomocí její matice $A$ jako
    $$\begin{pmatrix}
        x & y & 1
    \end{pmatrix}\cdot A \cdot \begin{pmatrix}
        x\\
        y\\
        1
    \end{pmatrix}=0.$$
\end{pozn}

\begin{definition}
    Nechť $k$ o rovnici \ref{kuzelosec} je kuželosečka. Každou kuželosečku lze
    vhodnou transformací soustavy souřadnic převést na jednu z následujících
    tvarů, které nazýváme \textbf{kanonickými rovnicemi kuželosečky}.
    \begin{enumerate}[$i.$]
    \item $\frac{x^2}{a^2} + \frac{y^2}{b^2}=1$ (elipsa, $a,b$ jsou její poloosy),
   	\item $\frac{x^2}{a^2} + \frac{y^2}{b^2}=-1$ (imaginární elipsa),
   	\item $\frac{x^2}{a^2} - \frac{y^2}{b^2}=1$ (hyperbola, $a,b$ jsou její poloosy),
   	\item $y^2=2px, p>0$ (parabola, $p$ je její parametr),
   	\item $y^2 = k^2x^2, k>0$ (dvě různoběžky),
   	\item $y^2 = -k^2x^2, k>0$ (bod),
   	\item $y^2 = r^2, r>0$ (dvě různé rovnoběžky),
   	\item $y^2 = -r^2, r>0$ (prázdná množina),
   	\item $y^2=0$ (dvojná přímka).
    \end{enumerate}
\end{definition}

\begin{priklad}
Je dána kuželosečka $4x^2+9y^2-16x+54y+61=0.$ Určete, jestli je to elipsa.
\end{priklad}

\begin{reseni}
Doplněním na čtverec pro $x$ a pro $y$.
\end{reseni}







\begin{definition}
    Kuželosečka je \textbf{singulární}, jestliže determinant její matice je nulový.
    V opačném případě je \textbf{regulární}.
\end{definition}

\begin{definition}
    Vyhovuje-li rovnici kuželosečky alespoň jeden bod, je to kuželosečka
    \textbf{bodově reálná}. V opačném případě je \textbf{formálně reálná}.
\end{definition}

\begin{definition}
    Elipsa, jejíž poloosy mají stejnou velikost, se nazývá \textbf{kružnice}.
\end{definition}

\begin{definition}
    Přímka, podle níž je kuželosečka souměrná, se nazývá její \textbf{osou}.
\end{definition}

\begin{definition}
\textbf{Elipsa} je množina všech bodů v rovině, které mají od dvou různých bodů v rovině,
které se nazývají \textbf{ohniska}, konstantní součet vzdálenosti.
\end{definition}

\begin{priklad}
Určete středovou rovnici elipsy $5x^2+3y^2+20x-24y+38=0.$
\end{priklad}

\begin{reseni}
Doplněním na čtverec.
\end{reseni}

\begin{definition}
\textbf{Hyperbola} je množina všech bodů v rovině, které mají od dvou různých
bodů v rovině, které se nazývají \textbf{ohniska}, konstantní absolutní
hodnotu rozdílu vzdálenosti.
\end{definition}

\begin{priklad}
Je dána hyperbola $k:9x^2-16y^2-36x+32y-124=0.$ Najděte její středovou rovnici,
střed, poloosy, ohniska, vrcholy, rovnice asymptot a hyperbolu načrtněte.
\end{priklad}

\begin{reseni}
Doplníme na čtverec a vydělíme, abychom na pravé straně rovnice měli jedničku.
Zbytek z vlastností hyperboly.
\end{reseni}

\begin{definition}
\textbf{Parabola} je množina všech bodů v rovině, které mají od pevného bodu, který
se nazývá \textbf{ohnisko} a pevné přímky, která se nazývá \textbf{řídící přímka},
na níž tento bod neleží, stejnou vzdálenost.
\end{definition}

\begin{priklad}
Určete vrchol, ohnisko, parametr a řídící přímku paraboly $k:y^2-7x-6y-19=0.$
\end{priklad}

\begin{reseni}
Doplněním na čtverec.
\end{reseni}

\begin{pozn}
    Pro přehled s rovnicemi, obrázky a popisem všech prvků těchto kuželoseček
    viz přílohu \ref{appb}.
\end{pozn}

\begin{definition}
    Přímka, která má s regulární kuželosečkou společný právě jeden bod $T$ a neobsahuje
    žádný bod vnitřní oblasti kuželosečky, se nazývá její \textbf{tečnou} a bod
    $T$ jejím \textbf{bodem dotyku}.
\end{definition}

\begin{definition}
Přímka, která má s regulární kuželosečkou společné právě dva body $T,T^\prime$, se
nazývá její \textbf{sečnou} a úsečka $TT^\prime$ \textbf{tětivou}.
\end{definition}

\begin{veta}
    Nechť \ref{kuzelosec} je obecná rovnice kuželosečky a nechť tečna $t$ této
    kuželosečky se jí dotýká v bodě $T[x_r, y_r].$ Pak tečna $t$ má rovnici
    \begin{equation}\label{kuztec}
        (a_{11}x_r+a_{12}y_r+a_{13})x+(a_{12}x_r+a_{22}y_r+a_{23})y+a_{13}x_r +
        a_{23}y_r+a_{33}=0.
    \end{equation}
\end{veta}

\begin{pozn}
    Rovnici \ref{kuztec} lze zapsat jako
    $$
    \begin{pmatrix}
        x & y & 1
    \end{pmatrix}\cdot A \cdot \begin{pmatrix}
        x_r \\
        y_r \\
        1
    \end{pmatrix}=0.
    $$
\end{pozn}

\begin{veta}[Rovnice tečen regulárních kuželoseček v základní poloze]
    Nechť je dána kuželosečka a její tečna s bodem dotyku $T[x_r,y_r].$ Pak tečna
    má rovnici
    \begin{enumerate}[$i.$]
    \item kružnice se středem v $S[m,n]:(x-m)(x_r-m)+(y-n)(y_r-n)=r^2,$
   	\item elipsa se středem v $S[m,n]:\frac{(x-m)(x_r-m)}{a^2}+\frac{(y-n)(y_r-n)}{b^2}=1$,
   	\item hyperbola se středem $S[m,n]:\frac{(x-m)(x_r-m)}{a^2}-\frac{(y-n)(y_r-n)}{b^2}=1$,
   	\item parabola s vrcholem $V[m,n]:(y-n)(y_r-n)=p(x+x_r-2m).$
    \end{enumerate}
\end{veta}

\begin{priklad}
Napište rovnice tečen ke kuželosečce $x^2+y^2-6x-4y+3=0$ v průsečíku s přímkou $y=x+3$.
\end{priklad}

\begin{reseni}
Nejprve najdeme průsečík $P[p_1,p_2]$ (vyřešíme soustavu těchto dvou rovnic).
Potom má rovnice tečen tvar
$$\begin{pmatrix}
    0 & 3 & 1
\end{pmatrix}\cdot \begin{pmatrix}
    1 & 0 & -3\\
    0 & 1 & -2 \\
    -3 & -2 & 3
\end{pmatrix}\cdot \begin{pmatrix}
    x \\
    y \\
    1
\end{pmatrix}=0.$$
\end{reseni}

\begin{definition}
    Směr v rovině určený vektorem $\vec u(u_1, u_2)\ne \vec o$ nazveme
    \textbf{asymptotickým směrem} kuželosečky o rovnici \ref{kuzelosec}, jestliže
    $$a_{11}u_1^2 + 2a_{12}u_1u_2+a_{22}u_2^2=0.$$
\end{definition}

\begin{priklad}
Určete asymptotické směry kuželosečky $x^2+4xy+y^2-7x-7=0.$
\end{priklad}

\begin{reseni}
Hledáme směr $\vec u(u_1,u_2).$ Po dosazení do rovnice kuželosečky (bereme jen
členy, ve kterých je $x,y$ dvakrát) dostáváme $u_1^2+4u_1u_2+u_2^2=0.$ Zvolíme jednu
z proměnných a rovnici vyřešíme.
\end{reseni}

\begin{definition}
    Bod $S$ je středem kuželosečky $k$ právě tehdy, když $\forall X\in k:\exists X^\prime\in k: S$ je středem $XX^\prime.$
\end{definition}

\begin{veta}
    Bod $S[s_1,s_2]$ je střed kuželosečky o rovnici \ref{kuzelosec} právě tehdy, když
    \begin{align*}
        a_{11}s_1 + a_{12}s_2+a_{13}&=0,\\
        a_{12}s_1 + a_{22}s_2 + a_{23}&=0,
    \end{align*}
\end{veta}

\begin{definition}
    Jestliže střed kuželosečky na ní leží, pak je to \textbf{singulární bod}.
\end{definition}

\begin{priklad}
Určete střed a singulární body kuželosečky $x^2+y^2+2x=0.$
\end{priklad}

\begin{reseni}
Střed kuželosečky vyhovuje rovnici
$$A\cdot \begin{pmatrix}
    s_1 \\
    s_2\\
    0
\end{pmatrix}=0,
$$
kde $A$ je matice dané kuželosečky. Pokud daný bod navíc vyhovuje rovnici
$$A\cdot \begin{pmatrix}
    s_1 \\
    s_2\\
    1
\end{pmatrix}=0,
$$
je to singulární bod.
\end{reseni}

\begin{priklad}
Je dána kuželosečka $y^2-xy-5x+7y+10=0.$ Určete:
\begin{enumerate}[$a.$]
\item zda je regulární / singulární,
\item asymptotické směry,
\item středy,
\item singulární body a
\item druh kuželosečky.
\end{enumerate}
\end{priklad}

\begin{reseni}
Nechť $A$ je matice dané kuželosečky, $\bar{A}$ zmenšená matice kuželosečky, $\vec u=(u_1,u_2)^T, \vec s=(s_1,s_2,0)^T,\vec t=(s_1,s_2,0)^T$.
\begin{enumerate}[$a.$]
\item Spočtením determinantu $A$.
\item Je řešením rovnice $\bar A\vec u=0.$
\item Je řešením rovnice $A\vec s=0.$
\item je řešením rovnice $A\vec t = 0.$
\item Doplněním na čtverec.
\end{enumerate}
\end{reseni}

\begin{definition}
\textbf{Kulová plocha} (resp. \textbf{koule}) se středem $S$ a poloměrem $r>0$ je množina všech bodů v prostoru,
jejichž vzdálenost od $S$ je $r$ (resp. menší nebo rovna $r$).
\end{definition}

\begin{definition}
\textbf{Tečná rovina} kulové plochy je taková rovina, která má s kulovou plochou právě
jeden společný bod.
\end{definition}

\section{Nezařazené vykřičníkové příklady}
\subsection{Algebraické výrazy}
\begin{priklad}
    Rozložte $2x^2 - 3x+1.$
\end{priklad}

\begin{reseni}
Doplněním na čtverec.
\end{reseni}

\begin{priklad}
Odstraňte odmocniny ze jmenovatele: $\frac{1}{2 \sqrt{x} }.$
\end{priklad}

\begin{reseni}
Platí
$$\frac{1}{2\sqrt{x} }=\frac{1}{2\sqrt{x} }\cdot \frac{\sqrt{x} }{\sqrt{x} }=\frac{\sqrt{x} }{2x},$$
je-li $x>0.$
\end{reseni}

\begin{priklad}
    Zakreslete na číselné ose
    \begin{enumerate}[$a.$]
    \item racionální číslo $-4/3.$
   	\item iracionální číslo $\sqrt{3}. $
    \end{enumerate}
\end{priklad}

\begin{reseni}\,
\begin{enumerate}[$a.$]
\item Z podobnosti trojúhelníků.
\item Z Pythagorovy věty.
\end{enumerate}
\end{reseni}

\begin{priklad}
    Upravte výraz $d(x)=(x+16)(x+17)(x+18)-(x+17)^2(x+19).$
\end{priklad}

\begin{reseni}
 Výhodnou volbou $x+17=t$ dostaneme $d(x)=-t(2t+1)=(-x-17)(2x+35).$
\end{reseni}

\begin{priklad}
    Ve výrazu $V(n+1)$ vyčleňte daný výraz $V(n)=n^3+2n.$
\end{priklad}

\begin{reseni}
 Platí
 \begin{align*}
   V(n+1) &= (n+1)^3 + 2(n+1) = n^3 + 3n^2 + 3n + 1+ 2n +2 \\
   & = \underbrace{n^3+ 2n}_{V(n)} + 3n^2 + 3n + 3 = V(n) + 3(n^2+n+1).
 \end{align*}
\end{reseni}

\begin{priklad}
V $\mathbb R$ zjednodušte
$$\frac{x^3+x^2-x-1}{\sqrt{x^2}+1 }.$$
\end{priklad}

\begin{reseni}
Platí $\sqrt{x^2}=|x| $.
\end{reseni}

\begin{priklad}
Rozložte $x^2-5x+6$.
\end{priklad}

\begin{reseni}
Doplněním na čtverec.
\end{reseni}

\begin{priklad}
Najděte nejmenší hodnotu výrazu $x^2+16x-17.$
\end{priklad}

\begin{reseni}
Doplněním na čtverec. Minimum nastane tehdy, když je čtverec nulový.
\end{reseni}


\subsection{Rovnice a nerovnice, matice}
\begin{priklad}
    V $\mathbb R$ řešte $x^2-5x \geq 0.$
\end{priklad}

\begin{priklad}
V $\mathbb R$ řešte
$$\frac{(4-x)(6+x)x}{2-x}\leq 0.$$
\end{priklad}

\begin{priklad}
V $\mathbb R$ řešte $$\frac{(x+1)(x-2)^2}{(3-x)^3(4+x)^4}\leq 0.$$
\end{priklad}

\begin{priklad}
V $\mathbb R$ řešte rovnici $|3x-5|=2x+10.$
\end{priklad}

\begin{reseni}
Rozdělíme na případy, kdy je výraz v absolutní hodnotě menší / větší než 0
a dále řešíme jako normálně.
\end{reseni}

\begin{priklad}
V $\mathbb R$ řešte $2|4+3x|\leq 6x+11.$
\end{priklad}

\begin{reseni}
Rozdělíme na případy, kdy je výraz v absolutní hodnotě menší / větší než 0
a dále řešíme jako normálně.
\end{reseni}

\begin{priklad}
V $\mathbb R$ řešte $|x+2|+|x-2|=2x+2.$
\end{priklad}

\begin{reseni}
Rozdělíme na případy, kdy je výraz v absolutní hodnotě menší / větší než 0
a dále řešíme jako normálně.
\end{reseni}

\begin{priklad}
V $\mathbb R$ řešte $|3x-2|<5+|x+1|.$
\end{priklad}

\begin{priklad}
V $\mathbb R$ řešte $x^2-6x+8 >0.$
\end{priklad}

\begin{priklad}
Danou matici převeďte na schodovitý tvar a určete její hodnost.
$$A=\begin{pmatrix}
    1 & 2 & -3 \\
    -3 & 1 & -2 \\
    2 & 3 & 2
\end{pmatrix}$$
\end{priklad}

\begin{reseni}
K jednotlivým řádkům přičítáme násobky jiných, aby nám vyšel schodovitý tvar.
\end{reseni}

\begin{priklad}
V $\mathbb R$ řešte:
\begin{align*}
    2x_1+5x_2-8x_2 & =8, \\
    4x_1 + 3x_2 - 9x_3 & =9,\\
    2x_1 + 3x_2 - 5x_3 & = 7, \\
    x_1 + 8x_2 - 7x_3 & = 12.
\end{align*}
\end{priklad}

\begin{reseni}
Převedeme na matici
$$
\left (
\begin{array}{c c c | c }
1 & 8& -7& 12\\
2& 5& -8& 8\\
4& 3& -9& 9\\
2& 3& -5& 7
\end{array}
\right )
$$
a řešíme Gaussovou eliminační metodou.
\end{reseni}

\begin{priklad}
V $\mathbb R$ řešte
\begin{align*}
    x_1+x_2+x_3&=1,\\
    x_1-x_3 &=0.
\end{align*}
\end{priklad}

\begin{reseni}
Převedeme na matici
$$
\left (
\begin{array}{c c c | c }
1 & 1 & 1 & 1\\
1  & 0 &-1  &0
\end{array}
\right )
$$
a řešíme Gaussovou eliminační metodou.
\end{reseni}

\begin{priklad}
Vypočtete determinant matice
$$A=\begin{pmatrix}
    1  &1 &1 &1\\
    1 &2 &3 &4 \\
    1 &3 &6 &10\\
    1 &4 &10 &20
\end{pmatrix}.$$
\end{priklad}

\begin{reseni}
Od druhého, třetího a čtvrtého řádku odečteme ten první (čímž se determinant nezmění)
a využijeme Laplaceův rozvoj podle prvního sloupce. Je tedy
$$\det A = \begin{vmatrix}
    1 &1 &1 &1\\
    0 &1 &2 &3 \\
    0 &2 &5 &9 \\
    0 &3 &9 &19
\end{vmatrix}=1 \cdot \begin{vmatrix}
1 &2 &3 \\
2 &5 &9\\
3 &9 &19
\end{vmatrix}=1$$
\end{reseni}

\begin{priklad}
V $\mathbb R$ řešte  soustavu rovnic
\begin{align*}
    x_1+x_2+x_3 &=1,\\
    x_1+x_2 &= 0,\\
    x_1+x_3 &=-1.
\end{align*}
\end{priklad}

\begin{reseni}
Řešme Cramerovým pravidlem. Soustavu přepišme jako matici
$$
\begin{pmatrix}
    1 & 1&1\\
    1 &1 &0 \\
    1 &0 &1
\end{pmatrix}.
$$
Platí $x_i=\frac{\det A_i}{\det A},$ kde $A_i$ značí matici, kterou jsme získali
z matice $A$ nahrazením $i$-tého sloupce za vektor \uv{pravé strany} soustavy rovnic.
\end{reseni}

\begin{priklad}
V $\mathbb R$ řešte soustavu rovnic s parametrem $a \in \mathbb R:$
\begin{align*}
    x-y &=2,\\
    ax+y &=4.
\end{align*}
\end{priklad}

\begin{reseni}
Buď Gaussovou eliminací (od druhého řádku odečteme $a$-násobek prvního řádku) nebo
Cramerovým pravidlem (to však funguje jen tehdy, když $\det A \ne 0$, případ $a=0$ tedy
musíme dořešit Gaussovou eliminací).
\end{reseni}

\begin{priklad}
V $\mathbb R$ řešte soustavu rovnic s parametrem $a\in \mathbb R:$
\begin{align*}
    ax_1-ax_2 &=0,\\
    -a^2x_2+x_3 &=a,\\
    ax_1 + x_3 &= a^2.
\end{align*}
\end{priklad}

\begin{reseni}
Buď Gaussovou eliminační metodou nebo Cramerovým pravidlem. Nesmíme zapomenout
rozdělit na případy, kdy by některé výrazy nemusely být definovány.
\end{reseni}

\begin{priklad}
V $\mathbb Z$ řešte $5x-13y=2.$
\end{priklad}

\begin{reseni}
\begin{enumerate}[1.]
\item způsob: Uhodnutí jednoho kořene: $x_0=3, y_0=1.$ Pak další řešení je tvaru
$x=x_0-\frac{b}{d}r, y=y_0+\frac{a}{d}r,$ kde $r \in \mathbb Z$ a $d$ značí
největšího společného dělitele koeficientů u~$x$ a~$y$. Je tedy
\begin{align*}
    x&=3-\frac{-13}{1}r=3+13r, \\
    y &= 1+\frac{5}{1}r = 1+5r,
\end{align*}
kde $r\in \mathbb Z.$
\item způsob: Modifikace Euklidova algoritmu: z rovnice osamostatníme tu neznámou,
jejíž koeficient je v absolutní hodnotě menší.
\begin{align*}
    x &= \frac{13y+2}{5}=\frac{10y}{5}+\frac{3y+2}{5}=2y+\underbrace{\frac{3y+2}{5}}_{\in \mathbb Z}\\
    \exists u \in \mathbb Z: u &= \frac{3y+2}{5}\implies 3y+2=5u \iff y=\frac{5u-2}{3}=u+\underbrace{\frac{2u-2}{2}}_{\in \mathbb Z} \\
     & \dots \\
     \exists w \in \mathbb Z: w &= \frac{v}{2}\implies v = 2w \implies 2 \, | \, v.
\end{align*}
Nyní zpětně dosazujeme:
\begin{align*}
    u&= \frac{3(2w)+2}{2}=\frac{6w+2}{2}=3w+1\\
    y&=\frac{5u-2}{3}=\frac{5 \frac{3w+1}{2}-2}{3}=5w+1 \\
    x &= \frac{13 \frac{5w+1}{1}+2}{5}=13w+3
\end{align*}
\end{enumerate}
\end{reseni}

\begin{priklad}
V $\mathbb R$ řešte rovnici $2x^4+3x^3-16x^2+3x+2=0.$
\end{priklad}

\begin{reseni}
Reciproká rovnice. Vydělíme $x^2$, neboť 0 není kořen a zavedeme substituci
$x+\frac{1}{x}=t$ (ostatní mocniny dopočteme). V rovnici vytkneme opakující se členy
tak, abychom mohli  subsituci použít. Nezapomeneme zpětně dosadit.
\end{reseni}

\begin{priklad}
Přibližně určete reálný kořen polynomu $x^3+x-3$.
\end{priklad}

\begin{reseni}
Postupně dosazujeme nějaké hodnoty, dokud nezjistíme, že mezi nějakými dvěma
musí graf funkce procházet nulou (jedna je záporná a jedna je kladná). Takto postupujeme
dál.
\end{reseni}

\begin{priklad}
V $\mathbb R$ řešte $1/(5^{2x-4})=125.$
\end{priklad}

\begin{reseni}
Převedeme na stejný základ: $5^{4-2x}=5^3$ a přesuneme se do rovnosti exponentů:
$4-2x=3$.
\end{reseni}

\begin{priklad}
V $\mathbb R$ řešte rovnici $4^x+2^x-6=0$.
\end{priklad}

\begin{reseni}
Použijeme substituci $2^x=t.$
\end{reseni}

\begin{priklad}
V $\mathbb R$ řešte $\log (4x+6)=1+\log(2x-1).$
\end{priklad}

\begin{reseni}
Určíme podmínky pro argument (argument je větší než 0). Potom úpravou chceme dostat
rovnost dvou logaritmů, abychom se mohli přesunou do rovnosti argumentů.
\end{reseni}

\begin{priklad}
V $\mathbb R$ řešte rovnici $x^{3+2\log_5 x}=25x^{2+\log_5 x}$.
\end{priklad}

\begin{reseni}
Upravíme a rovnici zlogaritmujeme.
\end{reseni}

\begin{priklad}
V $\mathbb R$ řešte nerovnici $3^{2x+5}\leq 3^{x+2}+2.$
\end{priklad}

\begin{reseni}
Užijeme výhodné substituce $a=3^{x+2}$.
\end{reseni}

\begin{priklad}
V $\mathbb R$ řešte soustavu rovnic
\begin{align*}
    x^{\log y} &= 4,\\
    xy &= 40.
\end{align*}
\end{priklad}

\begin{reseni}
Druhou rovnici zlogaritmujeme a využijeme substituce $a=\log x, b=\log y.$
\end{reseni}

\begin{priklad}
V $\mathbb R$ řešte rovnici $\sin x = \frac{1}{2}.$
\end{priklad}

\begin{reseni}
Z náčrtku jednotkové kružnice zjistíme $x_1 = \frac{\pi}{6}+2k\pi$ a $x_2=\pi-\frac{\pi}{6}+2k\pi.$
\end{reseni}

\begin{priklad}
V $\mathbb R$ řešte rovnici $\tg x = -\sqrt{3}. $
\end{priklad}

\begin{reseni}
Z náčrtku zjistíme $x_1=-\frac{\pi}{3}+k\pi$ a $x_2=\frac{2\pi}{3}+k\pi.$
\end{reseni}

\begin{priklad}
V $\mathbb R$ řešte nerovnici $\sin x \geq \frac{\sqrt{2} }{2}.$
\end{priklad}


\subsection{Důkazy}
\begin{priklad}\label{dkpr}
Dokažte, že je-li $n$ sudé, pak i $n^2$ je sudé.
\end{priklad}

\begin{reseni}
Přímým důkazem. Nechť $n$ je sudé přirozené číslo, tedy lze jej zapsat ve tvaru $2k, k \in \mathbb N.$
Pak $n^2 = 4k^2 = 2\cdot 2l, 2l\in \mathbb N, $ tedy $n^2$ je sudé.
\end{reseni}

\begin{priklad}
Dokažte, že je-li $n^2$ sudé, je i $n$ sudé.
\end{priklad}

\begin{reseni}
Nepřímým důkazem. Obměna: Je-li $n$ liché, je i $n^2$ liché. Obdobně jako v příkladu \ref{dkpr}.
\end{reseni}

\begin{priklad}
Dokažte, že pro všechna $n\in N$ platí: $n$ je sudé právě tehdy, když $n^3$ je sudé.
\end{priklad}

\begin{reseni}
Musíme dokázat oba směry implikace zvlášť.
\end{reseni}

\begin{priklad}
Dokažte, že každým bodem roviny lze vést k dané přímce nejvýše jednu kolmici.
\end{priklad}

\begin{reseni}
Sporem. Předpokládáme platnost negace a dojdeme ke sporu. Musí tedy platit původní výrok.
\end{reseni}

\begin{priklad}
Dokažte, že $\sqrt{2} $ je iracionální.
\end{priklad}

\begin{reseni}
Sporem.
\end{reseni}

\begin{priklad}
Dokažte pro všechna $n\in \mathbb N: $
$$1+2+3+\dots+n=\frac{n(n+1)}{2}.$$
\end{priklad}

\begin{reseni}
Matematickou indukcí.
\begin{enumerate}[1.]
\item $n=1$: $1=1$
\item Předpokládáme platnost pro $n$ a dokážeme, že pak platí i pro $n+1$.
\end{enumerate}
\end{reseni}

\subsection{Kombinatorická geometrie}
\begin{priklad}
Odvoďte vztah pro počet úhlopříček v konvexním $n$-úhelníku.
\end{priklad}

\begin{reseni}
Odvodíme rekurentní vztah. Je dán $n$-úhelník a zjistíme, kolik připude
úhlopříček, přidáme-li jeden vrchol. Přibude $n-3$ úhlopříček a z jedné strany
se stane úhlopříčka. Celkem tedy přibude $n+1-3+1=n-1$ úhlopříček. Nyní odvodíme
explicitní vztah:
\begin{align*}
    u_3 &= 0,\\
    u_4 &= 2=u_3+2,\\
    u_5 &= 5=u_4+3. \\
    u_6&=9=u_5+4,\\
    \dots &= \dots, \\
    u_{n-1}&=u_{n-2}+(n-3),\\
    u_n &= u_{n-1}+(n-2).
\end{align*}
Sečtením všech řádků a úpravou dostaneme $u_n=\frac{n(n-3)}{2}.$
\end{reseni}

\begin{priklad}
V rovině je dáno $n$ přímek tak, že žádné dvě nejsou rovnoběžné a žádné tři neprochází
jedním bodem. Určete, na kolik částí rozdělují
rovinu.
\end{priklad}

\begin{reseni}
Přidáme-li $n+1.$ přímku, protne $n$ přímek v $n$ bodech. Je rozdělena na
$n+1$ částí, takže přibude $n+1$ částí. Opět odvodíme
rekurentní a explicitní vzorec.
\end{reseni}


% dodatky
\appendix
\section{Funkce sinus, kosinus, arkussinus, akouskosinus}\label{appa}
\begin{figure}[h]%
    \centering
    {{\includegraphics[height=4cm]{images/jednotkova_kruznice.png} }}%
    \qquad
    {{\includegraphics[height=4cm]{images/prubeh.png} }}%
\end{figure}

\begin{figure}[h]%
    \includegraphics[width=0.5\linewidth]{arcsine.png}
    \hfill
    \includegraphics[width=0.5\linewidth]{arccosine.png}
\end{figure}

\begin{pozn}[Základní hodnoty goniometrických funkcí]\,\\
    \begin{tabularx}{\textwidth}{| p{0.087\textwidth} || p{0.081\textwidth} | p{0.081\textwidth} |
    p{0.081\textwidth} | p{0.081\textwidth} | p{0.081\textwidth} | p{0.081\textwidth} | p{0.081\textwidth}
    | p{0.081\textwidth} |}
    \hline
    $\beta$ [$^\circ$] & 0 & 30 & 45 & 60 & 90 & 180 & 270 & 360 \\
    \hline
    $\alpha$ [rad] & 0 & $\frac{\pi}{6}$ & $\frac{\pi}{4}$ & $\frac{\pi}{3}$ & $\frac{\pi}{2}$ & $\pi$ & $\frac{3\pi}{2}$ & $2\pi$\\
    \hline
    $\sin \alpha$ & 0 & $\frac{1}{2}$ & $\frac{\sqrt{2}}{2}$ & $\frac{\sqrt{3}}{2}$ & 1 & 0 & $-1$ & 0\\
    \hline
    $\cos \alpha$ & 1 & $\frac{\sqrt{3}}{2}$ & $\frac{\sqrt{2}}{2}$ & $\frac{1}{2}$ & 0 & $-1$ & 0 & 1\\
    \hline
    $\tg \alpha$ & 0 & $\frac{\sqrt{3}}{3}$ & 1 & $\sqrt{3}$ & -- & 0 & -- & 0\\
    \hline
    $\cotg \alpha$ & -- & $\sqrt{3}$ & 1 & $\frac{\sqrt{3}}{3}$ & 0 & -- & 0 & --\\
    \hline
    \end{tabularx}
\end{pozn}

\section{Neurčitý integrál}\label{appint}
\subsection*{Tabulkové integrály}
\begin{align*}
&\int \textcolor{blue}{0} \, dx = \textcolor{blue}{c} & & \int \textcolor{blue}{1}\,  dx = \textcolor{blue}{x}+c & & \int \textcolor{blue}{x^n} \, dx = \textcolor{blue}{\frac{x^{n+1}}{n+1}}+c \\
&\int \textcolor{blue}{\frac{1}{x}} \, dx = \textcolor{blue}{\ln |x|} + c & & \int \textcolor{blue}{e^x}\, dx = \textcolor{blue}{e^x}+c & &  \int \textcolor{blue}{a^x} \, dx = \textcolor{blue}{\frac{a^x}{\ln a}} + c \\
&\int \textcolor{blue}{\sin x} \, dx = \textcolor{blue}{-\cos x} + c & & \int \textcolor{blue}{\cos x} \, dx = \textcolor{blue}{\sin x} +c & &\int \frac{1}{\cos^2 x}\, dx = \tg x + c \\
&\int \textcolor{blue}{\frac{1}{\sin^2 x}} \, dx = \textcolor{blue}{-\cotg x}+c & & \int \textcolor{blue}{\frac{1}{\sqrt{1-x^2} }}\, dx = \textcolor{blue}{\frac{1}{\arcsin x}} + c & &\int \textcolor{blue}{\frac{1}{x^2+1}}\, dx = \textcolor{blue}{\arctg x} + c&
\end{align*}

\subsection*{Integrály řešené jinak než substitucí}
\begin{align*}
    &\textrm{integrál}                        & &                 & & \textrm{řešení} \\
    &\int \frac{A}{x-\alpha}\, dx             & & \longrightarrow & & A\ln|x-\alpha|+c \\
    &\int \frac{1}{(x^2+a^2)^n}\, dx          & & \longrightarrow & & \textrm{per partes vyjádříme } J_1 \textrm{ ($u=J_1, v^\prime=1$)}\\
    &\int \frac{Mx+N}{x^2+px+q}\, dx          & & \longrightarrow & & \textrm{do čitatele dostaneme derivaci jmenovatele, vede na }\ln \\
    &\int \frac{Mx+N}{(x^2+px+q)^n}\, dx      & & \longrightarrow & & \textrm{doplníme na čtverec, vede na }\arctg  \\
    &\int \sin^2 x\, dx, \int \cos^2 x \, dx  & & \longrightarrow & & \textrm{pomocí vzorce pro poloviční argument}
\end{align*}

\subsection*{Integrály řešené substitucí}
\begin{align*}
    &\textrm{integrál}                                & & \textrm{příklad}                                       & &                 & & \textrm{substituce}\\
    &\int \frac{A}{(x-\alpha)^n}\, dx                 & & \int \frac{5}{x-3}\, dx                                & & \longrightarrow & & x-\alpha=t \\
    &\int \frac{Mx+N}{(x^2+\alpha^2)^k}\, dx          & & \int \frac{3x-8}{(x^2+4)^3}\, dx                       & & \longrightarrow & & x^2+\alpha^2=t \\
    &\int R(\sin x, \cos x)\, dx                      & & \int \frac{dx}{1-\cos x+\sin x}                        & & \longrightarrow & & \tg \frac{x}{2}=t \\
    &\int R(\sin x, \cos x)\, dx \textrm{ -- lichá v prom.}\sin x                                               & & \int \frac{\sin^3 x}{\cos^2 x}\, dx                    & & \longrightarrow & & \cos x = t \\
    &\int R(\sin x, \cos x)\, dx \textrm{ -- lichá v prom.}\cos x                                               & & \int \frac{\cos^3 x}{1-\sin^3 x}\, dx                  & & \longrightarrow & & \sin x = t \\
    &\int R(\sin x, \cos x)\, dx \textrm{ -- sudá v obou prom.} & & \int \frac{2\cos x - \sin x}{\cos x - 2\sin x} \, dx   & & \longrightarrow & & \tg x = t \\
    &\int R(x^2,\sqrt[3]{x})\, dx                     & & \int \frac{x^2+\sqrt{x}+1}{x+\sqrt{x}}\, dx            & & \longrightarrow & & x=t^s \\
    &\int R(x,\sqrt[s_1]{x},\dots,\sqrt[s_k]{x})\, dx & & \int \frac{1+x-\sqrt[3]{x}}{x+\sqrt[6]{x^5}}\, dx      & & \longrightarrow & & x=t^{\textrm{nsn}(s_1,\dots,s_k)}\\
    &\int R(x,\sqrt[s]{ax+b})\, dx                    & & \int \frac{\sqrt{x+1}+1}{\sqrt{x+1}-1} \, dx           & & \longrightarrow & & ax+b=t^s \\
    &\int R\left(x,\sqrt[s]{\frac{ax+b}{cx+d}}\right)\, dx       & & \int \frac{1}{x}\cdot \sqrt{\frac{x+1}{x-1}} \, dx     & & \longrightarrow & & \frac{ax+b}{cx+d}=t^s \\
    &\int R(x,\sqrt{k^2-x^2}) \, dx                   & & \int x \sqrt{4-x^2}\, dx                               & & \longrightarrow & & x=k\sin t \\
    &\int R(x,\sqrt{x^2+k^2})\, dx                    & & \int x \sqrt{x^2+4} \, dx                              & & \longrightarrow & & x=\tg t \\
    &\int R(x,\sqrt{x^2-k^2} )\, dx                   & & \int \frac{1}{x}\sqrt{x^2-1}\, dx                      & & \longrightarrow & & x=\frac{k}{\sin t}
\end{align*}

\appendix
\section{Kuželosečky}\label{appb}


\backmatter
%!TEX root = ../main.tex

\clearpage
\phantomsection
\addcontentsline{toc}{section}{About the Author}
\begin{adjustwidth}{0.1\textwidth}{0.1\textwidth}
\begingroup
\null\vspace{0.2\textheight}
\begin{center}
{\bfseries\Large O autorovi}\par\vspace{2em}

Da, da, da, da, da \\
It's the motherfucking D-O-double-D \\
Da, da, da, da, da \\
You know I'm mobbin' with Honza Romanovský (Yeah, yeah, yeah)
\end{center}
\endgroup
\end{adjustwidth}
\clearpage


\end{document}
