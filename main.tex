\documentclass[11pt]{template/cauchy}
\usepackage[czech]{babel}

\usepackage{enumerate}
\usepackage{mlmodern}
\usepackage{template/mathsphystools}
\usepackage{amsthm,thmtools,xcolor}
\usepackage[scr]{rsfso}
% \usepackage{thmstyles}
\usepackage{tabularx}
\usepackage{graphicx}
\graphicspath{{images/}}

\declaretheoremstyle[
  headfont=\color{red}\normalfont\bfseries,
  bodyfont=\color{black}\normalfont,
]{def}

\declaretheoremstyle[
  headfont=\color{blue}\normalfont\bfseries,
  bodyfont=\color{black}\normalfont\itshape,
]{pr}

\declaretheoremstyle[
  headfont=\color{green}\normalfont\bfseries,
  bodyfont=\color{black}\normalfont\itshape,
]{veta}

\declaretheoremstyle[
  headfont=\color{brown}\normalfont\bfseries,
  bodyfont=\color{black}\normalfont,
]{pozn}

\declaretheoremstyle[
  headfont=\color{brown}\normalfont\bfseries,
  bodyfont=\color{black}\normalfont,
]{dusl}

\declaretheorem[
  style=def,
  name=Definice,
]{definition}

\declaretheorem[
  style=pr,
  name=Příklad,
]{example}

\declaretheorem[
  style=veta,
  name=Věta,
]{veta}

\declaretheorem[
  style=pozn,
  name=Poznámka,
]{pozn}

\declaretheorem[
  style=dusl,
  name=Důsledek,
]{dusledek}

\title[Title for the Header]{Matematika}
\subtitle{}
\author{Dominik Doležel \& Honza Romanovský}
\affiliation{Gymnázium Brno, třída Kapitána Jaroše}
\date{\today}
\begin{document}

%!TEX root = ../main.tex
\setcounter{page}{0}
\thispagestyle{fancy-blank}
\begingroup
% \vphantom{Optional note}
{\large \par}
\vspace*{35mm}
{\huge\bfseries\utitle\par}

\vspace*{4mm}
{\rule{\linewidth}{0.5mm}\par}
\vspace*{4mm}

{\large\bfseries\uauthor\par}\vspace*{1mm}

{\large\itshape{a kolektiv}\par}


\vspace{10em}
\begin{figure}[ht!]
\begin{center}
  \begin{tikzpicture}
  \begin{axis}[
      axis lines = middle,
      xlabel = \(x\),
      ylabel = {\(y\)},
      xmin=-1,
      xmax=1,
      ymin=-1,
      ymax=1.6,
      xtick = \empty,
      ytick = \empty,
  ]
  %Below the red parabola is defined
  \addplot [
      domain=-1:1,
      samples=500,
      color=red,
  ]
  {abs(x)^(2/3)+sqrt(1 - x^2)};

  \addplot [
      domain=-1:1,
      samples=500,
      color=red,
      ]
      {abs(x)^(2/3)-sqrt(1 - x^2)};

  \end{axis}
  \end{tikzpicture}
  \caption*{Graf $x^2+ \left [ y- x^\frac{2}{3} \right ]^2 = 1$}
\end{center}
\end{figure}
\vfill
{\large \par}
\endgroup
\clearpage


\frontmatter
\tableofcontents
\clearpage
% \listoffigures
% \listoftables
% \clearpage

% Každá otázka vypadá takto:
% \section*{Název otázky}
% \subsection{Základní pojmy}
% itemize se seznamem základních pojmů
% \section*{Příklady}

\mainmatter

% \input{01_nazev_otazky}

\section{Základní pojmy z teorie množin}
\begin{definition}
  \textbf{Množina} je sourhn objektů, chápaný jako celek. Tyto objekty nazýváme prvky množiny.
\end{definition}

Množina může být konečná, nekonečná nebo prázdná. Množinu lze zadat výčtem prvků nebo pomocí charakteristické vlastnosti (např. $\left \{ 2k, k \in \mathbb{N}\right\}$).

\begin{definition}
  \textbf{Podmnožina} množiny $A$ je taková množina $B$, že všechny její prvky patří do množiny $A$.
\end{definition}

Každá neprázdná množina má dvě \textbf{nevlastní podmnožiny}: množinu prázdnou a sebe sama. Všechny ostatní její podmnožiny nazýváme vlastní.

\begin{definition}
  Množiny $A$ a $B$ se rovnají právě tehdy, když $A$ je podmnožinou $B$ a zároveň $B$ je podmnožinou $A$.
\end{definition}

\begin{definition}
  Nechť $A \subseteq B$ a $B\neq \emptyset$. Množinu všech prvků množiny $B$, které nepatří do množiny $A$, nazýváme \textbf{doplněk} (komplement) množiny $A$ v množině $B$. Značíme $A_B^\prime.$
\end{definition}

\begin{definition}
  Nechť $A, B$ jsou dvě množiny. Jejich \textbf{sjednocením} nazveme takovou množinu, která obsahuje ty prkvy, které patří alespoň do jedné z množin $A, B$. Zapisujeme $A \cup B$.
\end{definition}

\begin{definition}
  Nechť $A, B$ jsou dvě množiny. Jejich \textbf{průnikem} nazveme takovou množinu, která obsahuje ty prvky, které patří zároveň do obou těchto množin $A, B$. Zapisujeme $A \cap B$.
\end{definition}

\begin{definition}
  \textbf{Vennův diagram} je grafické schematické znázornění všech možných vztahů (sjednocení, průnik, rozdíl, doplněk) několika podmnožin univerzální množiny, jež znázorňujeme pomocí uzavřených čar.
\end{definition}

\begin{definition}
  Dvě množiny jsou \textbf{disjunktní}, pokud nemají žádný společný prvek, tedy pokud je jejich průnikem prázdná množina.
\end{definition}

\begin{definition}
  Nechť $A, B$ jsou dvě množiny. \textbf{Rozdíl} množin je množina, která obsahuje všechny prvky množiny $A$ s výjimkou těch, jež jsou zároveň prvky množiny $B$. Zapisujeme $A - B$.
\end{definition}

\begin{veta}
  \textbf{De Morganovy zákony} jsou zákony určující vztahy mezi sjednocením, průnikem a doplňkem množiny. Nechť $A, B$ jsou dvě množiny, $^\prime$ doplněk množiny. Potom platí:
  $$ (A \cup B)^\prime = A^\prime \cap B^\prime$$
  $$ (A \cap B)^\prime = A^\prime \cup B^\prime$$
\end{veta}

\begin{proof}
  Důkaz prvního vztahu:

  \begin{minipage}{0.5\textwidth}
    \centering
        \includegraphics[width=0.5\linewidth]{vennsjed.png}
        \includegraphics[width=0.5\linewidth]{venndoplsjed.png}
  \end{minipage}
  \hfill
  \noindent\begin{minipage}{0.5\textwidth}
  \centering
        \includegraphics[width=0.5\linewidth]{venndoplAaB.png}
        \includegraphics[width=0.5\linewidth]{venndoplsjed.png}
  \end{minipage}

  Důkaz druhého analogicky.
\end{proof}

\begin{pozn}[Číselné množiny]
  Rozlišujeme následující základní číselné množiny:
  \begin{itemize}
    \item $\mathbb{N}$: přirozená čísla $(1, 2, 3, \dots)$,
    \item $\mathbb{Z}$: celá čísla $(\dots, -2, -1, 0, 1, 2, \dots)$,
    \item $\mathbb{Q}$: racionální čísla $(3/5, 0,\overline{3})$,
    \item $\mathbb{R}$: reálná čísla $(e, \pi)$,
    \item $\mathbb{C}$: komplexní čísla $(3+2i)$
  \end{itemize}
\end{pozn}

Iracionální čísla ($\mathbb{I}$) jsou doplněk racionálních v $\mathbb{R}$.

\begin{definition}
  \textbf{Celá čísla} jsou čísla, která vyjadřují počty prvků množin, čísla k nim opačná a číslo 0.
\end{definition}

\begin{definition}
  \textbf{Racionálním číslem} nazveme takové číslo $a = \frac{k}{l}, k, l \in \mathbb{Z}$, a $p,q$ jsou nesoudělná.
\end{definition}

\begin{pozn}
  Přirozená čísla zapisujeme pomocí číslic 0--9 a chápeme je takto:
  $$4503=4\cdot 10^3+5\cdot 10^2 + 0 \cdot 10^1 + 3\cdot 10^0.$$
  Každé racionální číslo je v desítkové soutavě vyjádřeno buď ukončeným desetinným rozvojem nebo neukončeným periodickým rozvojem. Iracionální číslo je vyjádřeno neukončeným neperiodickým rozvojem.
\end{pozn}

\begin{definition}
  \textbf{Reálnými čísly} nazýváme všechna čísla, která jsou velikostmi úseček.
\end{definition}

\begin{definition}
  Nechť $a,b \in \mathbb{R},$ kde $a<b$. Pak množiny takových $x\in \mathbb{R},$ že $a\leq x\leq b$ (resp. $a < x < b$, resp. $a < x$, resp. $a \leq x < b$ atd.) nazýváme uzavřeným (resp. otevřeným, resp. neomezeným zleva otevřeným, resp.zprava uzavřeným, zleva otevřeným atd.) \textbf{intervalem}. Zapisujeme $\left<a,b\right>$ (resp. $\left(a,b\right)$, resp. $(a, \infty)$, resp. $\left<a, b\right)$)
\end{definition}

\begin{definition}
  \textbf{Periodický rozvoj čísla} je rozvoj, u kterého se za desetinnou čárkou donekonečna opakuje táž číslice nebo skupina číslic. Čísla s takovýmto rozvojem se nazývají ryze periodická čísla, opakující se číslice nebo skupina opakujících se číslic se nazývá perioda. Zapisují se tak, že se nad opakující se skupinou napíše pruh:
  $$0,333 … = 0,\overline{3}$$
\end{definition}

\begin{definition}
  Množina komplexních čísel $\mathbb C$ je množina uspořádaných reálných dvojic $[x, y]$, na kterých je definována rovnost, sčítání a násobení následovně:
$$[a, b] = [c, d] \Leftrightarrow a = c \land b = d,$$
$$[a, b] + [c, d] = [a + c, b + d],$$
$$[a, b][c, d] = [ac - bd, ad + bc].$$
  Dvojici $[0, 1]$ označíme $i$ a budeme ji nazývat komplexní jednotkou. Zřejmě pak platí, že $i^2 = -1$.
\end{definition}

\begin{definition}
  Pokud je uspořádaná dvojice z předchozí definice ve tvaru $[0, b], b \in \R$, nazveme toto číslo \textbf{ryze imaginárním}.
\end{definition}

\begin{example}[SÚM 169/8]
  Označme $M$ množinu všech dvojciferných přirozených čísel delitelných šesti a $N$ všechn dělitelů čísla 210, kteří jsou různí od čísla 1 a 210. Určete, která z množin má větší počet prvků, a vypište všechny prvky, které mají obě množiny stejné.
  \begin{align*}
    M & = \left\{12, 18, 24, 30, 36, 42, 48, 54, 60, 66, 72, 78, 84, 90, 96\right\}\\
    210 & = 2\cdot 3 \cdot 5  \cdot 7 \textrm{ -- hledáme násobky všech podmnožin těchto čísel} \\
    N  & = \left\{2,3,5,6,7, 10, 14, 15, 21, 30, 35, 42, 70, 105\right\} \\
    |M| & = 15, |N| = 14, M \cap N = \left\{30, 42\right\}
  \end{align*}

  \rm Množina $M$ má více prvků a společná jsou čísla 30 a 42.
\end{example}

\begin{example}[SÚM 171/26]
  $M$ je množina šech reálných čísel $x$, která splňují nerovnosti $-2<x<5$, $N$ je mn. všech reálných čísel $y$, která splňují nerovnost $|y|<4$. Určete množinu $R=M\cup N$ a $S = M\cap N.$ \hfill $R = (-4,5), S=(-2,4).$
\end{example}

\begin{example}[SÚM 172/29f]
  Znázorněte a určete výsledný interval: $(a,a+2)\cap (a-1,a+1),$ kde $a>0.$\hfill$(a,a+1)$
\end{example}

\begin{example}[SÚM (172/33)]
  Je dána kružnice $k$ se středem v bodě  $S$ a poloměrem $r$. Množinu všech bodů uvnitř kružnice označte $A$. Nakreslete rovnostranný trojúhelník $ESD$, jehož jeden vrchol je ve středu dané kružnice a délky stran jsou rovny velikosti jejího průměru. Množinu vnitřních bodů tohoto trojúhelníka ozn. $B$. Díle sestrojte osu úhlu $ESD$ a množinu bodů této přímky označte $C$. Nakreslete samostatné obrázky pro:
  \begin{itemize}
    \item $(A\cap B)\cup C,$
    \item $(A\cup C) \cap (B\cup C),$
    \item $(A\cap B) \cup (B\cap C),$
    \item $(A\cup C) \cap B$.
  \end{itemize}
\end{example}

\begin{example}[SÚM 173/34]
  Pro která $x$ je interval:
  \begin{enumerate}[a.]
    \item $\left<2x,x+3\right>$ částí intervalu $(2,7)$? \hfill $x \in (1,3)$
    \item $(x,5)$ částí intervalu $\left(-1,x+1\right)$? \hfill $x\in (4,5)$
    \item $(x,x+3)$ částí intervalu $\left<5,8\right>$? \hfill $x=5$
    \item $\left<x,2x-1\right>$ částí intervalu $\left<-2,5\right>$? \hfill $x\in\left<-2,5\right>$
    \item $\left<3x,2x+1\right>$ částí intervalu $(3,6)$? \hfill $x\in \left\{\right\}$
  \end{enumerate}
\end{example}

\begin{example}[SÚM 173/35]
  Nechť $M = (a,b), N = (1,8), Q = (1,5)$. Určete $a,b \in \mathbb{R}$ tak, aby platilo $M\cap N = Q$.\hfill $a\in \left(-\infty, 1\right>, b=5$
\end{example}

\begin{example}[SÚM 173/37*]
  Je dán trojúhelník $ABC$. Uvažujme množinu $M$ všech bodů tohoto trojúhelníka, pro které platí $|AX| \geq |BX| \geq |CX|.$ Pomocí velikosti stran a úhlů troj. $ABC$ vyjádřete podmínky pro to, aby:
  \begin{enumerate}[a.]
    \item $X$ byla pětiúhelník, \hfill $\gamma > 90^\circ, \alpha < \beta$
    \item $X$  je jeden bod, \hfill $\alpha = 90^\circ$
    \item $X$ je prázdná.\hfill $\alpha > 90^\circ$
  \end{enumerate}
\end{example}


\begin{example}[SÚM 174/42]
  Jsou dány množiny $M=\left\{1,2; 3; 4\right\},N=\left\{x;y;z\right\}.$ Uveďte alespoň jeden příklad na zobrazení množiny
  \begin{enumerate}[a.]
    \item $M$ do $N$\hfill $1,2\rightarrow x; 3\rightarrow y; 4 \rightarrow y$
    \item $N$ do $M$ \hfill $x\rightarrow 1,2; y\rightarrow 3; z\rightarrow 3$
    \item $M$ na $N.$ \hfill $1,2\rightarrow x; 3 \rightarrow y; 4 \rightarrow z$
  \end{enumerate}
\end{example}

\begin{example}[SÚM 174/46]
  Kolik je všech zobrazení (pod)množiny $\left\{a,b,c,d\right\}$ do (na) množiny $\left\{1,2\right\}$?\hfill \rm 81
\end{example}

\begin{example}[SÚM 106/20]
  Převeďte na obyčejné zlomky:
  \begin{enumerate}[a.]
    \item $0,\overline{27}$\hfill $\frac{27}{99}\frac{3}{11}$
    \item $0,\overline{6}$ \hfill $\frac{2}{3}$
    \item $2,\overline{345}$ \hfill $2+\frac{345}{999}=\frac{781}{333}$
    \item $0,\overline{1234}$\hfill $\frac{1234}{9999}$
    \item $0,7\overline{2}$\hfill $\frac{7}{10}+\frac{2}{90}=\frac{13}{18}$
    \item $0,1\overline{36}$\hfill $\frac{1}{10}+\frac{36}{990}=\frac{3}{22}$
    \item $0,7\overline{27}$\hfill $\frac{7}{10}+\frac{27}{990}=\frac{8}{11}$
    \item $3,39\overline{85}$\hfill $3+\frac{39}{100}+\frac{85}{9900}=\frac{33646}{9900}$
  \end{enumerate}
\end{example}

\begin{example}[SÚM 107/21]
  Proveďte:
  \begin{enumerate}[a.]
    \item $0,\overline{4}+0,\overline{12}$ \hfill $\frac{4}{9}+\frac{12}{9}=\frac{16}{9}$
    \item $0,\overline{7}+0,\overline{35}$  \hfill $\frac{112}{99}$
    \item $0,\overline{47}+0,\overline{023}$ \hfill $\frac{5470}{10989}$
    \item $0,\overline{47}+0,0\overline{23}$ \hfill $\frac{493}{990}$
    \item $0,5\overline{354}+0,\overline{85}$\hfill $1,394021\dots$
    \item $2,\overline{35}-1,\overline{231}$\hfill$ \frac{4111}{3663}$
    \item $1,\overline{25}-0,\overline{773}$ \hfill $\frac{5261}{10989}$
  \end{enumerate}
\end{example}

\begin{example}[SÚM 107/22*]
  Proveďte:
  \begin{enumerate}[a.]
    \item $1,\overline{2}\cdot 1,\overline{18}$\hfill $\left(1+\frac{2}{9}\right)\left(1+\frac{18}{99}\right)=\frac{11}{9}\cdot \frac{117}{99}=\frac{13}{9}$
    \item $0,\overline{32}\cdot 1,\overline{3}$\hfill $\frac{128}{297}$
  \end{enumerate}
\end{example}

\begin{example}[SÚM 107/23*]
  Řešte rovnici:
  \begin{enumerate}
    \item $0,\overline{25}x + 0,\overline{31}x = 1,\overline{13}$ \hfill $x=2$
    \item $2,\overline{64}x - 3,\overline{48} = 1,\overline{48}x$  \hfill $x = 3$
  \end{enumerate}
\end{example}

\begin{example}[SMP 140/6abc]
  Pomocí Vennových diagramů zjednodušte zápisy množin: \\
  \begin{minipage}{0.5\textwidth}
    \begin{enumerate}[a.]
      \item $(A \cap B \cap C) \cup [B \cap (A^\prime \cup C)^\prime]$
      \item $[(A \cup B)^\prime \cup (B \cup C)] \cap (C \cup A)$
      \item $[(A \cup B^\prime) \cap C] \cup [(B^\prime \cup A^\prime)^\prime \cap C]$
    \end{enumerate}
  \end{minipage}
  \hfill
  \noindent\begin{minipage}{0.5\textwidth}
      \includegraphics[width=\linewidth]{vennovy}
      \captionof{figure}{}
  \end{minipage}
\end{example}

\begin{example}[SÚM 109/36]
  Dokažte, že číslo $\sqrt{5}$ je iracionální.

  Dk. sporem: Nechť $\sqrt{5} \in \Q \Rightarrow \sqrt{5} = \frac{a}{b}, a, b \in \Z, D(a, b) = 1$.
  $$\sqrt{5} = \frac{a}{b}$$
  $$5 = \frac{a^2}{b^2}$$
  $$5b^2 = a^2 \Rightarrow 5 \mid a^2 \Rightarrow 5 \mid a \Rightarrow \exists k: a = 5k$$
  $$5b^2 = (5k)^2$$
  $$5b^2 = 25k^2$$
  $$b^2 = 5k^2 \Rightarrow 5 \mid b^2 \Rightarrow 5 \mid b$$ -- spor s předpokladem, že $D(a, b) = 1$
  $$\sqrt{5} \in \mathbb{I}$$
\end{example}

\begin{example}[SÚM 109/37]
  Dokažte, že číslo $\sqrt{2} - 1$ je iracionální.

  Dk. sporem: Nechť $(\sqrt{2} - 1) \in \Q \Rightarrow \sqrt{2} - 1 = \frac{a}{b}, a, b \in \Z, D(a, b) = 1$.
  $$\sqrt{2} - 1 = \frac{a}{b}$$
  $$1 - 2\sqrt{2} = \frac{a^2}{b^2}$$
  $$\frac{b^2-a^2}{2b^2} = \sqrt{2} \Rightarrow \sqrt{2} = \frac{p}{q}, p,q \in \Z \Rightarrow \sqrt(2) \in \Q$$ -- spor
  $$(\sqrt{2} - 1) \in \mathbb{I}$$
\end{example}

\begin{example}[SÚM 109/38]
  Dokažte, že číslo $2\sqrt{5}$ je iracionální.

  Dk. sporem: Nechť $2\sqrt{5} \in \Q \Rightarrow 2\sqrt{5} = \frac{a}{b}, a, b \in \Z, D(a, b) = 1$.
  $$2\sqrt{5} = \frac{a}{b}$$
  $$10 = \frac{a^2}{b^2}$$
  $$10b^2 = a^2 \Rightarrow 10 \mid a^2 \Rightarrow 10 \mid a \Rightarrow \exists k: a = 10k$$
  $$10b^2 = (10k)^2$$
  $$10b^2 = 100k^2$$
  $$b^2 = 10k^2 \Rightarrow 10 \mid b^2 \Rightarrow 10 \mid b$$ -- spor s předpokladem, že $D(a, b) = 1$
  $$2\sqrt{5} \in \mathbb{I}$$
\end{example}

\begin{example}[SÚM 109/39]
  Dokažte, že jestliže přirozené číslo $m$ není druhou mocninou žádného přirozeného čísla, potom $\sqrt{m}$ je číslo iracionální.

  Dk. sporem: Nechť $m \in \N, m != n^2 \forall n \in \N, \sqrt{m} \in \Q \Rightarrow \sqrt{m} = \frac{a}{b}, a, b \in \N, D(a, b) = 1$.
  $$\sqrt{m} = \frac{a}{b}$$
  $$m = \frac{a^2}{b^2}, m \in \N \Rightarrow b^2 = 1$$
  $$m = a^2, a \in \N$$ -- spor s předpokladem, že $m != n^2 \forall n \in \N$
  QED
\end{example}

\begin{example}[SÚM 144/301]
  Dokažte, že:
  \begin{enumerate}[a.]
    \item součet dvou dvojciferných čísel přirozených, která se liší jen pořadím cifer, je dělitelný jedenácti: $$S = \overline{ab} + \overline{ba} = 10a + b + 10b + a = 11(a + b) \Rightarrow 11 \mid S$$
    \item rozdíl dvou dvojciferných čísel přirozených, která se liší jen pořadím cifer, je dělitelný devítí: $$S = \overline{ab} - \overline{ba} = 10a + b - 10b - a = 9(a - b) \Rightarrow 9 \mid S$$
    \item rozdíl přirozeného čísla trojciferného  a čísla, které vznikne z tohoto záměnou krajních cifer, je dělitelný 99: $$S = \overline{abc} - \overline{cba} = 100a + 10b + c - 100c - 10b - a = 99(a - c) \Rightarrow 99 \mid S$$
  \end{enumerate}
\end{example}

\begin{example}[SÚM 145/303]
  Dokažte, že tři mocniny čísla 2, jejichž exponenty jsou tři po sobě jdoucí přirozená čísla, mají součet dělitelný sedmi: $$S = 2^a + 2^{a+1} + 2^{a+2} = 2^a + 2^a*2^1 + 2^a*2^2 = 7*2^a \Rightarrow 7 \mid S$$
\end{example}

\begin{example}[SÚM 145/305]
  Dokažte, že součet třetích mocnin tří po sobě jdoucích přirozených čísel je dělitelný třemi: $$S = a^3 + (a+1)^3 + (a+2)^3 = a^3 + a^3 + 3a^2 + 3a + 1 + a^3 + 6a^2 + 12a + 8 = 3(a^3 + 3a^2 + 5a + 3) \Rightarrow 3 \mid S $$
\end{example}

\begin{example}[SÚM 145/306]
  Dokažte, že:
  \begin{enumerate}[a]
    \item číslo utvořené z rozdílu třetí mociny přirozeného čísla $n$ a tohoto čísla je dělitelné šesti: $$S = n^3 - n = n(n^2-1) = (n-1)n(n+1)$$ Jsou to tři po sobě jdoucí čísla $\Rightarrow$ právě 1 z nich je dělitelné třemi $\Rightarrow 3 \mid S$
    \item je-li číslo $n$ liché, je uvažovaný rozdíl dělitelný čísel 24: $$S = (n-1)n(n+1)$$ Jsou to tři po sobě jdoucí čísla a to prostřední je liché $\Rightarrow$ dělitelné třemi, z dalších čísel je jedno dělitelné 2 a jedno dělitelné čtyřmi: 2*4*3 = 24 $\Rightarrow 24 \mid S$
  \end{enumerate}
\end{example}

\begin{example}[SÚM 145/307]
  Dokažte, že je—li přirozené číslo $x$ liché, je výraz $V = x^3 + 3x^2 — x — 3$ dělitelný číslem 48: $$V = x^3 + 3x^2 — x — 3 = x^2(x+3) -(x+3) = (x^2 - 1)(x+3) = (x-1)(x+1)(x+3)$$ $\Rightarrow$ tři po sobě jdoucí sudá čísla $\Rightarrow$ jedno dělitelné 2, jedno 4 a jedno 6 $\Rightarrow 2*4*6 = 48 \Rightarrow 48 \mid V$
\end{example}

\begin{example}[SÚM 145/308]
  Dokažte, že výraz $V = 5x^3 + 15x^2 + 10x$ je dělitelný číslem 30 prokaždé přirozené číslo $x$: $$V = 5x^3 + 15x^2 + 10x = 5x(x^2 + 3x + 2) = 5x(x+2)(x+1)$$ $\Rightarrow$ $x$, $x+1$, $x+2$ tři po sobě jdoucí čísla $\Rightarrow$ jedno dělitelné 3, alespoň jedno dělitelné 2, $5x$ dělitelné 5 $\Rightarrow 2*3*5=30 \Rightarrow 30 \mid V$
\end{example}

\begin{example}[SÚM 145/312]
  Dokažte, že je-li $n$ číslo přirozené, je číslo $N = n^3 + 11n$ dělitelné šesti:
  mod 6:
  \begin{enumerate}
    \item $n = 6k$: $n^3 + 11n \equiv 0^3 + 11*0 \equiv 0 \Rightarrow 6 \mid N$
    \item $n = 6k + 1$: $n^3 + 11n \equiv 1^3 + 11*1 \equiv 12 \equiv 0 \Rightarrow 6 \mid N$
    \item $n = 6k + 2$: $n^3 + 11n \equiv 2^3 + 11*2 \equiv 30 \equiv 0 \Rightarrow 6 \mid N$
    \item $n = 6k + 3$: $n^3 + 11n \equiv 3^3 + 11*3 \equiv 60 \equiv 0 \Rightarrow 6 \mid N$
    \item $n = 6k + 4$: $n^3 + 11n \equiv 4^3 + 11*4 \equiv 108 \equiv 0 \Rightarrow 6 \mid N$
    \item $n = 6k + 5$: $n^3 + 11n \equiv 5^3 + 11*5 \equiv 180 \equiv 0 \Rightarrow 6 \mid N$
  \end{enumerate}
  $\Rightarrow 6 \mid N \forall n \in \N$
\end{example}

\begin{example}[SÚM 145/315]

\end{example}

\section{Výroková logika}
\begin{definition}
  \textbf{Výrokem} nazýváme každou oznamvací větu, která je buď pravdivá, nebo nepravdivá. \textbf{Pravdivostní hodnotou} výroku rozumíme jeho pravdivost / nepravdivost.
\end{definition}

\begin{definition}
  \textbf{Negací výroku} $V$ nazýváme výrok $V^\prime$, který má opačnou pravdivostní hodnotu než výrok $V$.
\end{definition}

\begin{pozn}
  \textbf{Kvantifikované výroky} jsou výroky, které uvádějí počet objektů. Pro to lze použít
  \begin{itemize}
    \item obecný kvantifikátor $\forall$ (pro všechno platí),
    \item existenční kvantifikátor $\exists$ (existuje alespoň jeden, že pro něj platí) a
    \item zesílený existenční kvantifikátor $\exists !$ (existuje právě jeden, že pro něj platí) a
  \end{itemize}
\end{pozn}

\begin{definition}
  \textbf{Složeným výrokem} rozumíme více výroků spojených logickými spojkami:
  \begin{center}
    \begin{tabular}{l | c c}
      název & zápis & význam \\
      \hline
      negace & $X^\prime$ & není pravda, že \\
      konjunkce & $X\land Y$ & $X$ a $Y$ platí současně \\
      alternativa & $X\lor Y$ & platí alespoň jedno z $X,Y$\\
      implikace & $X\implies Y$ & jestliže $X$, pak $Y$\\
      ekvivalence & $X\iff Y$ & $X$ platí právě tehdy, když platí $Y$
    \end{tabular}
  \end{center}
\end{definition}


\begin{example}[SMP 143/4]
  Jsou následující výroky tautologie?
  \rm
  \begin{enumerate}[a.]
    \item $\left[(A\implies B)\land A\right]\implies B$
    \begin{center}
      \begin{tabular}{c c | c c c}
        $A$ & $B$ & $A \implies B$ & $(A\implies B)\land A$ & $(A\implies B)\land A]\implies B$ \\
        \hline
        1 & 1 & 1 & 1 & 1 \\
        1 & 0 & 0 & 0 & 1 \\
        0 & 1 & 1 & 0 & 1 \\
        0 & 0 & 1 & 0 & 1
      \end{tabular}
    \end{center}
    Výrok je tautologií.
    \item $\left[(A\implies B)\land B^\prime\right]\implies A^\prime$
    \begin{center}
      \begin{tabular}{c c c c | c c c}
        $A$ & $B$ & $A^\prime$ & $B^\prime$ &  $A \implies B$ & $(A\implies B)\land B^\prime$ & $\left[(A\implies B)\land B^\prime\right]\implies A^\prime$ \\
        \hline
        1 & 1 & 0 & 0 & 1 & 0 & 1 \\
        1 & 0 & 0 & 1 & 0 & 0 & 1 \\
        0 & 1 & 1 & 0 & 1 & 0 & 1 \\
        0 & 0 & 1 & 1 & 1 & 1 & 1
      \end{tabular}
    \end{center}
    Výrok je tautologií.
  \end{enumerate}
\end{example}

\begin{example}[SMP 144/8]
  Na modelu kolejiště je možno uvést do pohybu tři vlakové soupravy A,B,C. V daném okamžiku je jejich
situace charakterizována formulí $$\left[(A^\prime \lor B^\prime) \implies C\right]\land\left[(A \lor C) \implies B^\prime\right].$$ Které soupravy jsou v pohybu?

\rm Napišme tabulku pravdivostních hodnot.
\begin{center}
  \begin{tabular}{c c c c c | c c c c c}
    $A$ & $B$ & $C$ & $A^\prime$ & $B^\prime$ & $A^\prime \lor B^\prime$ & $(A^\prime \lor B^\prime) \implies C$ & $A\lor C$ & $(A \lor C) \implies B^\prime $ & celkem\\
    \hline
    1 & 1 & 1 & 0 & 0 & 0 & 1 & 1 & 0 & 0 \\
    1 & 1 & 0 & 0 & 0 & 0 & 1 & 1 & 0 & 0 \\
    1 & 1 & 0 & 0 & 0 & 1 & 1 & 1 & 1 & 1 \\
    1 & 0 & 1 & 0 & 1 & 1 & 0 & 1 & 1 & 0 \\
    0 & 1 & 1 & 1 & 0 & 1 & 1 & 1 & 0 & 0 \\
    0 & 1 & 0 & 1 & 0 & 1 & 0 & 0 & 1 & 0 \\
    0 & 0 & 1 & 1 & 1 & 1 & 1 & 1 & 1 & 1 \\
    0 & 0 & 0 & 1 & 1 & 1 & 0 & 0 & 0 & 0
  \end{tabular}
\end{center}
Buď jsou v provozu soupravy $A$ a $C$ nebo jen souprava $C$.
\end{example}

\begin{example}[SMP 144/10]
  Květa si pozvala na oslavu svých osmnáctin přátele. Uvažuje takto:
  \begin{enumerate}[a.]
    \item Alena a Boris chodí vždycky spolu. Přijdou oba, nebo ani jeden.
    \item Přijde Boris nebo Dan, ale určitě ne oba.
    \item Když přijde Alena, pak přijde i Eva.
    \item Když Eva nepřijde, nepřijde Dan.
  \end{enumerate}
  S jakým největším počtem přátel může Květa počítat? Vyplývá z Květiny úvahy, že může nastat situace, kdy nepřijde ani jeden z pozvaných?

  \rm Přeložme tato tvrzení symbolicky.
  \begin{enumerate}[a.]
    \item $A\iff B$
    \item $B^\prime \lor D^\prime$
    \item $A\implies E$
    \item $E^\prime \implies D^\prime$
  \end{enumerate}
  Protože alespoň jeden z dvojice Boris, Dan nemůže přijít a zbytek podmínek si neodporují, na oslavu můžou přijít nejvýše tři lidé.

  Situace, že nepřijde nikdo nastat může.
\end{example}

\begin{example}[SMP 144/12]
  Trenér se věnuje trojici gymnastů -- Adamovi, Břéťovi a Čeňkovi. Rozhodněte, koho vyšle na kontrolní
závod, jestliže splní tyto tři podmínky:
\begin{itemize}
  \item Tělovýchovnou jednotu budou reprezentovat nejvýše dva závodníci, přitom pojede aspoň jeden.
  \item Pojede Adam nebo Čeňek, ale určitě ne oba součastně.
  \item Nepojede-li Čeňek, pak nepojede ani Břéťa.
\end{itemize}

\rm Rozdělme příklad na dva případy.
\begin{enumerate}[$i.$]
  \item pojede Adam: pak nepojede Čeněk a tedy ani Břéťa,
  \item pojede Čeněk: pak může jet i Břéťa.
\end{enumerate}

Buď pojede Adam sám nebo Čeněk sám nebo Čeněk s Břéťou.

\end{example}

\section{Dělitelnost přirozených čísel}
\begin{definition}
  Nechť $a,b\in\mathbb Z.$ Číslo $a$ dělí číslo $b$, jestliže $\exists c \in \mathbb Z: b=ac$. Zapisujeme $a\, | \, b$.
\end{definition}

\begin{definition}
  Nechť $a\in \mathbb R$. Číslo $|a|$ takové, že
  \begin{enumerate}[$i.$]
    \item $a\geq 0 \implies |a| = a$,
    \item $a<0 \implies |a| = - a$
  \end{enumerate}
  nazýváme \textbf{absolutní hodnotou} čísla $a$.
\end{definition}

\begin{veta}[O dělení se zbytkem]
  Nechť $a\in \mathbb Z, b\in \mathbb N.$ Pak $\exists ! q \in \mathbb Z, r\in \mathbb N_0:$
  $$a=bq+r, 0 \leq r < b.$$
\end{veta}

\begin{definition}
  Nechť $a,b\in \mathbb N$. Pak $c$ je \textbf{společným dělitelem} čísel $a,b$, jestliže $c \, | \, a \land c\, | \, b.$
\end{definition}

\begin{definition}
  $d\in \mathbb N$ je \textbf{největší společný dělitel} čísel $a,b \in \mathbb N,$ jestliže jsou splněny zároveň obě podmínky:
  \begin{enumerate}[$i.$]
    \item $d\, | \, a \land d \, | \, b$ a
    \item $\forall c \in \mathbb N: c \, | \, a \land c \, | \, b \implies c \, | \, d.$
  \end{enumerate}
  Takové číslo značíme $d=D(a,b)=(a,b).$
\end{definition}

\begin{definition}
  -- INSERT EUKLIDŮV ALGORITMUS --
\end{definition}

\begin{definition}
  Nechť $a,b\in \mathbb N.$ Tato čísla jsou \textbf{nesoudělná}, jestliže $D(a,b)=1$. V opačném případě jsou \textbf{soudělná}.
\end{definition}

\begin{veta}[Fundamentální věta aritmetiky]
  Nechť $a_1,a_2,b\in \mathbb N, b>1.$ Pak $b \, | \, a_1a_2 \land D(a_1,b)=1\implies b\, | \, a_2.$
\end{veta}

\begin{definition}
  Nechť $a,b\in \mathbb N.$ Pak $c$ je \textbf{společným násobek} čísel $a,b$, jestliže $a \, | \, c \land b\, | \, c.$
\end{definition}

\begin{definition}
  $n\in \mathbb N$ je \textbf{nejmenší společný násobek} čísel $a,b \in \mathbb N,$ jestliže jsou splněny zároveň obě podmínky:
  \begin{enumerate}[$i.$]
    \item $a\, | \, n \land b \, | \, n$ a
    \item $\forall m \in \mathbb N: a \, | \, m \land b \, | \, m \implies m \, | \, n.$
  \end{enumerate}
  Takové číslo značíme $n=n(a,b)=\left [ a,b\right ] .$
\end{definition}

\begin{veta}
  $\forall a,b \in \mathbb N: ab=D(a,b)\cdot n(a,b).$
\end{veta}

\begin{definition}
  Nechť $n\in \mathbb N, n>1.$ Má-li číslo $n$ pouze triviální dělitele ($1 \, | \, n, n \, | \, n$), nazýváme jej \textbf{prvočíslem}. V opačném případě hovoříme o \textbf{čísle složeném}.
\end{definition}

\begin{veta}
  Každé přirozené složené číslo $n$ má alespoň jednoho prvočíselného dělitele $p\leq \sqrt{n}$.
\end{veta}

\begin{veta}
  Prvočísel je nekonečně mnoho.
\end{veta}

\begin{veta}[Základní věta aritmetiky]
  Každé přirozené číslo $n>1$ lze zapsat ve tvaru:
  $$n=p_1^{m_1}\cdot p_2^{m_2} \cdot p_3^{m_3}\cdot \hdots \cdot p_r^{m_r},$$
  kde $p_i,i\in\{ 1, 2, \dots, r \}$ jsou navzájem různá prvočísla, $m_i\in \mathbb N_0$. Toto vyjádření je jednoznačné až na pořadí činitelů a říkáme mu \textbf{rozklad čísla} $n$ \textbf{na součin prvočinitelů}.
\end{veta}

\begin{veta}[Věta o iraciálnosti odmocnin]
  Nechť $n\in \mathbb N.$ Pak platí: Pokud $n$ není druhou mocninou přirozeného čísla, pak odmocnina z $n$ je iracionální.
\end{veta}

\subsection*{Kritéria dělitelnosti}
\begin{věta}
  Nechť $n\in \mathbb N, n=a_k\cdot 10^k+a_{k-1}\cdot 10^{k-1}+\dots + a\cdot 10 + a_0.$ Pak platí:
  \begin{enuemrate}[$i.$]
    \item $2 \, | \, n \iff 2 \, | \, a_0$,
    \item $4 \, | \, n \iff 4 \, | \, (10a_1 + a_0)$,
    \item $5 \, | \, n \iff 5 \, | \, a_0$,
    \item $8 \, | \, n \iff 8 \, | \, (10^2a_2 + 10a_1 + a_0)$ a
    \item $10 \, | \, n \iff a_0 = 0$.
  \end{enuemrate}
\end{věta}

\begin{proof}
  $$n = 10(a_k\cdot 10^{k-1}+a_{k-1}\cdot 10 ^{k-2}+\dots+a_1)+a_0 = 10l+a_0, l\in \mathbb N$$
\end{proof}

\begin{definition}
  Nechť $n\in \mathbb N, n=a_k\cdot 10^k+a_{k-1}\cdot 10^{k-1}+\dots + a\cdot 10 + a_0.$ Pak číslo
  $$S(n) = \sum_{i=0}^k a_i$$
  nazveme \textbf{ciferným součtem} čísla $n$.
\end{definition}

\begin{veta}
  Nechť $n\in \mathbb N, n=a_k\cdot 10^k+a_{k-1}\cdot 10^{k-1}+\dots + a\cdot 10 + a_0,$ $S(n)$ je ciferný součet čísla $n$. Pak platí:
  \begin{enumerate}[$i.$]
    \item $3\, | \, n \iff 3 \, | \, S(n)$ a
    \item $9\, | \, n \iff 9 \, | \, S(n)$
  \end{enumerate}
\end{veta}

\begin{proof}
  už se mi nechce
\end{proof}

\section{Základní pojmy z planimetrie, rovinné útvary, úhly v nich}
\begin{definition}
  Nechť $A,B,C\in \mathbb E_2$ jsou tři body. Jestliže všechny tři leží (resp. neleží) na jedné přímce, řekneme, že jsou \textbf{kolineární} (resp. \textbf{nekolineární}).
\end{definition}

\begin{definition}
  Nechť $p,q\in \mathscr P$. Jestliže platí:
  \begin{enumerate}[$i.$]
    \item $p=q$, pak přímky $p,q$ se nazývají \textbf{splývající rovnoběžky},
    \item $p \ne q \land p\cap q = \emptyset$, pak přímky $p,q$ se nazývají \textbf{různé rovnoběžky},
    \item $p \ne q \land p\cap q \ne \emptyset$, pak přímky $p,q$ se nazývají \textbf{různoběžky}.
  \end{enumerate}
\end{definition}

\begin{definition}
  Nechť $p\in \mathscr P,A \in p, B \in p, A \ne B.$ Množinu
  \[
    P(A)=\left \{ X\in \mathbb E_2; X=B \lor X\,\mu\, AB \lor B\, \mu\, AX \right \}
  \]
  (resp. množinu $P(A)\cup \{A\}$) nazýváme \textbf{otevřenou} (resp. \textbf{uzavřenou}) \textbf{polopřímkou} $AB$ s počátkem v bodě $A$. Množinu
  \[
    Q(A)=\left \{ X\in \mathbb E_2; A\,\mu\, BX \right \}
  \]
  (resp. množinu $Q(A)\cup \{A\}$) nazýváme \textbf{otevřenou} (resp. \textbf{uzavřenou}) \textbf{polopřímku opačnou} k polopřímce $AB$ s počátkem $A$.
\end{definition}

\begin{definition}
  Nechť body $A,B\in \mathbb E_2, A\ne B$. Průnik uzanřených polopřímek $AB$ a $BA$ nazveme \textbf{úsečkou} $AB$. Body $A,B$ se nazývají \textbf{krajní body úsečky} $AB$, bod $X\,\mu \, AB$ se nazývá \textbf{vnitřní bod} úsečky $AB$.
\end{definition}

\begin{definition}
Nechť $a\in \mathscr P, A\notin a, B\notin a, A\ne B.$ Pak \textbf{přímka} $a$ \textbf{odděluje body} $A,B$ a zapisujeme $a\, \nu\, AB.$ V opčaném případě \textbf{přímka} $a$ \textbf{neodděluje body} $A,B$. Zapisujeme $a\, \overline \nu \,AB.$
\end{definition}

\begin{definition}
  Nechť $a \in \mathscr P$. Pak všechny $X\in \mathbb E_2 - a$ lze rozdělit do dvou podmnožin $P(a), Q(a)$ tak, že:
  \begin{enumerate}[$i.$]
    \item přímka $a$ odděluje každé dva body z různých podmnožin:
      \[
        \forall x \in P(a), \forall Y \in Q(a): a \, \nu \, XY
      \]
    \item přímka $a$ neodděluje žádné dva body z jedné podmnožiny
      \[
        \forall X, Y \in P(a) \land \forall X,Y \in Q(a): a \,\overline \nu\, XY.
      \]
  \end{enumerate}
  Pak množinu $P(a)$ (resp. množinu $Q(a)$) nazveme \textbf{otevřenou polorovinou s hraniční přímkou} $a$ (resp. \textbf{otevřenou polorovinou s hraniční přímkou} $a$ \textbf{opačnou k} $P(a)$). Množinu $P(a)\cup a$ (resp. $Q(a) \cup a$) nazveme \textbf{uzavřenou polorovinou s hraniční přímkou} $a$ (resp. \textbf{uzavřenou polorovinou s hraniční přímkou} $a$ \textbf{opačnou k} $P(a)$).
\end{definition}

\begin{definition}
  Nechť $A,B,V \in \mathbb E_2$ jsou tři různé nekolineární body. Průnik polorovin $VBA \cap VAB$ nazveme \textbf{konvexním úhlem} (zapisujeme $\sphericalangle BVA$), $V$ jeho \textbf{vrcholem}, polopřímky $VA, VB$ jeho \textbf{rameny}. \textbf{Nekonvexním úhlem} $BVA$ nazveme sjednocení polorovin opačných k polorovinám $VBA,VAB.$
\end{definition}

\begin{definition}
  Nechť $A,B,C\in \mathbb E_2$ jsou tři různé nekolineární body. \textbf{Trojúhelníkem} $ABC$ (značíme $\triangle ABC$) nazýváme \textbf{vrcholy} tohoto trojúhelníka, úsečky $AB, BC, CA$ jeho \textbf{stranami}.
\end{definition}

\begin{definition}
  Nechť $A,B\in \mathbb E_2.$ Přiřaďme uspořádané dvojici bodů $(A,B)$ reálné číslo označené $|AB|,$ pro něž platí:
  \begin{enumerate}[$i.$]
    \item $|AB|\geq 0,$ přičemž $|AB|=0 \iff A=B,$
    \item $|AB|=|BA|,$
    \item $C\, \mu \, AB \implies |AB|=|CB|+|AC|,$
    \item nechť $AB$ je polopřímka, $m\in \mathbb R^+_0.$ Pak $\exists ! C\in AB$ tak, že $|AC|=m.$
  \end{enumerate}
\end{definition}

\begin{definition}
  Každému konvexnímu úhlu $\sphericalangle AVC$ přiřaďme \textbf{velikost úhlu} (ozn. $|\sphericalangle AVC|$) ve stupních tak, že
  \begin{enumerate}[$i.$]
    \item nulový úhel má velikost $0^\circ$, přímý $180^\circ$,
    \item každý jiný konvexní úhel má velikost $n^\circ, $ kde $0^\circ< n ^\circ< 180^\circ, u \in \mathbb R,$
    \item jestliže polopřímka $VB$ prochází mezi rameny konvexního úhlu $\sphericalangle AVC$, pak $|\sphericalangle AVC|=|\sphericalangle AVB|+ |\sphericalangle BVC|$ a
    \item nechť $VA$ je polopřímka, $u \in \mathbb R, u \in \left < 0 ^\circ,180^\circ \right >$. Pak existuje polopřímka $VB$ taková, že $|\sphericalangle AVB|=u^\circ.$
  \end{enumerate}
\end{definition}

\begin{pozn}
  \textbf{Radián} je středový úhel příslušný v jednotkové kružnici kruhovému oblouku délky 1. Z definice plyne:
  \[
    360^\circ = 2\pi.
  \]
\end{pozn}

\begin{definition}
  Nechť $p,q\in \mathscr P$ jsou dvě různoběžky. Řekneme, že $p,q$ jsou na sebe \textbf{kolmé} (a zapisujeme $p\perp q$), jestliže všechny čtyři úhly, které spolu svírají, jsou shodné (a tedy pravé).
\end{definition}

\begin{pozn}[Klasifikace úhlů podle velikosti]
  Nechť $\alpha$ je velikost úhlu $|\sphericalangle ABC|$. Potom je úhel $|\sphericalangle ABC|$
  \begin{itemize}
    \item nulový, pokud $\alpha=0^\circ$,
  \item ostrý, pokud $0^\circ <\alpha < 90^\circ,$
  \item pravý, pokud $\alpha = 90^\circ,$
  \item tupý, pokud $90^\circ < \alpha < 180^\circ,$
  \item přímý, pokud $\alpha = 180^\circ,$
  \item konvexní, pokud $0^\circ \leq \alpha \leq 180^\circ,$
  \item nekonvexní, pokud $180^\circ < \alpha < 360^\circ$,
  \item plný, pokud $\alpha = 360^\circ,$
  \item kosý, pokud je ostrý nebo tupý.
  \end{itemize}
\end{pozn}

\begin{definition}
  Nechť $A,B\in \mathbb E_2$. \textbf{Vzdáleností bodů} $A,B$ nazveme délku úsečky $AB.$
\end{definition}


\begin{definition}
  Nechť $P\in \mathbb E_2, p \in \mathscr P.$ \textbf{Vzdáleností bodu} $P$ \textbf{od přímky} $p$ nazveme reálné číslo označené $\rho(P,p)$ takové, že $\rho(P,p)=|PP_0|,$ kde $P_0$ je kolmý průmět bodu $p$ na přímku $p$.
\end{definition}

\begin{definition}
  Nechť $a,b \in \mathscr P, a \parallel b.$ \textbf{Vzdáleností dvou přímek} $a,b$ nazveme reálné číslo označené $\rho(a,b)$ takové, že $\rho(a,b)=\rho(A,b),$ kde $A\in a$ je libovolný bod.
\end{definition}

\begin{definition}
  Nechť $a,b\in \mathscr P.$ \textbf{Odchylkou přímek} $a,b$ nazveme reálné číslo $\varphi^\circ\in \left <0, 180\right>$, kde $\varphi$ je velikost úhlu, který spolu přímky $a,b$ svírají. U rovnoběžek klademe $\varphi = 0^\circ.$
\end{definition}

\begin{definition}
  Dvojice úhlů:
  \begin{itemize}
    \item \textbf{vrcholové úhly} -- dvojice úhlů, jejichž ramena jsou opačné polopřímky
    \item \textbf{vedlejší úhly} --	dvojice úhlů, jejichž jedno rameno je společné a druhá ramena jsou opačné polopřímky
    \item \textbf{souhlasné úhly} -- dvojice úhlů, jejichž první ramena leží na jedné přímce a druhá ramena jsou rovnoběžná, přitom směr příslušných ramen je stejný
    \item \textbf{střídavé úhly}	-- dvojice úhlů, jejichž první ramena leží na jedné přímce a druhá ramena jsou rovnoběžná, přitom směr příslušných ramen je opačný
    \item \textbf{přilehlé úhly} -- dvojice úhlů, jejichž první ramena leží na jedné přímce a jdou do opačných směrů a druhá ramena jsou rovnoběžná
  \end{itemize}
\end{definition}

\begin{definition}
  Dvě polopřímky, ležící na téže přímce nazývámě \textbf{souhlasnými}, jestliže jedna z nich je podmnožinou druhé. V opačném případě je nazveme \textbf{nesouhlasnými}.
\end{definition}

\begin{definition}
  Nechť $\triangle ABC$. \textbf{Vnitřním úhlem} $\triangle ABC$ při vrcholu A (B, C) nazýváme $\sphericalangle CAB (ABC, BCA)$, \textbf{vnějším úhlem} $\triangle ABC$ při vrcholu A (B, C) pak vedlejší úhel k úhlu $\sphericalangle CAB (ABC, BCA)$.
\end{definition}

\begin{veta}
  Součet všech vnitřních úhlů v trojúhelníku je $180^\circ$.
\end{veta}

\begin{proof}
  \begin{figure}[h]
      \includegraphics[width=\linewidth]{trojuhelnik_1.png}
  \end{figure}
\end{proof}

\begin{veta}
  Součet libovolných dvou vnitřních úhlů v trojúhelníku je menší něž $180^\circ$.
\end{veta}

\begin{proof}
  Plyne z předchozí věty a tvrzení, že vnitřní úhel v trojúhelníku je nenulový.
\end{proof}

\begin{veta}
  V každém trojúhelníku je velikost vnějšího úhlu při jednom vrcholu rovna součtu velikostí dvou zbylých vnitřních úhlů.
\end{veta}

\begin{proof}
  $$\alpha + \alpha^\prime = 180^\circ$$ (úhly vedlejší)
  $$\alpha + \beta + \gamma = 180^\circ$$
  $$\Rightarrow \alpha^\prime = \beta + \gamma$$
\end{proof}

\begin{definition}
  Trojúhelník se nazývá \textbf{rovnoramenný}, jestliže alespoň dvě jeho strany jsou shodné úsečky. Strany stejné délky nazveme \textbf{rameny}, tu třetí \textbf{základnou} a vrchol proti základně \textbf{vrcholem}.
  Trojúhelník se nazývá \textbf{rovnostranný}, jestliže všechny tři jeho strany jsou shodné úsečky.
  Trojúhelník se nazývá \textbf{pravoúhlý}, jestliže je jeden jeho vnitřní úhel pravý. Dvě strany, které jsou rameny pravého úhlu nazveme \textbf{odvěsnami}, tu třetí \textbf{přeponou}.
\end{definition}

\begin{definition}
  \textbf{Střední příčkou} trojúhelníku nazýváme úsečku spojující středy dvou stran trojúhelníka.
  \textbf{Těžnicí} trojúhelníku nazýváme úsečku spojující vrchol trojúhelníku se středem protější strany. Průsečík těžnic nazýváme \textbf{těžíště}.
  \textbf{Výškou} trojúhelníku nazýváme úsečku procházející vrcholem trojúhelníka, která je kolmá na přímku, na které leží protější strana trojúhelníku. Průsečík výšek nazýváme \textbf{ortocentrum}.
\end{definition}

\begin{veta}
  Každá střední příčka je rovnoběžná s protilehlou stranou a je dvakrát menší než protilehlá strana.
\end{veta}

\begin{proof}
  Trojúhelníky vrchol - střední příčka a vrchol - protilehlá strana jsou zjevně podobné s koeficientem 2.
\end{proof}

\begin{veta}
  Těžnice trojúhelníku se všechny protínají v jednom bodě, který je dělí v poměru $2:1$.
\end{veta}

\begin{proof}
  Doslova nevim, ve skriptách to není.
\end{proof}

\begin{veta}

\end{veta}

\section{Kartézský součin, binární relace, zobrazení}
\begin{definition}
  \textbf{Kartézským součinem množin} $A,B$ nazýváme množinu $A\times B$ všech uspořádaných dvojic $(a,b)$ takových, že $a\in A,b\in B$.
  \[
    A \times B = \left \{ (a,b); a\in A,b\in B \right \}.
  \]
\end{definition}

\begin{pozn}
  Kartézský součin $A\times A$ nazveme \textbf{kartézským čtvercem} množiny $A$.
\end{pozn}

\begin{pozn}
  Kartézský graf je graf, kde prvky na ose $x$ (resp. $y$) jsou prvky z množiny $A$ (resp. $B$) a uspořádanou dvojici $(a,b)$ zaneseme jako příslušný bod se souřadnicemi $[a,b].$
\end{pozn}

\begin{definition}
  Nechť $A,B$ jsou dvě množiny. Pak každou podmnožinu kartézského součinu $A\times B$ nazýváme \textbf{binární relací} mezi množinami $A,B$
  (v tomto pořadí). Je-li speciálně $A=B$, pak hovoříme o \textbf{binární relaci v množině} $A$.
\end{definition}

\begin{definition}
Nechť $\alpha\subseteq A\times B$ je binární relace. Je-li uspořádaná dvojice $(a,b), a \in A, b \in B$ prvkem množiny $\alpha$,
říkáme, že $a$ \textbf{je v relaci s} $b$ a píšeme $a \sim b$.
\end{definition}

\begin{priklad}
Znázorněte graf relace $U=\left \{ \left [ x,y \right ]\in \mathbb R^2: x-y+1=0  \right \}. $
\end{priklad}

\begin{reseni}
Řešením je přímka $y=x+1.$
\end{reseni}

\begin{definition}
  Relaci na množině $A$ nazveme
  \begin{enumerate}[$i.$]
    \item \textbf{reflexivní}, pokud $\forall a\in A: a\sim a$,
    \item \textbf{symetrickou}, pokud $\forall a,b \in A: a\sim b \implies b\sim a,$
    \item \textbf{tranzitivní}, pokud $\forall a,b,c\in A: a\sim b \land b\sim c \implies a\sim c.$
  \end{enumerate}
\end{definition}

\begin{definition}
  Relaci na množině $A$, která je zároveň reflexivní, symetrická a tranzitivní, nazveme \textbf{relací ekvivalence}.
\end{definition}

\begin{pozn}
  Každé relaci ekvivalence na množině $A$ přísluší \textbf{rozklad příslušný ekvivalenci} tak, že množinu $A$ rozdělíme na po dvou disjunktní podmnožiny,
  jejichž sjednocení dává množinu $A$ a navíc platí, že všechny prvky v jedné podmnožině jsou navzájem ekvivalentní a~žádné dva prvky z jiných podmnožin ekvivalentní nejsou. Tyto podmnožiny potom nazveme \textbf{třídami rozkladu}.
\end{pozn}

\begin{definition}
  Nechť $A,B$ jsou dvě množiny. \textbf{Zobrazením} $f$ \textbf{z množiny $A$ do množiny $B$} nazýváme relaci $f\subseteq A \times B,$ pro níž platí: $\forall x \in A: \exists \text{ max. 1 } y \in B: (x,y) \in f$. Prvek $x$ nazveme \textbf{vzorem} a $y$ \textbf{obrazem}.
\end{definition}

\begin{definition}
  Nechť $f\subseteq A\times B$ je zobrazení. \textbf{Definičním oborem} (resp. \textbf{oborem hodnot}) zobrazení $f$ nazveme množinu $D(f)\subseteq A$
  (resp. $H(f)\subseteq B$) všech prvků $a\in A$ (resp. $b\in B$) takových, že k nim existuje právě jedno $b\in B$ (resp. alespoň jedno $a\in A$) tak, že $b=f(a)$.
\end{definition}

\begin{definition}
  Pokud $D(f) = A$ (resp. $D(f)\ne A$), hovoříme o \textbf{zobrazení množiny} (resp. \textbf{zobrazení z množiny}) $A$. Pokud $H(f)=B$ (resp. $H(f)\ne B$), hovoříme o \textbf{zobrazení na množinu} (resp. \textbf{zobrazení do množiny}) $B$.
  Zobrazení množiny na množinu nazýváme \textbf{surjekcí}. Je-li $A=B$ (resp. $A=B \land ( A = D(f) \lor B = H(f))$), hovoříme o \textbf{zobrazení v množině} (resp. \textbf{zobrazení na množině}) $A$.
\end{definition}

\begin{definition}
  Nechť $f$ je zobrazení z $A$ do $B$. Zobrazení $f$ je \textbf{prosté} neboli \textbf{injektivní}, jestliže ke každému $b\in B$ existuje nejvýše jedno $a \in A$ takové, že $b=f(a).$
\end{definition}

\begin{definition}
  Zobrazení, které je současně surjekcí a injekcí, nazýváme \textbf{bijekce}.
\end{definition}

\begin{definition}
  Nechť $\alpha \subseteq A\times B$ je binární relace. \textbf{Inverzní relací} k relaci $\alpha$ nazveme relaci $\alpha^{-1} \subseteq B\times A$ takovou, že
  \[
    \alpha^{-1}=\left\{ (b,a)\in B\times A;  (a,b)\in \alpha\right\}.
  \]
  Je-li $\alpha^{-1}$ zobrazení, nazveme relaci $\alpha ^{-1}$ \textbf{inverzním zobrazením} k zobrazení $\alpha$.
\end{definition}

\begin{definition}
  Nechť $f:B\to C, g:A\to B$ jsou zobrazení. \textbf{Složeným zobrazením} ze zobrazení $f$ a $g$ nazveme zobrazení $h: A\to C$ takové, že
  \[
    h=\left\{ (a,c)\in A\times C;\exists b\in B: f(b)=c \land g(a)=b \right\}.
  \]
  Značíme $c=f(g(a))$, $h=f\, \circ \, g $ (čteme \uv{$f$ po $g$}).
\end{definition}

\section{Funkce a jejich základní vlastnosti}
\begin{definition}
  \textbf{Funkcí} $f$ nazýváme každé zobrazení z $\mathbb R$ do $\mathbb R$.
\end{definition}

\begin{definition}
  Nechť $f\subseteq \mathbb R \times \mathbb R$ je funkce. Množinu
  \[
    D(f) = \left  \{ x \in \mathbb R:\exists ! y \in \mathbb R:y=f(x) \right \}
  \]
  nazveme \textbf{definičním oborem} funkce $f$. Množinu
  \[
    H(f) = \left  \{ y \in \mathbb R:\exists ! x \in \mathbb R:y=f(x) \right \}
  \]
  nazveme \textbf{oborem hodnot} funkce $f$.
\end{definition}

\begin{pozn}
  \textbf{Grafem funkce} rozumíme množinu všech bodů $[x,f(x)]$, kde $x\in D(f).$
\end{pozn}


\begin{definition}
  Funkce $f$ se nazývá \textbf{sudá} (resp. \textbf{lichá}), jestliže platí
\begin{itemize}
  \item
  $\forall x \in D(f):(-x) \in D(f)$
\item
 $\forall x \in D(f): f(-x)=f(x)$, resp. $f(-x)=f(x)$.
\end{itemize}

\end{definition}

\begin{definition}
  Funkce $f$ se nazývá \textbf{prostá}, právě tehdy když platí
  \[
    \forall x_1,x_2\in D(f): x_1\ne x_2 \implies f(x_1)\ne f(x_2)
  \]
\end{definition}

\begin{definition}
  Nechť $f$ je funkce a $M$ alespoň dvouprvková množina z $D(f)$. Řekneme, že funkce $f$ je v množině $M$
  \begin{enumerate}[$i.$]
    \item \textbf{rostoucí} $\iff \forall x_1, x_2 \in M: x_1 < x_2 \implies f(x_1) < f(x_2),$
    \item \textbf{klesající} $\iff \forall x_1, x_2 \in M: x_1 < x_2 \implies f(x_1) > f(x_2),$
    \item \textbf{neklesající} $\iff \forall x_1, x_2 \in M: x_1 < x_2 \implies f(x_1) \leq f(x_2),$
    \item \textbf{nerostoucí} $\iff \forall x_1, x_2 \in M: x_1 < x_2 \implies f(x_1) \geq f(x_2).$
  \end{enumerate}
  Je-li $f$ neklesající nebo nerostoucí, je \textbf{monotónní}. POkud je klesající nebo rostoucí, je \textbf{ryze monotónní}.
\end{definition}

\begin{definition}
  Nechť $f$ je funkce, $M\subseteq D(f)$. Řekneme, že funkce $f$ je v množině $M$
  \begin{enumerate}[$i.$]
    \item \textbf{shora omezená} $\iff \exists k \in \mathbb R: \forall x \in M: f(x)\leq k,$
    \item \textbf{zdola omezená} $\iff \exists k \in \mathbb R: \forall x \in M: f(x)\geq k,$
    \item \textbf{omezená} $\iff $ je zdola i shora omezená.
  \end{enumerate}
\end{definition}


\begin{definition}
  Nechť $f$ je funkce, $M \subseteq D(f)$, v ní prvek $a \in M$.
  Řekneme, že funkce f má v bodě a:
  \begin{enumerate}[i.]
    \item \textbf{ostré maximum} na množině $M$ právě tehdy, když $\forall x \in M; x \not = a: f(x) < f(a)$
    \item \textbf{maximum} (neostré) na množině $M$ právě tehdy, když $\forall x \in M : f(x) \leq f(a)$
    \item \textbf{ostré minimum} na množině $M$ právě tehdy, když $\forall x \in M; x > a: f(x) > f(a)$
    \item \textbf{minimum} (neostré) na množině $M$ právě tehdy, když $\forall x \in M : f(x) \geq f(a)$
  \end{enumerate}
\end{definition}

\begin{definition}
  Nechť $f$ je funkce. Funkce $f$ se nazývá \textbf{periodická}, pokud $\forall p \in \mathbb R^{+}: \forall x \in D(f):$
  \begin{enumerate}
    \item $x \in D(f) \implies x \pm p \in D(f)$
    \item $f(x) = f(x \pm p)$
  \end{enumerate}
  Číslo $p$ se nazývá \textbf{periodou} této funkce. Periodu $p_0$ nazveme \textbf{nejmenší periodou} funkce, pokud pro všechny ostatní periody $p$ platí $p > p_0$. V opačném případě se funkce nazývá \textbf{neperiodická}.
\end{definition}

\begin{definition}
  Máme funkci $f: y = f(u)$ s definičním oborem $D(f)$ a funkci $g: u=g(x)$ s oborem hodnot $H(g)$. Jestliže je $H(g) \subseteq D(f)$, pak funkci $h: y = f(g(x))$ nazveme \textbf{složenou funkcí} (někdy píšeme též $h=f \circ g$).
\end{definition}

\begin{definition}
  \textbf{Dirichletova funkce} je definována vzorcem
  $$\mathbf{D} (x) = \begin{cases}
1 \text{ pokud } x \in \mathbb Q \\
0 \text{ jinak }
  \end{cases}  $$
\end{definition}

\begin{definition}
  Funkce \textbf{signum} je definováno následujícím způsobem $$\operatorname {sgn} x={\begin{cases}-1,&x<0\\0,&x=0\\1,&x>0\end{cases}}$$
\end{definition}

\begin{definition}
  Nechť $x \in \mathbb{R}$ je libovolné číslo. Pak existuje právě jedna dvojice $z \in \mathbb{Z}, a \in \left \langle 0;1 \right) \text{ tak, že } x = z + a$.
Číslo z nazýváme \textbf{celou částí} čísla x a zapisujeme $[x] = z (\lfloor x \rfloor = z)$.
\end{definition}

\begin{definition}
  O funkci $f:\mathbb {R} \rightarrow \mathbb {R}$ řekneme, že je \textbf{spojitá} v bodě $a$, pokud ke každému libovolně malému číslu $\varepsilon >0$ existuje takové číslo $\delta >0$, že pro všechna $x$, pro něž platí $|x-a|<\delta$, platí také $|f(x)-f(a)|<\varepsilon$.
\end{definition}

\begin{definition}
  Číslo $A\in \mathbb {R}$ je limitou funkce $f:\mathbb {R} \rightarrow \mathbb {R}$ v bodě $ a\in \mathbb {R}$, jestliže k libovolnému $ \varepsilon >0$ existuje takové $ \delta >0$, že pro všechna $x\in D(f)$ taková, že $ \left|x-a\right|<\delta$ ($x$ leží v prstencovém okolí bodu $a$) platí $\left|f(x)-A\right|<\varepsilon $.

  Limitu má smysl zkoumat jen v definičním oboru funkce neobsahujícím bod $a$, tj. libovolně blízko k bodu $a$ musí být funkce definována.
\end{definition}

\begin{definition}
  Nejběžnější moderní definice \textbf{derivace} funkce $f$ v bodě $a$, zapisujeme $f'(a)$ je
  $$f'(a)=\lim _{h\to 0}{\frac {f(a+h)-f(a)}{h}}=\lim _{x\to a}{\frac {f(x)-f(a)}{x-a}}.$$

  Co to znamená se mě neptejte (klidně se zeptejte). Vložte intuitivní definici derivace.
\end{definition}

\begin{definition}
  Nechť $f$ je funkce spojitá na intervalu $(a,b)$. Pak říkáme, že funkce $f$ je na intervalu $(a,b)$ \textbf{konvexní} (resp. \textbf{ryze konvexní}) právě tehdy, když pro libovolné číslo $\lambda \in (0,1)$ s vlastností $\forall x,y\in (a,b),x<y:f(\lambda x+(1-\lambda )y) < (\text{resp. } \leq) \lambda f(x)+(1-\lambda )f(y)$

  Pokud spojitá funkce není na intervalu konvexní (resp. ryze konvexní), je na něm ryze konkávní (resp. konkávní).
\end{definition}

\begin{definition}
  \textbf{Primitivní funkce} k funkci $f$ na intervalu $(a,b)$ je taková funkce $F$, že pro každé $x\in (a,b)$ je $F'(x)=f(x)$.

  Procesu hledání primitivní funkce se často říká \textbf{integrování} nebo \textbf{integrace} (od slova integrál).
\end{definition}

\section{Polynomy, kořeny polynomů}
\begin{definition}
  \textbf{Polynomem} nazýváme každý výraz $P(x)$ tvaru
  \[
    P(x)=a_nx^n + a_{n-1}x^{n-1}+\dots + a_2x^2+a_1x+a_0,
  \]
  kde $a_i\in \mathbb R, i\in \{ 0,1,\dots , n \},n\in \mathbb N$. Čísla $a_i$ se nazývají \textbf{koeficienty polynomu}, sčítance $a_ix^i$ \textbf{členy polynomu}. Je-li $a_n\ne 0$, číslo $n$ se nazývá \textbf{stupeň polynomu} a označuje se $n= \text{st}(P(x))$. Je-li $a_i=0$ pro všechna $i \in \{ 0,1,\dots ,n\}$, pak klademe $\text{st}(P(x))=-\infty$ a $P(x)$ se nazývá \textbf{nulovým polynomem}. Označujeme jej $O(x).$ Je-li $a_n=1$, nazývá se $P(x)$ \textbf{normovaným polynomem}.
\end{definition}

\begin{veta}[O dělení polynomů se zbytkem]
  Nechť $A(x), B(x) \in \mathbb R[x],B(x) \ne O(x).$ Pak existuje právě jedna dvojice polynomů $Q(x),R(x)\in \mathbb R[x]$ tak, že
  \[
    A(x)=B(x)\cdot Q(x)+R(x),
  \]
  kde $R(x)=O(x)$ nebo $\text{st}(R(x))<\text{st}(B(x)).$
\end{veta}

\begin{definition}
  Nechť $A(x), B(x) \in \mathbb R[x]$. Řekneme, že polynom $B(x)$ \textbf{dělí} polynom $A(x)$ právě tehdy, když existuje takový polynom $C(x) \in \mathbb R[x]$ tak, že
  \[
    A(x) = B(x)\cdot C(x).
  \]
  Zapisujeme $B(x) \, | \, A(x).$
\end{definition}

\begin{definition}
  Nechť $P(x)\in \mathbb R[x], P(x)=a_nx^n+a_{n-1}x^{n-1}+\dots +a_1x+a_0.$ Nechť $c\in \mathbb R$ je libovolné číslo. \textbf{Hodnotou polynomu} $P(x)$ \textbf{v čísle} (v bodě) $c$ nazýváme reálné číslo $P(c)$
  \[
    P(c) = a_nc^n+a_{n-1}c^{n-1}+\dots+a_1c+a_0.
  \]
  Číslo $c\in \mathbb R$ nazveme \textbf{kořenem polynomu} $P(x) \iff P(c) = 0$.
\end{definition}


\begin{definition}
  Nechť $P(x) \in \mathbb R [x]$ a $c\in \mathbb R$ je jeho kořen. Lineární polynom $x-c$ nazveme \textbf{kořenovým činitelem}.
\end{definition}

\begin{definition}
  Nechť $P(x) \in \mathbb R [x], c \in \mathbb R$ je jeho kořen a $k \in \mathbb N$. Číslo $C \in \mathbb R$ nazýváme \textbf{$k$-násobným kořenem} polynomu $P(x)$ právě tehdy, když platí
  \[
    (x-c)^k \, | \, P(x) \land (x-c)^{k+1} \nmid  P(x).
  \]
\end{definition}

\begin{veta}[Vi\`{e}tovy vztahy]
  Nechť $P(x)=a_nx^n+a_{n-1}x^{n-1}+\dots + a_1x+a_0$ je polynom stupně $n$, který má v množině $\mathbb R$ právě $n$ kořenů $x_1,x_2\dots,x_n$ (každý počítáme tolikrát, jaká je jeho násobnost). Pak platí:
  \begin{align*}
    \sum_{i=1}^n x_i = & -\frac{a_{n-1}}{a_n} \\
    \sum_{i,j=1; i<j}^{n}x_ix_j= & \frac{a_{n-2}}{a_n} \\
    \vdots & \\
    \prod_{i=1}^nx_i=&(-1)^n\frac{a_0}{a_n}
  \end{align*}
\end{veta}

\begin{pozn}
  \textbf{Hornerovo schéma} je numerická metoda pro vyhodnocení funkční hodnoty polynomu $P(x) \in \mathbb R [x]$ v bodě $ x_0 \in \mathbb R$. Příklad:
  Vyhodnoťte $f_{1}(x)=2x^{3}-6x^{2}+2x-1\,$ v bodě $x=3\;$.
  Opakovaným vytknutím $x$, může být $f_{1}$ zapsáno jako $x(x(2x-6)+2)-1\;$. Pro větší přehlednost užijeme k zápisu průběhu výpočtu tzv. syntetický diagram.
  \begin{tabular}{ c|c c c c }
    $x_{0}$ & $x^{3}$ & $x^{2}$ & $x^{1}$ & $x^{0}$\\
    \hline
    $3$ & $2$ & $0$ & $2$ & $5$
  \end{tabular}

  Do prvního místa opíšeme $a_n$. Čísla v řádku jsou součty koeficientu $a_k$ součinu hodnoty $x$, v níž polynom vyhodnocujeme (v tomto příkladě tedy $3$) s číslem v řádku o jeden sloupec vlevo (tedy pod $a_{k+1}$). Výsledek vyhodnocování je vpravo dole – v našem případě tedy $5$.

  Důsledkem věty o dělení polynomu polynomem je, že zbytek po vydělení f1 polynomem (x-3) je 5 a výsledkem tohoto dělení je polynom stupně 2 s koeficienty danými zbylými třemi čísly ve třetím řádku. Díky tomuto pozorování lze Hornerovo schéma použít i jako efektivní algoritmus k dělení polynomů.
\end{pozn}

\begin{veta}[Hledání racionálních kořenů polynomu s racionálními koeficienty]
  Mějme polynom $P(x) \in \mathbb Q [x]$. Potom najdeme jeho kořeny $\frac{r}{s} \in \mathbb Q$ takto:
  \begin{enumerate}[1.]
    \item Nalezneme všechny celočíselné dělitele $r$ absolutního členu polynomu $a_0$.
    \item Nalezneme všechny přirozené dělitele $s$ vedoucího členu $a_n$.
    \item Utvoříme všechny zlomky tvaru $\frac{r}{s}, (r,s) = 1$.
    \item Hornerovým schématem určíme $P(1)$, případně $P(-1)$ ($1$ a $-1$ také mohou být kořeny).
    \item Vyškrtáme ty zlomky $\frac{r}{s}$, které nesplňují podmínky $(r-s) \mid P(1) \land (r+s) \mid P(-1)$.
    \item U ostatních zlomků vyzkoušíme Hornerovým schématem, zda jsou kořeny daného polynomu.
  \end{enumerate}
\end{veta}

\begin{veta}[Rozklad polynomu v reálném a komplexním oboru]
  Nechť $P(x) \in \mathbb R [x], P(x) = a_n x^n + a_{n-1} x^{n-1} + ... + a_1 x + a_0$, kde $st(P(x)) \geq 1$. Pak $P(x)$ lze v $\mathbb R$ vyjádřit jako součin polynomů 1. a 2. stupně a koeficientu $a_n$:
  \[
    P(x) = a_n(x-c_1)^{k_1}(x-c_2)^{k_2} ... (x-c_k)^{k_k}(x^2+p_1 x + q_1)^{r_1}(x^2+p_2 x + q_2)^{r_2} ... (x^2+p_r x + q_r)^{r_r}
  \]
  kde $c_1,c_2, ..., c_k$ jsou všechny jeho reálné různé kořeny s násobnostmi $l_1, k_2, ..., k_k \in \mathbb N$; $p_1, p_2, ..., p_r$ a $q_1, q_2, ..., q_r$ jsou reálná čísla, $r_1, r_2, ..., r_n \in \mathbb N$.
  Polynomy $(x^2+p_1 x + q_1)^{r_1}(x^2+p_2 x + q_2)^{r_2} ... (x^2+p_r x + q_r)^{r_r}$ jsou kvadratické polynomy se záporným diskriminantem. Uvedený rozklad je až na pořadí činitelů jednoznačný a platí $st(P(x)) = k_1 + k_2 + ... + k_k + 2(r_1 + r_2 + ... + r_r)$.

  Jestliže $a_1 \pm ib_1, a_2 \pm ib_2, .., a_s \pm ib_s$ jsou všechny navzájem různé dvojice komplexně sdružených kořenů s násobností $r_1, r_2, ..., r_s$, $P(x)$ můžeme v $\mathbb C$ psát ve tvaru:
  \[
    P(x) = a_n(x-c_1)^{k_1}(x-c_2)^{k_2} ... (x-c_k)^{k_k}[(x-a_1)^2+b_1^2]^{r_1}[(x-a_2)^2+b_2^2]^{r_2} ... [(x-a_s)^2+b_s^2]^{r_s}
  \]
\end{veta}

\begin{definition}
  Nechť $A(x), B(x) \in \mathbb R [x]$. Polynom $C(x) \in \mathbb R [x]$ se nazývá \textbf{společný dělitel polynomů} $A(x), B(x)$, právě tehdy, když platí: $C(x) \mid A(x) \land C(x) \mid B(x)$.
  Polynom $D(x) \in \mathbb R [x]$ se nazývá \textbf{největší společný dělitel polynomů} $A(x), B(x)$, který označujeme $D(A(x), B(x))$ nebo $NSD(A(x), B(x))$, právě tehdy, když platí:
  \begin{enumerate}[1.]
    \item $D(x) \mid A(x) \land D(x) \mid B(x)$
    \item $\forall C(x) \in \mathbb R [x]: C(x) \mid A(x) \land C(x) \mid B(x) \implies C(x) \mid D(x)$
  \end{enumerate}
\end{definition}

\begin{pozn}
  Největší společný dělitel hledáme \textbf{Euklidovým algoritmem}: Nechť $A(x), B(x) \in \mathbb R [x]$.
  \begin{enumerate}[a.]
    \item $A(x) = 0(x) \implies D(A(x), B(x)) = B(x)$
    \item $st(A(x)) \geq st(B(x)) \geq 0$
  \end{enumerate}
  Proveďme následující posloupnost dělení se zbytkem. Toto dělení ukončíme, až dostaneme zbytek -- nulový polynom. Vzhledem k nerovnosti na pravé straně, tento nulový zbytek existuje.
  \[
    A(x) = B(x) \cdot Q_1(x) + R_1(x), \hfill st(R_1(x)) < st(B(x))\\
    B(x) = R_1(x) \cdot Q_2(x) + R_2(x), \hfill st(R_2(x)) < st(R_1(x))\\
    R_1(x) = R_2(x) \cdot Q_3(x) + R_3(x), \hfill st(R_3(x)) < st(R_2(x))\\
    ... ...\\
    R_{n-2}(x) = R_{n-1}(x) \cdot Q_n(x) + R_n(x), \hfill st(R_n(x)) < st(R_{n-1}(x))\\
    R_{n-1}(x) = R_{n}(x) \cdot Q_{n+1}(x), \hfill st(R_{n-1}(x)) < 0(x))
  \]
  Potom $(A(x), B(x)) = R_n(x)$, či libovolný násobek $R_n(x)$. \textbf{Normovaný největší společný dělitel polynomů} ale existuje právě jeden, ten se označuje $(A(x), B(x))$.
\end{pozn}

\begin{definition}
  Nehcť polynomy $A(x), B(x)$. Pokud $(A(x), B(x)) = 1$, řekneme, že polynomy $A(x), B(x)$ jsou \textbf{nesoudělné}. V opačném případě jsou \textbf{soudělné}.
\end{definition}

\begin{definition}
    Nechť $P(x) \in \mathbb R[x], P(x)=a_nx^n+a_{n-1}x^{n-1}+\dots+a_1x+a_0.$ \textbf{Derivací polynomu} $P(x)$ rozumíme polynom $P^\prime(x)$ definovaný takto:
    $$
        P^\prime(x)=\begin{cases}
        0, &\text{ je-li st} (P(x)) \leq 0,\\
        a_n n x^{n-1} + a_{n-1}(n-1)x^{n-2} + \dots + a_1, & \text{ je-li st} (P(x)) \geq 1.
        \end{cases}
    $$
\end{definition}

\begin{veta}
    Nechť $P(x) \in R[x], c \in \mathbb R$ je jeho $k$-násobný kořen, $k\in \mathbb N.$ Pak platí: $c$ je
    $k$-násobný kořen $P(x) \iff P(c)=P^\prime(c)=P^{\prime \prime}(c)=\dots P^{k-1}(c)=0 \land P^{k}(c)\ne 0$.
\end{veta}

\section{Polynomická funkce (především lineární a kvadratické)}

\begin{definition}\,
\begin{enumerate}
  \item Nechť $b \in \mathbb R$. Funkci $f:y = b$ nazveme \textbf{konstantní funkcí}.
  \item Nechť $a, b \in \mathbb R, a \neq 0$. Pak funkci $f:y= ax + b$ nazveme \textbf{lineární funkcí}.
\end{enumerate}
\end{definition}

\begin{pozn}\,
  \begin{itemize}
    \item Definičním oborem konstantní i lineární funkce je $\mathbb R$. \\
          Oborem hodnot konstantní funkce je ${b}$.\\
          Oborem hodnot lineární funkce je $\mathbb R$.
    \item Grafem konstantní funkce je přímka rovnoběžná s osou $x$. \\
          Grafem lineární funkce je přímka, která není rovnoběžná s osou $x$ ani s osou $y$.
  \end{itemize}
\end{pozn}

\begin{veta}
  Nechť $f: y = ax + b, a \neq 0$ je lineární funkce. Pak platí:
  \begin{enumerate}[1.]
    \item $b=0 \implies f$ je lichá\\
          $b \neq 0 \implies f$ není ani sudá, ani lichá
    \item $a > 0 \implies f$ je rostoucí\\
          $a < 0 \implies f$ je klesající
    \item $f$ není ani shora, ani zdola omezená
    \item $f$ nemá extrémy
    \item $f$ není periodická
  \end{enumerate}
\end{veta}

\begin{proof}
  \begin{enumerate}[1.]
    \item $b=0 \implies f:y= ax = f(x) \land -ax = -f(x) \implies f$ je lichá\\
          $b\neq 0 \implies f:y = ax + b = f(x) = -ax + b \implies f$ není sudá ani lichá
    \item $a>0 \implies x_1 < x_2 \implies ax_1 < ax_2 \implies ax_1 + b < ax_2 + b \implies f$ je rostoucí\\
          $a<0 \implies x_1 < x_2 \implies ax_1 > ax_2 \implies ax_1 + b > ax_2 + b \implies f$ je klesající
    \item Obor hodnot je $\mathbb R \implies f$ není omezená.
    \item Plyne z grafu.
    \item Plyne z z grafu nebo z toho, že $f$ je ryze monotónní
  \end{enumerate}
\end{proof}

\begin{definition}
  Nechť $a,b,c \in \mathbb R, a \neq 0$. Pak $f:y = ax^2 + bx+ x$ nazýváme \textbf{kvadratickou funkcí}.
\end{definition}

\begin{pozn}\,
  \begin{itemize}
    \item Definičním oborem kvadratické funkce je $\mathbb R$.
    \item Grafem kvadratické funkce je parabola.
  \end{itemize}
\end{pozn}

\begin{veta}
  Nechť $f:y = ax^2, a \neq 0$ je kvadratická funkce. Pak platí:\\
  \begin{tabularx}{\textwidth}{ >{\raggedright\arraybackslash}X >{\raggedright\arraybackslash}X >{\raggedright\arraybackslash}X }
    \, & $a>0$ & $a<0$ \\
    obor hodnot & $H(f) = \mathbb R^{+}_0$ & $H(f) = \mathbb R^{-}_0$ \\
    parita & sudá & sudá \\
    monotónnost & $x \in (-\infty, 0\rangle$ -- klesající & $x \in (-\infty, 0\rangle$ -- rostoucí \\
    \, & $x \in \langle 0, \infty)$ -- rostoucí & $x \in \langle 0, \infty)$ -- klesající \\
    omezenost & zdola omezená & zdola neomezená \\
    \, & shora neomezená & shora omezená \\
    extrémy &  ostré minimum v bodě $x_0=0$ & ostré maximum v bodě $x_0=0$ \\
    periodicita & neperiodická & neperiodická
  \end{tabularx}
\end{veta}


\nocite{*}
\clearpage
{\small\bibliography{references}}

\backmatter
%!TEX root = ../main.tex

\clearpage
\phantomsection
\addcontentsline{toc}{section}{About the Author}
\begin{adjustwidth}{0.1\textwidth}{0.1\textwidth}
\begingroup
\null\vspace{0.2\textheight}
\begin{center}
{\bfseries\Large O autorovi}\par\vspace{2em}

Da, da, da, da, da \\
It's the motherfucking D-O-double-D \\
Da, da, da, da, da \\
You know I'm mobbin' with Honza Romanovský (Yeah, yeah, yeah)
\end{center}
\endgroup
\end{adjustwidth}
\clearpage


\end{document}
