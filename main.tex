\documentclass[11pt]{template/cauchy}
\usepackage[czech]{babel}

\usepackage{enumerate}
\usepackage{mlmodern}
\usepackage{template/mathsphystools}
\usepackage{amsthm,thmtools,xcolor}
% \usepackage{thmstyles}
\usepackage{graphicx}
\graphicspath{{images/}}

\declaretheoremstyle[
  headfont=\color{red}\normalfont\bfseries,
  bodyfont=\color{black}\normalfont,
]{def}

\declaretheoremstyle[
  headfont=\color{blue}\normalfont\bfseries,
  bodyfont=\color{black}\normalfont\itshape,
]{pr}

\declaretheoremstyle[
  headfont=\color{green}\normalfont\bfseries,
  bodyfont=\color{black}\normalfont\itshape,
]{veta}

\declaretheoremstyle[
  headfont=\color{brown}\normalfont\bfseries,
  bodyfont=\color{black}\normalfont,
]{pozn}

\declaretheorem[
  style=def,
  name=Definice,
]{definition}

\declaretheorem[
  style=pr,
  name=Příklad,
]{example}

\declaretheorem[
  style=veta,
  name=Věta,
]{veta}

\declaretheorem[
  style=pozn,
  name=Poznámka,
]{pozn}


\title[Title for the Header]{Matematika}
\subtitle{}
\author{Dominik Doležel \& Honza Romanovský}
\affiliation{Gymnázium Brno, třída Kapitána Jaroše}
\date{\today}
\begin{document}

%!TEX root = ../main.tex
\setcounter{page}{0}
\thispagestyle{fancy-blank}
\begingroup
% \vphantom{Optional note}
{\large \par}
\vspace*{35mm}
{\huge\bfseries\utitle\par}

\vspace*{5mm}
{\Large\usubtitle\par}

\vspace*{4mm}
{\rule{\linewidth}{0.5mm}\par}
\vspace*{4mm}

{\large\bfseries\uauthor\par}\vspace*{1mm}

{\large\itshape\uaffiliation\newline}
{\large\itshape{Provizorní nápis}\par}

\vfill
{\large \par}
\endgroup
\clearpage


\frontmatter
\tableofcontents
\clearpage
% \listoffigures
% \listoftables
% \clearpage

% Každá otázka vypadá takto:
% \section*{Název otázky}
% \subsection{Základní pojmy}
% itemize se seznamem základních pojmů
% \section*{Příklady}

\mainmatter

% \input{01_nazev_otazky}

\section{Základní pojmy z teorie množin}
\begin{definition}
  \textbf{Množina} je sourhn objektů, chápaný jako celek. Tyto objekty nazýváme prvky množiny.
\end{definition}

Množina může být konečná, nekonečná nebo prázdná. Množinu lze zadat výčtem prvků nebo pomocí charakteristické vlastnosti (např. $\left \{ 2k, k \in \mathbb{N}\right\}$).

\begin{definition}
  \textbf{Podmnožina} množiny $A$ je taková množina $B$, že všechny její prvky patří do množiny $A$.
\end{definition}

Každá neprázdná množina má dvě \textbf{nevlastní podmnožiny}: množinu prázdnou a sebe sama. Všechny ostatní její podmnožiny nazýváme nevlastní.

\begin{definition}
  Množiny $A$ a $B$ se rovnají právě tehdy, když $A$ je podmnožinou $B$ a zároveň $B$ je podmnožinou $A$.
\end{definition}

\begin{definition}
  Nechť $A \subseteq B$ a $B\neq \emptyset$. Množinu všech prvků mn. $B$, které nepatří do mn. $A$, nazýváme \textbf{doplněk} (komplement) množiny $A$ v množině $B$. Značíme $A_B^\prime.$
\end{definition}

\begin{definition}
  Nechť $A, B$ jsou dvě množiny. Jejich \textbf{sjednocením} nazveme takovou množinu, která obsahuje ty prkvy, které patří alespoň do jedné z množin $A, B$. Zapisujeme $A \cup B.$
\end{definition}

----
Honzo neschovávej se už píšeme


průnik mn\\
vennovy diag \\
disjunktní mn\\
rozdíl mn\\
de morganova pravidla plus dk!!\\

\begin{pozn}[Číselné množiny]
  Rozlišujeme následující základní číselné množiny:
  \begin{itemize}
    \item $\mathbb{N}$: přirozená čísla $(1, 2, 3, \dots)$,
    \item $\mathbb{Z}$: celá čísla $(\dots, -2, -1, 0, 1, 2, \dots)$,
    \item $\mathbb{Q}$: racionální čísla $(3/5, 0,\overline{3})$,
    \item $\mathbb{R}$: reálná čísla $(e, \pi)$,
    \item $\mathbb{C}$: komplexní čísla $(3+2i)$
  \end{itemize}
\end{pozn}

Iracionální čísla ($\mathbb{I}$) jsou doplněk racionálních v $\mathbb{R}$.

\begin{definition}
  \textbf{Celá čísla} jsou čísla, která vyjadřují počty prvků množin, čísla k nim opačná a číslo 0.
\end{definition}

\begin{definition}
  \textbf{Racionálním číslem} nazveme takové číslo $a = \frac{k}{l}, k, l \in \mathbb{Z}$, a $p,q$ jsou nesoudělná.
\end{definition}

\begin{pozn}
  Přirozená čísla zapisujeme pomocí číslic 0--9 a chápeme je takto:
  $$4503=4\cdot 10^3+5\cdot 10^2 + 0 \cdot 10^1 + 3\cdot 10^0.$$
  Každé racionální číslo je v desítkové soutavě vyjádřeno buď nekonečným desetinným rozvojem nebo neukončeným periodickým rozvojem. Iracionální číslo je vyjádřeno neukončeným neperiodickým rozvojem.
\end{pozn}

\begin{definition}
  \textbf{Reálnými čísly} nazýváme všechna čísla, která jsou velikostmi úseček.
\end{definition}

\begin{definition}
  Nechť $a,b \in \mathbb{R},$ kde $a<b$. Pak množiny takových $x\in \mathbb{R},$ že $a\leq x\leq b$ (resp. $a < x < b$, resp. $a < x$, resp. $a \leq x < b$ atd.) nazýváme uzavřeným (resp. otevřeným, resp. neomezeným zleva otevřeným, resp.zprava uzavřeným, zleva otevřeným atd.) \textbf{intervalem}. Zapisujeme $\left<a,b\right>$ (resp. $\left(a,b\right)$, resp. $(a, \infty)$, resp. $\left<a, b\right)$)
\end{definition}

für Honza: periodický rozvoj čísel, množina komplexních čísel, imaginární čísla -- něco z toho už možná je, nevim

\begin{example}[SÚM 169/8]
  Označme $M$ množinu všech dvojciferných přirozených čísel delitelných šesti a $N$ všechn dělitelů čísla 210, kteří jsou různí od čísla 1 a 210. Určete, která z množin má větší počet prvků, a vypište všechny prvky, které mají obě množiny stejné.
  \begin{align*}
    M & = \left\{12, 18, 24, 30, 36, 42, 48, 54, 60, 66, 72, 78, 84, 90, 96\right\}\\
    210 & = 2\cdot 3 \cdot 5  \cdot 7 \textrm{ -- hledáme násobky všech podmnožin těchto čísel} \\
    N  & = \left\{2,3,5,6,7, 10, 14, 15, 21, 30, 35, 42, 70, 105\right\} \\
    |M| & = 15, |N| = 14, M \cap N = \left\{30, 42\right\}
  \end{align*}

  \rm Množina $M$ má více prvků a společná jsou čísla 30 a 42.
\end{example}

\begin{example}[SÚM 171/26]
  $M$ je množina šech reálných čísel $x$, která splňují nerovnosti $-2<x<5$, $N$ je mn. všech reálných čísel $y$, která splňují nerovnost $|y|<4$. Určete množinu $R=M\cup N$ a $S = M\cap N.$ \hfill $R = (-4,5), S=(-2,4).$
\end{example}

\begin{example}[SÚM 172/29f]
  Znázorněte a určete výsledný interval: $(a,a+2)\cap (a-1,a+1),$ kde $a>0.$\hfill$(a,a+1)$
\end{example}

\begin{example}[SÚM (172/33)]
  Je dána kružnice $k$ se středem v bodě  $S$ a poloměrem $r$. Množinu všech bodů uvnitř kružnice označte $A$. Nakreslete rovnostranný trojúhelník $ESD$, jehož jeden vrchol je ve středu dané kružnice a délky stran jsou rovny velikosti jejího průměru. Množinu vnitřních bodů tohoto trojúhelníka ozn. $B$. Díle sestrojte osu úhlu $ESD$ a množinu bodů této přímky označte $C$. Nakreslete samostatné obrázky pro:
  \begin{itemize}
    \item $(A\cap B)\cup C,$
    \item $(A\cup C) \cap (B\cup C),$
    \item $(A\cap B) \cup (B\cap C),$
    \item $(A\cup C) \cap B$.
  \end{itemize}
\end{example}

\begin{example}[SÚM 173/34]
  Pro která $x$ je interval:
  \begin{enumerate}[a.]
    \item $\left<2x,x+3\right>$ částí intervalu $(2,7)$? \hfill $x \in (1,3)$
    \item $(x,5)$ částí intervalu $\left(-1,x+1\right)$? \hfill $x\in (4,5)$
    \item $(x,x+3)$ částí intervalu $\left<5,8\right>$? \hfill $x=5$
    \item $\left<x,2x-1\right>$ částí intervalu $\left<-2,5\right>$? \hfill $x\in\left<-2,5\right>$
    \item $\left<3x,2x+1\right>$ částí intervalu $(3,6)$? \hfill $x\in \left\{\right\}$
  \end{enumerate}
\end{example}

\begin{example}[SÚM 173/35]
  Nechť $M = (a,b), N = (1,8), Q = (1,5)$. Určete $a,b \in \mathbb{R}$ tak, aby platilo $M\cap N = Q$.\hfill $a\in \left(-\infty, 1\right>, b=5$
\end{example}

\begin{example}[SÚM 173/37*]
  Je dán trojúhelník $ABC$. Uvažujme množinu $M$ všech bodů tohoto trojúhelníka, pro které platí $|AX| \geq |BX| \geq |CX|.$ Pomocí velikosti stran a úhlů troj. $ABC$ vyjádřete podmínky pro to, aby:
  \begin{enumerate}[a.]
    \item $X$ byla pětiúhelník, \hfill $\gamma > 90^\circ, \alpha < \beta$
    \item $X$  je jeden bod, \hfill $\alpha = 90^\circ$
    \item $X$ je prázdná.\hfill $\alpha > 90^\circ$
  \end{enumerate}
\end{example}


\begin{example}[SÚM 174/42]
  Jsou dány množiny $M=\left\{1,2;3;4\right\},N=\left\{x;y;z\right\}.$ Uveďte alespoň jeden příklad na zobrazení množiny
  \begin{enumerate}[a.]
    \item $M$ do $N$\hfill $1,2\rightarrow x, 3\rightarrow y, 4 \rightarrow y$
    \item $N$ do $M$ \hfill $x\rightarrow 1,2, y\rightarrow 3, z\rightarrow 3$
    \item $M$ na $N.$ \hfill $1,2\rightarrow x, 3 \rightarrow y, 4 \rightarrow z$
  \end{enumerate}
\end{example}

\begin{example}[SÚM 174/46]
  Kolik je všech zobrazení (pod)množiny $\left\{a,b,c,d\right\}$ do (na) množiny $\left\{1,2\right\}$?\hfill \rm 81
\end{example}

\begin{example}[SÚM 106/20]
  Převeďte na obyčejné zlomky:
  \begin{enumerate}[a.]
    \item $0,\overline{27}$\hfill $\frac{27}{99}\frac{3}{11}$
    \item $0,\overline{6}$ \hfill $\frac{2}{3}$
    \item $2,\overline{345}$ \hfill $2+\frac{345}{999}=\frac{781}{333}$
    \item $0,\overline{1234}$\hfill $\frac{1234}{9999}$
    \item $0,7\overline{2}$\hfill $\frac{7}{10}+\frac{2}{90}=\frac{13}{18}$
    \item $0,1\overline{36}$\hfill $\frac{1}{10}+\frac{36}{990}=\frac{3}{22}$
    \item $0,7\overline{27}$\hfill $\frac{7}{10}+\frac{27}{990}=\frac{8}{11}$
    \item $3,39\overline{85}$\hfill $3+\frac{39}{100}+\frac{85}{9900}=\frac{33646}{9900}$
  \end{enumerate}
\end{example}

\begin{example}[SÚM 107/21]
  Proveďte:
  \begin{enumerate}[a.]
    \item $0,\overline{4}+0,\overline{12}$ \hfill $\frac{4}{9}+\frac{12}{9}=\frac{16}{9}$
    \item $0,\overline{7}+0,\overline{35}$  \hfill $\frac{112}{99}$
    \item $0,\overline{47}+0,\overline{023}$ \hfill $\frac{5470}{10989}$
    \item $0,\overline{47}+0,0\overline{23}$ \hfill $\frac{493}{990}$
    \item $0,5\overline{354}+0,\overline{85}$\hfill $1,394021\dots$
    \item $2,\overline{35}-1,\overline{231}$\hfill$ \frac{4111}{3663}$
    \item $1,\overline{25}-0,\overline{773}$ \hfill $\frac{5261}{10989}$
  \end{enumerate}

\begin{example}[SÚM 107/22*]
  Proveďte:
  \begin{enumerate}[a.]
    \item $1,\overline{2}\cdot 1,\overline{18}$\hfill $\left(1+\frac{2}{9}\right)\left(1+\frac{18}{99}\right)=\frac{11}{9}\cdot \frac{117}{99}=\frac{13}{9}$
    \item $0,\overline{32}\cdot 1,\overline{3}$\hfill $\frac{128}{297}$
  \end{enumerate}
\end{example}

\begin{example}[SÚM 107/23*]
  Řešte rovnici:
  \begin{enumerate}
    \item $0,\overline{25}x + 0,\overline{31}x = 1,\overline{13}$ \hfill $x=2$
    \item $2,\overline{64}x - 3,\overline{48} = 1,\overline{48}$  \hfill $x = 3$
  \end{enumerate}
\end{example}

končím ruším nesleduju tě

\end{example}

\section{Výroková logika}
\begin{definition}
  \textbf{Výrokem} nazýváme každou oznamovací větu, která je buď pravdivá, nebo nepravdivá. \textbf{Pravdivostní hodnotou} výroku rozumíme jeho pravdivost / nepravdivost.
\end{definition}

\begin{definition}
  \textbf{Negací výroku} $V$ nazýváme výrok $V^\prime$, který má opačnou pravdivostní hodnotu než výrok $V$.
\end{definition}

\begin{pozn}
  \textbf{Kvantifikované výroky} jsou výroky, které uvádějí počet objektů. Pro to lze použít
  \begin{itemize}
    \item obecný kvantifikátor $\forall$ (pro všechno platí),
    \item existenční kvantifikátor $\exists$ (existuje alespoň jeden, že pro něj platí) a
    \item zesílený existenční kvantifikátor $\exists !$ (existuje právě jeden, že pro něj platí) a
  \end{itemize}
\end{pozn}

\begin{definition}
  \textbf{Složeným výrokem} rozumíme více výroků spojených logickými spojkami:
  \begin{center}
    \begin{tabular}{l | c c}
      název & zápis & význam \\
      \hline
      negace & $X^\prime$ & není pravda, že \\
      konjunkce & $X\land Y$ & $X$ a $Y$ platí současně \\
      alternativa & $X\lor Y$ & platí alespoň jedno z $X,Y$\\
      implikace & $X\implies Y$ & jestliže $X$, pak $Y$\\
      ekvivalence & $X\iff Y$ & $X$ platí právě tehdy, když platí $Y$
    \end{tabular}
  \end{center}
\end{definition}


\begin{pozn}
  Pravdivostní hodnoty výrokových formulí s logickou spojkou:
  \begin{center}
    \begin{tabular}{c c | c c | c c c c}
      $X$ & $Y$ & $X^\prime$ & $Y^\prime$ & $X\land Y$ & $X\lor Y$ & $X\implies Y$ & $X\iff Y$ \\
      \hline
      1 & 1 & 0 & 0 & 1 & 1 & 1 & 1 \\
      1 & 0 & 0 & 1 & 0 & 1 & 0 & 0 \\
      0 & 1 & 1 & 0 & 0 & 1 & 1 & 0 \\
      0 & 0 & 1 & 1 & 0 & 0 & 1 & 1 \\
    \end{tabular}
  \end{center}
\end{pozn}

\begin{definition}
  Výrazy sestavené z výrokových proměnných, závorek a logických spojek nazýváme \textbf{výrokové formule}.
\end{definition}

\begin{definition}
  Výroková formule, která nabývá pravdivostní hodnoty 1 bez ohledu na pravdivostní hodnoty elementárních výroků, se nazývá \textbf{tautologie}.
\end{definition}

\begin{veta}
  Pro každé dva výroky $X,Y$ platí:
  \begin{enumerate}[$i.$]
    \item $(X\lor Y)^\prime = X^\prime \land Y^\prime$,
    \item $(X\land Y)^\prime = X^\prime \lor Y^\prime$,
    \item $(X\implies Y)^\prime = X\land Y^\prime$ a
    \item $(X\iff Y)^\prime = (X\land Y^\prime) \lor (X^\prime \land Y)$
  \end{enumerate}
\end{veta}

\begin{definition}
  Nechť $X\implies Y$ je implikace. Pak
  \begin{enumerate}[$i.$]
    \item implikaci $Y\implies X$ nazýváme \textbf{obrácením} a
    \item implikaci $Y^\prime \implies X^\prime$ nazýváme \textbf{obměnou}
  \end{enumerate}
  původní implikace.
\end{definition}

\begin{veta}
  Implikace a její obměna mají touž pravdivostní hodnotu.
\end{veta}

\begin{pozn}
  Implikace a její obrácení nemusí vždy mít touž pravdivostní hodnotu.
\end{pozn}

\begin{definition}
  \textbf{Výroková forma} je tvrzení obsahující proměnné. Po dosazení konstant za proměnné dostáváme výrok.
\end{definition}

\begin{pozn}
  Důležitému netriviálnímu a dostatečně obecnému výroku nebo výrokové formě s matematickým obsahem říkáme \textbf{věta}.
\end{pozn}

\begin{example}[SMP 143/4]
  Jsou následující výroky tautologie?
  \rm
  \begin{enumerate}[a.]
    \item $\left[(A\implies B)\land A\right]\implies B$
    \begin{center}
      \begin{tabular}{c c | c c c}
        $A$ & $B$ & $A \implies B$ & $(A\implies B)\land A$ & $(A\implies B)\land A]\implies B$ \\
        \hline
        1 & 1 & 1 & 1 & 1 \\
        1 & 0 & 0 & 0 & 1 \\
        0 & 1 & 1 & 0 & 1 \\
        0 & 0 & 1 & 0 & 1
      \end{tabular}
    \end{center}
    Výrok je tautologií.
    \item $\left[(A\implies B)\land B^\prime\right]\implies A^\prime$
    \begin{center}
      \begin{tabular}{c c c c | c c c}
        $A$ & $B$ & $A^\prime$ & $B^\prime$ &  $A \implies B$ & $(A\implies B)\land B^\prime$ & $\left[(A\implies B)\land B^\prime\right]\implies A^\prime$ \\
        \hline
        1 & 1 & 0 & 0 & 1 & 0 & 1 \\
        1 & 0 & 0 & 1 & 0 & 0 & 1 \\
        0 & 1 & 1 & 0 & 1 & 0 & 1 \\
        0 & 0 & 1 & 1 & 1 & 1 & 1
      \end{tabular}
    \end{center}
    Výrok je tautologií.
  \end{enumerate}
\end{example}

\begin{example}[SMP 144/8]
  Na modelu kolejiště je možno uvést do pohybu tři vlakové soupravy A,B,C. V daném okamžiku je jejich
situace charakterizována formulí $$\left[(A^\prime \lor B^\prime) \implies C\right]\land\left[(A \lor C) \implies B^\prime\right].$$ Které soupravy jsou v pohybu?

\rm Napišme tabulku pravdivostních hodnot.
\begin{center}
  \begin{tabular}{c c c c c | c c c c c}
    $A$ & $B$ & $C$ & $A^\prime$ & $B^\prime$ & $A^\prime \lor B^\prime$ & $(A^\prime \lor B^\prime) \implies C$ & $A\lor C$ & $(A \lor C) \implies B^\prime $ & celkem\\
    \hline
    1 & 1 & 1 & 0 & 0 & 0 & 1 & 1 & 0 & 0 \\
    1 & 1 & 0 & 0 & 0 & 0 & 1 & 1 & 0 & 0 \\
    1 & 1 & 0 & 0 & 0 & 1 & 1 & 1 & 1 & 1 \\
    1 & 0 & 1 & 0 & 1 & 1 & 0 & 1 & 1 & 0 \\
    0 & 1 & 1 & 1 & 0 & 1 & 1 & 1 & 0 & 0 \\
    0 & 1 & 0 & 1 & 0 & 1 & 0 & 0 & 1 & 0 \\
    0 & 0 & 1 & 1 & 1 & 1 & 1 & 1 & 1 & 1 \\
    0 & 0 & 0 & 1 & 1 & 1 & 0 & 0 & 0 & 0
  \end{tabular}
\end{center}
Buď jsou v provozu soupravy $A$ a $C$ nebo jen souprava $C$.
\end{example}

\begin{example}[SMP 144/10]
  Květa si pozvala na oslavu svých osmnáctin přátele. Uvažuje takto:
  \begin{enumerate}[a.]
    \item Alena a Boris chodí vždycky spolu. Přijdou oba, nebo ani jeden.
    \item Přijde Boris nebo Dan, ale určitě ne oba.
    \item Když přijde Alena, pak přijde i Eva.
    \item Když Eva nepřijde, nepřijde Dan.
  \end{enumerate}
  S jakým největším počtem přátel může Květa počítat? Vyplývá z Květiny úvahy, že může nastat situace, kdy nepřijde ani jeden z pozvaných?

  \rm Přeložme tato tvrzení symbolicky.
  \begin{enumerate}[a.]
    \item $A\iff B$
    \item $(B^\prime \land D) \lor (B \land D^\prime)$
    \item $A\implies E$
    \item $E^\prime \implies D^\prime$
  \end{enumerate}
  Protože alespoň jeden z dvojice Boris, Dan nemůže přijít a zbytek podmínek si neodporují, na oslavu můžou přijít nejvýše tři lidé.

  Situace, že nepřijde nikdo nastat nemůže, protože výdy přijde buď Boris, nebo Dan.
\end{example}

\begin{example}[SMP 144/12]
  Trenér se věnuje trojici gymnastů -- Adamovi, Břéťovi a Čeňkovi. Rozhodněte, koho vyšle na kontrolní
závod, jestliže splní tyto tři podmínky:
\begin{itemize}
  \item Tělovýchovnou jednotu budou reprezentovat nejvýše dva závodníci, přitom pojede aspoň jeden.
  \item Pojede Adam nebo Čeňek, ale určitě ne oba součastně.
  \item Nepojede-li Čeňek, pak nepojede ani Břéťa.
\end{itemize}

\rm Rozdělme příklad na dva případy.
\begin{enumerate}[$i.$]
  \item pojede Adam: pak nepojede Čeněk a tedy ani Břéťa,
  \item pojede Čeněk: pak může jet i Břéťa.
\end{enumerate}

Buď pojede Adam sám nebo Čeněk sám nebo Čeněk s Břéťou.

\end{example}

\begin{example}[SMP 143/13]
  V souvislosti s otevřením další části metra uvažuje komise o účelnosti autobusových linek A, B, C, D. Je přitom třeba vzít v úvahu tyto tři skutečnosti:
  \begin{itemize}
    \item  Aspoň jedna z linek A, B, C bude zrušena.
    \item Bude-li zrušena linka A, pak bude zachována linka B nebo D.
    \item Nebude-li zrušena linka C, pak bude zrušena linka A nebo D.
  \end{itemize}
  Najděte všechna řešení a rozhodněte, zda stačí zachovat jen jednu z linek.

  \rm Pokud linka $X$ jezdí, píšeme $X$, v opačném případě $X^\prime.$ Podmínky převedeme jako:
  \begin{itemize}
    \item $A^\prime\lor B^\prime \lor C^\prime$
    \item $A^\prime \implies (B\lor D)$
    \item $C\implies (A^\prime\lor D^\prime)$
  \end{itemize}
  Sestavme tabulku (\uv{celkem} značí konkunkci všech tří podmínek výše):
  \begin{widetext}
    \begin{center}
      \begin{tabular}{c c c c | c c c c c | c}
        $A$ & $B$ & $C$ & $D$ & $A^\prime\lor B^\prime \lor C^\prime$ & $B\lor D$ & $A^\prime \implies (B\lor D)$ & $A^\prime \lor D^\prime$ & $C\implies (A^\prime\lor D^\prime)$ & celkem \\
        \hline
        1 & 1 & 1 & 1 & 0 & 1 & 1 & 0 & 0 & 0 \\
        1 & 1 & 1 & 0 & 0 & 1 & 1 & 1 & 1 & 0 \\
        1 & 1 & 0 & 1 & 1 & 1 & 1 & 0 & 0 & 0 \\
        1 & 1 & 0 & 0 & 1 & 1 & 1 & 1 & 1 & 1 \\
        1 & 0 & 1 & 1 & 1 & 1 & 1 & 0 & 0 & 0 \\
        1 & 0 & 1 & 0 & 1 & 0 & 1 & 1 & 0 & 1 \\
        1 & 0 & 0 & 1 & 1 & 1 & 1 & 0 & 1 & 0 \\
        1 & 0 & 0 & 0 & 1 & 0 & 1 & 1 & 1 & 1 \\
        0 & 1 & 1 & 1 & 1 & 1 & 1 & 1 & 1 & 1 \\
        0 & 1 & 1 & 0 & 1 & 1 & 1 & 1 & 1 & 1 \\
        0 & 1 & 0 & 1 & 1 & 1 & 1 & 1 & 1 & 1 \\
        0 & 1 & 0 & 0 & 1 & 1 & 1 & 1 & 1 & 1 \\
        0 & 0 & 0 & 1 & 1 & 1 & 1 & 1 & 1 & 1 \\
        0 & 0 & 1 & 0 & 1 & 0 & 0 & 1 & 1 & 0 \\
        0 & 0 & 1 & 1 & 1 & 1 & 1 & 1 & 1 & 1 \\
        0 & 0 & 0 & 0 & 1 & 0 & 0 & 1 & 1 & 0 \\
      \end{tabular}
    \end{center}
  \end{widetext}
  Výsledek je patrný z tabulky. Může nastat situace, že bude jezdit jediná linka, a to buď $A$, nebo $D$.
\end{example}

\begin{example}[ŘMÚ 274/28.5]
  \begin{enumerate}[a.]
    \item K implikaci \uv{jestliže funkce $f$ má v každém bodě intervalu $(a,b)$ kladnou derivaci, pak funkce $f$ je v intervalu $(a,b)$ rostoucí} utvořte negaci, obměněnou a obrácenou implikaci a stanovte jejich pravdivostní hodnoty.

    {\rm negace: \uv{funkce $f$ má v každém bodě intervalu $(a,b)$ kladnou derivaci a funkce $f$ není v intervalu $(a,b)$ rostoucí}, obměna: \uv{jestliže funkce $f$ není v intervalu $(a,b)$ rostoucí, pak funkce $f$ alespoň v jednom bodu intervalu $(a,b)$ nemá kladnou derivaci}, obrácení: \uv{jestliže funkce $f$ je v intervalu $(a,b)$ rostoucí, pak funkce $f$ má v každém bodě intervalu $(a,b)$ kladnou derivaci}}
    \item Negujte výroky:
    \begin{enumerate}[1.]
      \item Aspoň jeden příklad jsem vyřešil správně.
      \item V tomto sadě je aspoň dvacet jabloní.
      \item Nejvýše šest žáků naší třídy prospělo s vyznamenáním.
      \item Každý den vstávám v sedm hodin.
    \end{enumerate}
    \vspace{3em}
    {\rm
      \begin{enumerate}[1.]
        \item Všechny příklady jsem vyřešil špatně.
        \item V tomto sadě je nejvýše devatenáct jadbloní.
        \item Alespoň sedm žáků naší třídy prospělo s vyznamenáním.
        \item Alespoň v jeden den nevstávám v sedm hodin.
      \end{enumerate}
    }
    \item Když si dám kávu, dám si i moučník. Nedám-li si zmrzlinu, nedám si moučník.
    \begin{enumerate}[1.]
      \item Vyplývá z uvedeného, že dám-li si zmrzlinu, pak si nedám kávu?\hfill {\rm ne}
      \item Vyplývá z uvedeného, že když si dám kávu, dám si i zmrzlinu?\hfill {\rm ano}
    \end{enumerate}
  \end{enumerate}
\end{example}

\begin{example}[ŘMÚ 275/28.6]
  \begin{enumerate}[a.]
    \item K implikaci \uv{Je-li dané přirozené číslo dělitelné dvěma a zároveň třemi, pak je dělitelné šesti}  utvořte
negaci, obměněnou a obrácenou implikaci.
\item Nebude-li ráno pršet, pojedeme na chalupu. Zjišťujeme, že ráno prší. Usuzujeme správně, když z uvedeného
odvodíme, že na chalupu nepojedeme?
\item Pro práci strojů A, B, C platí dvě podmínky: Nesmí pracovat pouze stroj B. Když pracuje stroj A, pak musí
být v chodu i stroj B. Navrhněte síť, která bude pro tuto trojici strojů kontrolním zařízením a bude
signalizovat nesplnění aspoň jedné z uvedených podmínek.
  \end{enumerate}
  \vspace{3em}
  \rm
  \begin{enumerate}[a.]
    negace: \uv{Dané přirozené číslo je dělitelné dvěma a zároveň třemi a není dělitelné šesti}, obměna: \uv{Jetliže dané přirozené číslo není dělitelné šesti, pak není dělitelné dvojkou nebo trojkou}, obrácení: \uv{Jestliže je dané přirozené číslo dělitelné šesti, pak je dělitelné dvěma a třemi}.
    \item ne
    \item Podmínky lze zapsat jako:
      $$B\implies (A\lor C), A\implies B.$$
    Z tabulky pravdivostních hodnot lze vyvodit, že nevyhovují jen možnosti $A, B^\prime, C$ a $A, B^\prime, C^\prime$.
  \end{enumerate}
\end{example}

\begin{example}[SÚM 240/1]
  Počet úhlopříček vypuklého $n$-úhelníka se vypočítá podle vzorce $P_n=\frac{n(n-3)}{2},$ kde $n$ je přirozené číslo větší než tři. Dokažte jeho správnost matematickou indukcí.

  \rm \begin{enumerate}[$i.$]
    \item $n=4:$
    $$P_4 = \frac{4(4-3)}{2}=2 \rightarrow \textrm{platí}$$
    \item úvaha: Přidáním $n$-tého bodu se zvýší počet úhlopříček o $n-3$ (úhlopříčky mezi novým bodem a těmi starými, se kterými nemá společnou stranu) a $1$ (strana, jejíž koncové body jsou ty body, se kterými má $n$-tý bod společnou stranu, se přemění v úhlopříčku) $\rightarrow$ celkem přibyde $n-2$ úhlopříček. \\
    Chceme: $P_n=\frac{n(n-3)}{2}\implies P_{n+1}=\frac{(n+1)(n-2)}{2}$. Předpokládejme, že $P_n=\frac{n(n-3)}{2}$.

    \begin{align*}
      P_{n+1} & =P_n + (n-1) = \frac{n(n-3)}{2}+n-1 \\
      & =\frac{n(n-3)+2n-2}{2} = \frac{n^2-n-2}{2}\\
      & = \frac{(n+1)(n-2)}{2} \rightarrow \textrm{platí}
    \end{align*}
\end{enumerate}
\end{example}

\begin{example}[SÚM 240/5]
  Dokažte matematickou indukcí, že číslo $Q_n=5^{n+1}+6^{2n-1}$ je dělitelné číslem 31 pro každé přirozené číslo $n$.

  \rm \begin{enumerate}[$i.$]
    \item $n=1:$
    $$Q_1=5^2+6^1=31 \rightarrow \textrm{platí}$$
    \item chceme: $31 \, | \, Q_n \implies 31 \, | \, Q_{n+1}$. Předpokládejme, že $31 \, | \, Q_n.$

    \begin{align*}
      Q_{n+1}&=5^{n+2}+6^{2(n+1)-1}=5^{n+2}+6^{2n+1}\\
      & = 5\cdot 5^{n+2} + 6^{2n+1} = 5V_n - 5\cdot 6^{2n-1}+6^{2n+1} \\
      & = 5V_n + 6 ^{2n-1}(6^2-5)=5V_n + 31 \cdot 6^{2n-1} \rightarrow \textrm{platí}
    \end{align*}
\end{enumerate}
\end{example}

\begin{example}[SÚM 241/8]
  Dokažte, že součet třetích mocnin tří po sobě jdoucích přirozených čísel je dělitelný devíti.

  \rm Zapišme součet tří po sobě jdoucích přirozených čísel jako $a_n = (a-1) ^3 + a^3 + (a+1)^3 = 3a^3+6a.$
  \begin{enumerate}[$i.$]
    \item $a = 2:$
    $$a_2 = 1^3 + 2^3 + 3 ^3 = 36 \rightarrow \textrm{platí}$$
    \item chceme: $9\, | \, a_n \implies 9 \, | \, a_{n+1}$. Předpokládejme, že $9\, | \, a_n.$
    \begin{align*}
      a_{n+1} & = a^3 + (a+1)^3 + (a+2)^3 = \\
      & = a^3 + a^3 + 3a^2 + 3a + 1 + a^3 + 3\cdot 2\cdot a^2 + 3\cdot a\cdot 2^2 + 2^3\\
      &= 3a^3 + 9a^2 + 15a+9\\
      &= a_n + 9a^2 + 9a+ 9 = a_n + 9(a^2+a+1) \rightarrow \textrm{platí}
    \end{align*}
\end{enumerate}
\end{example}

\begin{example}[SÚM 241/9]
  Dokažte matematickou indukcí, že číslo $V_n = n^3 + 11n$ je dělitelné šesti pro každé přirozené číslo n.
  \begin{enumerate}[$i.$]
    \item $a = 1:$
    $$a_1 = 1^3 + 11\cdot1 = 12 \rightarrow \textrm{platí}$$
    \item chceme: $6\, | \, V_n \implies 6 \, | \, V_{n+1}$. Předpokládejme, že $6\, | \, V_n$.
    \begin{align*}
      V_{n+1} & = (n+1)^3 + 11\cdot(n+1) = \\
      & = n^3 + 3n^2 + 3n + 1 + 11n + 11\\
      &= n^3 + 11n + 3n^2 + 3n + 12\\
      &= V_n + 3n^2 + 3n + 12 = V_n + 12+ 3n(n+1) \rightarrow \textrm{platí}
    \end{align*}
\end{enumerate}
\end{example}

\begin{example}[SÚM 242/18]
  Matematickou indukcí dokažte tyto vzorce:
\begin{enumerate}[a.]
  \item $\frac{1}{1\cdot3}+\frac{1}{3\cdot5}+...+\frac{1}{(2n-1)\cdot(2n+1)} = \frac{n}{2n+1}$
    \begin{enumerate}[$i.$]
      \item $a = 1:$
      $$a_1 = \frac{1}{1\cdot3} = \frac{1}{2\cdot1 + 1}\rightarrow \textrm{platí}$$
      \item chceme: $S_n = \frac{n}{2n+1} \implies S_{n+1} = \frac{n+1}{2n+3}$. Předpokládejme, že $S_n = \frac{n}{2n+1}$.
      \begin{align*}
        S_{n+1} & = S_n + \frac{1}{(2n+1)\cdot(2n+3)} = \\
        & = \frac{n}{2n+1} + \frac{1}{(2n+1)\cdot(2n+3)}\\
        &= \frac{n\cdot(2n+3)+1}{(2n+1)\cdot(2n+3)}\\
        &= \frac{2n^2+3n+1}{(2n+1)\cdot(2n+3)} = \frac{(2n+1)\cdot(n+1)}{(2n+1)\cdot(2n+3)} = \frac{n+1}{2n+3} \rightarrow \textrm{platí}
      \end{align*}
    \end{enumerate}
  \item $\frac{1}{1\cdot4} + \frac{1}{4\cdot7}+...+\frac{1}{(3n-2)(3n+1)} = \frac{n}{3n+1}$
    \begin{enumerate}[$i.$]
      \item $a = 1:$
      $$a_1 = \frac{1}{1\cdot4} = \frac{1}{3\cdot1 + 1}\rightarrow  \textrm{platí}$$
      \item chceme: $S_n = \frac{n}{3n+1} \implies S_{n+1} = \frac{n+1}{3n+4}$. Předpokládejme, že $S_n = \frac{n}{3n+1}$.
      \begin{align*}
        S_{n+1} & = S_n + \frac{1}{(3n+1)(3n+4)} = \\
        & = \frac{n}{3n+1} + \frac{1}{(3n+1)(3n+4)}\\
        &= \frac{n\cdot(3n+4)+1}{(3n+1)(3n+4)}\\
        &= \frac{3n^2+4n+1}{(3n+1)(3n+4)} = \frac{(3n+1)(n+1)}{(3n+1)(3n+4)} = \frac{n+1}{3n+4} \rightarrow \textrm{platí}
      \end{align*}
    \end{enumerate}
  \item $\frac{1}{1\cdot 5}+ \frac{1}{5\cdot9}+...+\frac{1}{(4n-3)(4n+1)} = \frac{n}{4n+1}$
    \begin{enumerate}[$i.$]
      \item $a = 1:$
      $$a_1 = \frac{1}{1\cdot5} = \frac{1}{4\cdot1 + 1}\rightarrow  \textrm{platí}$$
      \item chceme: $S_n = \frac{n}{4n+1} \implies S_{n+1} = \frac{n+1}{4n+5}$. Předpokládejme, že $S_n = \frac{n}{4n+1}$.
      \begin{align*}
        S_{n+1} & = S_n + \frac{1}{(4n+1)(4n+5)} = \\
        & = \frac{n}{4n+1} + \frac{1}{(4n+1)(4n+5)}\\
        &= \frac{n\cdot(4n+5)+1}{(4n+1)(4n+5)}\\
        &= \frac{4n^2+5n+1}{(4n+1)(4n+5)} = \frac{(4n+1)(n+1)}{(4n+1)(4n+5)} = \frac{n+1}{4n+5} \rightarrow \textrm{platí}
      \end{align*}
    \end{enumerate}
  \item $1^2 + 3^2 + 5^2 + ... + (2n-1)^2 = \frac{n(2n-1)(2n+1)}{3}$
    \begin{enumerate}[$i.$]
      \item $a = 1:$
      $$a_1 = 1^2 = \frac{1\cdot1\cdot3}{3}\rightarrow  \textrm{platí}$$
      \item chceme: $S_n = \frac{n(2n-1)(2n+1)}{3} \implies S_{n+1} = \frac{(n+1)(2n+1)(2n+3)}{3}$. Předpokládejme, že $S_n = \frac{n(2n-1)(2n+1)}{3}$.
      \begin{align*}
        S_{n+1} & = S_n + (2n-1)^2 = \\
        & = \frac{n(2n-1)(2n+1)}{3} + \frac{12n^2-12n+3}{3}\\
        &= \frac{4n^3-n+12n^2-12n+3}{3}\\
        &= \frac{4n^3+12n^2-13n+3}{3} = \frac{(n+1)(2n+1)(2n+3)}{3} \rightarrow \textrm{platí}
      \end{align*}
    \end{enumerate}
    \item $1^3 + 2^3 + 3^3 + ... + n^3 = (\frac{n(n+1)}{2})^2$
      \begin{enumerate}[$i.$]
        \item $a = 1:$
        $$a_1 = 1^3 = (\frac{1\cdot2}{2})^2 \rightarrow  \textrm{platí}$$
        \item chceme: $S_n = (\frac{n(n+1)}{2})^2 \implies S_{n+1} =   (\frac{(n+1)(n+2)}{2})^2$. Předpokládejme, že $S_n = (\frac{n(n+1)}{2})^2$.
        \begin{align*}
          S_{n+1} & = S_n + n^3 = \\
          & = (\frac{n(n+1)}{2})^2 + \frac{4n^3}{4}\\
          &= \frac{n^4 + 2n^3 + n^2}{4} + \frac{4n^3}{4}\\
          &= \frac{n^4 + 6n^3 + n^2}{4} \rightarrow \textrm{neplatí?}
        \end{align*}
      \end{enumerate}
\end{enumerate}
\end{example}

\begin{example}[SÚM 242/19]

\end{example}

\begin{example}[SÚM 242/20]

\end{example}

\begin{example}[SMP 127/4]
  Matematickou indukcí dokažte Moivreovu větu: $(cos α + i sin α)^n = cos nα + i sin nα$.
  \begin{enumerate}[$i.$]
    \item $a = 1:$
    $$(cosø + isinø)^1 = cos(1ø) + isin(1ø)$$
    \item chceme: $S_n = (\frac{n(n+1)}{2})^2 \implies S_{n+1} =   (\frac{(n+1)(n+2)}{2})^2$. Předpokládejme, že $S_n = (\frac{n(n+1)}{2})^2$.
    \begin{align*}
      (cosø + isinø)^k = cos(kø) + isin(kø). (cosø + isinø)^{k+1} = cos((k + 1)ø) + isin((k + 1)ø).
   (cosø + isinø)^{k+1} = (cosø + isinø)^k x (cosø + isinø)
        = (cos(kø) + isin(kø)) x (cosø + isinø)
                 = cos(kø)cos(ø) + icos(kø)sin(ø) + isin(kø)cos(ø) - sin(kø)sin(ø)
                 = cos(kø + ø) + isin(kø + ø)
                = cos((k + 1)ø) + isin((k + 1)ø) \rightarrow \textrm{platí}
    \end{align*}
  \end{enumerate}
\end{example}

\begin{example}

\end{example}

\section{Dělitelnost přirozených čísel}
\begin{definition}
  Nechť $a,b\in\mathbb Z.$ Číslo $a$ dělí číslo $b$, jestliže $\exists c \in \mathbb Z: b=ac$. Zapisujeme $a\, | \, b$.
\end{definition}

\begin{definition}
  Nechť $a\in \mathbb R$. Číslo $|a|$ takové, že
  \begin{enumerate}[$i.$]
    \item $a\geq 0 \implies |a| = a$,
    \item $a<0 \implies |a| = - a$
  \end{enumerate}
  nazýváme \textbf{absolutní hodnotou} čísla $a$.
\end{definition}

\begin{veta}[O dělení se zbytkem]
  Nechť $a\in \mathbb Z, b\in \mathbb N.$ Pak $\exists ! q \in \mathbb Z, r\in \mathbb N_0:$
  $$a=bq+r, 0 \leq r < b.$$
\end{veta}

\begin{definition}
  Nechť $a,b\in \mathbb N$. Pak $c$ je \textbf{společným dělitelem} čísel $a,b$, jestliže $c \, | \, a \land c\, | \, b.$
\end{definition}

\begin{definition}
  $d\in \mathbb N$ je \textbf{největší společný dělitel} čísel $a,b \in \mathbb N,$ jestliže jsou splněny zároveň obě podmínky:
  \begin{enumerate}[$i.$]
    \item $d\, | \, a \land d \, | \, b$ a
    \item $\forall c \in \mathbb N: c \, | \, a \land c \, | \, b \implies c \, | \, d.$
  \end{enumerate}
  Takové číslo značíme $d=D(a,b)=(a,b).$
\end{definition}

\begin{definition}
  -- INSERT EUKLIDŮV ALGORITMUS --
\end{definition}

\begin{definition}
  Nechť $a,b\in \mathbb N.$ Tato čísla jsou \textbf{nesoudělná}, jestliže $D(a,b)=1$. V opačném případě jsou \textbf{soudělná}.
\end{definition}

\begin{veta}[Fundamentální věta aritmetiky]
  Nechť $a_1,a_2,b\in \mathbb N, b>1.$ Pak $b \, | \, a_1a_2 \land D(a_1,b)=1\implies b\, | \, a_2.$
\end{veta}

\begin{definition}
  Nechť $a,b\in \mathbb N.$ Pak $c$ je \textbf{společným násobek} čísel $a,b$, jestliže $a \, | \, c \land b\, | \, c.$
\end{definition}

\begin{definition}
  $n\in \mathbb N$ je \textbf{nejmenší společný násobek} čísel $a,b \in \mathbb N,$ jestliže jsou splněny zároveň obě podmínky:
  \begin{enumerate}[$i.$]
    \item $a\, | \, n \land b \, | \, n$ a
    \item $\forall m \in \mathbb N: a \, | \, m \land b \, | \, m \implies m \, | \, n.$
  \end{enumerate}
  Takové číslo značíme $n=n(a,b)=\left [ a,b\right ] .$
\end{definition}

\begin{veta}
  $\forall a,b \in \mathbb N: ab=D(a,b)\cdot n(a,b).$
\end{veta}

\begin{definition}
  Nechť $n\in \mathbb N, n>1.$ Má-li číslo $n$ pouze triviální dělitele ($1 \, | \, n, n \, | \, n$), nazýváme jej \textbf{prvočíslem}. V opačném případě hovoříme o \textbf{čísle složeném}.
\end{definition}

\begin{veta}
  Každé přirozené složené číslo $n$ má alespoň jednoho prvočíselného dělitele $p\leq \sqrt{n}$.
\end{veta}

\begin{veta}
  Prvočísel je nekonečně mnoho.
\end{veta}

\begin{veta}[Základní věta aritmetiky]
  Každé přirozené číslo $n>1$ lze zapsat ve tvaru:
  $$n=p_1^{m_1}\cdot p_2^{m_2} \cdot p_3^{m_3}\cdot \hdots \cdot p_r^{m_r},$$
  kde $p_i,i\in\{ 1, 2, \dots, r \}$ jsou navzájem různá prvočísla, $m_i\in \mathbb N_0$. Toto vyjádření je jednoznačné až na pořadí činitelů a říkáme mu \textbf{rozklad čísla} $n$ \textbf{na součin prvočinitelů}.
\end{veta}

\begin{veta}[Věta o iraciálnosti odmocnin]
  Nechť $n\in \mathbb N.$ Pak platí: Pokud $n$ není druhou mocninou přirozeného čísla, pak odmocnina z $n$ je iracionální.
\end{veta}

\subsection*{Kritéria dělitelnosti}
\begin{veta}
  Nechť $n\in \mathbb N, n=a_k\cdot 10^k+a_{k-1}\cdot 10^{k-1}+\dots + a\cdot 10 + a_0.$ Pak platí:
  \begin{enumerate}[$i.$]
    \item $2 \, | \, n \iff 2 \, | \, a_0$,
    \item $4 \, | \, n \iff 4 \, | \, (10a_1 + a_0)$,
    \item $5 \, | \, n \iff 5 \, | \, a_0$,
    \item $8 \, | \, n \iff 8 \, | \, (10^2a_2 + 10a_1 + a_0)$ a
    \item $10 \, | \, n \iff a_0 = 0$.
  \end{enumerate}
\end{veta}

\begin{proof}
  $$n = 10(a_k\cdot 10^{k-1}+a_{k-1}\cdot 10 ^{k-2}+\dots+a_1)+a_0 = 10l+a_0, l\in \mathbb N$$
\end{proof}

\begin{definition}
  Nechť $n\in \mathbb N, n=a_k\cdot 10^k+a_{k-1}\cdot 10^{k-1}+\dots + a\cdot 10 + a_0.$ Pak číslo
  $$S(n) = \sum_{i=0}^k a_i$$
  nazveme \textbf{ciferným součtem} čísla $n$.
\end{definition}

\begin{veta}
  Nechť $n\in \mathbb N, n=a_k\cdot 10^k+a_{k-1}\cdot 10^{k-1}+\dots + a\cdot 10 + a_0,$ $S(n)$ je ciferný součet čísla $n$. Pak platí:
  \begin{enumerate}[$i.$]
    \item $3\, | \, n \iff 3 \, | \, S(n)$ a
    \item $9\, | \, n \iff 9 \, | \, S(n)$
  \end{enumerate}
\end{veta}

\begin{proof}
  už se mi nechce
\end{proof}


\nocite{*}
\clearpage
{\small\bibliography{references}}

\backmatter
%!TEX root = ../main.tex

\clearpage
\phantomsection
\addcontentsline{toc}{section}{About the Author}
\begin{adjustwidth}{0.1\textwidth}{0.1\textwidth}
\begingroup
\null\vspace{0.2\textheight}
\begin{center}
{\bfseries\Large O autorovi}\par\vspace{2em}

Da, da, da, da, da \\
It's the motherfucking D-O-double-D \\
Da, da, da, da, da \\
You know I'm mobbin' with Honza Romanovský (Yeah, yeah, yeah)
\end{center}
\endgroup
\end{adjustwidth}
\clearpage


\end{document}
