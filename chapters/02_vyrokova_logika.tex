\section{Výroková logika}
\begin{definition}
  \textbf{Výrokem} nazýváme každou oznamovací větu, která je buď pravdivá, nebo nepravdivá. \textbf{Pravdivostní hodnotou} výroku rozumíme jeho pravdivost / nepravdivost.
\end{definition}

\begin{definition}
  \textbf{Negací výroku} $V$ nazýváme výrok $V^\prime$, který má opačnou pravdivostní hodnotu než výrok $V$.
\end{definition}

\begin{pozn}
  \textbf{Kvantifikované výroky} jsou výroky, které uvádějí počet objektů. Pro to lze použít
  \begin{itemize}
    \item obecný kvantifikátor $\forall$ (pro všechno platí),
    \item existenční kvantifikátor $\exists$ (existuje alespoň jeden, že pro něj platí) a
    \item zesílený existenční kvantifikátor $\exists !$ (existuje právě jeden, že pro něj platí).
  \end{itemize}
\end{pozn}

\begin{definition}
  \textbf{Složeným výrokem} rozumíme více výroků spojených logickými spojkami:
  \begin{center}
    \begin{tabular}{l | c c}
      název & zápis & význam \\
      \hline
      negace & $X^\prime$ & není pravda, že \\
      konjunkce & $X\land Y$ & $X$ a $Y$ platí současně \\
      alternativa & $X\lor Y$ & platí alespoň jedno z $X,Y$\\
      implikace & $X\implies Y$ & jestliže $X$, pak $Y$\\
      ekvivalence & $X\iff Y$ & $X$ platí právě tehdy, když platí $Y$
    \end{tabular}
  \end{center}
\end{definition}


\begin{pozn}
  Pravdivostní hodnoty výrokových formulí s logickou spojkou:
  \begin{center}
    \begin{tabular}{c c | c c | c c c c}
      $X$ & $Y$ & $X^\prime$ & $Y^\prime$ & $X\land Y$ & $X\lor Y$ & $X\implies Y$ & $X\iff Y$ \\
      \hline
      1 & 1 & 0 & 0 & 1 & 1 & 1 & 1 \\
      1 & 0 & 0 & 1 & 0 & 1 & 0 & 0 \\
      0 & 1 & 1 & 0 & 0 & 1 & 1 & 0 \\
      0 & 0 & 1 & 1 & 0 & 0 & 1 & 1 \\
    \end{tabular}
  \end{center}
\end{pozn}

\begin{definition}
  Zápisy sestavené ze značek, výrokových proměnných, závorek a logických spojek nazýváme \textbf{výrokové formule}.
\end{definition}

\begin{definition}
  Výroková formule, která nabývá pravdivostní hodnoty 1 bez ohledu na pravdivostní hodnoty elementárních výroků, se nazývá \textbf{tautologie}.
\end{definition}

\begin{veta}
  Pro každé dva výroky $X,Y$ platí:
  \begin{enumerate}[$i.$]
    \item $(X\lor Y)^\prime = X^\prime \land Y^\prime$,
    \item $(X\land Y)^\prime = X^\prime \lor Y^\prime$,
    \item $(X\implies Y)^\prime = X\land Y^\prime$ a
    \item $(X\iff Y)^\prime = (X\land Y^\prime) \lor (X^\prime \land Y)$.
  \end{enumerate}
\end{veta}

\begin{definition}
  Nechť $X\implies Y$ je implikace. Pak
  \begin{enumerate}[$i.$]
    \item implikaci $Y\implies X$ nazýváme \textbf{obrácením} a
    \item implikaci $Y^\prime \implies X^\prime$ nazýváme \textbf{obměnou}
  \end{enumerate}
  původní implikace.
\end{definition}

\begin{veta}
  Implikace a její obměna mají touž pravdivostní hodnotu.
\end{veta}

\begin{pozn}
  Implikace a její obrácení nemusí vždy mít touž pravdivostní hodnotu.
\end{pozn}

\begin{definition}
  \textbf{Výroková forma} je tvrzení obsahující proměnné. Po dosazení konstant za proměnné dostáváme výrok.
\end{definition}

\begin{pozn}
  Důležitému netriviálnímu a dostatečně obecnému výroku nebo výrokové formě s matematickým obsahem říkáme \textbf{věta}.
\end{pozn}

\begin{comment}


\begin{example}[SMP 143/4]
  Jsou následující výroky tautologie?
  \rm
  \begin{enumerate}[a.]
    \item $\left[(A\implies B)\land A\right]\implies B$
    \begin{center}
      \begin{tabular}{c c | c c c}
        $A$ & $B$ & $A \implies B$ & $(A\implies B)\land A$ & $(A\implies B)\land A]\implies B$ \\
        \hline
        1 & 1 & 1 & 1 & 1 \\
        1 & 0 & 0 & 0 & 1 \\
        0 & 1 & 1 & 0 & 1 \\
        0 & 0 & 1 & 0 & 1
      \end{tabular}
    \end{center}
    Výrok je tautologií.
    \item $\left[(A\implies B)\land B^\prime\right]\implies A^\prime$
    \begin{center}
      \begin{tabular}{c c c c | c c c}
        $A$ & $B$ & $A^\prime$ & $B^\prime$ &  $A \implies B$ & $(A\implies B)\land B^\prime$ & $\left[(A\implies B)\land B^\prime\right]\implies A^\prime$ \\
        \hline
        1 & 1 & 0 & 0 & 1 & 0 & 1 \\
        1 & 0 & 0 & 1 & 0 & 0 & 1 \\
        0 & 1 & 1 & 0 & 1 & 0 & 1 \\
        0 & 0 & 1 & 1 & 1 & 1 & 1
      \end{tabular}
    \end{center}
    Výrok je tautologií.
  \end{enumerate}
\end{example}

\begin{example}[SMP 144/8]
  Na modelu kolejiště je možno uvést do pohybu tři vlakové soupravy A,B,C. V daném okamžiku je jejich
situace charakterizována formulí $$\left[(A^\prime \lor B^\prime) \implies C\right]\land\left[(A \lor C) \implies B^\prime\right].$$ Které soupravy jsou v pohybu?

\rm Napišme tabulku pravdivostních hodnot.
\begin{center}
  \begin{tabular}{c c c c c | c c c c c}
    $A$ & $B$ & $C$ & $A^\prime$ & $B^\prime$ & $A^\prime \lor B^\prime$ & $(A^\prime \lor B^\prime) \implies C$ & $A\lor C$ & $(A \lor C) \implies B^\prime $ & celkem\\
    \hline
    1 & 1 & 1 & 0 & 0 & 0 & 1 & 1 & 0 & 0 \\
    1 & 1 & 0 & 0 & 0 & 0 & 1 & 1 & 0 & 0 \\
    1 & 1 & 0 & 0 & 0 & 1 & 1 & 1 & 1 & 1 \\
    1 & 0 & 1 & 0 & 1 & 1 & 0 & 1 & 1 & 0 \\
    0 & 1 & 1 & 1 & 0 & 1 & 1 & 1 & 0 & 0 \\
    0 & 1 & 0 & 1 & 0 & 1 & 0 & 0 & 1 & 0 \\
    0 & 0 & 1 & 1 & 1 & 1 & 1 & 1 & 1 & 1 \\
    0 & 0 & 0 & 1 & 1 & 1 & 0 & 0 & 0 & 0
  \end{tabular}
\end{center}
Buď jsou v provozu soupravy $A$ a $C$ nebo jen souprava $C$.
\end{example}

\begin{example}[SMP 144/10]
  Květa si pozvala na oslavu svých osmnáctin přátele. Uvažuje takto:
  \begin{enumerate}[a.]
    \item Alena a Boris chodí vždycky spolu. Přijdou oba, nebo ani jeden.
    \item Přijde Boris nebo Dan, ale určitě ne oba.
    \item Když přijde Alena, pak přijde i Eva.
    \item Když Eva nepřijde, nepřijde Dan.
  \end{enumerate}
  S jakým největším počtem přátel může Květa počítat? Vyplývá z Květiny úvahy, že může nastat situace, kdy nepřijde ani jeden z pozvaných?

  \rm Přeložme tato tvrzení symbolicky.
  \begin{enumerate}[a.]
    \item $A\iff B$
    \item $(B^\prime \land D) \lor (B \land D^\prime)$
    \item $A\implies E$
    \item $E^\prime \implies D^\prime$
  \end{enumerate}
  Protože alespoň jeden z dvojice Boris, Dan nemůže přijít a zbytek podmínek si neodporují, na oslavu můžou přijít nejvýše tři lidé.

  Situace, že nepřijde nikdo nastat nemůže, protože výdy přijde buď Boris, nebo Dan.
\end{example}

\begin{example}[SMP 144/12]
  Trenér se věnuje trojici gymnastů -- Adamovi, Břéťovi a Čeňkovi. Rozhodněte, koho vyšle na kontrolní
závod, jestliže splní tyto tři podmínky:
\begin{itemize}
  \item Tělovýchovnou jednotu budou reprezentovat nejvýše dva závodníci, přitom pojede aspoň jeden.
  \item Pojede Adam nebo Čeňek, ale určitě ne oba součastně.
  \item Nepojede-li Čeňek, pak nepojede ani Břéťa.
\end{itemize}

\rm Rozdělme příklad na dva případy.
\begin{enumerate}[$i.$]
  \item pojede Adam: pak nepojede Čeněk a tedy ani Břéťa,
  \item pojede Čeněk: pak může jet i Břéťa.
\end{enumerate}

Buď pojede Adam sám nebo Čeněk sám nebo Čeněk s Břéťou.

\end{example}

\begin{example}[SMP 143/13]
  V souvislosti s otevřením další části metra uvažuje komise o účelnosti autobusových linek A, B, C, D. Je přitom třeba vzít v úvahu tyto tři skutečnosti:
  \begin{itemize}
    \item  Aspoň jedna z linek A, B, C bude zrušena.
    \item Bude-li zrušena linka A, pak bude zachována linka B nebo D.
    \item Nebude-li zrušena linka C, pak bude zrušena linka A nebo D.
  \end{itemize}
  Najděte všechna řešení a rozhodněte, zda stačí zachovat jen jednu z linek.

  \rm Pokud linka $X$ jezdí, píšeme $X$, v opačném případě $X^\prime.$ Podmínky převedeme jako:
  \begin{itemize}
    \item $A^\prime\lor B^\prime \lor C^\prime$
    \item $A^\prime \implies (B\lor D)$
    \item $C\implies (A^\prime\lor D^\prime)$
  \end{itemize}
  Sestavme tabulku (\uv{celkem} značí konkunkci všech tří podmínek výše):
  \begin{widetext}
    \begin{center}
      \begin{tabular}{c c c c | c c c c c | c}
        $A$ & $B$ & $C$ & $D$ & $A^\prime\lor B^\prime \lor C^\prime$ & $B\lor D$ & $A^\prime \implies (B\lor D)$ & $A^\prime \lor D^\prime$ & $C\implies (A^\prime\lor D^\prime)$ & celkem \\
        \hline
        1 & 1 & 1 & 1 & 0 & 1 & 1 & 0 & 0 & 0 \\
        1 & 1 & 1 & 0 & 0 & 1 & 1 & 1 & 1 & 0 \\
        1 & 1 & 0 & 1 & 1 & 1 & 1 & 0 & 0 & 0 \\
        1 & 1 & 0 & 0 & 1 & 1 & 1 & 1 & 1 & 1 \\
        1 & 0 & 1 & 1 & 1 & 1 & 1 & 0 & 0 & 0 \\
        1 & 0 & 1 & 0 & 1 & 0 & 1 & 1 & 0 & 1 \\
        1 & 0 & 0 & 1 & 1 & 1 & 1 & 0 & 1 & 0 \\
        1 & 0 & 0 & 0 & 1 & 0 & 1 & 1 & 1 & 1 \\
        0 & 1 & 1 & 1 & 1 & 1 & 1 & 1 & 1 & 1 \\
        0 & 1 & 1 & 0 & 1 & 1 & 1 & 1 & 1 & 1 \\
        0 & 1 & 0 & 1 & 1 & 1 & 1 & 1 & 1 & 1 \\
        0 & 1 & 0 & 0 & 1 & 1 & 1 & 1 & 1 & 1 \\
        0 & 0 & 0 & 1 & 1 & 1 & 1 & 1 & 1 & 1 \\
        0 & 0 & 1 & 0 & 1 & 0 & 0 & 1 & 1 & 0 \\
        0 & 0 & 1 & 1 & 1 & 1 & 1 & 1 & 1 & 1 \\
        0 & 0 & 0 & 0 & 1 & 0 & 0 & 1 & 1 & 0 \\
      \end{tabular}
    \end{center}
  \end{widetext}
  Výsledek je patrný z tabulky. Může nastat situace, že bude jezdit jediná linka, a to buď $A$, nebo $D$.
\end{example}

\begin{example}[ŘMÚ 274/28.5]
  \begin{enumerate}[a.]
    \item K implikaci \uv{jestliže funkce $f$ má v každém bodě intervalu $(a,b)$ kladnou derivaci, pak funkce $f$ je v intervalu $(a,b)$ rostoucí} utvořte negaci, obměněnou a obrácenou implikaci a stanovte jejich pravdivostní hodnoty.

    {\rm negace: \uv{funkce $f$ má v každém bodě intervalu $(a,b)$ kladnou derivaci a funkce $f$ není v intervalu $(a,b)$ rostoucí}, obměna: \uv{jestliže funkce $f$ není v intervalu $(a,b)$ rostoucí, pak funkce $f$ alespoň v jednom bodu intervalu $(a,b)$ nemá kladnou derivaci}, obrácení: \uv{jestliže funkce $f$ je v intervalu $(a,b)$ rostoucí, pak funkce $f$ má v každém bodě intervalu $(a,b)$ kladnou derivaci}}
    \item Negujte výroky:
    \begin{enumerate}[1.]
      \item Aspoň jeden příklad jsem vyřešil správně.
      \item V tomto sadě je aspoň dvacet jabloní.
      \item Nejvýše šest žáků naší třídy prospělo s vyznamenáním.
      \item Každý den vstávám v sedm hodin.
    \end{enumerate}
    \vspace{3em}
    {\rm
      \begin{enumerate}[1.]
        \item Všechny příklady jsem vyřešil špatně.
        \item V tomto sadě je nejvýše devatenáct jadbloní.
        \item Alespoň sedm žáků naší třídy prospělo s vyznamenáním.
        \item Alespoň v jeden den nevstávám v sedm hodin.
      \end{enumerate}
    }
    \item Když si dám kávu, dám si i moučník. Nedám-li si zmrzlinu, nedám si moučník.
    \begin{enumerate}[1.]
      \item Vyplývá z uvedeného, že dám-li si zmrzlinu, pak si nedám kávu?\hfill {\rm ne}
      \item Vyplývá z uvedeného, že když si dám kávu, dám si i zmrzlinu?\hfill {\rm ano}
    \end{enumerate}
  \end{enumerate}
\end{example}

\begin{example}[ŘMÚ 275/28.6]
  \begin{enumerate}[a.]
    \item K implikaci \uv{Je-li dané přirozené číslo dělitelné dvěma a zároveň třemi, pak je dělitelné šesti}  utvořte
negaci, obměněnou a obrácenou implikaci.
\item Nebude-li ráno pršet, pojedeme na chalupu. Zjišťujeme, že ráno prší. Usuzujeme správně, když z uvedeného
odvodíme, že na chalupu nepojedeme?
\item Pro práci strojů A, B, C platí dvě podmínky: Nesmí pracovat pouze stroj B. Když pracuje stroj A, pak musí
být v chodu i stroj B. Navrhněte síť, která bude pro tuto trojici strojů kontrolním zařízením a bude
signalizovat nesplnění aspoň jedné z uvedených podmínek.
  \end{enumerate}
  \vspace{3em}
  \rm
  \begin{enumerate}[a.]
    negace: \uv{Dané přirozené číslo je dělitelné dvěma a zároveň třemi a není dělitelné šesti}, obměna: \uv{Jetliže dané přirozené číslo není dělitelné šesti, pak není dělitelné dvojkou nebo trojkou}, obrácení: \uv{Jestliže je dané přirozené číslo dělitelné šesti, pak je dělitelné dvěma a třemi}.
    \item ne
    \item Podmínky lze zapsat jako:
      $$B\implies (A\lor C), A\implies B.$$
    Z tabulky pravdivostních hodnot lze vyvodit, že nevyhovují jen možnosti $A, B^\prime, C$ a $A, B^\prime, C^\prime$.
  \end{enumerate}
\end{example}

\begin{example}[SÚM 240/1]
  Počet úhlopříček vypuklého $n$-úhelníka se vypočítá podle vzorce $P_n=\frac{n(n-3)}{2},$ kde $n$ je přirozené číslo větší než tři. Dokažte jeho správnost matematickou indukcí.

  \rm \begin{enumerate}[$i.$]
    \item $n=4:$
    $$P_4 = \frac{4(4-3)}{2}=2 \rightarrow \textrm{platí}$$
    \item úvaha: Přidáním $n$-tého bodu se zvýší počet úhlopříček o $n-3$ (úhlopříčky mezi novým bodem a těmi starými, se kterými nemá společnou stranu) a $1$ (strana, jejíž koncové body jsou ty body, se kterými má $n$-tý bod společnou stranu, se přemění v úhlopříčku) $\rightarrow$ celkem přibyde $n-2$ úhlopříček. \\
    Chceme: $P_n=\frac{n(n-3)}{2}\implies P_{n+1}=\frac{(n+1)(n-2)}{2}$. Předpokládejme, že $P_n=\frac{n(n-3)}{2}$.

    \begin{align*}
      P_{n+1} & =P_n + (n-1) = \frac{n(n-3)}{2}+n-1 \\
      & =\frac{n(n-3)+2n-2}{2} = \frac{n^2-n-2}{2}\\
      & = \frac{(n+1)(n-2)}{2} \rightarrow \textrm{platí}
    \end{align*}
\end{enumerate}
\end{example}

\begin{example}[SÚM 240/5]
  Dokažte matematickou indukcí, že číslo $Q_n=5^{n+1}+6^{2n-1}$ je dělitelné číslem 31 pro každé přirozené číslo $n$.

  \rm \begin{enumerate}[$i.$]
    \item $n=1:$
    $$Q_1=5^2+6^1=31 \rightarrow \textrm{platí}$$
    \item chceme: $31 \, | \, Q_n \implies 31 \, | \, Q_{n+1}$. Předpokládejme, že $31 \, | \, Q_n.$

    \begin{align*}
      Q_{n+1}&=5^{n+2}+6^{2(n+1)-1}=5^{n+2}+6^{2n+1}\\
      & = 5\cdot 5^{n+2} + 6^{2n+1} = 5V_n - 5\cdot 6^{2n-1}+6^{2n+1} \\
      & = 5V_n + 6 ^{2n-1}(6^2-5)=5V_n + 31 \cdot 6^{2n-1} \rightarrow \textrm{platí}
    \end{align*}
\end{enumerate}
\end{example}

\begin{example}[SÚM 241/8]
  Dokažte, že součet třetích mocnin tří po sobě jdoucích přirozených čísel je dělitelný devíti.

  \rm Zapišme součet tří po sobě jdoucích přirozených čísel jako $a_n = (a-1) ^3 + a^3 + (a+1)^3 = 3a^3+6a.$
  \begin{enumerate}[$i.$]
    \item $a = 2:$
    $$a_2 = 1^3 + 2^3 + 3 ^3 = 36 \rightarrow \textrm{platí}$$
    \item chceme: $9\, | \, a_n \implies 9 \, | \, a_{n+1}$. Předpokládejme, že $9\, | \, a_n.$
    \begin{align*}
      a_{n+1} & = a^3 + (a+1)^3 + (a+2)^3 = \\
      & = a^3 + a^3 + 3a^2 + 3a + 1 + a^3 + 3\cdot 2\cdot a^2 + 3\cdot a\cdot 2^2 + 2^3\\
      &= 3a^3 + 9a^2 + 15a+9\\
      &= a_n + 9a^2 + 9a+ 9 = a_n + 9(a^2+a+1) \rightarrow \textrm{platí}
    \end{align*}
\end{enumerate}
\end{example}

\begin{example}[SÚM 241/9]
  Dokažte matematickou indukcí, že číslo $V_n = n^3 + 11n$ je dělitelné šesti pro každé přirozené číslo n.
  \begin{enumerate}[$i.$]
    \item $a = 1:$
    $$a_1 = 1^3 + 11\cdot1 = 12 \rightarrow \textrm{platí}$$
    \item chceme: $6\, | \, V_n \implies 6 \, | \, V_{n+1}$. Předpokládejme, že $6\, | \, V_n$.
    \begin{align*}
      V_{n+1} & = (n+1)^3 + 11\cdot(n+1) = \\
      & = n^3 + 3n^2 + 3n + 1 + 11n + 11\\
      &= n^3 + 11n + 3n^2 + 3n + 12\\
      &= V_n + 3n^2 + 3n + 12 = V_n + 12+ 3n(n+1) \rightarrow \textrm{platí}
    \end{align*}
\end{enumerate}
\end{example}

\begin{example}[SÚM 242/18]
  Matematickou indukcí dokažte tyto vzorce:
\begin{enumerate}[a.]
  \item $\frac{1}{1\cdot3}+\frac{1}{3\cdot5}+...+\frac{1}{(2n-1)\cdot(2n+1)} = \frac{n}{2n+1}$
    \begin{enumerate}[$i.$]
      \item $a = 1:$
      $$a_1 = \frac{1}{1\cdot3} = \frac{1}{2\cdot1 + 1}\rightarrow \textrm{platí}$$
      \item chceme: $S_n = \frac{n}{2n+1} \implies S_{n+1} = \frac{n+1}{2n+3}$. Předpokládejme, že $S_n = \frac{n}{2n+1}$.
      \begin{align*}
        S_{n+1} & = S_n + \frac{1}{(2n+1)\cdot(2n+3)} = \\
        & = \frac{n}{2n+1} + \frac{1}{(2n+1)\cdot(2n+3)}\\
        &= \frac{n\cdot(2n+3)+1}{(2n+1)\cdot(2n+3)}\\
        &= \frac{2n^2+3n+1}{(2n+1)\cdot(2n+3)} = \frac{(2n+1)\cdot(n+1)}{(2n+1)\cdot(2n+3)} = \frac{n+1}{2n+3} \rightarrow \textrm{platí}
      \end{align*}
    \end{enumerate}
  \item $\frac{1}{1\cdot4} + \frac{1}{4\cdot7}+...+\frac{1}{(3n-2)(3n+1)} = \frac{n}{3n+1}$
    \begin{enumerate}[$i.$]
      \item $a = 1:$
      $$a_1 = \frac{1}{1\cdot4} = \frac{1}{3\cdot1 + 1}\rightarrow  \textrm{platí}$$
      \item chceme: $S_n = \frac{n}{3n+1} \implies S_{n+1} = \frac{n+1}{3n+4}$. Předpokládejme, že $S_n = \frac{n}{3n+1}$.
      \begin{align*}
        S_{n+1} & = S_n + \frac{1}{(3n+1)(3n+4)} = \\
        & = \frac{n}{3n+1} + \frac{1}{(3n+1)(3n+4)}\\
        &= \frac{n\cdot(3n+4)+1}{(3n+1)(3n+4)}\\
        &= \frac{3n^2+4n+1}{(3n+1)(3n+4)} = \frac{(3n+1)(n+1)}{(3n+1)(3n+4)} = \frac{n+1}{3n+4} \rightarrow \textrm{platí}
      \end{align*}
    \end{enumerate}
  \item $\frac{1}{1\cdot 5}+ \frac{1}{5\cdot9}+...+\frac{1}{(4n-3)(4n+1)} = \frac{n}{4n+1}$
    \begin{enumerate}[$i.$]
      \item $a = 1:$
      $$a_1 = \frac{1}{1\cdot5} = \frac{1}{4\cdot1 + 1}\rightarrow  \textrm{platí}$$
      \item chceme: $S_n = \frac{n}{4n+1} \implies S_{n+1} = \frac{n+1}{4n+5}$. Předpokládejme, že $S_n = \frac{n}{4n+1}$.
      \begin{align*}
        S_{n+1} & = S_n + \frac{1}{(4n+1)(4n+5)} = \\
        & = \frac{n}{4n+1} + \frac{1}{(4n+1)(4n+5)}\\
        &= \frac{n\cdot(4n+5)+1}{(4n+1)(4n+5)}\\
        &= \frac{4n^2+5n+1}{(4n+1)(4n+5)} = \frac{(4n+1)(n+1)}{(4n+1)(4n+5)} = \frac{n+1}{4n+5} \rightarrow \textrm{platí}
      \end{align*}
    \end{enumerate}
  \item $1^2 + 3^2 + 5^2 + ... + (2n-1)^2 = \frac{n(2n-1)(2n+1)}{3}$
    \begin{enumerate}[$i.$]
      \item $a = 1:$
      $$a_1 = 1^2 = \frac{1\cdot1\cdot3}{3}\rightarrow  \textrm{platí}$$
      \item chceme: $S_n = \frac{n(2n-1)(2n+1)}{3} \implies S_{n+1} = \frac{(n+1)(2n+1)(2n+3)}{3}$. Předpokládejme, že $S_n = \frac{n(2n-1)(2n+1)}{3}$.
      \begin{align*}
        S_{n+1} & = S_n + (2n-1)^2 = \\
        & = \frac{n(2n-1)(2n+1)}{3} + \frac{12n^2-12n+3}{3}\\
        &= \frac{4n^3-n+12n^2-12n+3}{3}\\
        &= \frac{4n^3+12n^2-13n+3}{3} = \frac{(n+1)(2n+1)(2n+3)}{3} \rightarrow \textrm{platí}
      \end{align*}
    \end{enumerate}
    \item $1^3 + 2^3 + 3^3 + ... + n^3 = (\frac{n(n+1)}{2})^2$
      \begin{enumerate}[$i.$]
        \item $a = 1:$
        $$a_1 = 1^3 = (\frac{1\cdot2}{2})^2 \rightarrow  \textrm{platí}$$
        \item chceme: $S_n = (\frac{n(n+1)}{2})^2 \implies S_{n+1} =   (\frac{(n+1)(n+2)}{2})^2$. Předpokládejme, že $S_n = (\frac{n(n+1)}{2})^2$.
        \begin{align*}
          S_{n+1} & = S_n + n^3 = \\
          & = (\frac{n(n+1)}{2})^2 + \frac{4n^3}{4}\\
          &= \frac{n^4 + 2n^3 + n^2}{4} + \frac{4n^3}{4}\\
          &= \frac{n^4 + 6n^3 + n^2}{4} \rightarrow \textrm{neplatí?}
        \end{align*}
      \end{enumerate}
\end{enumerate}
\end{example}

\begin{example}[SÚM 242/19]

\end{example}

\begin{example}[SÚM 242/20]

\end{example}

\begin{example}[SMP 127/4]
  Matematickou indukcí dokažte Moivreovu větu: $(\cos \alpha + i \sin \alpha)^n = \cos n\alpha + i \sin n\alpha$.
  \begin{enumerate}[$i.$]
    \item $a = 1:$
    $$(\cos \alpha + i\sin \alpha)^1 = \cos(1\alpha) + i\sin(1\alpha)$$
    \item chceme: $S_n = (\frac{n(n+1)}{2})^2 \implies S_{n+1} =   (\frac{(n+1)(n+2)}{2})^2$. Předpokládejme, že $S_n = (\frac{n(n+1)}{2})^2$.
    \begin{align*}
      (\cos \alpha + i\sin \alpha)^k & = \cos(k\alpha) + i\sin(k\alpha)\cdot  (\cos  \alpha + i\sin\alpha)^{k+1}\\
      &  = \cos((k + 1)\alpha) + i\sin((k + 1)\alpha)\cdot
   (\cos\alpha + i\sin\alpha)^{k+1}\\
   & = (\cos\alpha + i\sin\alpha)^k \cdot (\cos\alpha + i\sin\alpha)\\
   & = (\cos(k\alpha) + i\sin(k\alpha)) \cdot (\cos\alpha + i\sin\alpha)\\
   & = \cos(k\alpha)\cos(\alpha) + i\cos(k\alpha)\sin(\alpha) + i\sin(k\alpha)\cos(\alpha) - \sin(k\alpha)\sin(\alpha)\\
   & = \cos(k\alpha + \alpha) + i\sin(k\alpha + \alpha)\\
  & = \cos((k + 1)\alpha) + i\sin((k + 1)\alpha) \rightarrow \rm{platí}
    \end{align*}
  \end{enumerate}
\end{example}

\begin{example}

\end{example}
\end{comment}
