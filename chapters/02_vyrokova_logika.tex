\section{Výroková logika}
\begin{definition}
  \textbf{Výrokem} nazýváme každou oznamvací větu, která je buď pravdivá, nebo nepravdivá. \textbf{Pravdivostní hodnotou} výroku rozumíme jeho pravdivost / nepravdivost.
\end{definition}

\begin{definition}
  \textbf{Negací výroku} $V$ nazýváme výrok $V^\prime$, který má opačnou pravdivostní hodnotu než výrok $V$.
\end{definition}

\begin{pozn}
  \textbf{Kvantifikované výroky} jsou výroky, které uvádějí počet objektů. Pro to lze použít
  \begin{itemize}
    \item obecný kvantifikátor $\forall$ (pro všechno platí),
    \item existenční kvantifikátor $\exists$ (existuje alespoň jeden, že pro něj platí) a
    \item zesílený existenční kvantifikátor $\exists !$ (existuje právě jeden, že pro něj platí) a
  \end{itemize}
\end{pozn}

\begin{definition}
  \textbf{Složeným výrokem} rozumíme více výroků spojených logickými spojkami:
  \begin{center}
    \begin{tabular}{l | c c}
      název & zápis & význam \\
      \hline
      negace & $X^\prime$ & není pravda, že \\
      konjunkce & $X\land Y$ & $X$ a $Y$ platí současně \\
      alternativa & $X\lor Y$ & platí alespoň jedno z $X,Y$\\
      implikace & $X\implies Y$ & jestliže $X$, pak $Y$\\
      ekvivalence & $X\iff Y$ & $X$ platí právě tehdy, když platí $Y$
    \end{tabular}
  \end{center}
\end{definition}


\begin{example}[SMP 143/4]
  Jsou následující výroky tautologie?
  \rm
  \begin{enumerate}[a.]
    \item $\left[(A\implies B)\land A\right]\implies B$
    \begin{center}
      \begin{tabular}{c c | c c c}
        $A$ & $B$ & $A \implies B$ & $(A\implies B)\land A$ & $(A\implies B)\land A]\implies B$ \\
        \hline
        1 & 1 & 1 & 1 & 1 \\
        1 & 0 & 0 & 0 & 1 \\
        0 & 1 & 1 & 0 & 1 \\
        0 & 0 & 1 & 0 & 1
      \end{tabular}
    \end{center}
    Výrok je tautologií.
    \item $\left[(A\implies B)\land B^\prime\right]\implies A^\prime$
    \begin{center}
      \begin{tabular}{c c c c | c c c}
        $A$ & $B$ & $A^\prime$ & $B^\prime$ &  $A \implies B$ & $(A\implies B)\land B^\prime$ & $\left[(A\implies B)\land B^\prime\right]\implies A^\prime$ \\
        \hline
        1 & 1 & 0 & 0 & 1 & 0 & 1 \\
        1 & 0 & 0 & 1 & 0 & 0 & 1 \\
        0 & 1 & 1 & 0 & 1 & 0 & 1 \\
        0 & 0 & 1 & 1 & 1 & 1 & 1
      \end{tabular}
    \end{center}
    Výrok je tautologií.
  \end{enumerate}
\end{example}

\begin{example}[SMP 144/8]
  Na modelu kolejiště je možno uvést do pohybu tři vlakové soupravy A,B,C. V daném okamžiku je jejich
situace charakterizována formulí $$\left[(A^\prime \lor B^\prime) \implies C\right]\land\left[(A \lor C) \implies B^\prime\right].$$ Které soupravy jsou v pohybu?

\rm Napišme tabulku pravdivostních hodnot.
\begin{center}
  \begin{tabular}{c c c c c | c c c c c}
    $A$ & $B$ & $C$ & $A^\prime$ & $B^\prime$ & $A^\prime \lor B^\prime$ & $(A^\prime \lor B^\prime) \implies C$ & $A\lor C$ & $(A \lor C) \implies B^\prime $ & celkem\\
    \hline
    1 & 1 & 1 & 0 & 0 & 0 & 1 & 1 & 0 & 0 \\
    1 & 1 & 0 & 0 & 0 & 0 & 1 & 1 & 0 & 0 \\
    1 & 1 & 0 & 0 & 0 & 1 & 1 & 1 & 1 & 1 \\
    1 & 0 & 1 & 0 & 1 & 1 & 0 & 1 & 1 & 0 \\
    0 & 1 & 1 & 1 & 0 & 1 & 1 & 1 & 0 & 0 \\
    0 & 1 & 0 & 1 & 0 & 1 & 0 & 0 & 1 & 0 \\
    0 & 0 & 1 & 1 & 1 & 1 & 1 & 1 & 1 & 1 \\
    0 & 0 & 0 & 1 & 1 & 1 & 0 & 0 & 0 & 0
  \end{tabular}
\end{center}
Buď jsou v provozu soupravy $A$ a $C$ nebo jen souprava $C$.
\end{example}

\begin{example}[SMP 144/10]
  Květa si pozvala na oslavu svých osmnáctin přátele. Uvažuje takto:
  \begin{enumerate}[a.]
    \item Alena a Boris chodí vždycky spolu. Přijdou oba, nebo ani jeden.
    \item Přijde Boris nebo Dan, ale určitě ne oba.
    \item Když přijde Alena, pak přijde i Eva.
    \item Když Eva nepřijde, nepřijde Dan.
  \end{enumerate}
  S jakým největším počtem přátel může Květa počítat? Vyplývá z Květiny úvahy, že může nastat situace, kdy nepřijde ani jeden z pozvaných?

  \rm Přeložme tato tvrzení symbolicky.
  \begin{enumerate}[a.]
    \item $A\iff B$
    \item $B^\prime \lor D^\prime$
    \item $A\implies E$
    \item $E^\prime \implies D^\prime$
  \end{enumerate}
  Protože alespoň jeden z dvojice Boris, Dan nemůže přijít a zbytek podmínek si neodporují, na oslavu můžou přijít nejvýše tři lidé.

  Situace, že nepřijde nikdo nastat může.
\end{example}

\begin{example}[SMP 144/12]
  Trenér se věnuje trojici gymnastů -- Adamovi, Břéťovi a Čeňkovi. Rozhodněte, koho vyšle na kontrolní
závod, jestliže splní tyto tři podmínky:
\begin{itemize}
  \item Tělovýchovnou jednotu budou reprezentovat nejvýše dva závodníci, přitom pojede aspoň jeden.
  \item Pojede Adam nebo Čeňek, ale určitě ne oba součastně.
  \item Nepojede-li Čeňek, pak nepojede ani Břéťa.
\end{itemize}

\rm Rozdělme příklad na dva případy.
\begin{enumerate}[$i.$]
  \item pojede Adam: pak nepojede Čeněk a tedy ani Břéťa,
  \item pojede Čeněk: pak může jet i Břéťa.
\end{enumerate}

Buď pojede Adam sám nebo Čeněk sám nebo Čeněk s Břéťou.

\end{example}
