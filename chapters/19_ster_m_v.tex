o\section{Stereometrie -- metrické vlastnosti}
\begin{definition}
    Nechť $A, B \in \mathbb E_3.$ \textbf{Vzdálenost bodů} $A,B$ nazýváme délku
    úsečky $AB$ a označujeme $\rho(A,B).$
\end{definition}

\begin{definition}
    Nechť $A\in \mathbb E_3$ je bod a $\alpha \subseteq \mathbb E_3$ je rovina.
    \textbf{Kolmým průmětem} bodu $A$ do roviny $\alpha$ nazýváme bod $A_0$
    splňující
    \begin{enumerate}[$i.$]
    \item $A\in \alpha \implies A_0=A,$
   	\item $A\notin \alpha \implies A_0 \in p\cap \alpha, p \perp\alpha, a \in p.$
    \end{enumerate}
    \textbf{Vzdáleností bodu $A$ od roviny $\alpha$} nazýváme reálné číslo označené
    $\rho(A,\alpha) $ a definované
    $$\rho(A,\alpha) = \rho(A, A_0) = |AA_0|,$$
    kde $A_0$ je kolmý průmět bodu $A$ do roviny $\alpha.$
\end{definition}

\begin{definition}
    Nechť $A\in \mathbb E_3$ je bod a $p \subseteq \mathbb E_3$ je přímka.
    \textbf{Kolmým průmětem} bodu $A$ na přímku $p$ nazýváme bod $A_0$
    splňující
    \begin{enumerate}[$i.$]
    \item $A\in \alpha \implies A_0=A,$
    \item $A\notin p \implies A_0 \in p\cap \alpha, \alpha \perp, A \in \alpha.$
    \end{enumerate}
    \textbf{Vzdáleností bodu $A$ od přímky $\alpha$} nazýváme reálné číslo označené
    $\rho(A,p) $ a definované
    $$\rho(A,p) = \rho(A, A_0) = |AA_0|,$$
    kde $A_0$ je kolmý průmět bodu $A$ na přímku $p.$
\end{definition}

\begin{definition}
    Nechť $\alpha, \beta \subseteq \mathbb E_3$ jsou dvě rovnoběžné roviny. Pak
    \textbf{vzdáleností dvou rovnoběžných rovin} $\alpha, \beta$ nazýváme reálné
    číslo označené $\rho(\alpha, \beta)$ a definované
    $$\rho(\alpha, \beta)=\rho(A,\beta),$$
    kde $A\in\alpha$ je libovolný bod.
\end{definition}

\begin{definition}
    Nechť $p\subseteq \mathbb E_3$ je přímka, $\alpha \subseteq \mathbb E_3$ je rovina.
    Nechť $p\parallel \alpha.$ \textbf{Vzdáleností přímky $p$ od roviny $\alpha$
    s ní rovnoběžné} nazýváme reálné číslo označené $\rho(p,\alpha)$ a definované
    $$\rho(p,\alpha)=\rho(A,\alpha),$$
    kde $A\in p$ je libovolný bod.
\end{definition}

\begin{definition}
    Nechť $p,q\subseteq \mathbb E_3$ jsou rovnoběžné přímky. \textbf{Vzdáleností
    dvou rovnoběžných přímek $p,q$} nazýváme reálné číslo označené $\rho(p,q)$ a
    definované
    $$\rho(p,q) = \rho(A,q),$$
    kde $A\in p$ je libovolný bod.
\end{definition}

\begin{definition}
    Nechť $p,q\subseteq \mathbb E_3$ jsou mimoběžné přímky. \textbf{Vzdáleností dvou
    mimoběžných přímek} $p,q$ nazýváme reálné číslo označené $\rho(p,q)$ a definované
    $$\rho(p,q)=\rho(\alpha, \beta),$$
    kde $\alpha \parallel \beta, p\subseteq\alpha, q\subseteq \beta.$
\end{definition}

\begin{definition}
    Nechť $p,q\subseteq \mathbb E_3$ jsou dvě komplanární přímky. \textbf{Odchylkou
    dvou komplanárních přímek} $p,q$ nazýváme reálné číslo označené $|\sphericalangle
    p,q|$ a definované
    \begin{enumerate}[$i.$]
    \item je-li $p\parallel q, |\sphericalangle p,q|=0^\circ,$
   	\item je-li $p\nparallel q$ odchylka $p,q$ je velikost ostrého nebo pravého úhlu,
        který $p,q$ svírají.
    \end{enumerate}
\end{definition}

\begin{veta}
    Nechť $p^\prime, q^\prime; p,q\subseteq \mathbb E_3: p\parallel p^\prime\land
    q\parallel q^\prime.$ Pak platí: $|\sphericalangle p^\prime q^\prime|.$
\end{veta}

\begin{definition}
    Nechť $p,q\subseteq \mathbb E_3$ jsou dvě mimoběžné přímky. \textbf{Odchylkou
    dvou mimoběžnách přímek} $p,q$ nazýváme reálné číslo označené $|\sphericalangle
    p,q|$ a definované
    $$|\sphericalangle p,q| = |\sphericalangle p^\prime, q^\prime|,$$
    kde $p^\prime \parallel p \land q^\prime \parallel q,$ kde $p^\prime, q^\prime$
    jsou komplanární a různoběžné.
\end{definition}

\begin{definition}
    Nechť $p\subseteq \mathbb E_3$ je přímka a $\alpha\subseteq \mathbb E_3$ je
    rovina.. \textbf{Odchylkou přímky $p$ od roviny $\alpha$} nazýváme
    reálné číslo označené $|\sphericalangle
    p,\alpha|$ a definované
    \begin{enumerate}[$i.$]
    \item je-li $p\parallel \alpha, |\sphericalangle p,\alpha|=0^\circ,$
    \item je-li $p\nparallel \alpha$ odchylka $p,\alpha$ je velikost ostrého nebo
        pravého úhlu,
        který svírají přímky $p,q$, kde $q$ je průsečnice rovin $\alpha, \beta$,
        přičemž $\beta \perp \alpha, p\in\beta.$
    \end{enumerate}
\end{definition}

\begin{definition}
    Nechť $\alpha,\beta\subseteq \mathbb E_3$ jsou dvě roviny. \textbf{Odchylkou
    rovin} $\alpha, \beta$ nazýváme reálné číslo označené $|\sphericalangle
    \alpha, \beta|$ a definované
    \begin{enumerate}[$i.$]
    \item je-li $\alpha\parallel \beta, |\sphericalangle \alpha,\beta|=0^\circ,$
   	\item je-li $\alpha\nparallel \beta$ odchylka $\alpha,\beta$ je velikost
        ostrého nebo pravého úhlu,
        který svírají přímky $p,q,$ kde $p\subseteq \alpha, q\subseteq \beta$ a obě
        přímky jsou kolmé k průsečnici rovin $\alpha, \beta.$
    \end{enumerate}
\end{definition}

\begin{definition}
  \textbf{Čtyřstěn} (zvaný též trojboký jehlan, tetraedr) je nejjednodušší mnohostěn, typ trojrozměrného tělesa. Je vymezen nejmenším možným počtem bodů, který může trojrozměrné těleso definovat, tzn. čtyřmi různými body v prostoru.

  \textbf{Pravidelný čtyřstěn} je tvořen čtyřmi stejnými rovnostrannými trojúhelníky.
\end{definition}

\begin{definition}
  \textbf{Hranol} je mnohostěn se dvěma stejnými monohúhelníkovými základnami, které jsou spojeny pásem rovnoběžníků.
\end{definition}

\begin{definition}
  \textbf{Rovnoběžnostěn} je čtyřboký hranol, jehož podstavou je rovnoběžník. Mezi rovnoběžnostěny patří např. kvádr, krychle nebo klenec (kosočtverco-stěn).
\end{definition}

\begin{definition}
  \textbf{Kvádr} je trojrozměrné těleso – rovnoběžnostěn, jehož stěny tvoří šest pravoúhlých čtyřúhelníků, zpravidla obdélníků.
\end{definition}

\begin{definition}
  \textbf{Krychle} je trojrozměrné těleso, jehož stěny tvoří 6 stejných čtverců.
\end{definition}

\begin{definition}
  \textbf{Válec} je trojrozměrné těleso, vymezené dvěma rovnoběžnými podstavami a pláštěm. Plášť je rozvinutelná plocha, všechny povrchové (tvořící) přímky pláště jsou rovnoběžné a pokud jsou k podstavám kolmé, hovoříme o \textbf{kolmém válci}. Je-li podstavou kruh, pak válec označíme jako \textbf{kruhový válec}.
\end{definition}

\begin{definition}
  \textbf{Jehlan} je trojrozměrné těleso. Jeho základnu (nebo také podstavu) tvoří mnohoúhelník. Vrcholy základny jsou spojeny s jedním bodem mimo rovinu základny – tento bod se obvykle nazývá \textbf{(hlavní) vrchol jehlanu}.
\end{definition}

\begin{definition}
  \textbf{Komolý jehlan} je prostorové těleso – část jehlanu, která leží mezi dvěma rovnoběžnými rovinami procházející tímto jehlanem.
\end{definition}

\begin{definition}
  \textbf{Kužel} je trojrozměrný geometrický tvar ohraničený kuželovou plochou a rovinou, která protíná kuželovou plochu tak, že vytváří uzavřenou křivku. Je-li podstavou kužele kruh, pak se kužel nazývá \textbf{kruhový kužel}.
\end{definition}

\begin{definition}
  \textbf{Komolý kužel} je prostorové těleso – část kužele, která leží mezi dvěma rovnoběžnými rovinami procházejícími tímto kuželem.
\end{definition}

\begin{veta}
  \begin{tabularx}{\textwidth}{ l | l  l  l }
      \, & $V$ & $S$ & $S_{pláště}$ \\
      \hline
      pravidelný čtyřstěn & $\frac{\sqrt{2}}{12}a^3$ & $\sqrt{3}a^2$ \\
      hranol & $S_{podstavy}\cdot v$ & $2S_{podstavy}+S_{pláště} & o_{podstavy} \cdot v$ \\
      rovnoběžnostěn & $\left| ( \mathbf{a} \times \mathbf{b} ) \cdot \mathbf{c} \right| = \left| ( \mathbf{b} \times \mathbf{c} ) \cdot \mathbf{a} \right| = \left| ( \mathbf{c} \times \mathbf{a} ) \cdot \mathbf{b} \right|$ & $2 \Bigg[ \Big((\mathbf{a}\times\mathbf{b})\cdot(\mathbf{a}\times\mathbf{b})\Big)^{1/2}+ \Big((\mathbf{b}\times\mathbf{c})\cdot(\mathbf{b}\times\mathbf{c})\Big)^{1/2}+ \Big((\mathbf{c}\times\mathbf{a})\cdot(\mathbf{c}\times\mathbf{a})\Big)^{1/2} \Bigg] $ & \\
      kvádr & $abc$ & $2(ab+bc+ca)$ & \\
      krychle & $a^3$ & $6a$ & \\
      válec & $S_{podstavy}\cdot v$ & $2S_{podstavy} + S_{pláště} & o_{podstavy} \cdot v$ \\
      kužel & $\frac{1}{3}\cdot S_{podstavy}\cdot v$
  \end{tabularx}
\end{veta}
