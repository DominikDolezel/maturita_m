\section{Stereometrie -- metrické vlastnosti}
\begin{definition}
    Nechť $A, B \in \mathbb E_3.$ \textbf{Vzdálenost bodů} $A,B$ nazýváme délku
    úsečky $AB$ a označujeme $\rho(A,B).$
\end{definition}

\begin{definition}
    Nechť $A\in \mathbb E_3$ je bod a $\alpha \subseteq \mathbb E_3$ je rovina.
    \textbf{Kolmým průmětem} bodu $A$ do roviny $\alpha$ nazýváme bod $A_0$
    splňující
    \begin{enumerate}[$i.$]
    \item $A\in \alpha \implies A_0=A,$
   	\item $A\notin \alpha \implies A_0 \in p\cap \alpha, p \perp\alpha, a \in p.$
    \end{enumerate}
    \textbf{Vzdáleností bodu $A$ od roviny $\alpha$} nazýváme reálné číslo označené
    $\rho(A,\alpha) $ a definované
    $$\rho(A,\alpha) = \rho(A, A_0) = |AA_0|,$$
    kde $A_0$ je kolmý průmět bodu $A$ do roviny $\alpha.$
\end{definition}

\begin{definition}
    Nechť $A\in \mathbb E_3$ je bod a $p \subseteq \mathbb E_3$ je přímka.
    \textbf{Kolmým průmětem} bodu $A$ na přímku $p$ nazýváme bod $A_0$
    splňující
    \begin{enumerate}[$i.$]
    \item $A\in \alpha \implies A_0=A,$
    \item $A\notin p \implies A_0 \in p\cap \alpha, \alpha \perp, A \in \alpha.$
    \end{enumerate}
    \textbf{Vzdáleností bodu $A$ od přímky $\alpha$} nazýváme reálné číslo označené
    $\rho(A,p) $ a definované
    $$\rho(A,p) = \rho(A, A_0) = |AA_0|,$$
    kde $A_0$ je kolmý průmět bodu $A$ na přímku $p.$
\end{definition}

\begin{definition}
    Nechť $\alpha, \beta \subseteq \mathbb E_3$ jsou dvě rovnoběžné roviny. Pak
    \textbf{vzdáleností dvou rovnoběžných rovin} $\alpha, \beta$ nazýváme reálné
    číslo označené $\rho(\alpha, \beta)$ a definované
    $$\rho(\alpha, \beta)=\rho(A,\beta),$$
    kde $A\in\alpha$ je libovolný bod.
\end{definition}

\begin{definition}
    Nechť $p\subseteq \mathbb E_3$ je přímka, $\alpha \subseteq \mathbb E_3$ je rovina.
    Nechť $p\parallel \alpha.$ \textbf{Vzdáleností přímky $p$ od roviny $\alpha$
    s ní rovnoběžné} nazýváme reálné číslo označené $\rho(p,\alpha)$ a definované
    $$\rho(p,\alpha)=\rho(A,\alpha),$$
    kde $A\in p$ je libovolný bod.
\end{definition}

\begin{definition}
    Nechť $p,q\subseteq \mathbb E_3$ jsou rovnoběžné přímky. \textbf{Vzdáleností
    dvou rovnoběžných přímek $p,q$} nazýváme reálné číslo označené $\rho(p,q)$ a
    definované
    $$\rho(p,q) = \rho(A,q),$$
    kde $A\in p$ je libovolný bod.
\end{definition}

\begin{definition}
    Nechť $p,q\subseteq \mathbb E_3$ jsou mimoběžné přímky. \textbf{Vzdáleností dvou
    mimoběžných přímek} $p,q$ nazýváme reálné číslo označené $\rho(p,q)$ a definované
    $$\rho(p,q)=\rho(\alpha, \beta),$$
    kde $\alpha \parallel \beta, p\subseteq\alpha, q\subseteq \beta.$
\end{definition}

\begin{definition}
    Nechť $p,q\subseteq \mathbb E_3$ jsou dvě komplanární přímky. \textbf{Odchylkou
    dvou komplanárních přímek} $p,q$ nazýváme reálné číslo označené $|\sphericalangle
    p,q|$ a definované
    \begin{enumerate}[$i.$]
    \item je-li $p\parallel q, |\sphericalangle p,q|=0^\circ,$
   	\item je-li $p\nparallel q$ odchylka $p,q$ je velikost ostrého nebo pravého úhlu,
        který $p,q$ svírají.
    \end{enumerate}
\end{definition}

\begin{veta}
    Nechť $p^\prime, q^\prime; p,q\subseteq \mathbb E_3: p\parallel p^\prime\land
    q\parallel q^\prime.$ Pak platí: $|\sphericalangle p^\prime q^\prime|.$
\end{veta}

\begin{definition}
    Nechť $p,q\subseteq \mathbb E_3$ jsou dvě mimoběžné přímky. \textbf{Odchylkou
    dvou mimoběžnách přímek} $p,q$ nazýváme reálné číslo označené $|\sphericalangle
    p,q|$ a definované
    $$|\sphericalangle p,q| = |\sphericalangle p^\prime, q^\prime|,$$
    kde $p^\prime \parallel p \land q^\prime \parallel q,$ kde $p^\prime, q^\prime$
    jsou komplanární a různoběžné.
\end{definition}

\begin{definition}
    Nechť $p\subseteq \mathbb E_3$ je přímka a $\alpha\subseteq \mathbb E_3$ je
    rovina.. \textbf{Odchylkou přímky $p$ od roviny $\alpha$} nazýváme
    reálné číslo označené $|\sphericalangle
    p,\alpha|$ a definované
    \begin{enumerate}[$i.$]
    \item je-li $p\parallel \alpha, |\sphericalangle p,\alpha|=0^\circ,$
    \item je-li $p\nparallel \alpha$ odchylka $p,\alpha$ je velikost ostrého nebo
        pravého úhlu,
        který svírají přímky $p,q$, kde $q$ je průsečnice rovin $\alpha, \beta$,
        přičemž $\beta \perp \alpha, p\in\beta.$
    \end{enumerate}
\end{definition}

\begin{definition}
    Nechť $\alpha,\beta\subseteq \mathbb E_3$ jsou dvě roviny. \textbf{Odchylkou
    rovin} $\alpha, \beta$ nazýváme reálné číslo označené $|\sphericalangle
    \alpha, \beta|$ a definované
    \begin{enumerate}[$i.$]
    \item je-li $\alpha\parallel \beta, |\sphericalangle \alpha,\beta|=0^\circ,$
   	\item je-li $\alpha\nparallel \beta$ odchylka $\alpha,\beta$ je velikost
        ostrého nebo pravého úhlu,
        který svírají přímky $p,q,$ kde $p\subseteq \alpha, q\subseteq \beta$ a obě
        přímky jsou kolmé k průsečnici rovin $\alpha, \beta.$
    \end{enumerate}
\end{definition}

\begin{definition}
    Nechť jsou dány nekomplanární body $A,B,C,D.$ Pak
  \textbf{čtyřstěn} je množina bodů ohraničená trojúhelníky $\triangle ABC, \triangle
  ABD, \triangle ACD, \triangle BCD.$

  \textbf{Pravidelný čtyřstěn} je tvořen čtyřmi stejnými rovnostrannými trojúhelníky.
\end{definition}

\begin{definition}
    Mějme v prostoru rovinu $\rho,$ v ní konvexní mnohoúhelník $A_1\dots A_n$ a nechť
    $A_1^\prime$ je bod, který v rovině $\rho$ neleží. Nechť $T: \mathbb E_3 \to
    \mathbb E_3$ je takové posunutí, že $A_1^\prime=T(A_1).$ Při tomto zobrazení se rovina
    $\rho$ zobrazí na rovinu $\rho^\prime,$ tyto dvě roviny jsou rovnoběžné. Množinu všech
    bodů $X$, všech úseček $BB^\prime$ takových, že $B\in A_1\dots A_n$ a $B^\prime$ je
    obraz bodu $B$ v posunutí $T$, nazýváme \textbf{hranolem}.
\end{definition}

\begin{pozn}
    -- ještě by se tam mohly dopsat: boční stěna, podstavy, plášť, stěna, vrcholy; kosý,
    kolmý, n-boký
\end{pozn}

\begin{definition}
    Hranol, jehož podstavy jsou rovnoběžníky, nazýváme \textbf{rovnoběžnostěn}.
\end{definition}

\begin{definition}
    Rovnoběžnostěn, jehož všechny stěny jsou pravoúhelníky (resp. čtverce) nazýváme
    \textbf{kvádr} (resp. \textbf{krychle}).
\end{definition}

\begin{definition}
    Mějme v prostoru rovinu $\rho,$ v ní kruh $K$ ohraničený kružnicí $k$, na níž
    leží bod $A$ a nechť
    $A^\prime$ je bod, který v rovině $\rho$ neleží. Nechť $T: \mathbb E_3 \to
    \mathbb E_3$ je takové posunutí, že $A^\prime=T(A).$ Označme $T(\rho)=\rho^\prime,
    T(k)=k^\prime, T(K)=K^\prime.$ Všechny body všech úseček $XX^\prime$, kde $X\in K$
    a $X^\prime$ je obraz bodu $X$, tytvoří \textbf{válec.}
\end{definition}

\begin{pozn}
    ---- mohly by se přidat: plášť, podstava, kolmý, kosý
\end{pozn}

\begin{definition}
Mějme v prostoru rovinu $\rho,$ v ní konvexní mnohoúhelník $A_1\dots A_n$ a nechť
$V$ je bod, který v rovině $\rho$ neleží. Úsečky $VX$, kde $X$ jsou všechny body
mnohoúhelníka $A_1\dots A_n$, nazýváme \textbf{jehlanem}.
\end{definition}

\begin{pozn}
    ---- add hlavní vrchol, vrcholy, podstava, boční stěny, hrany, podstavné hrany
\end{pozn}

\begin{definition}
    Nechť je dán jehlan s hlavním vrcholem $V$ a podstavou $A_1\dots A_n$ v rovině
    $\rho.$ Nechť $k \in \mathbb R, k \ne 1, k >0.$ Zaveďme $\mathscr H_{V,k}(A_1\dots A_n)
    =A_1^\prime\dots A_n^\prime.$ Těleso ohraničené podstavami $A_1\dots A_n$,
    $A_1^\prime\dots A_n^\prime$ a stěnami $A_iA_{i+1}A_i^\prime A_{i+1}^\prime$ se
    nazývá \textbf{komolý jehlan}.
\end{definition}

\begin{definition}
    Nechť je dán vrchol $V$ a kruhová podstava $K$ v rovině
    $\rho.$ Množina všech bodů úseček $VX,$ kde $X\in K,$ se nazývá \textbf{kužel}.
\end{definition}

\begin{definition}
Nechť je dán kužel s hlavním vrcholem $V$ a podstavou $K$ v rovině
$\rho.$ Nechť $k \in \mathbb R, k \ne 1, k >0.$ Zaveďme $\mathscr H_{V,k}(K)
=K^\prime.$ Množina všech bodů úseček $X\mathscr H(X),$ kde $X\in K$, je
\textbf{komolý kužel}.
\end{definition}

\begin{veta}
  Označme $V$ objem tělesa, $S$ jeho povrch, dále obsahy $S_{\rm pláště}$, $S_{\rm podstavy}$, $\mathbf{a},\mathbf{b},\mathbf{c}$ vektory stran, $a$, $b$, $c$ jejich délky. Pak se rovnají:

  \begin{center}
   \footnotesize
    \begin{tabularx}{\textwidth}{ l | l  l  l }

      \, & $V$ & $S$ & $S_{\text{pláště}}$ \\
      \hline
      prav. čtyřstěn & $\frac{\sqrt{2}}{12}a^3$ & $\sqrt{3}a^2$ & \\
      hranol & $S_{\rm podstavy} \cdot v$ & $2S_{\rm podstavy}+S_{\rm pláště}$ & $o_{\rm podstavy} \cdot v$ \\
      rovnoběžnostěn & $| ( \mathbf{a} \times \mathbf{b} ) \cdot \mathbf{c} | $ & {\rm z vektorového součinu} & \, \\
      kvádr & $abc$ & $2(ab+bc+ca)$ & \, \\
      krychle & $a^3$ & $6a^2$ & \, \\
      válec & $S_{podstavy}\cdot v$ & $2S_{\rm podstavy} + S_{\rm pláště}$ & $o_{\rm podstavy} \cdot v$ \\
      jehlan & $\frac{1}{3}\cdot S_{\rm podstavy}\cdot v$ & $S_{\rm podstavy} + S_{\rm pláště}$ & \, \\
      komolý jehlan & $\frac{1}{3}(S_{\rm p1} + \sqrt{S_{\rm p1} S_{\rm p2}} + S_{\rm p2})\cdot v$ & \, & \,\\
      kužel & $\frac{1}{3}\cdot S_{\rm podstavy}\cdot v$ & \, & \, \\
      komolý kužel & $\frac{1}{3}\pi(r_1^2 + r_1r_2 + r_2^2)\cdot v$ & \, & \,\\
      koule & $\frac{4}{3}\pi r^3$ & $4\pi r^2$ & \, \\
      kulová úseč & $\frac{\pi v^2}{3}(3r-v)$ & $2\pi r v$ & \,
    \end{tabularx}
  \end{center}
  \normalsize
\end{veta}

\begin{veta}[Cavalieriho princip]
    Nechť tělesa $T_1, T_2$ leží mezi dvěma rovinami $\rho_1, \rho_2$ a každá rovina
    $\rho \parallel \rho_1\parallel \rho_2$ protne tělesa $T_1, T_2$ v konvexních
    rovinných útvarech s obsahy $P_1, P_2.$ Jestliže pro každou rovinu $\rho$ platí
    $P_1=P_2$, mají $T_1$ a $T_2$ stejný objem.
\end{veta}

\begin{definition}
\textbf{Mnohostěn} je konvexní část prostoru, hranice je tvořena konečným počtem
mnohoúhelníků.
\end{definition}

\begin{veta}[Eulerova věta]
    Nechť $s$ je počet stěn, $h$ počet hran a $v$ počet vrcholů daného tělesa. Pak
    platí
    \begin{align*}
        s-h+v &=2,\\
        s+v &=h+2.
    \end{align*}
\end{veta}

\begin{pozn}[Pravidelné mnohostěny]\,
\begin{center}
\begin{tabular}{l|l|c|c|c|r}
    prav. mnohostěn & tvar stěny & $v$ & $s$ & $h$ & \, \\
    čtyřstěn        & rovnostranný trojúhelník & 4 & 4&6 & tetraedr \\
    šestistěn       & čtverec & 8 & 6&12 & hexaedr \\
    osmistěn        & rovnostranný trojúhelník & 6 & 8&12 & oktaedr \\
    dvanáctistěn    & pravidelný pětiúhelník & 20 & 12&30 & dodekaedr \\
    dvacetistěn     & rovnostranný trojúhelník & 12 & 20&30 & ikosaedr \\
\end{tabular}
\end{center}

\end{pozn}

\begin{definition}
    O tělesu $M$ v prostoru říkáme, že má $p$ za osu rotace a že je \textbf{rotačním
    tělesem}, jestliže se zobrazí samo na sebe při každém otočení kolem přímky $p.$
\end{definition}

\begin{pozn}
\begin{itemize}
\item \textbf{rotační válec} -- pravoúhelník otáčený kolem své strany,
\item \textbf{rotační kužel} -- pravidelný trojúhelník otáčený kolem odvěsny,
\item \textbf{koule} -- půlkruh nad průměrem, hranice se nazývá \textbf{kulová plocha},
\item \textbf{torus} -- kruh kolem osy, která leží mimo něj
\end{itemize}
\end{pozn}

--- doplnit kulovou úseč a kulový vrchlík, což teď nemůžu najít ---
