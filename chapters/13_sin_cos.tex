\section{Funkce sinus, kosinus, arkussinus, arkuskosinus}

\begin{pozn}
  Pro definice úhlu a jeho velikosti viz def. \ref{uheldef}, \ref{uhelvel} a~\ref{raddef}.
\end{pozn}

\begin{comment}
\begin{definition}
    Nechť $\sphericalangle AVB$ je úhel. Pak mu přiřazujeme číslo $a\in \mathbb R$,
    které nazýváme \textbf{velikostí úhlu} $\sphericalangle AVB$ takto:\\
    Nechť $k(V,r), X\in k\cap \overrightarrow{VA}, Y\in k\cap \overrightarrow{VB}.$
    Označme $s$ velikost oblouku $XY.$ Velikost úhlu pak definujeme jako
    $$\alpha = \frac{s}{r}.$$
\end{definition}

\begin{pozn}
    Je-li $s=r,$ má úhel $\alpha$ velikost 1 \textbf{radián}. Velikost úhlu lze vyjadřovat
    též ve stupních, přičemž platí $2\pi=360^\circ$.
\end{pozn}

\begin{veta}
    Nechť $\sphericalangle AVB$ je úhel, $\alpha$ jeho velikost v radiánech a $\beta$
    jeho velikost ve stupních. Pak platí:
    \begin{align*}
        \alpha = \beta\cdot \frac{\pi}{180^\circ}, & & \beta = \alpha\cdot \frac{180^\circ}{\pi}.
    \end{align*}
\end{veta}

PRESUNUTO DO PLANIMETRIE
\end{comment}

\begin{definition}
    \textbf{Orientovaným úhlem} $\sphericalangle AVB$ nazýváme úhel s uspořádanou dvojicí
    polopřímek $\overrightarrow{VA}, \overrightarrow{VB}$ se společným počátkem v bodě $V.$
    Polopřímku $\overrightarrow{VA}$, resp. $\overrightarrow{VB}$ nazýváme \textbf{počátečním},
    resp. \textbf{koncovým ramenem}, bod $V$ \textbf{vrcholem}.
\end{definition}

\begin{definition}
    Nechť $\sphericalangle AVB$ je orientovaný úhel o základní velikosti $|\sphericalangle AVB| = \alpha$ dle def. \ref{raddef}. Pak
    \textbf{velikostí orientovaného úhlu} $\sphericalangle AVB$ rozumíme každé z
    čísel $\alpha +2k\pi, k \in \mathbb Z.$
\end{definition}

\begin{definition}
  Nechť je dána jednotková kružnice. Hodnotou funkce \textbf{sinus} (resp. \textbf{kosinus})
  v bodě $x$ nazveme $y$-ovou (resp. $x$-ovou) souřadnici průsečíku této kružnice s ramenem
  úhlu svírajícím s $x$-ovou osou úhel $\theta = x$.
\end{definition}

\begin{veta}
    Vlastnosti funkce sinus:\\
    Nechť $k\in \mathbb Z.$
    \begin{enumerate}[$i.$]
        \item $D(f)= \mathbb R$, $H(f)= \left < -1,1 \right > $.
       	\item Je lichá.
        \item Je rostoucí v intervalu $\left < -\frac{\pi}{2}+2k\pi, \frac{\pi}{2}+2k\pi \right > $.
        a klesající v $\left < \frac{\pi}{2}+2k\pi, \frac{3\pi}{2}+2k\pi \right > $
        \item Je omezená.
        \item Maximum má v bodech $\frac{\pi}{2}+2k\pi$, minimum v bodech $\frac{3\pi}{2}+2k\pi.$
        \item Je periodická s nejmenší periodou $2\pi.$
    \end{enumerate}
    Vlastnoti funkce kosinus:
    \begin{enumerate}[$i.$]
        \item $D(f)= \mathbb R$, $H(f)= \left < -1,1 \right > $.
       	\item Je sudá.
        \item Je rostoucí v intervalu $\left < \pi+2k\pi, 2\pi+2k\pi \right > $.
        a klesající v $\left < 0+2k\pi, \pi+2k\pi \right > $
        \item Je omezená.
        \item Maximum má v bodech $2k\pi$, minimum v bodech $\pi+2k\pi.$
        \item Je periodická s nejmenší periodou $2\pi.$
    \end{enumerate}
\end{veta}

\begin{priklad}
Určete $\sin \frac{5\pi}{3}.$
\end{priklad}

\begin{reseni}
Počítejme:
$$\sin \frac{5\pi}{3}=\sin \left ( 2\pi-\frac{\pi}{3} \right ) =\sin \left ( -\frac{\pi}{3} \right ). $$
\end{reseni}

\begin{priklad}
Načrtněte graf funkce $y=\cos (2x-\frac{\pi}{3}).$
\end{priklad}

\begin{reseni}
Po vytknutí dvojky zjistíme, že funkce je dvojnásobně zhuštěná a posunutá o $\frac{\pi}{6}$ doprava.
\end{reseni}

\begin{definition}
Funkce \textbf{arkussinus}, označená $f^{-1}: y=\arcsin x$, se nazývá funkce inverzní
k funkci $f: y=\sin x$, kde $D(f)=\left < -\frac{\pi}{2},\frac{\pi}{2} \right >$.
\end{definition}
\begin{pozn}
    Funkce arkussinus má následující vlastnosti:
    \begin{enumerate}
    \item $D(f^{-1}) = \left < -1, 1 \right >$, $H(f^{-1}) = \left < -\frac{\pi}{2},\frac{\pi}{2} \right >,$
    \item $f^{-1}$ je rostoucí v celém $D(f^{-1}).$
    \end{enumerate}
\end{pozn}


\begin{definition}
Funkce \textbf{arkuskosinus}, označená $g^{-1}: y=\arccos x$, se nazývá funkce inverzní
k funkci $g: y=\cos x$, kde $D(g)=\left < 0,\pi \right >$.
\end{definition}
\begin{pozn}
Funkce arkuskosinus má následující vlastnosti:
\begin{enumerate}
    \item $D(g^{-1}) = \left < -1, 1 \right >$, $H(g^{-1}) = \left < 0, \pi \right >,$
    \item $g^{-1}$ je klesající v celém $D(g^{-1}).$
\end{enumerate}
\end{pozn}

\begin{veta}[Trigonometrická jednička]
  $\forall x \in \mathbb{R}:\sin^2 x + \cos^2 x = 1$
\end{veta}

\begin{veta}
    V pravoúhlém trojúhelníku $ABC$ s přeponou $AB$ platí:
    \begin{align*}
        \sin \alpha = \frac{ |BC| }{ |AB| }, & & \cos \alpha = \frac{|AC|}{|AB|}.
    \end{align*}
\end{veta}

\begin{pozn}
    \textbf{Goniometrická (ne)rovnice} je (ne)rovnice, v níž se vyskytují goniometrické funkce.
\end{pozn}

\begin{pozn}
    Grafy jednotlivých funkcí a tabulka se základními hodnotami jsou k nalezení
    v příloze \ref{appa}.
\end{pozn}
