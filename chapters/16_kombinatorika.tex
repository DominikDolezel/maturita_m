\section{Kombinatorika}
\begin{veta}[Pravidlo součtu]
    Nechť $M$ je konečná množina, $M_1, M_2, \dots M_k, k
    \in \mathbb N$ její podmnožiny takové, že
    \begin{enumerate}[$i.$]
    \item $M_1\cup M_2 \cup \dots \cup M_k = M,$
   	\item $M_i \cap M_j = \emptyset$ pro libovolná
    $i,j \in \left \{ 1,2,\dots,k \right \},i\ne j $.
    \end{enumerate}
    Pak platí $|M|=|M_1|+|M_2|+\dots +|M_k|,$ kde
    symbolem $|A|$ značíme počet prvků množiny $A$.
\end{veta}

\begin{veta}[Pravidlo součinu]
    Nechť $M_1, M_2, \dots, M_k, k\in \mathbb N$ jsou konečné množiny takové,
    že $|M_1| = m_1, |M_2|=m_2, \dots, |M_k| = m_k.$ Pak platí
    $$|M_1\times M_2 \times \dots \times M_k| = m_1m_2\dots m_k,$$
    kde  $M_1\times M_2\times \dots \times M_k= \left \{ \left [ a_1, a_2, \dots, a_k \right ]  \right \}
    , a_1\in M_1, a_2 \in M_2, \dots, a_k \in M_k.$
\end{veta}

\begin{veta}[Dirichletův princip]
    Má-li být alespoň $nk+1$ předmětů rozděleno do $k$ přihrádek, pak
    alespoň v jedné přihrádce je alespoň $n+1$ předmětů.
\end{veta}

\begin{definition}
    Nechť $M$ je neprázdná množina. \textbf{Rozklad množiny} $M$ značíme
    $\mathscr R(M)$ a definujeme jako neprázdný systém neprázdných podmnožiny
    $M_1, M_2, \dots, M_k, k\in \mathbb N,$ pro které platí
    \begin{enumerate}[$i.$]
    \item $M_1\cup M_2 \cup \dots \cup M_k = M,$
   	\item $M_i \cap M_j = \emptyset$ pro libovolná
    $i,j \in \left \{ 1,2,\dots,k \right \},i\ne j $.
    \end{enumerate}
    Množiny $M_1, M_2, \dots, M_k$ nazýváme \textbf{třídami rozkladu} $\mathscr R(M).$
\end{definition}

\begin{definition}
    Nechť je dáno $n_1$ prvků prvního druhu, $n_2$ prvků druhého druhu, \dots,
    $n_d$ prvků $d$-tého druhu, přičemž prvky téhož druhu považujeme za nerozlišitelné.
    Nechť $n_1+n_2+\dots+n_d=n.$ \textbf{Pořadím (permutací) s opakováním} z  $n_1$
    prvků prvního druhu, $n_2$ prvků druhého druhu, \dots, $n_d$ prvků $d$-tého druhu
    nazveme každou uspořádanou $n$-tici, která obsahuje $n_1$ prvků prvního druhu,
    $n_2$ prvků druhého druhu, \dots, $n_d$ prvků $d$-tého druhu. Počet všech pořadí
    označme $P_o(n_1, n_1, \dots, n_d).$
\end{definition}

\begin{veta}
    Nechť $n_1, n_2, \dots, n_d \in \mathbb N, n, n_1, n_2, \dots, n_d=n.$ Pak platí:
    $$P_o(n_1, n_1, \dots, n_d)=\frac{n!}{n_1!n_2!\dots n_d!}=
    \frac{\left ( \sum_{i=1}^d n_i \right )! }{\prod_{i=1}^d n_i!}.$$
\end{veta}

\begin{definition}
    Nechť je dáno $n_1$ prvků prvního druhu, $n_2$ prvků druhého druhu, \dots,
    $n_d$ prvků $d$-tého druhu. Nechť $k\in \mathbb N, k \leq n, \forall
    i \in \left \{ 1, 2, \dots, d \right \} .$ \textbf{$k$-prvkovou variací s
    opakováním} z prvků daných $d$ druhů nazveme každou ušpořádanou $k$-tici
    vytvořenou z prvků těchto $d$ druhů. Počet těchto variací ozančme $V_o(k,d).$
\end{definition}

\begin{veta}
    $\forall k, d, \in \mathbb N$ platí:
    $$V_o(k,d)=d^k.$$
\end{veta}
