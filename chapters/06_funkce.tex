\section{Funkce a jejich základní vlastnosti}
\begin{definition}
  \textbf{Funkcí} $f$ nazýváme každé zobrazení z $\mathbb R$ do $\mathbb R$.
\end{definition}

\begin{definition}
  Nechť $f\subseteq \mathbf R \times \mathbb R$ je funkce. Množinu
  \[
    D(f) = \left  \{ x \in \mathbb R:\exists ! y \in \mathbb R:y=f(x) \right \}
  \]
  nazveme \textbf{definičním oborem} funkce $f$. Množinu
  \[
    H(f) = \left  \{ y \in \mathbb R:\exists ! x \in \mathbb R:y=f(x) \right \}
  \]
  nazveme \textbf{oborem hodnot} funkce $f$.
\end{definition}

\begin{pozn}
  \textbf{Grafem funkce} rozumíme množinu všech bodů $[x,f(x)]$, kde $x\in D(f).$
\end{pozn}


\begin{definition}
  Funkce $f$ se nazývá \textbf{sudá} (resp. \textbf{lichá}), jestliže platí 
\begin{itemize}
  \item 
  $\forall x \in D(f):(-x) \in D(f)$
\item 
 $\forall x \in D(f): f(-x)=f(x)$, resp. $f(-x)=f(x)$.
\end{itemize}

\end{definition}

\begin{definition}
  Funkce $f$ se nazývá \textbf{prostá}, právě tehdy když platí
  \[
    \forall x_1,x_2\in D(f): x_1\ne x_2 \implies f(x_1)\ne f(x_2)
  \]
\end{definition}

\begin{definition}
  Nechť $f$ je funkce a $M$ alespoň dvouprvková množina z $D(f)$. Řekneme, že funkce $f$ je v množině $M$
  \begin{enumerate}[$i.$]
    \item \textbf{rostoucí} $\iff \forall x_1, x_2 \in M: x_1 < x_2 \implies f(x_1) < f(x_2),$
    \item \textbf{klesající} $\iff \forall x_1, x_2 \in M: x_1 < x_2 \implies f(x_1) > f(x_2),$
    \item \textbf{neklesající} $\iff \forall x_1, x_2 \in M: x_1 < x_2 \implies f(x_1) \leq f(x_2),$
    \item \textbf{nerostoucí} $\iff \forall x_1, x_2 \in M: x_1 < x_2 \implies f(x_1) \geq f(x_2).$
  \end{enumerate}
  Je-li $f$ neklesající nebo nerostoucí, je \textbf{monotónní}. POkud je klesající nebo rostoucí, je \textbf{ryze monotónní}.
\end{definition}

\begin{definition}
  Nechť $f$ je funkce, $M\subseteq D(f)$. Řekneme, že funkce $f$ je v množině $M$
  \begin{enumerate}[$i.$]
    \item \textbf{shora omezená} $\iff \exists k \in \mathbb R: \forall x \in M: f(x)\leq k,$
    \item \textbf{zdola omezená} $\iff \exists k \in \mathbb R: \forall x \in M: f(x)\geq k,$
    \item \textbf{omezená} $\iff $ je zdola i shora omezená.
  \end{enumerate}
\end{definition}


\begin{definition}

\end{definition}
