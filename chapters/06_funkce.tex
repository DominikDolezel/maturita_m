\section{Funkce a jejich základní vlastnosti}
\begin{definition}
  \textbf{Funkcí} $f$ nazýváme každé zobrazení z $\mathbb R$ do $\mathbb R$.
\end{definition}

\begin{pozn}
  Obecněji by se funkce dala definovat také jako zobrazení z $\mathbb C$ do $\mathbb C$.
\end{pozn}

\begin{definition}
  Nechť $f\subseteq \mathbb R \times \mathbb R$ je funkce. Množinu
  \[
    D(f) = \left  \{ x \in \mathbb R:\exists ! y \in \mathbb R:y=f(x) \right \}
  \]
  nazveme \textbf{definičním oborem} funkce $f$. Množinu
  \[
    H(f) = \left  \{ y \in \mathbb R:\exists ! x \in \mathbb R:y=f(x) \right \}
  \]
  nazveme \textbf{oborem hodnot} funkce $f$.
\end{definition}

\begin{priklad}
Určete definiční obor a obor hodnot funkce $y=x^2+x+1.$
\end{priklad}

\begin{reseni}
Definiční obor je $\mathbb R$, obor hodnot (množinu všech vyhovujících $a$)
získáme řešením rovnice $a=x^2+x+1.$
\end{reseni}

\begin{pozn}
  \textbf{Grafem funkce} rozumíme množinu všech bodů $[x,f(x)]$, kde $x\in D(f).$
\end{pozn}

\begin{priklad}
Nakreslete graf funkce $y=2-|x+1|.$
\end{priklad}

\begin{priklad}
Nakreslete graf funkce $y=-2x^2+x-1$.
\end{priklad}

\begin{reseni}
Doplněním na čtverec, odkud jsou posuny jasně patrné.
\end{reseni}

\begin{definition}
  Funkce $f$ se nazývá \textbf{lichá} (resp. \textbf{sudá}), jestliže platí
  \begin{enumerate}[$i.$]
    \item $\forall x \in D(f): -x \in D(f)$ a
  	\item $\forall x \in D(f): f(-x)=-f(x)$, resp. $f(-x)=f(x)$.
  \end{enumerate}
\end{definition}

\begin{pozn}
  Graf liché funkce je souměrný podle počátku, graf sudé funkce je souměrný podle souřadné osy $y$.
\end{pozn}

\begin{priklad}
Určete paritu funkce $y=x^2$.
\end{priklad}

\begin{reseni}
\begin{enumerate}[1.]
\item $D(f)=\mathbb R$ -- platí,
\item $f(-x)=(-x)^2=x^2=f(x)$ -- platí, tedy funkce je sudá.
\end{enumerate}
\end{reseni}

\begin{definition}
  Funkce $f$ se nazývá \textbf{prostá}, právě tehdy když
  \[
    \forall x_1,x_2\in D(f): x_1\ne x_2 \implies f(x_1)\ne f(x_2).
  \]
\end{definition}

\begin{definition}
  Nechť $f$ je funkce a $M$ alespoň dvouprvková množina z $D(f)$. Řekneme, že funkce $f$ je na množině $M$
  \begin{enumerate}[$i.$]
    \item \textbf{rostoucí} $\iff \forall x_1, x_2 \in M: x_1 < x_2 \implies f(x_1) < f(x_2),$
    \item \textbf{klesající} $\iff \forall x_1, x_2 \in M: x_1 < x_2 \implies f(x_1) > f(x_2),$
    \item \textbf{neklesající} $\iff \forall x_1, x_2 \in M: x_1 < x_2 \implies f(x_1) \leq f(x_2),$
    \item \textbf{nerostoucí} $\iff \forall x_1, x_2 \in M: x_1 < x_2 \implies f(x_1) \geq f(x_2).$
  \end{enumerate}
  Je-li $f$ neklesající nebo nerostoucí, je \textbf{monotónní}. Pokud je klesající nebo rostoucí, je \textbf{ryze monotónní}.
\end{definition}

\begin{priklad}
Určete monotónnost funkce $y=x^2$ na množině $\mathbb R_0^+.$
\end{priklad}
\begin{reseni}
Je-li $x_1,x_2 >0$, pak $x_1 < x_2 \implies x_1^2 < x_1x_2.$ Obdobně $x_1x_2<x_2^2$.
Celkem tedy $x_1^2<x_2^2 \implies f(x_1)<f(x_2),$ tedy $f$ je rostoucí v $\mathbb R_0^+.$
\end{reseni}

\begin{definition}
  Nechť $f$ je funkce, $M\subseteq D(f)$. Řekneme, že funkce $f$ je na množině $M$
  \begin{enumerate}[$i.$]
    \item \textbf{shora omezená} $\iff \exists k \in \mathbb R: \forall x \in M: f(x)\leq k,$
    \item \textbf{zdola omezená} $\iff \exists k \in \mathbb R: \forall x \in M: f(x)\geq k,$
    \item \textbf{omezená} $\iff $ je zdola i shora omezená.
  \end{enumerate}
\end{definition}

\begin{definition}
  Nechť $f$ je funkce, $M \subseteq D(f)$, v ní prvek $a \in M$.
  Řekneme, že funkce $f$ má v bodě $a$:
  \begin{enumerate}[$i.$]
    \item \textbf{ostré maximum} na množině $M$ právě tehdy, když $\forall x \in M\smallsetminus \left \{ a \right \}  : f(x) < f(a)$,
    \item \textbf{maximum} (neostré) na množině $M$ právě tehdy, když $\forall x \in M \smallsetminus \left \{ a \right \}: f(x) \leq f(a)$,
    \item \textbf{ostré minimum} na množině $M$ právě tehdy, když $\forall x \in M\smallsetminus \left \{ a \right \}: f(x) > f(a)$,
    \item \textbf{minimum} (neostré) na množině $M$ právě tehdy, když $\forall x \in M\smallsetminus \left \{ a \right \} : f(x) \geq f(a)$.
  \end{enumerate}
\end{definition}

\begin{definition}
  Nechť $f$ je funkce. Funkce $f$ se nazývá \textbf{periodická}, pokud $\exists p \in \mathbb R^{+}: \forall x \in D(f):$
  \begin{enumerate}
    \item $x \in D(f) \implies x \pm p \in D(f)$ a
    \item $f(x) = f(x \pm p)$.
  \end{enumerate}
  Číslo $p$ se nazývá \textbf{periodou} této funkce. Periodu $p_0$ nazveme \textbf{nejmenší periodou} funkce, pokud pro všechny ostatní periody $p$ platí $p > p_0$. V opačném případě se funkce nazývá \textbf{neperiodická}.
\end{definition}

\begin{definition}
  Nechť máme funkci $f: y = f(u)$ s definičním oborem $D(f)$ a funkci $g: u=g(x)$ s oborem hodnot $H(g)$. Jestliže je $H(g) \subseteq D(f)$, pak funkci $h: y = f(g(x))$ nazveme \textbf{složenou funkcí} (někdy píšeme též $h=f \circ g$).
\end{definition}

\begin{definition}
  \textbf{Dirichletova funkce} je definována následovně:
  $$\mathbf{D} (x) = \begin{cases}
1, & \text{ je-li } x \in \mathbb Q, \\
0 & \text{ jinak}.
  \end{cases}  $$
\end{definition}

\begin{definition}
  Funkce \textbf{signum} je definována následovně: $$\operatorname {sgn} x={\begin{cases}-1,& \textrm{je-li }x<0,\\0,&\textrm{je-li }x=0,\\1,&\textrm{je-li }x>0.\end{cases}}$$
\end{definition}

\begin{definition}
  Nechť $x \in \mathbb{R}$ je libovolné číslo. Pak existuje právě jedna dvojice $z \in \mathbb{Z}, a \in \left \langle 0;1 \right) \text{ tak, že } x = z + a$.
Číslo $z$ nazýváme \textbf{celou částí} čísla $x$ a zapisujeme $[x] = z$, někdy taky $\lfloor x \rfloor = z$.
\end{definition}

\begin{pozn}
    Pro definici limity a funkce spojité viz kapitolu \ref{limita},
    dále definici derivace def. \ref{derivace}, funkce konvexní a
    konkávní def. \ref{konvkonk} a konečně definici integrálu def. \ref{integral}.
\end{pozn}
