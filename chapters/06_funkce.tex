\section{Funkce a jejich základní vlastnosti}
\begin{definition}
  \textbf{Funkcí} $f$ nazýváme každé zobrazení z $\mathbb R$ do $\mathbb R$.
\end{definition}

\begin{definition}
  Nechť $f\subseteq \mathbb R \times \mathbb R$ je funkce. Množinu
  \[
    D(f) = \left  \{ x \in \mathbb R:\exists ! y \in \mathbb R:y=f(x) \right \}
  \]
  nazveme \textbf{definičním oborem} funkce $f$. Množinu
  \[
    H(f) = \left  \{ y \in \mathbb R:\exists ! x \in \mathbb R:y=f(x) \right \}
  \]
  nazveme \textbf{oborem hodnot} funkce $f$.
\end{definition}

\begin{pozn}
  \textbf{Grafem funkce} rozumíme množinu všech bodů $[x,f(x)]$, kde $x\in D(f).$
\end{pozn}


\begin{definition}
  Funkce $f$ se nazývá \textbf{sudá} (resp. \textbf{lichá}), jestliže platí
\begin{itemize}
  \item
  $\forall x \in D(f):(-x) \in D(f)$
\item
 $\forall x \in D(f): f(-x)=f(x)$, resp. $f(-x)=f(x)$.
\end{itemize}

\end{definition}

\begin{definition}
  Funkce $f$ se nazývá \textbf{prostá}, právě tehdy když platí
  \[
    \forall x_1,x_2\in D(f): x_1\ne x_2 \implies f(x_1)\ne f(x_2)
  \]
\end{definition}

\begin{definition}
  Nechť $f$ je funkce a $M$ alespoň dvouprvková množina z $D(f)$. Řekneme, že funkce $f$ je v množině $M$
  \begin{enumerate}[$i.$]
    \item \textbf{rostoucí} $\iff \forall x_1, x_2 \in M: x_1 < x_2 \implies f(x_1) < f(x_2),$
    \item \textbf{klesající} $\iff \forall x_1, x_2 \in M: x_1 < x_2 \implies f(x_1) > f(x_2),$
    \item \textbf{neklesající} $\iff \forall x_1, x_2 \in M: x_1 < x_2 \implies f(x_1) \leq f(x_2),$
    \item \textbf{nerostoucí} $\iff \forall x_1, x_2 \in M: x_1 < x_2 \implies f(x_1) \geq f(x_2).$
  \end{enumerate}
  Je-li $f$ neklesající nebo nerostoucí, je \textbf{monotónní}. POkud je klesající nebo rostoucí, je \textbf{ryze monotónní}.
\end{definition}

\begin{definition}
  Nechť $f$ je funkce, $M\subseteq D(f)$. Řekneme, že funkce $f$ je v množině $M$
  \begin{enumerate}[$i.$]
    \item \textbf{shora omezená} $\iff \exists k \in \mathbb R: \forall x \in M: f(x)\leq k,$
    \item \textbf{zdola omezená} $\iff \exists k \in \mathbb R: \forall x \in M: f(x)\geq k,$
    \item \textbf{omezená} $\iff $ je zdola i shora omezená.
  \end{enumerate}
\end{definition}


\begin{definition}
  Nechť $f$ je funkce, $M \subseteq D(f)$, v ní prvek $a \in M$.
  Řekneme, že funkce f má v bodě a:
  \begin{enumerate}[i.]
    \item \textbf{ostré maximum} na množině $M$ právě tehdy, když $\forall x \in M; x \not = a: f(x) < f(a)$
    \item \textbf{maximum} (neostré) na množině $M$ právě tehdy, když $\forall x \in M : f(x) \leq f(a)$
    \item \textbf{ostré minimum} na množině $M$ právě tehdy, když $\forall x \in M; x > a: f(x) > f(a)$
    \item \textbf{minimum} (neostré) na množině $M$ právě tehdy, když $\forall x \in M : f(x) \geq f(a)$
  \end{enumerate}
\end{definition}

\begin{definition}
  Nechť $f$ je funkce. Funkce $f$ se nazývá \textbf{periodická}, pokud $\forall p \in \mathbb R^{+}: \forall x \in D(f):$
  \begin{enumerate}
    \item $x \in D(f) \implies x \pm p \in D(f)$
    \item $f(x) = f(x \pm p)$
  \end{enumerate}
  Číslo $p$ se nazývá \textbf{periodou} této funkce. Periodu $p_0$ nazveme \textbf{nejmenší periodou} funkce, pokud pro všechny ostatní periody $p$ platí $p > p_0$. V opačném případě se funkce nazývá \textbf{neperiodická}.
\end{definition}

\begin{definition}
  Máme funkci $f: y = f(u)$ s definičním oborem $D(f)$ a funkci $g: u=g(x)$ s oborem hodnot $H(g)$. Jestliže je $H(g) \subseteq D(f)$, pak funkci $h: y = f(g(x))$ nazveme \textbf{složenou funkcí} (někdy píšeme též $h=f \circ g$).
\end{definition}

\begin{definition}
  \textbf{Dirichletova funkce} je definována vzorcem
  $$\mathbf{D} (x) = \begin{cases}
1 \text{ pokud } x \in \mathbb Q \\
0 \text{ jinak }
  \end{cases}  $$
\end{definition}

\begin{definition}
  Funkce \textbf{signum} je definováno následujícím způsobem $$\operatorname {sgn} x={\begin{cases}-1,&x<0\\0,&x=0\\1,&x>0\end{cases}}$$
\end{definition}

\begin{definition}
  Nechť $x \in \mathbb{R}$ je libovolné číslo. Pak existuje právě jedna dvojice $z \in \mathbb{Z}, a \in \left \langle 0;1 \right) \text{ tak, že } x = z + a$.
Číslo z nazýváme \textbf{celou částí} čísla x a zapisujeme $[x] = z (\lfloor x \rfloor = z)$.
\end{definition}

\begin{definition}
  O funkci $f:\mathbb {R} \rightarrow \mathbb {R}$ řekneme, že je \textbf{spojitá} v bodě $a$, pokud ke každému libovolně malému číslu $\varepsilon >0$ existuje takové číslo $\delta >0$, že pro všechna $x$, pro něž platí $|x-a|<\delta$, platí také $|f(x)-f(a)|<\varepsilon$.
\end{definition}

\begin{definition}
  Číslo $A\in \mathbb {R}$ je limitou funkce $f:\mathbb {R} \rightarrow \mathbb {R}$ v bodě $ a\in \mathbb {R}$, jestliže k libovolnému $ \varepsilon >0$ existuje takové $ \delta >0$, že pro všechna $x\in D(f)$ taková, že $ \left|x-a\right|<\delta$ ($x$ leží v prstencovém okolí bodu $a$) platí $\left|f(x)-A\right|<\varepsilon $.

  Limitu má smysl zkoumat jen v definičním oboru funkce neobsahujícím bod $a$, tj. libovolně blízko k bodu $a$ musí být funkce definována.
\end{definition}

\begin{definition}
  Nejběžnější moderní definice \textbf{derivace} funkce $f$ v bodě $a$, zapisujeme $f'(a)$ je
  $$f'(a)=\lim _{h\to 0}{\frac {f(a+h)-f(a)}{h}}=\lim _{x\to a}{\frac {f(x)-f(a)}{x-a}}.$$

  Co to znamená se mě neptejte (klidně se zeptejte). Vložte intuitivní definici derivace.
\end{definition}

\begin{definition}
  Nechť $f$ je funkce spojitá na intervalu $(a,b)$. Pak říkáme, že funkce $f$ je na intervalu $(a,b)$ \textbf{konvexní} (resp. \textbf{ryze konvexní}) právě tehdy, když pro libovolné číslo $\lambda \in (0,1)$ s vlastností $\forall x,y\in (a,b),x<y:f(\lambda x+(1-\lambda )y) < (\text{resp. } \leq) \lambda f(x)+(1-\lambda )f(y)$

  Pokud spojitá funkce není na intervalu konvexní (resp. ryze konvexní), je na něm ryze konkávní (resp. konkávní).
\end{definition}

\begin{definition}
  \textbf{Primitivní funkce} k funkci $f$ na intervalu $(a,b)$ je taková funkce $F$, že pro každé $x\in (a,b)$ je $F'(x)=f(x)$.

  Procesu hledání primitivní funkce se často říká \textbf{integrování} nebo \textbf{integrace} (od slova integrál).
\end{definition}
