\section{Vektorové prostory}
\begin{definition}
    Nechť $P, X, Q, Y\in \mathbb E_3$ jsou čtyři libovolné body. Řekneme, že body
    $P, X, Q, Y$ tvoří
    vrcholy zobecněného rovnoběžníku, jestliže střed úsečky $PQ$ splývá se středem úsečky
    $XY$.
\end{definition}

\begin{definition}
    Na množině $\mathscr U$ všech orientovaných úseček s pevným počátečním bodem
    $P\in \mathbb E_3$
    definujeme operaci sčítání takto:
    $$\forall \overrightarrow{PX}, \overrightarrow{PY} \in \mathscr U:
        \overrightarrow{PX} +  \overrightarrow{PY} = \overrightarrow{PQ},$$
    je-li bod $Q\in \mathbb E_3$
    vrcholem zobecněného rovnoběžníku $PXQY$.
\end{definition}

\begin{definition}
    Na množině $\mathscr U$ všech orientovaných úseček s pevným počátečním bodem
    $P\in \mathbb E_3$
    definujeme operaci \textbf{násobení orientovaných úseček reálným číslem}
    (tzv. vnější násobení) takto: je-li $p\in \mathbb R$ a $\overrightarrow{PX}\in \mathscr U$,
    pak $p$-násobkem orientované
    úsečky $\overrightarrow{PX}$ nazveme orientovanou úsečku $\overrightarrow{PY}$
    (a zapisujeme $\overrightarrow{PY}=p \cdot \overrightarrow{PX}$), přičemž
    platí:
    \begin{enumerate}[$i.$]
    \item $p=0 \implies Y=P,$
   	\item $p\ne 0 \implies Y = \mathscr H_{P,p}(X).$
    \end{enumerate}
\end{definition}

\begin{definition}
    Nechť je dána množina $G$ s operací $*:G\times G \to G$ (je na $G$ uzavřená).
    Pak dvojice $(G,*)$ je \textbf{grupa}, jestliže platí:
    \begin{enumerate}[$i.$]
        \item $\forall a,b,c, \in G: a*(b*c) = (a*b)*c$ (asociativita),
       	\item $\exists e \in G$ takové, že $\forall a\in G: a*e=e*a=a$ (existence neutrálního prvku),
       	\item $\forall a\in G: \exists a^{-1}\in G$ takové, že $a*a^{-1}=a^{-1}a=e$ (existence inverzního prvku).
    \end{enumerate}
    Pokud navíc
    \begin{enumerate}[$iv.$]
        \item $\forall a,b \in G: a*b=b*a$ (komutativita),
    \end{enumerate}
    je $(G,*)$ \textbf{komutativní} (též \textbf{Abelovská}) \textbf{grupa}.
\end{definition}

\begin{definition}\label{vekt_prost}
    Nechť je dána množina $V$, těleso $T$ a dvě operace $\bigoplus: V\times V \to V$ a~$\bigotimes: T\times V \to V$ (jsou na $V$ uzavřené). Pak čtveřice $(V,T,\bigoplus,
    \bigotimes)$ je \textbf{vektorový prostor} nad tělesem $T$, jestliže $\forall p,q \in T, \vec u,
    \vec v, \vec w \in V$ platí:
    \begin{enumerate}[$i.$]
    \item $\vec u \bigoplus \vec v = \vec v \bigoplus \vec u$ (komutativita sčítání),
   	\item $(\vec u \bigoplus \vec v)\bigoplus \vec w = \vec u \bigoplus (\vec v \bigoplus \vec w)$ (asociativita sčítání),
   	\item $\exists \vec o\in V$ takové, že $\forall \vec u\in V:\vec u \bigoplus \vec o = \vec o \bigoplus \vec u = \vec u$ (existence nulového prvku),
   	\item $\forall \vec u \in V: \exists (-\vec u) \in V: \vec u \bigoplus (-\vec u) = (-\vec u) \bigoplus \vec u = \vec o$ (existence opačného prvku),
   	\item $p\bigotimes (\vec u \bigoplus \vec v)=p\bigotimes \vec u \bigoplus p\bigotimes \vec v$ (distributivita),
   	\item $(p+q)\bigotimes \vec u=p\bigotimes \vec u \bigoplus q\bigotimes \vec u$ (distributivita),
   	\item $(p\cdot q)\bigotimes \vec u = p\bigotimes (q\bigotimes \vec u)$ (asociativita vnějšího násobení),
   	\item $\exists 1 \in T$ taková, že $\forall \vec u \in V: 1\bigotimes \vec u = \vec u$ (existence neutrálního prvku vzhledem k~násobení).
    \end{enumerate}
\end{definition}

\begin{pozn}
    Výčet prvních čtyř podmínek z definice \ref{vekt_prost} lze zjednodušit jako:
    $(V,\bigoplus)$ je komutativní grupa.
\end{pozn}

\begin{pozn}
    Místo znaků $\bigoplus$ (resp. $\bigotimes$) píšeme znaky $+$ (resp. $\cdot$).
    Byly použity, aby bylo jednoznačně odlišeno sčítání vektorů a čísel (resp. násobení
    vektorů skalárem a násobení čísel). Z kontextu je však jasně zřejmé, kterou operaci
    použít. V dalším textu budeme již používat znaky $+$ (resp $\cdot$).
\end{pozn}

\begin{pozn}
    V definici jsou skaláry obecně z tělesa (viz poznámku \ref{teleso}). V dalším textu
    pro jednoduchost uvažujeme $T=\mathbb R$.
\end{pozn}

\begin{pozn}
    Množina $\mathscr U_n$ všech orientovaných úseček s počátečním bodem $P\in \mathbb E_n$
    je vektorovým prostorem
    \textbf{vázaných vektorů}.
\end{pozn}

\begin{pozn}
    Množina $\mathbb R^{(n)}$ všech uspořádaných $n$-tic tvoří \textbf{aritmetický}
    vektorový prostor.
\end{pozn}

\begin{definition}
    Nechť $A,B,C,D \in \mathbb E_n$ jsou body. Řekneme, že orientované úsečky
    $\overrightarrow{AB}$ a $\overrightarrow{CD}$ jsou \textbf{ekvipolentní}, jestliže
    střed úsečky $AD$ je i středem úsečky $BC$. Zapisujeme $\overrightarrow{AB}\,\varepsilon\,\overrightarrow{CD}.$
\end{definition}

\begin{pozn}
    Pro definici relace ekvivalce viz def. \ref{reflsymtran}, \ref{ekvivalence} a poznámku \ref{rozkladprislusnyekvivalenci}.
\end{pozn}

\begin{definition}
    Nechť je dána množina $M$ všech orientovaných úsešek v $\mathbb E_n$ a
    relace ekvipolence $\varepsilon \subseteq M\times M.$ Pak třída rozkladu množiny
    $M$, který přísluší relaci $\varepsilon$, je \textbf{volný vektor}.
\end{definition}

\begin{definition}
    Nechť $V$ je množina všech volných vektorů v $\mathbb E_n$. Pak na množině $V$ definujeme
    operace sčítání a vnější násobení takto:
    \begin{enumerate}[$i.$]
    \item $\forall \vec u, \vec v \in V: \vec u + \vec v = \vec w,$ kde $\vec w = \left \{ \overrightarrow{XY}; \overrightarrow{XY} \, \varepsilon \, \overrightarrow{PC} \right \} $, kde $\overrightarrow{PA}\in\vec u, \overrightarrow{PB}\in \vec v$ a $\overrightarrow{PC}=\overrightarrow{PA}+\overrightarrow{PB}$ je součet orientovaných úseček $\overrightarrow{PA},\overrightarrow{PB}.$
   	\item $\forall p \in \mathbb R, \forall \vec u \in V: p\cdot \vec u = \vec z, \vec z = \left \{ \overrightarrow{XY}; \overrightarrow{XY} \, \varepsilon \, \overrightarrow{PC} \right \} $, přitom $\overrightarrow{PE}\in \vec u$ a $\overrightarrow{PF} = p\cdot \overrightarrow{PE}$ je vnější součin orientované úsečky $\overrightarrow{PE}$ a čísla $p$ (tj. pomocí stejnolehlosti).
    \end{enumerate}
\end{definition}

\begin{definition}
Nechť $\vec u \subseteq M$ je libovolný volný vektor a $\overrightarrow{AB}\subseteq \vec u$ orientovaná úsečka, tedy
$u =\left  \{ \overrightarrow{XY} : \overrightarrow{XY} \, \varepsilon \, \overrightarrow{AB} \right \}$. Pak orientovanou úsečku $\overrightarrow{AB}$ nazveme \textbf{umístěním} vektoru $\vec u$.
\end{definition}

\begin{definition}
    Pokud $\overrightarrow{PA}\in \vec u$, nazýváme orientovanou úsečku $\overrightarrow{PA}$ též reprezentantem vektoru $\vec u$.
\end{definition}

\begin{definition}
    Nechť $V$ je vektorový prostor, $\vec u_1,\dots, \vec u_k\in V$ vektory, $p_1,\dots,
    p_k\in \mathbb R.$ Vektor
    $$\vec x = p_1\vec u_1 + p_2\vec u_2 + \dots + p_k\vec u_k = \sum_{i=1}^{k} p_i\vec u_i$$
    nazýváme \textbf{lineární kombinací} vektorů $\vec u_1,\dots, \vec u_k.$
\end{definition}

\begin{definition}
    Podmožina $W$ vektorového prostoru $V$ se nazývá \textbf{podprostor} vektorového
    prostoru $V$, jestliže $W$ je vektorový prostor.
\end{definition}

\begin{definition}
    Množina všech lineárních kombinací vektorů množiny $S$ označená $\left < S \right >$ se nazývá \textbf{lineární
    obal} množiny $S$ a její prvky \textbf{generátory} $\left < S \right >$.
\end{definition}

\begin{priklad}
Je dán aritmetický vektorový prostor $\mathbb R^{2}$ a jeho podmnožina $S=\left \{ (1,2), (4,3) \right \}. $
Rozhodněte, zda vektor $\vec x = (7,4)$ patří do množiny generované $S$.
\end{priklad}

\begin{reseni}
Vlastně řešíme rovnici $(7,4)=a(1,2)+b(4,3).$
\end{reseni}

\begin{definition}
    Nechť $S= \left \{ \vec u_1, \dots, u_k \right \} $ je množina vektorů vektorového prostoru $V$. Množina $S$ je
   	\begin{enumerate}[$i.$]
    \item \textbf{lineární nezávislá}, jestliže
    $$p_1\vec u_1 + p_2\vec u_2 + \dots + p_k\vec u_k = \vec o \iff p_1 = p_2 = \dots = p_k = 0;$$
   	\item \textbf{lineárně závislá}, jestliže existuje $p_i\ne 0$ takové, že
    $$ p_1\vec u_1 + p_2\vec u_2 + \dots + p_k\vec u_k = \vec o.$$
    \end{enumerate}
\end{definition}

\begin{veta}[Kriterium lineární závislosti]
    Vektory jsou závislé, jestliže alespoň jeden z nich lze vyjádřit jako lineární
    kombinaci ostatních.
\end{veta}

\begin{priklad}
V $\mathbb R^{(3)}$ jsou dány vektory $\vec a = (2,1,5), \vec b = (3,0,4), \vec c=(1,2,6),
\vec c^\prime = (1,2,5).$ Rozhodněte, zda jsou vektory
\begin{enumerate}[$a.$]
\item $\vec a, \vec b,\vec c,$
\item $\vec a, \vec b,\vec c^\prime$
\end{enumerate}
lineárně závislé.
\end{priklad}

\begin{reseni}
Vlastně řešíme rovnici $p\vec a + q\vec b + r\vec c= \vec o.$ Existuje-li nenulové
řešení, jsou lineárně závislé.
\end{reseni}

\begin{definition}
    Nechť je dán vektorový prostor $V$. Pak množina $\left < W \right >,$ kde $
    W\subseteq V$, se nazývá \textbf{podprostor} prostoru $V$ \textbf{generovaný}
    množinou $W$ a prvky množiny $W$ \textbf{generátory} tohoto podprostoru.
\end{definition}

\begin{definition}
    Nechť $(\vec u_1, \dots \vec u_k)$ je konečná posloupnost vektorů vektorového
    prostoru $V$. Tato posloupnost tvoří \textbf{bázi} vektorového prostoru $V$, jestliže
    \begin{enumerate}[$i.$]
    \item $\vec u_1, \dots, \vec u_k$ jsou generátory $V$ a
   	\item $\vec u_1, \dots, \vec u_k$ jsou lineárně nezávislé.
    \end{enumerate}
\end{definition}

\begin{definition}
Nechť $(\vec e_1,\dots, \vec e_k)$ je báze vektorového prostoru $V$. Pak číslo $k \in
\mathbb N_0$ je \textbf{dimenzí} vektorového prostoru $V$ a píšeme $\dim V=k$.
\end{definition}

\begin{priklad}
Nalezněte dimenzi a bázi vektorového prostoru $\left < S \right >$ (tj. generovaného množinou $S$), je-li $S = \left
\{ \vec u_1,\vec u_2,\vec u_3,\vec u_4,\vec u_5 \right \}, $
kde
$\vec u_1=(2,3,5,-4,1),$ $\vec u_2=(1,-1,2,3,5),$ $\vec u_3=(3,7,8,-11,-3),$ $\vec u_4=(1,-1,1,-2,3),$ $\vec u_5=(1,4,3,-7,-4).$
Dále vyjádřete vektory $\vec u_1$ až $\vec u_5$ jako lineární kombinaci vektorů báze.
\end{priklad}

\begin{reseni}
Vlastně hledáme hodnost matice
$$
\begin{pmatrix}
    \vec u_1\\
   \vec u_2\\
  \vec u_3\\
 \vec u_4\\
\vec u_5
\end{pmatrix}.
$$
Gaussovou eliminací pokračujeme tak dlouho, dokud nedostaneme matici do schodovitého tvaru.
Řádky takové matice jsou pak generátory $\left < S \right > .$
Vyjadřujeme-li vektor jako lineární kombinaci báze, řešíme vlastně rovnici
$\vec u_i = p_1\vec e_1 + p_2\vec e_2 +\dots + p_n\vec e_n.$
\end{reseni}

\begin{priklad}V pravidelném čtyřstěnu $ABCD$ vyjádřete vektory $\overrightarrow{AS},\overrightarrow{AU},\overrightarrow{AT}$ jako
lineární kombinaci vektorů $\overrightarrow{AB},\overrightarrow{AC},\overrightarrow{AD},$
je-li $S$ střed strany $BC$, $U$ těžiště trojúhelníka $BCD$ a $T$ je těžiště
$ABCD$.
\end{priklad}

\begin{definition}\label{izom}
    Nechť $(\vec e_1,\dots, \vec e_k)$ je báze vektorového prostoru $V$. Zobrazení
    $\varphi: A\to B$ se nazývá \textbf{homomorfismus} vzhledem k operaci $*$, jestliže
    $$\forall x,y \in A:\varphi(x*y)=\varphi(x)*\varphi(y).$$
    Pokud je $\varphi$ navíc bijektivní, nazývá se \textbf{izomorfismus}.
\end{definition}

\begin{pozn}
    Dva vektorové prostory $V,W$ nad týmž tělesem $T$ jsou izomorfní, jestliže vztah pro zobrazení $\varphi:V\to W$ z def. \ref{izom} platí pro
    operace sčítání i vnější násobení, tzn. platí:
    \begin{enumerate}[$i.$]
    \item $\forall x,y \in V: \varphi(\vec x+\vec y)=\varphi(\vec x)+\varphi(\vec y),$
    	\item $\forall x \in V, \forall p \in T: \varphi(p\cdot \vec x) = p\cdot \varphi(\vec x).$
    \end{enumerate}
\end{pozn}

\begin{definition}
    Nechť $(\vec e_1, \dots, \vec e_n)$ je báze vektorového prostoru $V$. Nechť
    $\vec x \in V$ je libovolný vektor takový, že
    $$\vec x = x_1\vec e_1 + x_2\vec e_2 + \dots + x_n\vec e_n,$$
    kde $x_i \in \mathbb R, i\in \left \{ 1,\dots,n \right \} .$ Pak uspořádanou
    $n$-tici $(x_1, x_2,\dots,x_n)$ nazýváme \textbf{souřadnice vektoru} $\vec x$ v
    bázi $(\vec e_1, \dots, \vec e_n)$.
\end{definition}

\begin{priklad}
Ve vektorovém prostoru $\mathbb R^{(4)}$ je dána báze
\begin{equation*}
    \vec e_1=(1,1,1,1), \vec e_2 = (1,1,1,0), \vec e_3 = (1,1,0,0), \vec e_4 = (1,0,0,0).
\end{equation*}
Určete souřadnice vektoru $\vec x = (1,0,1,0)$ v této bázi.
\end{priklad}

\begin{reseni}
Vlastně řešíme rovnici $\vec x= x_1\vec e_1 + x_2\vec e_2 + x_3\vec e_3 + x_4\vec e_4.$
\end{reseni}

\begin{definition}
    Nechť je dán bod $P\in \mathbb E_3$ a báze $(\vec e_1, \vec e_2, \vec e_3)$ vektorového
    prostoru volných vektorů $\mathscr V_3$. Pak uspořádaná čtveřice
    $(P, \vec e_1, \vec e_2, \vec e_3)$ je
    \textbf{afinní soustava souřadnic} v $\mathbb E_3$ s~počátkem $P$.
\end{definition}

\begin{definition}
    Nechť je dán bod $P\in \mathbb E_3$ a báze $(\vec e_1, \vec e_2, \vec e_3)$ vektorového
    prostoru volných vektorů $\mathscr V_n$ taková, že platí:\\
    Jsou-li $X,Y,Z\in \mathbb E_3$ takové body, že
    $\overrightarrow{PX}\in\vec e_1,\overrightarrow{PY}\in\vec e_2,
    \overrightarrow{PZ}\in\vec e_3$, souřadné osy $\overrightarrow{PX},\overrightarrow{PY},
    \overrightarrow{PZ}$ jsou navzájem kolmé (tedy báze je ortogonální) a jejich velikost je jedna (tedy báze je navíc ortonormální),
    pak uspořádaná čtveřice $(P, \vec e_1, \vec e_2, \vec e_3)$ je \textbf{kartézská
    soutava souřadnic} v~$\mathbb E_3$ s počátkem $P$.
\end{definition}

\begin{definition}
\textbf{Velikostí vektoru} $\overrightarrow{AB}$ rozumíme délku úsečky $AB$ a zapisujeme
$|\overrightarrow{AB}| = |AB|.$ Vektor o velikosti jedna nazýváme \textbf{jednotkovým
vektorem}.
\end{definition}

\begin{definition}
\textbf{Skalární součin} je zobrazení $V\times V\to \mathbb R$ (označené $\cdot$) s následujícími vlastnostmi:
\begin{enumerate}[$i.$]
\item $\vec u \cdot \vec v = \vec v \cdot \vec u,$
\item $(\vec u+\vec v)\cdot \vec w = \vec u\cdot \vec w + \vec v\cdot \vec w,$
\item $(p\cdot \vec u)\cdot \vec v = p\cdot(\vec u \cdot \vec v),$
\item $\vec u\cdot \vec v \geq 0, \vec u \cdot \vec u = 0 \iff \vec u = \vec o$
\end{enumerate}
pro všechna $\vec u, \vec v, \vec w \in V.$
\end{definition}

\begin{veta}
    Pro nenulové vektory $\vec u, \vec v \in V$ platí
    $$\vec u \cdot \vec v = |\vec u|\cdot |\vec v|\cdot \cos \varphi,$$
    kde $\varphi$ je konvexní úhel, který svírají vektory $\vec u, \vec v.$
\end{veta}

\begin{definition}
    Vektory jsou \textbf{ortogonální} (kolmé), jestliže jejich skalární součin
    je roven nule.
\end{definition}

\begin{definition}
    Množina vektorů je \textbf{ortonormální}, jestliže je ortogonální a všechny vektory mají
    velikost 1.
\end{definition}

\begin{veta}[Gramm-Schmidtův ortogonalizační proces]
    Je dána množina vektorů $M=\left \{ \vec u_1,\dots, \vec u_n \right \}$.
    Hledáme ortogonální bázi $\left \{ \vec e_1, \dots, \vec e_n \right \}$
    generující $\left < M \right >.$
    \begin{enumerate}[1.]
    \item Položme $\vec e_1 = \vec u_1.$
   	\item $k$-tý vektor volíme tak, aby
    \begin{equation}\label{gs}
    \vec e_k = p_1\vec e_1 + p_2\vec e_2 + \dots +p_{k-1} \vec e_{k-1} + \vec u_k.
    \end{equation}
    Dosazením do (\ref{gs}) dostaneme
    $$p_i = -\frac{\vec u_k\cdot \vec e_i}{\vec e_i\cdot \vec e_i}.$$
    (Skalární součiny každých dvou různých bázových vektorů jsou nula.)
    \end{enumerate}
\end{veta}

\begin{priklad}
Ve vektorovém prostoru $\mathbb R^{(5)}$ je dán systém generátorů $\vec u_1,\dots,\vec u_4$:
$$
    \vec u_1 = (0,1,0,0,0), \vec u_2 = (1,1,1,2,1), \vec u_3 = (1,1,0,-1,0), \vec u_4 = (0,0,1,3,1).
$$
najděte ortogonální a ortonormální bázi tohoto vektorového prostoru.
\end{priklad}

\begin{reseni}
Položme $\vec e_1=\vec u_1.$ Dále $\vec e_2 = p_1\vec e_1 + \vec u_2$. Po vynásobení
této matice $\vec e_1$ (ten je kolmý na $\vec e_2$) dostáváme
$$\underbrace{\vec e_2\cdot \vec e_1}_{0}=p_1\vec e_1\cdot\vec e_1+\vec u_2\cdot\vec e_1,$$
takže
$$p_1=-\frac{\vec u_2\cdot\vec e_1}{\vec e_1\cdot\vec e_1} = -\frac{1}{1}=-1,$$
a pak
$$\vec e_2 = -1(0,1,0,0,0)+(1,1,1,2,1)=(1,0,1,2,1).$$
Obdobně pak řešíme rovnici $\vec e_3=q_1\vec e_1 + q_2\vec e_2 + \vec u_3$. Vynásobíme
ji postupně $\vec e_1$, $\vec e_2$, takže dostaneme dvě rovnice, na jejichž levé straně
je součin dvou kolmých vektorů. Obdobně pokračujeme dál. Ortonormální bázi
dostaneme tak, že každý z vektorů ortogonální báze vydělíme jeho velikostí.
\end{reseni}

\begin{priklad}
Rozhodněte, zda jsou tělesové úhlopříčky krychle navzájem kolmé.
\end{priklad}

\begin{reseni}
Vyjádříme si je jako vektory a spočítáme skalární součin. Pokud je roven nule, jsou kolmé.
\end{reseni}

\begin{definition}
\textbf{Vektorový součin} je zobrazení $V\times V\to  V$ (označené $\times$) s následujícími vlastnostmi:
\begin{enumerate}[$i.$]
\item $\vec u\times \vec v = -(\vec v \times \vec u),$
\item $(\vec u + \vec v)\times \vec w = \vec u \times \vec w + \vec v \times \vec w \land \vec u\times (\vec v + \vec w) = \vec u \times \vec v+\vec u\times \vec w,$
\item $(p\cdot \vec u)\times \vec v = p\cdot(\vec u \times \vec v) = \vec u \times (p\cdot \vec v)$
\end{enumerate}
pro všechna $\vec u, \vec v, \vec w \in V.$
\end{definition}

\begin{priklad}
Vypočtěte obsah trojúhelníku $ABC$, je-li $A[-1,1], B[5,4], C[2,7].$
\end{priklad}

\begin{reseni}
Abychom mohli spočítat vektorový součin (tady polovinu vektorového součinu), musíme
mít trojrozměrné vektory. Doplňme je tedy: $A[-1,1, 0], B[5,4, 0], C[2,7,0].$ Pak
je obsah polovina velikosti vektorového součinu dvou vektorů (třeba $\overrightarrow{AB}$ a $\overrightarrow{BC}$).
\end{reseni}

\begin{veta}
    Pro každé tři lineárně nezávislé vektory $\vec u, \vec v, \vec w \in V$ takové,
   že $\vec u\times \vec v = \vec w,$ platí:
   \begin{enumerate}[$i.$]
   \item $|\vec w| = |\vec u|\cdot |\vec v|\cdot \sin \varphi,$ kde $\varphi$ je úhel, který svárají,
  	\item $\vec w\perp \vec u, \vec w \perp \vec v,$
  	\item $\vec u, \vec v, \vec w$ jsou kladně orientované (v pravotočivé bázi).
   \end{enumerate}
\end{veta}

\begin{pozn}
    Velikost vektorového součinu je obsah trojúhelníka, jehož dvě strany jsou
    násobené vektory.
\end{pozn}

\begin{definition}
    Pro všechny vektory $\vec u, \vec v, \vec w \in V$ je
    $$(\vec u \times \vec v) \cdot \vec w$$
    \textbf{smíšený součin} vektorů $\vec u, \vec v, \vec w.$
\end{definition}
