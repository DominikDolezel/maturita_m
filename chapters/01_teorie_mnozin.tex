\section{Základní pojmy z teorie množin}
\begin{definition}
  \textbf{Množina} je sourhn objektů, chápaný jako celek. Tyto objekty nazýváme prvky množiny.
\end{definition}

Množina může být konečná, nekonečná nebo prázdná. Množinu lze zadat výčtem prvků nebo pomocí charakteristické vlastnosti (např. $\left \{ 2k, k \in \mathbb{N}\right\}$).

\begin{definition}
  \textbf{Podmnožina} množiny $A$ je taková množina $B$, že všechny její prvky patří do množiny $A$.
\end{definition}

Každá neprázdná množina má dvě \textbf{nevlastní podmnožiny}: množinu prázdnou a sebe sama. Všechny ostatní její podmnožiny nazýváme vlastní.

\begin{definition}
  Množiny $A$ a $B$ se rovnají právě tehdy, když $A$ je podmnožinou $B$ a zároveň $B$ je podmnožinou $A$.
\end{definition}

\begin{definition}
  Nechť $A \subseteq B$ a $B\neq \emptyset$. Množinu všech prvků množiny $B$, které nepatří do množiny $A$, nazýváme \textbf{doplněk} (komplement) množiny $A$ v množině $B$. Značíme $A_B^\prime.$
\end{definition}

\begin{definition}
  Nechť $A, B$ jsou dvě množiny. Jejich \textbf{sjednocením} nazveme takovou množinu, která obsahuje ty prkvy, které patří alespoň do jedné z množin $A, B$. Zapisujeme $A \cup B$.
\end{definition}

\begin{definition}
  Nechť $A, B$ jsou dvě množiny. Jejich \textbf{průnikem} nazveme takovou množinu, která obsahuje ty prvky, které patří zároveň do obou těchto množin $A, B$. Zapisujeme $A \cap B$.
\end{definition}

\begin{definition}
  \textbf{Vennův diagram} je grafické schematické znázornění všech možných vztahů (sjednocení, průnik, rozdíl, doplněk) několika podmnožin univerzální množiny, jež znázorňujeme pomocí uzavřených čar.
\end{definition}

\begin{definition}
  Dvě množiny jsou \textbf{disjunktní}, pokud nemají žádný společný prvek, tedy pokud je jejich průnikem prázdná množina.
\end{definition}

\begin{definition}
  Nechť $A, B$ jsou dvě množiny. \textbf{Rozdíl} množin je množina, která obsahuje všechny prvky množiny $A$ s výjimkou těch, jež jsou zároveň prvky množiny $B$. Zapisujeme $A - B$.
\end{definition}

\begin{veta}
  \textbf{De Morganovy zákony} jsou zákony určující vztahy mezi sjednocením, průnikem a doplňkem množiny. Nechť $A, B$ jsou dvě množiny, $^\prime$ doplněk množiny. Potom platí:
  $$ (A \cup B)^\prime = A^\prime \cap B^\prime$$
  $$ (A \cap B)^\prime = A^\prime \cup B^\prime$$
\end{veta}

\begin{proof}
  Důkaz prvního vztahu:

  \begin{minipage}{0.5\textwidth}
    \centering
        \includegraphics[width=0.5\linewidth]{vennsjed.png}
        \includegraphics[width=0.5\linewidth]{venndoplsjed.png}
  \end{minipage}
  \hfill
  \noindent\begin{minipage}{0.5\textwidth}
  \centering
        \includegraphics[width=0.5\linewidth]{venndoplAaB.png}
        \includegraphics[width=0.5\linewidth]{venndoplsjed.png}
  \end{minipage}

  Důkaz druhého analogicky.
\end{proof}

\begin{pozn}[Číselné množiny]
  Rozlišujeme následující základní číselné množiny:
  \begin{itemize}
    \item $\mathbb{N}$: přirozená čísla $(1, 2, 3, \dots)$,
    \item $\mathbb{Z}$: celá čísla $(\dots, -2, -1, 0, 1, 2, \dots)$,
    \item $\mathbb{Q}$: racionální čísla $(3/5, 0,\overline{3})$,
    \item $\mathbb{R}$: reálná čísla $(e, \pi)$,
    \item $\mathbb{C}$: komplexní čísla $(3+2i)$
  \end{itemize}
\end{pozn}

Iracionální čísla ($\mathbb{I}$) jsou doplněk racionálních v $\mathbb{R}$.

\begin{definition}
  \textbf{Celá čísla} jsou čísla, která vyjadřují počty prvků množin, čísla k nim opačná a číslo 0.
\end{definition}

\begin{definition}
  \textbf{Racionálním číslem} nazveme takové číslo $a = \frac{k}{l}, k, l \in \mathbb{Z}$, a $p,q$ jsou nesoudělná.
\end{definition}

\begin{pozn}
  Přirozená čísla zapisujeme pomocí číslic 0--9 a chápeme je takto:
  $$4503=4\cdot 10^3+5\cdot 10^2 + 0 \cdot 10^1 + 3\cdot 10^0.$$
  Každé racionální číslo je v desítkové soutavě vyjádřeno buď ukončeným desetinným rozvojem nebo neukončeným periodickým rozvojem. Iracionální číslo je vyjádřeno neukončeným neperiodickým rozvojem.
\end{pozn}

\begin{definition}
  \textbf{Reálnými čísly} nazýváme všechna čísla, která jsou velikostmi úseček.
\end{definition}

\begin{definition}
  Nechť $a,b \in \mathbb{R},$ kde $a<b$. Pak množiny takových $x\in \mathbb{R},$ že $a\leq x\leq b$ (resp. $a < x < b$, resp. $a < x$, resp. $a \leq x < b$ atd.) nazýváme uzavřeným (resp. otevřeným, resp. neomezeným zleva otevřeným, resp.zprava uzavřeným, zleva otevřeným atd.) \textbf{intervalem}. Zapisujeme $\left<a,b\right>$ (resp. $\left(a,b\right)$, resp. $(a, \infty)$, resp. $\left<a, b\right)$)
\end{definition}

\begin{definition}
  \textbf{Periodický rozvoj čísla} je rozvoj, u kterého se za desetinnou čárkou donekonečna opakuje táž číslice nebo skupina číslic. Čísla s takovýmto rozvojem se nazývají ryze periodická čísla, opakující se číslice nebo skupina opakujících se číslic se nazývá perioda. Zapisují se tak, že se nad opakující se skupinou napíše pruh:
  $$0,333 … = 0,\overline{3}$$
\end{definition}

\begin{definition}
  Množina komplexních čísel $\mathbb C$ je množina uspořádaných reálných dvojic $[x, y]$, na kterých je definována rovnost, sčítání a násobení následovně:
$$[a, b] = [c, d] \Leftrightarrow a = c \land b = d,$$
$$[a, b] + [c, d] = [a + c, b + d],$$
$$[a, b][c, d] = [ac - bd, ad + bc].$$
  Dvojici $[0, 1]$ označíme $i$ a budeme ji nazývat komplexní jednotkou. Zřejmě pak platí, že $i^2 = -1$.
\end{definition}

\begin{definition}
  Pokud je uspořádaná dvojice z předchozí definice ve tvaru $[0, b], b \in \R$, nazveme toto číslo \textbf{ryze imaginárním}.
\end{definition}

\begin{example}[SÚM 169/8]
  Označme $M$ množinu všech dvojciferných přirozených čísel delitelných šesti a $N$ všechn dělitelů čísla 210, kteří jsou různí od čísla 1 a 210. Určete, která z množin má větší počet prvků, a vypište všechny prvky, které mají obě množiny stejné.
  \begin{align*}
    M & = \left\{12, 18, 24, 30, 36, 42, 48, 54, 60, 66, 72, 78, 84, 90, 96\right\}\\
    210 & = 2\cdot 3 \cdot 5  \cdot 7 \textrm{ -- hledáme násobky všech podmnožin těchto čísel} \\
    N  & = \left\{2,3,5,6,7, 10, 14, 15, 21, 30, 35, 42, 70, 105\right\} \\
    |M| & = 15, |N| = 14, M \cap N = \left\{30, 42\right\}
  \end{align*}

  \rm Množina $M$ má více prvků a společná jsou čísla 30 a 42.
\end{example}

\begin{example}[SÚM 171/26]
  $M$ je množina šech reálných čísel $x$, která splňují nerovnosti $-2<x<5$, $N$ je mn. všech reálných čísel $y$, která splňují nerovnost $|y|<4$. Určete množinu $R=M\cup N$ a $S = M\cap N.$ \hfill $R = (-4,5), S=(-2,4).$
\end{example}

\begin{example}[SÚM 172/29f]
  Znázorněte a určete výsledný interval: $(a,a+2)\cap (a-1,a+1),$ kde $a>0.$\hfill$(a,a+1)$
\end{example}

\begin{example}[SÚM (172/33)]
  Je dána kružnice $k$ se středem v bodě  $S$ a poloměrem $r$. Množinu všech bodů uvnitř kružnice označte $A$. Nakreslete rovnostranný trojúhelník $ESD$, jehož jeden vrchol je ve středu dané kružnice a délky stran jsou rovny velikosti jejího průměru. Množinu vnitřních bodů tohoto trojúhelníka ozn. $B$. Díle sestrojte osu úhlu $ESD$ a množinu bodů této přímky označte $C$. Nakreslete samostatné obrázky pro:
  \begin{itemize}
    \item $(A\cap B)\cup C,$
    \item $(A\cup C) \cap (B\cup C),$
    \item $(A\cap B) \cup (B\cap C),$
    \item $(A\cup C) \cap B$.
  \end{itemize}
\end{example}

\begin{example}[SÚM 173/34]
  Pro která $x$ je interval:
  \begin{enumerate}[a.]
    \item $\left<2x,x+3\right>$ částí intervalu $(2,7)$? \hfill $x \in (1,3)$
    \item $(x,5)$ částí intervalu $\left(-1,x+1\right)$? \hfill $x\in (4,5)$
    \item $(x,x+3)$ částí intervalu $\left<5,8\right>$? \hfill $x=5$
    \item $\left<x,2x-1\right>$ částí intervalu $\left<-2,5\right>$? \hfill $x\in\left<-2,5\right>$
    \item $\left<3x,2x+1\right>$ částí intervalu $(3,6)$? \hfill $x\in \left\{\right\}$
  \end{enumerate}
\end{example}

\begin{example}[SÚM 173/35]
  Nechť $M = (a,b), N = (1,8), Q = (1,5)$. Určete $a,b \in \mathbb{R}$ tak, aby platilo $M\cap N = Q$.\hfill $a\in \left(-\infty, 1\right>, b=5$
\end{example}

\begin{example}[SÚM 173/37*]
  Je dán trojúhelník $ABC$. Uvažujme množinu $M$ všech bodů tohoto trojúhelníka, pro které platí $|AX| \geq |BX| \geq |CX|.$ Pomocí velikosti stran a úhlů troj. $ABC$ vyjádřete podmínky pro to, aby:
  \begin{enumerate}[a.]
    \item $X$ byla pětiúhelník, \hfill $\gamma > 90^\circ, \alpha < \beta$
    \item $X$  je jeden bod, \hfill $\alpha = 90^\circ$
    \item $X$ je prázdná.\hfill $\alpha > 90^\circ$
  \end{enumerate}
\end{example}


\begin{example}[SÚM 174/42]
  Jsou dány množiny $M=\left\{1,2; 3; 4\right\},N=\left\{x;y;z\right\}.$ Uveďte alespoň jeden příklad na zobrazení množiny
  \begin{enumerate}[a.]
    \item $M$ do $N$\hfill $1,2\rightarrow x; 3\rightarrow y; 4 \rightarrow y$
    \item $N$ do $M$ \hfill $x\rightarrow 1,2; y\rightarrow 3; z\rightarrow 3$
    \item $M$ na $N.$ \hfill $1,2\rightarrow x; 3 \rightarrow y; 4 \rightarrow z$
  \end{enumerate}
\end{example}

\begin{example}[SÚM 174/46]
  Kolik je všech zobrazení (pod)množiny $\left\{a,b,c,d\right\}$ do (na) množiny $\left\{1,2\right\}$?\hfill \rm 81
\end{example}

\begin{example}[SÚM 106/20]
  Převeďte na obyčejné zlomky:
  \begin{enumerate}[a.]
    \item $0,\overline{27}$\hfill $\frac{27}{99}\frac{3}{11}$
    \item $0,\overline{6}$ \hfill $\frac{2}{3}$
    \item $2,\overline{345}$ \hfill $2+\frac{345}{999}=\frac{781}{333}$
    \item $0,\overline{1234}$\hfill $\frac{1234}{9999}$
    \item $0,7\overline{2}$\hfill $\frac{7}{10}+\frac{2}{90}=\frac{13}{18}$
    \item $0,1\overline{36}$\hfill $\frac{1}{10}+\frac{36}{990}=\frac{3}{22}$
    \item $0,7\overline{27}$\hfill $\frac{7}{10}+\frac{27}{990}=\frac{8}{11}$
    \item $3,39\overline{85}$\hfill $3+\frac{39}{100}+\frac{85}{9900}=\frac{33646}{9900}$
  \end{enumerate}
\end{example}

\begin{example}[SÚM 107/21]
  Proveďte:
  \begin{enumerate}[a.]
    \item $0,\overline{4}+0,\overline{12}$ \hfill $\frac{4}{9}+\frac{12}{9}=\frac{16}{9}$
    \item $0,\overline{7}+0,\overline{35}$  \hfill $\frac{112}{99}$
    \item $0,\overline{47}+0,\overline{023}$ \hfill $\frac{5470}{10989}$
    \item $0,\overline{47}+0,0\overline{23}$ \hfill $\frac{493}{990}$
    \item $0,5\overline{354}+0,\overline{85}$\hfill $1,394021\dots$
    \item $2,\overline{35}-1,\overline{231}$\hfill$ \frac{4111}{3663}$
    \item $1,\overline{25}-0,\overline{773}$ \hfill $\frac{5261}{10989}$
  \end{enumerate}
\end{example}

\begin{example}[SÚM 107/22*]
  Proveďte:
  \begin{enumerate}[a.]
    \item $1,\overline{2}\cdot 1,\overline{18}$\hfill $\left(1+\frac{2}{9}\right)\left(1+\frac{18}{99}\right)=\frac{11}{9}\cdot \frac{117}{99}=\frac{13}{9}$
    \item $0,\overline{32}\cdot 1,\overline{3}$\hfill $\frac{128}{297}$
  \end{enumerate}
\end{example}

\begin{example}[SÚM 107/23*]
  Řešte rovnici:
  \begin{enumerate}
    \item $0,\overline{25}x + 0,\overline{31}x = 1,\overline{13}$ \hfill $x=2$
    \item $2,\overline{64}x - 3,\overline{48} = 1,\overline{48}x$  \hfill $x = 3$
  \end{enumerate}
\end{example}

\begin{example}[SMP 140/6abc]
  Pomocí Vennových diagramů zjednodušte zápisy množin: \\
  \begin{minipage}{0.5\textwidth}
    \begin{enumerate}[a.]
      \item $(A \cap B \cap C) \cup [B \cap (A^\prime \cup C)^\prime]$
      \item $[(A \cup B)^\prime \cup (B \cup C)] \cap (C \cup A)$
      \item $[(A \cup B^\prime) \cap C] \cup [(B^\prime \cup A^\prime)^\prime \cap C]$
    \end{enumerate}
  \end{minipage}
  \hfill
  \noindent\begin{minipage}{0.5\textwidth}
      \includegraphics[width=\linewidth]{vennovy}
      \captionof{figure}{}
  \end{minipage}
\end{example}

\begin{example}[SÚM 109/36]
  Dokažte, že číslo $\sqrt{5}$ je iracionální.

  Dk. sporem: Nechť $\sqrt{5} \in \Q \Rightarrow \sqrt{5} = \frac{a}{b}, a, b \in \Z, D(a, b) = 1$.
  $$\sqrt{5} = \frac{a}{b}$$
  $$5 = \frac{a^2}{b^2}$$
  $$5b^2 = a^2 \Rightarrow 5 \mid a^2 \Rightarrow 5 \mid a \Rightarrow \exists k: a = 5k$$
  $$5b^2 = (5k)^2$$
  $$5b^2 = 25k^2$$
  $$b^2 = 5k^2 \Rightarrow 5 \mid b^2 \Rightarrow 5 \mid b$$ -- spor s předpokladem, že $D(a, b) = 1$
  $$\sqrt{5} \in \mathbb{I}$$
\end{example}

\begin{example}[SÚM 109/37]
  Dokažte, že číslo $\sqrt{2} - 1$ je iracionální.

  Dk. sporem: Nechť $(\sqrt{2} - 1) \in \Q \Rightarrow \sqrt{2} - 1 = \frac{a}{b}, a, b \in \Z, D(a, b) = 1$.
  $$\sqrt{2} - 1 = \frac{a}{b}$$
  $$1 - 2\sqrt{2} = \frac{a^2}{b^2}$$
  $$\frac{b^2-a^2}{2b^2} = \sqrt{2} \Rightarrow \sqrt{2} = \frac{p}{q}, p,q \in \Z \Rightarrow \sqrt(2) \in \Q$$ -- spor
  $$(\sqrt{2} - 1) \in \mathbb{I}$$
\end{example}

\begin{example}[SÚM 109/38]
  Dokažte, že číslo $2\sqrt{5}$ je iracionální.

  Dk. sporem: Nechť $2\sqrt{5} \in \Q \Rightarrow 2\sqrt{5} = \frac{a}{b}, a, b \in \Z, D(a, b) = 1$.
  $$2\sqrt{5} = \frac{a}{b}$$
  $$10 = \frac{a^2}{b^2}$$
  $$10b^2 = a^2 \Rightarrow 10 \mid a^2 \Rightarrow 10 \mid a \Rightarrow \exists k: a = 10k$$
  $$10b^2 = (10k)^2$$
  $$10b^2 = 100k^2$$
  $$b^2 = 10k^2 \Rightarrow 10 \mid b^2 \Rightarrow 10 \mid b$$ -- spor s předpokladem, že $D(a, b) = 1$
  $$2\sqrt{5} \in \mathbb{I}$$
\end{example}

\begin{example}[SÚM 109/39]
  Dokažte, že jestliže přirozené číslo $m$ není druhou mocninou žádného přirozeného čísla, potom $\sqrt{m}$ je číslo iracionální.

  Dk. sporem: Nechť $m \in \N, m != n^2 \forall n \in \N, \sqrt{m} \in \Q \Rightarrow \sqrt{m} = \frac{a}{b}, a, b \in \N, D(a, b) = 1$.
  $$\sqrt{m} = \frac{a}{b}$$
  $$m = \frac{a^2}{b^2}, m \in \N \Rightarrow b^2 = 1$$
  $$m = a^2, a \in \N$$ -- spor s předpokladem, že $m != n^2 \forall n \in \N$
  QED
\end{example}

\begin{example}[SÚM 144/301]
  Dokažte, že:
  \begin{enumerate}[a.]
    \item součet dvou dvojciferných čísel přirozených, která se liší jen pořadím cifer, je dělitelný jedenácti: $$S = \overline{ab} + \overline{ba} = 10a + b + 10b + a = 11(a + b) \Rightarrow 11 \mid S$$
    \item rozdíl dvou dvojciferných čísel přirozených, která se liší jen pořadím cifer, je dělitelný devítí: $$S = \overline{ab} - \overline{ba} = 10a + b - 10b - a = 9(a - b) \Rightarrow 9 \mid S$$
    \item rozdíl přirozeného čísla trojciferného  a čísla, které vznikne z tohoto záměnou krajních cifer, je dělitelný 99: $$S = \overline{abc} - \overline{cba} = 100a + 10b + c - 100c - 10b - a = 99(a - c) \Rightarrow 99 \mid S$$
  \end{enumerate}
\end{example}

\begin{example}[SÚM 145/303]
  Dokažte, že tři mocniny čísla 2, jejichž exponenty jsou tři po sobě jdoucí přirozená čísla, mají součet dělitelný sedmi: $$S = 2^a + 2^{a+1} + 2^{a+2} = 2^a + 2^a*2^1 + 2^a*2^2 = 7*2^a \Rightarrow 7 \mid S$$
\end{example}

\begin{example}[SÚM 145/305]
  Dokažte, že součet třetích mocnin tří po sobě jdoucích přirozených čísel je dělitelný třemi: $$S = a^3 + (a+1)^3 + (a+2)^3 = a^3 + a^3 + 3a^2 + 3a + 1 + a^3 + 6a^2 + 12a + 8 = 3(a^3 + 3a^2 + 5a + 3) \Rightarrow 3 \mid S $$
\end{example}

\begin{example}[SÚM 145/306]
  Dokažte, že:
  \begin{enumerate}[a]
    \item číslo utvořené z rozdílu třetí mociny přirozeného čísla $n$ a tohoto čísla je dělitelné šesti: $$S = n^3 - n = n(n^2-1) = (n-1)n(n+1)$$ Jsou to tři po sobě jdoucí čísla $\Rightarrow$ právě 1 z nich je dělitelné třemi $\Rightarrow 3 \mid S$
    \item je-li číslo $n$ liché, je uvažovaný rozdíl dělitelný čísel 24: $$S = (n-1)n(n+1)$$ Jsou to tři po sobě jdoucí čísla a to prostřední je liché $\Rightarrow$ dělitelné třemi, z dalších čísel je jedno dělitelné 2 a jedno dělitelné čtyřmi: 2*4*3 = 24 $\Rightarrow 24 \mid S$
  \end{enumerate}
\end{example}

\begin{example}[SÚM 145/307]
  Dokažte, že je—li přirozené číslo $x$ liché, je výraz $V = x^3 + 3x^2 — x — 3$ dělitelný číslem 48: $$V = x^3 + 3x^2 — x — 3 = x^2(x+3) -(x+3) = (x^2 - 1)(x+3) = (x-1)(x+1)(x+3)$$ $\Rightarrow$ tři po sobě jdoucí sudá čísla $\Rightarrow$ jedno dělitelné 2, jedno 4 a jedno 6 $\Rightarrow 2*4*6 = 48 \Rightarrow 48 \mid V$
\end{example}

\begin{example}[SÚM 145/308]
  Dokažte, že výraz $V = 5x^3 + 15x^2 + 10x$ je dělitelný číslem 30 prokaždé přirozené číslo $x$: $$V = 5x^3 + 15x^2 + 10x = 5x(x^2 + 3x + 2) = 5x(x+2)(x+1)$$ $\Rightarrow$ $x$, $x+1$, $x+2$ tři po sobě jdoucí čísla $\Rightarrow$ jedno dělitelné 3, alespoň jedno dělitelné 2, $5x$ dělitelné 5 $\Rightarrow 2*3*5=30 \Rightarrow 30 \mid V$
\end{example}

\begin{example}[SÚM 145/312]
  Dokažte, že je-li $n$ číslo přirozené, je číslo $N = n^3 + 11n$ dělitelné šesti:
  mod 6:
  \begin{enumerate}
    \item $n = 6k$: $n^3 + 11n \equiv 0^3 + 11*0 \equiv 0 \Rightarrow 6 \mid N$
    \item $n = 6k + 1$: $n^3 + 11n \equiv 1^3 + 11*1 \equiv 12 \equiv 0 \Rightarrow 6 \mid N$
    \item $n = 6k + 2$: $n^3 + 11n \equiv 2^3 + 11*2 \equiv 30 \equiv 0 \Rightarrow 6 \mid N$
    \item $n = 6k + 3$: $n^3 + 11n \equiv 3^3 + 11*3 \equiv 60 \equiv 0 \Rightarrow 6 \mid N$
    \item $n = 6k + 4$: $n^3 + 11n \equiv 4^3 + 11*4 \equiv 108 \equiv 0 \Rightarrow 6 \mid N$
    \item $n = 6k + 5$: $n^3 + 11n \equiv 5^3 + 11*5 \equiv 180 \equiv 0 \Rightarrow 6 \mid N$
  \end{enumerate}
  $\Rightarrow 6 \mid N \forall n \in \N$
\end{example}

\begin{example}[SÚM 145/315]

\end{example}
