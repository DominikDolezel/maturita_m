\section{Nezařazené vykřičníkové příklady}
\subsection{Algebraické výrazy}
\begin{priklad}
    Rozložte $2x^2 - 3x+1.$
\end{priklad}

\begin{reseni}
Doplněním na čtverec.
\end{reseni}

\begin{priklad}
    Zakreslete na číselné ose
    \begin{enumerate}[$a.$]
    \item racionální číslo $-4/3.$
   	\item iracionální číslo $\sqrt{3}. $
    \end{enumerate}
\end{priklad}

\begin{reseni}
\begin{enumerate}[$a.$]
\item Z podobnosti trojúhelníků.
\item Z Pythagorovy věty.
\end{enumerate}
\end{reseni}

\begin{priklad}
    Upravte výraz $d(x)=(x+16)(x+17)(x+18)-(x+17)^2(x+19).$
\end{priklad}

\begin{reseni}
 Výhodnou volbou $x+17=t$ dostaneme $d(x)=-t(2t+1)=(-x-17)(2x+35).$
\end{reseni}

\begin{priklad}
    Ve výrazu $V(n+1)$ vyčleňte daný výraz $V(n)=n^3+2n.$
\end{priklad}

\begin{reseni}
 TODO
\end{reseni}

\subsection{Rovnice a nerovnice}
\begin{priklad}
    V $\mathbb R$ řešte $x^2-5x \geq 0.$
\end{priklad}

\subsection{Mocniny a odmocniny}
\begin{priklad}
Odstraňte odmocniny ze jmenovatele: $\frac{1}{2 \sqrt{x} }.$
\end{priklad}

\begin{reseni}
Platí
$$\frac{1}{2\sqrt{x} }=\frac{1}{2\sqrt{x} }\cdot \frac{\sqrt{x} }{\sqrt{x} }=\frac{\sqrt{x} }{2x},$$
je-li $x>0.$
\end{reseni}

\subsection{Důkazy}
\begin{priklad}
Dokažte, že je-li $n$ sudé, pak i $n^2$ je sudé.
\end{priklad}

\begin{reseni}\label{dkpr}
Přímým důkazem. Nechť $n$ je sudé přirozené číslo, tedy lze jej zapsat ve tvaru $2k, k \in \mathbb N.$
Pak $n^2 = 4k^2 = 2\cdot 2l, 2l\in \mathbb N, $ tedy $n^2$ je sudé.
\end{reseni}

\begin{priklad}
Dokažte, že je-li $n^2$ sudé, je i $n$ sudé.
\end{priklad}

\begin{reseni}
Nepřímým důkazem. Obměna: Je-li $n$ liché, je i $n^2$ liché. Obdobně jako v příkladu \ref{dkpr}.
\end{reseni}

\begin{priklad}
Dokažte, že pro všechna $n\in N$ platí: $n$ je sudé právě tehdy, když $n^3$ je sudé.
\end{priklad}

\begin{reseni}
Musíme dokázat oba směry implikace zvlášť.
\end{reseni}

\begin{priklad}
Dokažte, že každým bodem roviny lze vést k dané přímce nejvýše jednu kolmici.
\end{priklad}

\begin{reseni}
Sporem. Předpokládáme platnost negace a dojdeme ke sporu. Musí tedy platit původní výrok.
\end{reseni}

\begin{priklad}
Dokažte, že $\sqrt{2} $ je iracionální.
\end{priklad}

\begin{reseni}
Sporem.
\end{reseni}

\begin{priklad}
Dokažte pro všechna $n\in \mathbb N: $
$$1+2+3+\dots+n=\frac{n(n+1)}{2}.$$
\end{priklad}

\begin{reseni}
Matematickou indukcí.
\begin{enumerate}[1.]
\item $n=1: 1=1$
\item Předpokládáme platnost pro $n$ a dokážeme, že pak platí i pro $n+1$.
\end{enumerate}
\end{reseni}
