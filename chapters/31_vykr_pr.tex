\section{Nezařazené vykřičníkové příklady}
\subsection{Algebraické výrazy}
\begin{priklad}
    Rozložte $2x^2 - 3x+1.$
\end{priklad}

\begin{reseni}
Doplněním na čtverec.
\end{reseni}

\begin{priklad}
    Zakreslete na číselné ose
    \begin{enumerate}[$a.$]
    \item racionální číslo $-4/3.$
   	\item iracionální číslo $\sqrt{3}. $
    \end{enumerate}
\end{priklad}

\begin{reseni}
\begin{enumerate}[$a.$]
\item Z podobnosti trojúhelníků.
\item Z Pythagorovy věty.
\end{enumerate}
\end{reseni}

\begin{priklad}
    Upravte výraz $d(x)=(x+16)(x+17)(x+18)-(x+17)^2(x+19).$
\end{priklad}

\begin{reseni}
 Výhodnou volbou $x+17=t$ dostaneme $d(x)=-t(2t+1)=(-x-17)(2x+35).$
\end{reseni}

\begin{priklad}
    Ve výrazu $V(n+1)$ vyčleňte daný výraz $V(n)=n^3+2n.$
\end{priklad}

\begin{reseni}
 TODO
\end{reseni}

\begin{priklad}
V $\mathbb R$ zjednodušte
$$\frac{x^3+x^2-x-1}{\sqrt{x^2}+1 }.$$
\end{priklad}

\begin{reseni}
Platí $\sqrt{x^2}=|x| $.
\end{reseni}

\begin{priklad}
Rozložte $x^2-5x+6$.
\end{priklad}

\begin{reseni}
Doplněním na čtverec.
\end{reseni}

\begin{priklad}
Najděte nejmenší hodnotu výrazu $x^2+16x-17.$
\end{priklad}

\begin{reseni}
Doplněním na čtverec. Minimum nastane tehdy, když je čtverec nulový.
\end{reseni}


\subsection{Rovnice a nerovnice, matice}
\begin{priklad}
    V $\mathbb R$ řešte $x^2-5x \geq 0.$
\end{priklad}

\begin{priklad}
V $\mathbb R$ řešte
$$\frac{(4-x)(6+x)x}{2-x}\leq 0.$$
\end{priklad}

\begin{priklad}
V $\mathbb R$ řešte $$\frac{(x+1)(x-2)^2}{(3-x)^3(4+x)^4}\leq 0.$$
\end{priklad}

\begin{priklad}
V $\mathbb R$ řešte rovnici $|3x-5|=2x+10.$
\end{priklad}

\begin{priklad}
Rozdělíme na případy, kdy je výraz v absolutní hodnotě menší / větší než 0
a dále řešíme jako normálně.
\end{priklad}

\begin{priklad}
V $\mathbb R$ řešte $2|4+3x|\leq 6x+11.$
\end{priklad}

\begin{reseni}
Rozdělíme na případy, kdy je výraz v absolutní hodnotě menší / větší než 0
a dále řešíme jako normálně.
\end{reseni}

\begin{priklad}
V $\mathbb R$ řešte $|x+2|+|x-2|=2x+2.$
\end{priklad}

\begin{reseni}
Rozdělíme na případy, kdy je výraz v absolutní hodnotě menší / větší než 0
a dále řešíme jako normálně.
\end{reseni}

\begin{priklad}
V $\mathbb R$ řešte $|3x-2|<5+|x+1|.$
\end{priklad}

\begin{priklad}
V $\mathbb R$ řešte $x^2-6x+8 >0.$
\end{priklad}

\begin{priklad}
Danou matici převeďte na schodovitý tvar a určete její hodnost.
$$A=\begin{pmatrix}
    1 & 2 & -3 \\
    -3 & 1 & -2 \\
    2 & 3 & 2
\end{pmatrix}$$
\end{priklad}

\begin{reseni}
K jednotlivým řádkům přičítáme násobky jiných, aby nám vyšel schodovitý tvar.
\end{reseni}

\begin{priklad}
V $\mathbb R$ řešte:
\begin{align*}
    2x_1+5x_2-8x_2 & =8 \\
    4x_1 + 3x_2 - 9x_3 & =9\\
    2x_1 + 3x_2 - 5x_3 & = 7 \\
    x_1 + 8x_2 - 7x_3 & = 12
\end{align*}
\end{priklad}

\begin{reseni}
Převedeme na matici
$$
\left (
\begin{array}{c c c | c }
1 & 8& -7& 12\\
2& 5& -8& 8\\
4& 3& -9& 9\\
2& 3& -5& 7
\end{array}
\right )
$$
a řešíme Gaussovou eliminační metodou.
\end{reseni}

\begin{priklad}
V $\mathbb R$ řešte
\begin{align*}
    x_1+x_2+x_3&=1\\
    x_1-x_3 &=0
\end{align*}
\end{priklad}

\begin{reseni}
Převedeme na matici
$$
\left (
\begin{array}{c c c | c }
1 & 1 & & 1\\
1  & 0 &-1  &0
\end{array}
\right )
$$
a řešíme Gaussovou eliminační metodou.
\end{reseni}

\begin{priklad}
Vypočtete determinant matice
$$A=\begin{pmatrix}
    1  &1 &1 &1\\
    1 &2 &3 &4 \\
    1 &3 &6 &10\\
    1 &4 &10 &20
\end{pmatrix}.$$
\end{priklad}

\begin{reseni}
Od druhého, třetího a řtvrtého řádku odečteme ten první (čímž se determinant nezmění)
a využijeme Laplaceův rozvoj podle prvního sloupce. Je tedy
$$\det A = \begin{vmatrix}
    1 &1 &1 &1\\
    0 &1 &2 &3 \\
    0 &2 &5 &9 \\
    0 &3 &9 &19
\end{vmatrix}=1 \cdot \begin{vmatrix}
1 &2 &3 \\
2 &5 &9\\
3 &9 &19
\end{vmatrix}=1$$
\end{reseni}

\subsection{Mocniny a odmocniny}
\begin{priklad}
Odstraňte odmocniny ze jmenovatele: $\frac{1}{2 \sqrt{x} }.$
\end{priklad}

\begin{reseni}
Platí
$$\frac{1}{2\sqrt{x} }=\frac{1}{2\sqrt{x} }\cdot \frac{\sqrt{x} }{\sqrt{x} }=\frac{\sqrt{x} }{2x},$$
je-li $x>0.$
\end{reseni}

\subsection{Důkazy}
\begin{priklad}
Dokažte, že je-li $n$ sudé, pak i $n^2$ je sudé.
\end{priklad}

\begin{reseni}\label{dkpr}
Přímým důkazem. Nechť $n$ je sudé přirozené číslo, tedy lze jej zapsat ve tvaru $2k, k \in \mathbb N.$
Pak $n^2 = 4k^2 = 2\cdot 2l, 2l\in \mathbb N, $ tedy $n^2$ je sudé.
\end{reseni}

\begin{priklad}
Dokažte, že je-li $n^2$ sudé, je i $n$ sudé.
\end{priklad}

\begin{reseni}
Nepřímým důkazem. Obměna: Je-li $n$ liché, je i $n^2$ liché. Obdobně jako v příkladu \ref{dkpr}.
\end{reseni}

\begin{priklad}
Dokažte, že pro všechna $n\in N$ platí: $n$ je sudé právě tehdy, když $n^3$ je sudé.
\end{priklad}

\begin{reseni}
Musíme dokázat oba směry implikace zvlášť.
\end{reseni}

\begin{priklad}
Dokažte, že každým bodem roviny lze vést k dané přímce nejvýše jednu kolmici.
\end{priklad}

\begin{reseni}
Sporem. Předpokládáme platnost negace a dojdeme ke sporu. Musí tedy platit původní výrok.
\end{reseni}

\begin{priklad}
Dokažte, že $\sqrt{2} $ je iracionální.
\end{priklad}

\begin{reseni}
Sporem.
\end{reseni}

\begin{priklad}
Dokažte pro všechna $n\in \mathbb N: $
$$1+2+3+\dots+n=\frac{n(n+1)}{2}.$$
\end{priklad}

\begin{reseni}
Matematickou indukcí.
\begin{enumerate}[1.]
\item $n=1: 1=1$
\item Předpokládáme platnost pro $n$ a dokážeme, že pak platí i pro $n+1$.
\end{enumerate}
\end{reseni}
