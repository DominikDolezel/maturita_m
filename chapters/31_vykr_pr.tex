\section{Nezařazené vykřičníkové příklady}
\subsection{Algebraické výrazy}
\begin{priklad}
    Rozložte $2x^2 - 3x+1.$
\end{priklad}

\begin{reseni}
Doplněním na čtverec.
\end{reseni}

\begin{priklad}
Odstraňte odmocniny ze jmenovatele: $\frac{1}{2 \sqrt{x} }.$
\end{priklad}

\begin{reseni}
Platí
$$\frac{1}{2\sqrt{x} }=\frac{1}{2\sqrt{x} }\cdot \frac{\sqrt{x} }{\sqrt{x} }=\frac{\sqrt{x} }{2x},$$
je-li $x>0.$
\end{reseni}

\begin{priklad}
    Zakreslete na číselné ose
    \begin{enumerate}[$a.$]
    \item racionální číslo $-4/3.$
   	\item iracionální číslo $\sqrt{3}. $
    \end{enumerate}
\end{priklad}

\begin{reseni}
\begin{enumerate}[$a.$]
\item Z podobnosti trojúhelníků.
\item Z Pythagorovy věty.
\end{enumerate}
\end{reseni}

\begin{priklad}
    Upravte výraz $d(x)=(x+16)(x+17)(x+18)-(x+17)^2(x+19).$
\end{priklad}

\begin{reseni}
 Výhodnou volbou $x+17=t$ dostaneme $d(x)=-t(2t+1)=(-x-17)(2x+35).$
\end{reseni}

\begin{priklad}
    Ve výrazu $V(n+1)$ vyčleňte daný výraz $V(n)=n^3+2n.$
\end{priklad}

\begin{reseni}
 TODO
\end{reseni}

\begin{priklad}
V $\mathbb R$ zjednodušte
$$\frac{x^3+x^2-x-1}{\sqrt{x^2}+1 }.$$
\end{priklad}

\begin{reseni}
Platí $\sqrt{x^2}=|x| $.
\end{reseni}

\begin{priklad}
Rozložte $x^2-5x+6$.
\end{priklad}

\begin{reseni}
Doplněním na čtverec.
\end{reseni}

\begin{priklad}
Najděte nejmenší hodnotu výrazu $x^2+16x-17.$
\end{priklad}

\begin{reseni}
Doplněním na čtverec. Minimum nastane tehdy, když je čtverec nulový.
\end{reseni}


\subsection{Rovnice a nerovnice, matice}
\begin{priklad}
    V $\mathbb R$ řešte $x^2-5x \geq 0.$
\end{priklad}

\begin{priklad}
V $\mathbb R$ řešte
$$\frac{(4-x)(6+x)x}{2-x}\leq 0.$$
\end{priklad}

\begin{priklad}
V $\mathbb R$ řešte $$\frac{(x+1)(x-2)^2}{(3-x)^3(4+x)^4}\leq 0.$$
\end{priklad}

\begin{priklad}
V $\mathbb R$ řešte rovnici $|3x-5|=2x+10.$
\end{priklad}

\begin{priklad}
Rozdělíme na případy, kdy je výraz v absolutní hodnotě menší / větší než 0
a dále řešíme jako normálně.
\end{priklad}

\begin{priklad}
V $\mathbb R$ řešte $2|4+3x|\leq 6x+11.$
\end{priklad}

\begin{reseni}
Rozdělíme na případy, kdy je výraz v absolutní hodnotě menší / větší než 0
a dále řešíme jako normálně.
\end{reseni}

\begin{priklad}
V $\mathbb R$ řešte $|x+2|+|x-2|=2x+2.$
\end{priklad}

\begin{reseni}
Rozdělíme na případy, kdy je výraz v absolutní hodnotě menší / větší než 0
a dále řešíme jako normálně.
\end{reseni}

\begin{priklad}
V $\mathbb R$ řešte $|3x-2|<5+|x+1|.$
\end{priklad}

\begin{priklad}
V $\mathbb R$ řešte $x^2-6x+8 >0.$
\end{priklad}

\begin{priklad}
Danou matici převeďte na schodovitý tvar a určete její hodnost.
$$A=\begin{pmatrix}
    1 & 2 & -3 \\
    -3 & 1 & -2 \\
    2 & 3 & 2
\end{pmatrix}$$
\end{priklad}

\begin{reseni}
K jednotlivým řádkům přičítáme násobky jiných, aby nám vyšel schodovitý tvar.
\end{reseni}

\begin{priklad}
V $\mathbb R$ řešte:
\begin{align*}
    2x_1+5x_2-8x_2 & =8 \\
    4x_1 + 3x_2 - 9x_3 & =9\\
    2x_1 + 3x_2 - 5x_3 & = 7 \\
    x_1 + 8x_2 - 7x_3 & = 12
\end{align*}
\end{priklad}

\begin{reseni}
Převedeme na matici
$$
\left (
\begin{array}{c c c | c }
1 & 8& -7& 12\\
2& 5& -8& 8\\
4& 3& -9& 9\\
2& 3& -5& 7
\end{array}
\right )
$$
a řešíme Gaussovou eliminační metodou.
\end{reseni}

\begin{priklad}
V $\mathbb R$ řešte
\begin{align*}
    x_1+x_2+x_3&=1\\
    x_1-x_3 &=0
\end{align*}
\end{priklad}

\begin{reseni}
Převedeme na matici
$$
\left (
\begin{array}{c c c | c }
1 & 1 & & 1\\
1  & 0 &-1  &0
\end{array}
\right )
$$
a řešíme Gaussovou eliminační metodou.
\end{reseni}

\begin{priklad}
Vypočtete determinant matice
$$A=\begin{pmatrix}
    1  &1 &1 &1\\
    1 &2 &3 &4 \\
    1 &3 &6 &10\\
    1 &4 &10 &20
\end{pmatrix}.$$
\end{priklad}

\begin{reseni}
Od druhého, třetího a řtvrtého řádku odečteme ten první (čímž se determinant nezmění)
a využijeme Laplaceův rozvoj podle prvního sloupce. Je tedy
$$\det A = \begin{vmatrix}
    1 &1 &1 &1\\
    0 &1 &2 &3 \\
    0 &2 &5 &9 \\
    0 &3 &9 &19
\end{vmatrix}=1 \cdot \begin{vmatrix}
1 &2 &3 \\
2 &5 &9\\
3 &9 &19
\end{vmatrix}=1$$
\end{reseni}

\begin{priklad}
V $\mathbb R$ řešte  soustavu rovnic
\begin{align*}
    x_1+x_2+x_3 &=1,
    x_1+x_2 &= 0,
    x_1+x_3 &=.
\end{align*}
\end{priklad}

\begin{reseni}
Řešme Cramerovým pravidlem. Soustavu přepišme jako matici
$$
\begin{pmatrix}
    1 & 1&1\\
    1 &1 &0 \\
    1 &0 &1
\end{pmatrix}.
$$
Platí $x_i=\frac{\det A_i}{\det A},$ kde $A_i$ značí matici, kterou jsme získali
z matice $A$ nahrazením $i$-tého sloupce za vektor \uv{pravé strany} soustavy rovnic.
\end{reseni}

\begin{priklad}
V $\mathbb R$ řešte soustavu rovnic s parametrem $a \in \mathbb R:$
\begin{align*}
    x-y &=2,\\
    ax+y &=4.
\end{align*}
\end{priklad}

\begin{reseni}
Buď Gaussovou eliminací (od druhého řádku odečteme $a$-násobek prvního řádku) nebo
Cramerovým pravidlem (to však funguje jen tehdy, když $\det A \ne 0$, případ $a=0$ tedy
musíme dořešit Gaussovou eliminací).
\end{reseni}

\begin{priklad}
V $\mathbb R$ řešte soustavu rovnic s parametrem $a\in \mathbb R:$
\begin{align*}
    ax_1-+ax_2 &=0,\\
    -a^2x_2+x_3 &=a,\\
    ax_1 + x_3 &= a^2.
\end{align*}
\end{priklad}

\begin{reseni}
Buď Gaussovou eliminační metodou nebo Cramerovým pravidlem. Nesmíme zapomenout
rozdělit na případy, kdy by některé výrazy nemusely být definovány.
\end{reseni}

\begin{priklad}
V $\mathbb Z$ řešte $5x-13y=2.$
\end{priklad}

\begin{reseni}
\begin{enumerate}[1.]
\item způsob: uhodnutí jednoho kořene: $y_0=3, y_0=1.$ Pak další řešení je tvaru
$x=x_0-\frac{b}{d}r, y=y_0+\frac{a}{d}r,$ kde $r \in \mathbb Z$ a $d$ značí
největšího společného dělitele koeficientů u $x$ a $y$. Je tedy
\begin{align*}
    x&=3-\frac{-13}{1}r=3+13r, \\
    y &= 1+\frac{5}{1}r = 1+5r,
\end{align*}
kde $r\in \mathbb Z.$
\item způsob: modifikace Euklidova algoritmu: z rovnice osamostatníme tu neznámou,
jejíž koeficient je v absolutní hodnotě menší.
\begin{align*}
    x &= \frac{13y+2}{5}=\frac{10y}{5}+\frac{3y+2}{5}=2y+\underbrace{\frac{3y+2}{5}}_{\in \mathbb Z}\\
    \exists u \in \mathbb Z: u &= \frac{3y+2}{5}\implies 3y+2=5 \iff y=\frac{5u-2}{3}=u+\underbrace{\frac{2u-2}{2}}_{\in \mathbb Z} \\
     & \dots \\
     \exists w \in \mathbb Z: w &= \frac{v}{2}\implies v = 2w \implies \, | \, v.
\end{align*}
Nyní zpětně dosazujeme:
\begin{align*}
    u&= \frac{3(2w)+2}{2}=\frac{6w+2}{2}=3w+1\\
    y&=\frac{5u-2}{3}=\frac{5 \frac{3w+1}{2}-2}{3}=5w+1 \\
    x &= \frac{13 \frac{5w+1}{1}+2}{5}=13w+3
\end{align*}
\end{enumerate}
\end{reseni}

\begin{priklad}
V $\mathbb R$ řešte rovnici $2x^4+3x^3-16x^2+3x+2=0.$
\end{priklad}

\begin{reseni}
Reciproká rovnice. Vydělíme $x^2$, neboť 0 není kořen a zavedeme substituci
$x+\frac{1}{x}=t$ (ostatní mocniny dopočteme). V rovnici vytkneme opakující se členy
tak, abychom mohli  subsituci použít. Nezapomeneme zpětně dosadit.
\end{reseni}

\begin{priklad}
Přibližně určete reálný kořen polynomu $x^3+x-3$.
\end{priklad}

\begin{reseni}
Postupně dosazujeme nějaké hodnoty, dokud nezjistíme, že mezi nějakými dvěma
musí graf funkce procházet nulou (jedna je záporná a jedna je kladná). Takto postupujeme
dál.
\end{reseni}

\begin{priklad}
V $\mathbb R$ řešte $1/(5^{2x-4})=125.$
\end{priklad}

\begin{reseni}
Převedeme na stejný základ: $5^{4-2x}=5^3$ a přesuneme se do rovnosti exponentů:
$4-2x=3$.
\end{reseni}

\begin{priklad}
V $\mathbb R$ řešte rovnici $4^x+2^x-6=0$.
\end{priklad}

\begin{reseni}
Použijeme substituci $2^x=t.$
\end{reseni}

\begin{priklad}
V $\mathbb R$ řešte $\log (4x+6)=1+\log(2x-1).$
\end{priklad}

\begin{reseni}
Určíme podmínky pro argument (argument je větší než 0). Potom úpravou chceme dostat
rovnost dvou logaritmů, abychom se mohli přesunou do rovnosti argumentů.
\end{reseni}

\begin{priklad}
V $\mathbb R$ řešte rovnici $x^{3+2\log_5 x}=25x^{2+\log_5 x}$.
\end{priklad}

\begin{reseni}
Upravíme a rovnici zlogaritmujeme.
\end{reseni}

\begin{priklad}
V $\mathbb R$ řešte nerovnici $3^{2x+5}\leq 3^{x+2}+2.$
\end{priklad}

\begin{reseni}
Užijeme výhodné substituce $a=3^{x+2}$.
\end{reseni}

\begin{priklad}
V $\mathbb R$ řešte soustavu rovnic
\begin{align*}
    x^{\log y} &= 4\\
    xy &= 40.
\end{align*}
\end{priklad}

\begin{reseni}
Druhou rovnici zlogaritmujeme a využijeme substituce $a=\log x, b=\log y.$
\end{reseni}

\begin{priklad}
V $\mathbb R$ řešte rovnici $\sin x = \frac{1}{2}.$
\end{priklad}

\begin{reseni}
Z náčrtku jednotkové kružnice zjistíme $x_1 = \frac{\pi}{6}+2k\pi$ a $x_2=\pi-\frac{\pi}{6}+2k\pi.$
\end{reseni}

\begin{priklad}
V $\mathbb R$ řešte rovnici $\tg x = -\sqrt{3}. $
\end{priklad}

\begin{reseni}
Z náčrtku zjistíme $x_1=-\frac{\pi}{3}+k\pi$ a $x_2=\frac{2\pi}{3}+k\pi.$
\end{reseni}

\begin{priklad}
V $\mathbb R$ řešte nerovnici $\sin x \geq \frac{\sqrt{2} }{2}.$
\end{priklad}


\subsection{Důkazy}
\begin{priklad}
Dokažte, že je-li $n$ sudé, pak i $n^2$ je sudé.
\end{priklad}

\begin{reseni}\label{dkpr}
Přímým důkazem. Nechť $n$ je sudé přirozené číslo, tedy lze jej zapsat ve tvaru $2k, k \in \mathbb N.$
Pak $n^2 = 4k^2 = 2\cdot 2l, 2l\in \mathbb N, $ tedy $n^2$ je sudé.
\end{reseni}

\begin{priklad}
Dokažte, že je-li $n^2$ sudé, je i $n$ sudé.
\end{priklad}

\begin{reseni}
Nepřímým důkazem. Obměna: Je-li $n$ liché, je i $n^2$ liché. Obdobně jako v příkladu \ref{dkpr}.
\end{reseni}

\begin{priklad}
Dokažte, že pro všechna $n\in N$ platí: $n$ je sudé právě tehdy, když $n^3$ je sudé.
\end{priklad}

\begin{reseni}
Musíme dokázat oba směry implikace zvlášť.
\end{reseni}

\begin{priklad}
Dokažte, že každým bodem roviny lze vést k dané přímce nejvýše jednu kolmici.
\end{priklad}

\begin{reseni}
Sporem. Předpokládáme platnost negace a dojdeme ke sporu. Musí tedy platit původní výrok.
\end{reseni}

\begin{priklad}
Dokažte, že $\sqrt{2} $ je iracionální.
\end{priklad}

\begin{reseni}
Sporem.
\end{reseni}

\begin{priklad}
Dokažte pro všechna $n\in \mathbb N: $
$$1+2+3+\dots+n=\frac{n(n+1)}{2}.$$
\end{priklad}

\begin{reseni}
Matematickou indukcí.
\begin{enumerate}[1.]
\item $n=1$: $1=1$
\item Předpokládáme platnost pro $n$ a dokážeme, že pak platí i pro $n+1$.
\end{enumerate}
\end{reseni}

\subsection{Kombinatorická geometrie}
\begin{priklad}
Odvoďte vztah pro počet úhlopříček v konvexním $n$-úhelníku.
\end{priklad}

\begin{reseni}
Odvodíme rekurentní vztah. Je dán $n$-úhelník a zjistíme, kolik připude
úhlopříček, přidáme-li jeden vrchol. Přibude $n-3$ úhlopříček a z jedné strany
se stane úhlopříčka. Celkem tedy přibude $n+1-3+1=n-1$ úhlopříček. Nyní odvodíme
explicitní vztah:
\begin{align*}
    u_3 &= 0,\\
    u_4 &= 2=u_3+2,\\
    u_5 &= 5=u_4+3. \\
    u_6&=9=u_5+4,\\
    \dots &= \dots, \\
    u_{n-1}&=u_{n-2}+(n-3),\\
    u_n &= u_{n-1}+(n-2).
\end{align*}
Sečtením všech řádků a úpravou dostaneme $u_n=\frac{n(n-3)}{2}.$
\end{reseni}

\begin{priklad}
V rovině je dáno $n$ přímek tak, že žádné dvě nejsou rovnoběžné a žádné tři neprochází
jedním bodem. Určete, na kolik částí rozdělují
rovinu.
\end{priklad}

\begin{reseni}
Přidáme-li $n+1.$ přímku, protne $n$ přímek v $n$ bodech. Je rozdělena na
$n+1$ částí, takže přibude $n+1$ částí. Opět odvodíme
rekurentní a explicitní vzorec.
\end{reseni}
