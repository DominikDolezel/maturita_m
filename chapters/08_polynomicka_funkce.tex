\section{Polynomická funkce (především lineární a kvadratické)}

\begin{definition}
\begin{enumerate}
  \item Nechť $b \in \mathbb R$. Funkci $f:y = b$ nazveme \textbf{konstantní funkcí}.
  \item Nechť $a, b \in \mathbb R, a \neq 0$. Pak funkci $f:y= ax + b$ nazveme \textbf{lineární funkcí}. 
\end{enumerate}
\end{definition}

\begin{pozn}
  \begin{itemize}
    \item Definičním oborem konstantní i lineární funkce je $\mathbb R$. \\
          Oborem hodnot konstantní funkce je ${b}$.\\
          Oborem hodnot lineární funkce je $\mathbb R$. 
    \item Grafem konstantní funkce je přímka rovnoběžná s osou $x$. \\
          Grafem lineární funkce je přímka, která není rovnoběžná s osou $x$ ani s osou $y$.
  \end{itemize}
\end{pozn}

\begin{veta}
  Nechť $f: y = ax + b, a \neq 0$ je lineární funkce. Pak platí: 
  \begin{enumerate}[1.]
    \item $b=0 \implies f$ je lichá\\
          $b \neq 0 \implies f$ není ani sudá, ani lichá
    \item $a > 0 \implies f$ je rostoucí\\ 
          $a < 0 \implies f$ je klesající
    \item $f$ není ani shora, ani zdola omezená
    \item $f$ nemá extrémy 
    \item $f$ není periodická
  \end{enumerate}
\end{veta}

\begin{proof}
  \begin{enumerate}[1.]
    \item $b=0 \implies f:y= ax = f(x) \land -ax = -f(x) \implies f$ je lichá\\
          $b\neq 0 \implies f:y = ax + b = f(x) = -ax + b \implies f$ není sudá ani lichá
    \item $a>0 \implies x_1 < x_2 \implies ax_1 < ax_2 \implies ax_1 + b < ax_2 + b \implies f$ je rostoucí\\
          $a<0 \implies x_1 < x_2 \implies ax_1 > ax_2 \implies ax_1 + b > ax_2 + b \impliesf$ je klesající
    \item Obor hodnot je $\mathbb R \implies f$ není omezená.
    \item Plyne z grafu.
    \item Plynez z grafu nebo z toho, že $f$ je ryze monotónní
  \end{enumerate}
\end{proof}
