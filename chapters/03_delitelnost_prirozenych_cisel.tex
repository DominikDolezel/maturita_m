\section{Dělitelnost přirozených čísel}
\begin{definition}
  Nechť $a,b\in\mathbb Z.$ Číslo $a$ dělí číslo $b$, jestliže $\exists c \in \mathbb Z: b=ac$. Zapisujeme $a\, | \, b$.
\end{definition}

\begin{definition}
  Nechť $a\in \mathbb R$. Číslo $|a|$ takové, že
  \begin{enumerate}[$i.$]
    \item $a\geq 0 \implies |a| = a$,
    \item $a<0 \implies |a| = - a$
  \end{enumerate}
  nazýváme \textbf{absolutní hodnotou} čísla $a$.
\end{definition}

\begin{veta}[O dělení se zbytkem]
  Nechť $a\in \mathbb Z, b\in \mathbb N.$ Pak $\exists ! q \in \mathbb Z, r\in \mathbb N_0:$
  $$a=bq+r, 0 \leq r < b.$$
\end{veta}

\begin{definition}
  Nechť $a,b\in \mathbb N$. Pak $c$ je \textbf{společným dělitelem} čísel $a,b$, jestliže $c \, | \, a \land c\, | \, b.$
\end{definition}

\begin{definition}
  $d\in \mathbb N$ je \textbf{největší společný dělitel} čísel $a,b \in \mathbb N,$ jestliže jsou splněny zároveň obě podmínky:
  \begin{enumerate}[$i.$]
    \item $d\, | \, a \land d \, | \, b$ a
    \item $\forall c \in \mathbb N: c \, | \, a \land c \, | \, b \implies c \, | \, d.$
  \end{enumerate}
  Takové číslo značíme $d=D(a,b)=(a,b).$
\end{definition}

\begin{definition}[Euklidův algoritmus]
  Je dán následující algoritmus.  \\
  \texttt{Mějme dána dvě přirozená čísla, uložená v proměnných $u$ a $w$.} \\
  \texttt{Dokud $w$ není nulové, opakuj:}
  \begin{itemize}
    \item \texttt{Do $r$ ulož zbytek po dělení čísla $u$ číslem $w$}
    \item \texttt{Do $u$ ulož $w$}
    \item \texttt{Do $w$ ulož $r$}
  \end{itemize}
  \texttt{Konec algoritmu, v $u$ je uložen největší společný dělitel původních čísel.}
\end{definition}

\begin{definition}
  Nechť $a,b\in \mathbb N.$ Tato čísla jsou \textbf{nesoudělná}, jestliže $D(a,b)=1$. V opačném případě jsou \textbf{soudělná}.
\end{definition}

\begin{veta}[Fundamentální věta aritmetiky]
  Nechť $a_1,a_2,b\in \mathbb N, b>1.$ Pak $b \, | \, a_1a_2 \land D(a_1,b)=1\implies b\, | \, a_2.$
\end{veta}

\begin{definition}
  Nechť $a,b\in \mathbb N.$ Pak $c$ je \textbf{společným násobek} čísel $a,b$, jestliže $a \, | \, c \land b\, | \, c.$
\end{definition}

\begin{definition}
  $n\in \mathbb N$ je \textbf{nejmenší společný násobek} čísel $a,b \in \mathbb N,$ jestliže jsou splněny zároveň obě podmínky:
  \begin{enumerate}[$i.$]
    \item $a\, | \, n \land b \, | \, n$ a
    \item $\forall m \in \mathbb N: a \, | \, m \land b \, | \, m \implies m \, | \, n.$
  \end{enumerate}
  Takové číslo značíme $n=n(a,b)=\left [ a,b\right ] .$
\end{definition}

\begin{veta}
  $\forall a,b \in \mathbb N: ab=D(a,b)\cdot n(a,b).$
\end{veta}

\begin{definition}
  Nechť $n\in \mathbb N, n>1.$ Má-li číslo $n$ pouze triviální dělitele ($1 \, | \, n, n \, | \, n$), nazýváme jej \textbf{prvočíslem}. V opačném případě hovoříme o \textbf{čísle složeném}.
\end{definition}

\begin{veta}
  Každé přirozené složené číslo $n$ má alespoň jednoho prvočíselného dělitele $p\leq \sqrt{n}$.
\end{veta}

\begin{veta}
  Prvočísel je nekonečně mnoho.
\end{veta}

\begin{veta}[Základní věta aritmetiky]
  Každé přirozené číslo $n>1$ lze zapsat ve tvaru:
  $$n=p_1^{m_1}\cdot p_2^{m_2} \cdot p_3^{m_3}\cdot \hdots \cdot p_r^{m_r},$$
  kde $p_i,i\in\{ 1, 2, \dots, r \}$ jsou navzájem různá prvočísla, $m_i\in \mathbb N_0$. Toto vyjádření je jednoznačné až na pořadí činitelů a říkáme mu \textbf{rozklad čísla} $n$ \textbf{na součin prvočinitelů}.
\end{veta}

\begin{veta}[Věta o iraciálnosti odmocnin]
  Nechť $n\in \mathbb N.$ Pak platí: Pokud $n$ není druhou mocninou přirozeného čísla, pak odmocnina z $n$ je iracionální.
\end{veta}

\subsection*{Kritéria dělitelnosti}
\begin{veta}
  Nechť $n\in \mathbb N, n=a_k\cdot 10^k+a_{k-1}\cdot 10^{k-1}+\dots + a\cdot 10 + a_0.$ Pak platí:
  \begin{enumerate}[$i.$]
    \item $2 \, | \, n \iff 2 \, | \, a_0$,
    \item $4 \, | \, n \iff 4 \, | \, (10a_1 + a_0)$,
    \item $5 \, | \, n \iff 5 \, | \, a_0$,
    \item $8 \, | \, n \iff 8 \, | \, (10^2a_2 + 10a_1 + a_0)$ a
    \item $10 \, | \, n \iff a_0 = 0$.
  \end{enumerate}
\end{veta}

\begin{proof}
  $$n = 10(a_k\cdot 10^{k-1}+a_{k-1}\cdot 10 ^{k-2}+\dots+a_1)+a_0 = 10l+a_0, l\in \mathbb N$$
\end{proof}

\begin{definition}
  Nechť $n\in \mathbb N, n=a_k\cdot 10^k+a_{k-1}\cdot 10^{k-1}+\dots + a\cdot 10 + a_0.$ Pak číslo
  $$S(n) = \sum_{i=0}^k a_i$$
  nazveme \textbf{ciferným součtem} čísla $n$.
\end{definition}

\begin{veta}
  Nechť $n\in \mathbb N, n=a_k\cdot 10^k+a_{k-1}\cdot 10^{k-1}+\dots + a\cdot 10 + a_0,$ $S(n)$ je ciferný součet čísla $n$. Pak platí:
  \begin{enumerate}[$i.$]
    \item $3\, | \, n \iff 3 \, | \, S(n)$ a
    \item $9\, | \, n \iff 9 \, | \, S(n)$
  \end{enumerate}
\end{veta}

\begin{proof}
  Dokážeme položku $ii$. Důkaz té první proběhne obdobně.
  \begin{align*}
    10^k & = (9+1)^k = 9l+1, l \in \mathbb N \\
    n & = a_k(9l_k+1)+a_{k-1}(9l_{k-1}+1)+\dots + a_1(9l_1+1)+a_0 \\
      & = 9(a_kl_k+a_{k-1}l_{k-1}+\dots+a_1l_1)+(a_k+a_{k-1}+\dots+a_0) \\
      & = 9m+S(n), m \in \mathbb N
  \end{align*}
  \begin{enumerate}
    \item $\implies: 9 \, | \, n \land 9 \, | \, 9m \implies 9 \, | \, S(n)$,
    \item $\impliedby: 9 \, | \, S(n) \land 9 \, | \, 9m \implies 9 \, | \, n.$ \qedhere
  \end{enumerate}
\end{proof}

\begin{veta}
  Nechť $n\in \mathbb N, n=a_k\cdot 10^k + a_{k-1}\cdot 10^{k-1}+\dots+a_1\cdot 10+ a_0$. Pak platí
  \[
    11 \, | \, n \iff 11 \, | \, (a_0-a_1+a_2-a_3+\dots+(-1)^k a_k).
  \]
\end{veta}

\begin{veta}
  Nechť $n\in \mathbb N, n-10k+a_0,k\in \mathbb N.$ Pak platí
  \begin{enumerate}[$i.$]
    \item $7 \, | \, n \iff 7 \, | \, (k+5a_0),$
    \item $13 \, | \, n \iff 13 \, | \, (k+4a_0) $
  \end{enumerate}
\end{veta}

\begin{proof}
  $n=10k+a_0$
  \begin{enumerate}[$i.$]
    \item $b=k+5a_0$
      \[
        10b=10k+50a_0=10k+a_0+49a_0=n+49a_0
      \]
      \begin{enumerate}
        \item $\implies: 7 \, | \, n \land 7 \, | \, 49a_0 \implies 7 \, | \, 10b \implies 7 \, | \, b \textrm{, neboť } D(7,10)=1$
        \item $\impliedby: 7 \, | \, b \implies 7 \, | \, 10 b \land 7 \, | \, 49 a_0 \implies 7 \, | \, n$
      \end{enumerate}
    \item $b=k+4a_0$
      \[
        10b=10k+40a_0=10k+a_0+39a_0=n+39a_0
      \]
      \begin{enumerate}
        \item $\implies: 13 \, | \, n \land 13 \, | \, 39a_0 \implies 13 \, | \, 10b \implies 13 \, | \, b \textrm{, neboť } D(7,10)=1$
        \item $\impliedby: 13 \, | \, b \implies  13 \, | \, 10 b \land 13 \, | \, 13a_0 \implies 13 \, | \, n$ \qedhere
      \end{enumerate}
  \end{enumerate}
\end{proof}

\begin{veta}[Binomická věta]\label{bin_v}
  Nechť $x,y \in \mathbb N_0$. Pak
  $$(x+y)^n=\sum_{k=0}^n \binom{n}{k}x^{n-k}y^k.$$
\end{veta}

\begin{dusledek}
  Důsledkem věty \ref{bin_v} je následující fakt:
  \[
    (a+b)^n=a^n+kb=la+b^n,\,\,\, k,l\in \mathbb N.
  \]
\end{dusledek}

\begin{example}[SÚM 140/272]
  Dokažte, že platí vztah $n=\displaystyle\frac{ab}{d},$ kde $a,b\in \mathbb N$ a $d = D(a,b),$ $n=n(a,b).$
  \rm
  \begin{equation*}
    n=\displaystyle\frac{ab}{d}\iff nd=ab
  \end{equation*}
  Označme $a=da^\prime, b=db^\prime,$ kde $D(a^\prime, b^\prime)=1$. Pak
  \[
    ab=da^\prime b^\prime = d(a^\prime b^\prime d) = dn.
  \]
\end{example}


\begin{example}[SÚM 141/273]
  Nejmenší společný násobek čísla 21900 a trojciferného čísla $x$ je 13140. Určete číslo $x$.

  \rm
  Označme $a=2190=da^\prime$, $x=dx^\prime$ je hledané trojciferné číslo a $d = D(a,x),$ navíc platí $D(a^\prime,x^\prime)=1.$ Potom platí
  \begin{align*}
     ax&=nd \\
     a^\prime dx^\prime d &= nd\\
    n&=da^\prime x ^\prime = 13140 \implies x ^\prime= \displaystyle\frac{13140}{2190}=6 \\
    a^\prime d & = 2190
  \end{align*}
  Rozložme číslo 2190 na součin prvočinitelů.
  \[
    2190=2\cdot 3 \cdot 5\cdot 73
  \]
  Aby byla splněna podmínka $D(a^\prime, x^\prime)=1$, číslo $a^\prime$ nesmí být násobek čísel $2,3$, a proto musí být jejich násobek číslo $d$ (tzn. $d$ je násobek šesti). Navíc, aby bylo číslo $x$ trojciferné, musí platit $17 \leq d \leq 166.$ Aby byly obě podmínky splněny, musí jedině $d=30$. Potom $x=180.$
\end{example}
