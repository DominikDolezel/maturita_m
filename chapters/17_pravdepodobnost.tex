\section{Pravděpodobnost}
\begin{pozn}
    Rozlišujeme dva typy pokusů:
   	\begin{itemize}
    \item náhodný (předem neznáme výsledek),
   	\item determinovaný (předem známe výsledek).
    \end{itemize}
\end{pozn}

\begin{definition}
    \textbf{Náhodným jevem} rozumíme jakékoliv tvrzení o výsledku náhodného pokusu,
    o kterém můžeme po provedení říci, zda je nebo není pravdivé.
\end{definition}

\begin{definition}
    Nechť $A,B$ jsou jevy.
    \begin{enumerate}[$i.$]
    \item Řekneme, že \textbf{jev $A$ má za důsledek jev $B$} a zapisujeme
    $A\subseteq B$ nebo $A\implies B$ právě tehdy, když jev $B$ nastane vždy, když
    nastane jev $A.$
   	\item Řekneme, že jevy $A$ a $B$ \textbf{jsou si rovny} a zapisujeme $A=B$ právě
    tehdy, když $A\subseteq B \land B\subseteq A,$ tedy $B$ nastane právě tehdy,
    když nastane $A.$
   	\item Jev, který nastane při každé realizaci pokusu nazveme jevem \textbf{jistým}
    a označíme $\Omega.$ Jev, který nemůže nikdy nastat nazveme jevem \textbf{nemožným}
    a označíme $\emptyset.$
    \end{enumerate}
\end{definition}

\begin{definition}
    Nechť $A_1, A_2, \dots, A_n$ jsou jevy. Pak \textbf{sjednocením} (resp. \textbf{
    průnikem}) jevů $A_1, A_2, \dots, A_n$ rozumíme takový jev $A$, který nastane
    právě tehdy, když nastane alespoň jeden (resp. všechny z) jevů $A_1, A_2,
    \dots, A_n$.
\end{definition}

\begin{definition}
    Nechť $A,B$ jsou jevy.
    \begin{enumerate}[$i.$]
    \item \textbf{Opačným jevem} k jevu $A$ nazveme takový jev $\overline A,$ který
    nastane právě tehdy, kdyý nenastane jev $A.$
   	\item \textbf{Rozdílem jevů} $A,B$ nazýváme jev $A-B,$ který nastane právě tehdy,
    když nastane jev $A$ a nenastane jev $B$.
    \end{enumerate}
\end{definition}

\begin{definition}
    Jevy nazveme \textbf{neslučitelné} (disjunktní), jestliže $A\cap B = \emptyset.$
\end{definition}

\begin{definition}
    Nechť je dán náhodný pokus. Jev $A$ nazveme \textbf{elementárním jevem} právě tehdy,
    když neexistují žádné dva jevy $B,C; A\ne B\ne C$ takové, že $A=B\cup C,$ tj.
    $A$ nelze vyjádřit jako sjednocení dvou jevů různých od $A.$ Množinu všech
    elementárních jevů nazveme \textbf{jevovým polem}. Je to množina všech možných
    výsledků daného náhodného pokusu. Tuto množinu označíme $\Omega.$
\end{definition}

\begin{definition}
    Nechť $\Omega$ je konečná neprázdná množina stejně možných výsledků daného
    náhodného pokusu, tzn. $\Omega$ je jevové pole. Nechť $A\subseteq \Omega$ je jev.
    Označme $|A|,$ resp. $|\Omega|$ počet prvků množiny $A$, resp. $\Omega.$ Pak
    \textbf{pravděpodobností} jevu $A$ nazýváme číslo
    $$P(A)=\frac{|A|}{|\Omega|}.$$
\end{definition}

\begin{priklad}
Hážeme dvěma mincemi. Určete pravděpodobnost následujících jevů:
\begin{enumerate}[a.]
\item na obou mincích padne hlava,
\item aspoň na jedné minci padne orel,
\item na obou mincích padne totéž.
\end{enumerate}
\end{priklad}

\begin{reseni}
Vypíšeme si všechny možnosti a podělíme ty, které vyhovují celkovým počtem.
\end{reseni}

\begin{veta}
    Nechť $A,B\subseteq \Omega$ jsou jevy. Pak
    $$P(A\cup B)=P(A)+P(B)-P(A\cap B).$$
\end{veta}

\begin{definition}
    Nechť $A,B\subseteq \Omega$ jsou jevy. Nazveme je \textbf{nezávislé} právě tehdy,
    když $P(A\cap B)=P(A)\cdot P(B).$
\end{definition}

\begin{priklad}
Zjistěte, zda jsou dané jevy nezávislé.
\begin{enumerate}[a.]
\item Na kostce padne sudé číslo. Na kostce padne 5 nebo 6.
\item Na ksotce padne sudé číslo. Na kostce padne liché číslo.
\end{enumerate}
\end{priklad}

\begin{reseni}
Spořítáme pravděpodobnosti jednotlivých jevů a pravděpodobnost toho, že nastanou
oba současně. Pokud je $P(A\cap B)=P(A)\cdot P(B)$, pak jsou nezávislé.
\end{reseni}

\begin{definition}
    Nechť $A,B\subseteq \Omega$ jsou jevy takové, že $P(B)>0.$ \textbf{Podmíněnou
    pravděpodobností} jevu $A$ za předpokladu nastoupení jevu $B$ nazýváme reálné
    číslo dané vzorcem
    $$P(A\, | \, B) = \frac{P(A\cap B)}{P(B)}.$$
\end{definition}

\begin{priklad}
Házíme dvakrát kostkou. Nechť
\begin{align*}
    &A \textrm{ } \dots  \textrm{ součet obou čísel je dělitelný čtyřmi,}\\
    &B \textrm{ } \dots  \textrm{ druhým hodem padne šestka}.
\end{align*}
Určete $P(A\, |\, B).$
\end{priklad}

\begin{reseni}
Platí
$$P(A\, |\, B)=\frac{P(A\cap B)}{P(B)}=\frac{\frac{2}{36}}{\frac{6}{36}}=\frac{1}{3}.$$
\end{reseni}

\begin{veta}[Formule úplné pravděpodobnosti]
    Nechť $A,B_i \subseteq \Omega, i \in \left \{ 1, 2, \dots, n \right \} $ jsou jevy
    takové, že $A\subseteq \bigcup_{i=1}^n B_i$ a $\forall i\ne j: B_i\cap B_j\ne
    \emptyset.$ Pak $P(A)=P(B_1)\cdot P(A\, |\, B_1) + P(B_2)\cdot P(A \, |\, B_2)+
    \dots + P(B_n)\cdot P(A \,|\, B_n)=\sum_{i=1}^n P(B_i)\cdot P(A\, |\, B_i).$
\end{veta}

\begin{priklad}
V osudí $A$ jsou dvě černé a tři bílé kuličky. V osudí $B$ jsou dvě černé a jedna
bílá kulička. Zvolíme náhodně jedno osudí a vyáthneme jednu kuličku. Jaká
je pravděpodobnost, že je je bílá?
\end{priklad}

\begin{reseni}
Označme jevy
\begin{center}
    $A$ \dots zvolíme osudí $A$,
    $B$ \dots zvolíme osudí $B$,
    $X$ \dots vytáhli jsme bílou kuličku.
\end{center}
Pak platí
$P(X \, | \, A) = \frac{3}{5}, P(X \, | B)=\frac{1}{3}.$ Je tedy
$$P(X)=P(A)\cdot P(X \, | A)+P(B)\cdot P(X\, | \, B)=\frac{7}{15}.$$
\end{reseni}

\begin{veta}[Bayesův vzorec]
    Nechť $A,B_i \subseteq \Omega, i \in \left \{ 1, 2, \dots, n \right \} $ jsou
    takové jevy, že $A\subseteq \bigcup_{i=1} B_i, \forall i,j, i\ne j: B_i\cap B_j=
    \emptyset.$ Nechť $\exists i \in \left \{ 1, 2, \dots, n \right \} : P(B_i) >0.$
    Pak platí: $\forall k\in \left \{ 1, 2, \dots, n \right \}:$
    $$P(B_k \, |\, A)=\frac{P(B_k)\cdot P(A\, |\, B_k)}{\sum_{i=1}^n P(B_i)\cdot
    P(A\, |\, B_i)}. $$
\end{veta}

\begin{priklad}
Na fakultě studuje 60 \% děvčat, z chlapců studuje matematiku 25 \%,
z děvčat 10 \%. Náhodně vybereme studenta/ku studující matiku. Určete
pravděpodobnost toho, že to bude dívka.
\end{priklad}

\begin{reseni}
Platí
\begin{align*}
P(D)=0,6 & & P(H)=0,4 \\
P(M \, | \, D)=0,1 & & P(M \, | \, H)=0,25.
\end{align*}
Hledáme $P(D \, | \, M).$ Počítejme
\begin{align*}
 P(D)\cdot P(M \, | \, D)&=P(M \cap D)=P(M)\cdot P(D \, | M)\\
 P(D, \,|\, M) &= \frac{P(D)\cdot P(M \, | \, D)}{P(M)}=\frac{P(D)\cdot P(M \, | \, D)}{P(D)\cdot P(M\, | \, D)+P(H)\cdot P(M \, | \, H)}=\frac{3}{8}.
\end{align*}
\end{reseni}

\begin{veta}[Bernoulliho věta]
    Provádíme-li sérii $n$ nezávislých pokusů, kdy pravděpodobnost příslušného
    pokusu je $p,$ pak pravděpodobnost toho, že právě $k$ pokusů bude úspěšných, platí
    $$P(k,n)=\binom{n}{k}\cdot p^k \cdot (1-p)^{n-k}.$$
\end{veta}

\begin{priklad}
Uvažuje $n$ hodů mincí. Jev \uv{padne hlava} označme $Z$, \uv{padne orel} označme $N$.
Určete pravděpodobnost toho, že v sérii sto hodů padne hlava právě padesátkrát.
\end{priklad}

\begin{reseni}
Platí
$$P=\binom{100}{50}\cdot \left ( \frac{1}{2} \right )^{50}\cdot \left ( \frac{1}{2} \right )^{50}.  $$
\end{reseni}
