\section{Derivace funkce}
\begin{definition}\label{derivace}
Nechť $a \in \mathbb R$ a $f$ je funkce. Jestliže existuje limita
$$\lim_{h\to 0} \frac{f(a+h)-f(a)}{h},$$
pak tuto limitu nazýváme \textbf{derivací funkce} $f$ \textbf{v bodě} $a$ a značíme ji $f^\prime (a).$
Obdobně definujeme \textbf{derivaci zprava} a \textbf{derivaci zleva funkce} $f$ \textbf{v bodě}
$a$ předpisy
\begin{align*}
    f_+^\prime(a) = \lim_{h\to 0+} \frac{f(a+h)-f(a)}{h} & & a & & f_-^\prime(a) = \lim_{h\to 0-} \frac{f(a+h)-f(a)}{h}.
\end{align*}
Derivaci zleva a derivaci zprava souhrnně nazýváme \textbf{jednostrannými derivacemi}.
\end{definition}

\begin{pozn}
    Derivaci lze obdobně zavést jako limitu
    $$\lim_{x\to a} \frac{f(x)-f(a)}{x-a}.$$
\end{pozn}

\begin{pozn}
     Při počítání derivace funkce $f$ v bodě $a\in \mathbb R$ mohou nastat
     tyto případy:
     $$
     \textrm{derivace v bodě } a \begin{cases}
        \textrm{neexistuje,} \\
        \textrm{existuje a je} \begin{cases}
            \textbf{vlastní}\textrm{, tj. je rovna reálnému číslu}, \\
            \textbf{nevlastní}\textrm{, tj. je rovna } +\infty \textrm{ nebo } - \infty.
        \end{cases}
     \end{cases}
     $$
\end{pozn}

\begin{pozn}
    Hodnoda derivace v daném bodě je směrnice tečny funkce v tomto bodě.
\end{pozn}

\begin{veta}
    Jestliže existuje derivace funkce $f$ v bodě $a$ (vlastní či nevlastní),
    pak je určena jednoznačně.
\end{veta}

\begin{definition}
Nechť existuje vlastní derivace funkce $f(x)$ pro všechna $x\in M, M\subset D(f).$
Pak funkci $f^\prime(x)$, která každému bodu $x\in M$ přiřadí derivaci funkce $f$
v tomto bodě, nazýváme \textbf{derivaci funkce} $f$ \textbf{na množině} $M.$
\end{definition}

\begin{veta}[Bolzanova]
Má-li funkce $f$ v bodě $a$ vlastní derivaci, je v tomto bodě spojitá.
\end{veta}

\begin{pozn}
    Má-li funkce $f:M\to \mathbb R$ na množině $M\subset \mathbb R$ vlastní derivaci
    v každém bodě $x\in M,$ pak zobrazení $f^\prime:M\to \mathbb R,$ které přiřadí
    bodu $x\in M$ hodnotu $f^\prime(x),$ je reálnou funkcí definovanou na množině $M$.
\end{pozn}

\begin{pozn}[Derivace základních funkcí]
Platí:
\begin{enumerate}[$i.$]
\item $(c)^\prime = 0, c \in \mathbb R,$
\item $(x^r)^\prime = r\cdot x^{r-1}, r\in \mathbb R, x \in \mathbb R^+,$
\item $(\sin x)^\prime = \cos x, x \in \mathbb R,$
\item $(\cos x) ^\prime = -\sin x, x \in \mathbb R,$
\item $(e^x)^\prime = e^x,x \in \mathbb R,$
\item $(\tg x)^\prime = \frac{1}{\cos^2 x}, x \in \mathbb R-\left \{ \frac{\pi}{2}+k\pi, k\in \mathbb Z \right \} $
\item $(\cotg x)^\prime = -\frac{1}{\sin^2 x}, x \in \mathbb R-\left \{ k\pi, k \in \mathbb Z \right \} $
\item $(\ln x)^\prime = \frac{1}{x}, x \in \mathbb R^+,$
\item $(\arcsin x)^\prime = \frac{1}{\sqrt{1-x^2}}, x \in(-1,1),$
\item $(\arccos x)^\prime=-\frac{1}{\sqrt{1-x^2} }, x \in (-1,1),$
\item $(\arctg x)^\prime = \frac{1}{x^2+1}, x \in \mathbb R,$
\item $(\arccotg x)^\prime = -\frac{1}{x^2+1}, x \in \mathbb R,$
\item $(a^x)^\prime = a^x \ln a, a>0, a\ne 1, x\in \mathbb R,$
\item $(\log_a x)^\prime=\frac{1}{x\ln a}, a>0, a\ne 1, x \in \mathbb R^+.$
\end{enumerate}
\end{pozn}

\begin{priklad}
  Spočtěte derivaci funkce $f(x)=x.$
\end{priklad}

\begin{reseni}
  Je $f^\prime(x)=1.$
\end{reseni}

\begin{veta}[Věta o derivaci součtu, rozdílu, součinu a podílu funkcí]
Nechť existují derivace funkcí $f$ a $g$ v bodě $a\in \mathbb R.$ Pak
také funkce $f\pm g, fg, f/g$ a $cf,$ kde $c\in \mathbb R$, mají v bodě $a$ derivaci a platí:
\begin{enumerate}[$i.$]
\item $(f\pm g)^\prime (a) = f^\prime(a) \pm g^\prime(a),$
\item $(fg)^\prime (a) = f^\prime (a)g(a)+ f(a)g^\prime(a),$
\item $\left ( \frac{f}{g} \right )^\prime(a)=\frac{f^\prime (a)g(a)-f(a)g^\prime(a)}{g^2(a)} $ pro $g(a)\ne 0,$
\item $(cf)^\prime(a)=cf^\prime (a).$
\end{enumerate}
\end{veta}

\begin{priklad}
  Spočtěte derivaci funkce $f(x)=xe^x.$
\end{priklad}

\begin{reseni}
  Je $f^\prime(x)=1e^x+xe^x=e^x(x+1)$.
\end{reseni}

\begin{veta}[Derivace inverzní funkce]
Nechť funcke $f:x=f(y)$ je spojitá a ryze monotónní na intervalu $I.$ Nechť $y_0$
je vnitřní bod intervalu $I$ a nechť má $f$ v $y_0$ derivaci $f^\prime(y_0).$
Pak inverzní funkce $f^{-1}:y=f^{-1}(x)$ má v bodě $x_0=f(y_0)$ derivaci a platí
$$
\left ( f^{-1} \right )^\prime(x_0)= \begin{cases}
\frac{1}{f^\prime(y_0)}, & \textrm{je-li } f^\prime(y_0)\ne 0,\\
+\infty & \textrm{je-li } f^\prime (y_0)=0 \textrm{ a funkce } f \textrm{ je na } I \textrm{ rostoucí},\\
-\infty & \textrm{je-li } f^\prime (y_0)=0 \textrm{ a funkce } f \textrm{ je na } I \textrm{ klesající}.
\end{cases}
$$
\end{veta}

\begin{veta}[Derivace složené funkce]
Nechť $f, g$ jsou funkce. Nechť existuje derivace funkce $g$ v bodě $a$ a
derivace funkce $f$ v bodě $b=g(a).$ Pak i složená funkce $F=f\circ g$ má derivaci
v bodě $a$ a platí
$$F^\prime(a)=(f\circ g)^\prime (a) = f^\prime(b)\cdot g^\prime(a)=f^\prime(g(a))\cdot g^\prime(a).$$
\end{veta}

\begin{priklad}
  Spočtěte derivaci funkce $f(x)=\left (x^3+2x^2+x-1\right )^8.$
\end{priklad}

\begin{reseni}
  Zavedeme substituci $x^3+2x^2+x-1=u$. Pak $f^\prime(x)=8u^7u^\prime$. Dále $u^\prime=3x^2-4x+1.$
  Je tedy $f^\prime(x)=8\left(x^3+2x^2+x-1\right)^7(3x^2-4x+1).$
\end{reseni}

\begin{pozn}
    Funkce tvaru $f(x)^{g(x)}$ derivujeme jako složenou funkci $e^{g(x)\ln f(x)},$
    protože $f(x)^{g(x)}=e^{g(x)\ln f(x)}$ a funkce $e^r, r \in \mathbb R$ je prostá
    a spojitá.
\end{pozn}

\begin{priklad}
  Spočtěte derivaci funkce $f(x)=x^x.$
\end{priklad}

\begin{reseni}
  Platí $f(x)=e^{x\ln x},$ takže $f^\prime(x)=\left (e^{x\ln x}\right)^\prime=e^{x\ln x}(x\ln x)^\prime=
  x^x\left(\ln x+ x\cdot \frac{1}{x}\right).$
\end{reseni}

\begin{definition}
Nechť $n \in \mathbb N$. Potom \textbf{$n$-tou derivací} funkce $f$ rozumíme funkci,
kterou označujeme $f^{(n)}$ a definujeme
$$f^{(n)}=\left ( f^{(n-1)} \right )^\prime, $$
přičemž $f^{(0)}=f.$
\end{definition}

\begin{definition}
Přímka $t$ o rovnici
$$y-f(x_0)=f^\prime(x_0)(x-x_0)$$
se nazývá \textbf{tečna ke grafu funkce} $f$ v dotykovém bodě $T=(x_0,f(x_0))$.
Přímka $n$, která prochází bodem $T$ a je kolmá k přímce $t$, se nazývá \textbf{normála
ke grafu funkce} $f$ v bodě $T.$
\end{definition}

\begin{veta}
    Má-li tečna grafu funkce $t$ v dotykovém bodě $T=(x_0, f(x_0))$ rovnici
    $$y-y_0=f^\prime(x_0)(x-x_0), \textrm{ kde } y_0=f(x_0),$$
    pak normála $n$ má rovnici
    $$y-y_0=-\frac{1}{f^\prime(x_0)}(x-x_0), \textrm{ pokud } f^\prime(x_0)\ne 0.$$
\end{veta}

\begin{pozn}
    Funkci lze vyjádřit buď jako $y=f(x)$ (explicitně -- tedy vždy máme přímo vyjádřeno
    $y$), nebo ve tvaru $f(x,y)=0$ (implicitně -- z této rovnice obecně nelze vyjádřit
    $y$). Například $x^2+y^2-r^2=0$ je implicitní vyjádření kružnice,
    zatímco $y=\pm \sqrt{r^2-x^2} $ je explicitní vyjádření.
\end{pozn}

\begin{veta}[Derivace funkce dané implicitně]\label{implfce}
Nechť máme funkci danou implicitně rovnicí $F(x,y)=0$. Pak
$$F^\prime (x,y)=\frac{\partial F(x,y)}{\partial x} + \frac{\partial F(x,y)}{\partial y}\cdot y^\prime.$$
\end{veta}

\begin{pozn}
Věta \ref{implfce} je složitá. Lze ji shrnout asi takhle:\\
Funkci zderivujeme podle $x$ ($y$ považujeme za konstantu) tak, jak jsme zvyklí. Pak
funkci zderivujeme ještě podle $y$ a předtím, než to přičteme k tomu,
co jsme předtím derivovali podle $x$, to vynásobíme $y^\prime$. \\
To funguje,
protože se vlastně díváme na $y$ jako na funkci $x$ ($y=f(x)$), takže
derivace $y=f^\prime(x)\cdot (f(x))^\prime$ (např. $\left ( y^2 \right )^\prime =2yy^\prime$).
\end{pozn}

\begin{priklad}
  Spočtěte derivaci funkce rovnice kružnice $x^2+y^2=1.$
\end{priklad}

\begin{reseni}
  Derivujeme funkci danou implicitně: $f(x,y)^\prime: 2x+2yy^\prime=0.$
\end{reseni}

\begin{veta}[Cauchy-Bolzanova věta]
Nechť je funkce $f$ spojitá na uzavřeném intervalu $\left < a,b \right > $ a platí
$f(a)\cdot f(b)<0.$ Pak existuje alespoň jedno číslo $x_0\in (a,b)$ takové, že $f(x_0)=0.$
\end{veta}

\begin{priklad}
  Určete znaménka funkce $f(x)=\frac{1-x^2}{x}.$
\end{priklad}

\begin{veta}[Rolleho věta]
Nechť funkce $f$ má následující vlastnosti:
\begin{enumerate}[$i.$]
\item je spojitá na uzavřeném ohraničeném intervalu $\left < a,b \right > $,
\item má derivaci na otevřeném intervalu $(a,b)$,
\item platí $f(a)=f(b)$.
\end{enumerate}
Pak existuje alespoň jedno číslo $x_0\in (a,b)$ takové, že $f^\prime(x_0)=0.$
\end{veta}

\begin{veta}[Lagrangeova věta]
Nechť funkce $f$ má následující vlastnosti:
\begin{enumerate}[$i.$]
\item je spojitá na uzavřeném ohraničeném intervalu $\left < a,b \right > $,
\item má derivaci na otevřeném intervalu $(a,b)$.
\end{enumerate}
Pak existuje alespoň jedno číslo $x_0\in (a,b)$ takové, že
$$f^\prime(x_0)=\frac{f(b)-f(a)}{b-a}.$$
\end{veta}

\begin{veta}[Cauchyho věta]
Nechť funkce $f$ a $g$ mají následující vlastnosti:
\begin{enumerate}[$i.$]
\item jsou spojité na uzavřeném ohraničeném intervalu $\left < a,b \right > $,
\item mají derivaci na otevřeném intervalu $(a,b)$, přičemž $g^\prime(x)\ne 0$ na $\left < a,b \right > $.
\end{enumerate}
Pak existuje alespoň jedno číslo $x_0\in (a,b)$ takové, že
$$\frac{f^\prime(x_0)}{g^\prime(x_0)}=\frac{f(b)-f(a)}{g(b)-g(a)}.$$
\end{veta}
