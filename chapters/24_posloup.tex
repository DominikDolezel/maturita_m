\section{Posloupnosti}
\begin{definition}
Nechť $A$ je množina. Zobrazení $a:\mathbb N\to A$ (nebo $a:\left \{ 1,2,...,k \right \} \to A$)
se nazývá \textbf{posloupnost} prvků množiny $A$, $k\in \mathbb N.$
\end{definition}

\begin{definition}
Nechť $\left \{ a_n \right \}_{n=1}^\infty $ je posloupnost reálných čísel. Pak je
\begin{enumerate}[$i.$]
\item \textbf{rostoucí} právě tehdy, když $\forall n \in \mathbb N: a_n< a_{n+1},$
\item \textbf{klesající} právě tehdy, když $\forall n \in \mathbb N: a_n> a_{n+1},$
\item \textbf{nerostoucí} právě tehdy, když $\forall n \in \mathbb N: a_n\geq a_{n+1},$
\item \textbf{neklesající} právě tehdy, když $\forall n \in \mathbb N: a_n\leq a_{n+1}.$
\end{enumerate}
Posloupnost, která má jednu z těchto vlastností, se nazývá \textbf{monotónní}.
Posloupnost rostoucí nebo klesající nazýváme \textbf{ryze monotónní}.
\end{definition}

\begin{definition}
    Nechť $\left \{ a_n \right \}_{n=1}^\infty $ je posloupnost reálných čísel. Pak je
    \begin{enumerate}[$i.$]
    \item \textbf{prostá} právě tehdy, když $\forall m,n\in \mathbb N: m\ne n \implies a_m\ne a_n,$
   	\item \textbf{stacionární} (konstantní) právě tehdy, když $\forall n \in \mathbb N:a_n = a_{n+1}.$
    \end{enumerate}
\end{definition}

\begin{definition}
    Nechť $\left \{ a_n \right \}_{n=1}^\infty $ je posloupnost reálných čísel. Pak je
    \begin{enumerate}[$i.$]
    \item \textbf{zdola omezená}, pokud existuje $k\in \mathbb R$ takové, že
    $\forall n \in \mathbb N: a_n \geq k,$
   	\item \textbf{shora omezená}, pokud existuje $k\in \mathbb R$ takové, že
    $\forall n \in \mathbb N: a_n \leq k.$
    \end{enumerate}
    Posloupnost je \textbf{omezená}, pokud je omezení shora i zdola.
    Pokud není omezení ani shora, ani zdola, je \textbf{neomezená}.
\end{definition}

\begin{definition}
Nechť $\left \{ a_n \right \}_{n=1}^\infty $ je posloupnost reálných čísel a
$\left \{ k_1,k_2,\dots,k_n,\dots \right \} $ je rostoucí posloupnost přirozených čísel.
Pak posloupnost $\left \{{a_k}_n \right \}_{n=1}^\infty = \left \{ {a_k}_1, {a_k}_2, \dots, {a_k}_n,\dots \right \}  $
je \textbf{posloupnost vybraná z posloupnosti} $\left \{ a_n \right \}_{n=1}^\infty $.
\end{definition}

\begin{definition}
Posloupnost $\left \{ a_n \right \}_{n=1}^\infty $ je \textbf{aritmetická} právě
tehdy, když
$$\exists d \in \mathbb R: \forall n \in \mathbb N: a_n=a_{n-1}+d$$
a číslo $d$ se nazývá \textbf{diference}.
\end{definition}

\begin{veta}
    Nechť $\left \{ a_n \right \}_{n=1}^\infty $ je aritmetická posloupnost s diferencí
    $d$. Pak platí:
    \begin{enumerate}[$i.$]
    \item $\forall n\in \mathbb N: a_n = a_1 + (n-1)d,$
   	\item $\forall r,s \in \mathbb N: a_s = a_r + (s-r)d.$
    \end{enumerate}
\end{veta}

\begin{proof}
\begin{enumerate}[$i.$]
\item Matematickou indukcí:
\begin{enumerate}[1.]
\item $n=1: a_1 = a_1+(1-1)d$ platí
\item $a_n = a_1+(n-1)d \implies a_{n-1}=a_1+(n)d,$ takže $a_{n+1}=a_n+d=a_1+(n-1)d+d=a_1(n)d$
\end{enumerate}
\item Užitím již dokázeného vztahu:
\begin{align*}
    a_s &= a_1 + (s-1)d,\\
    a_r &= a_1 + (r-1)d.
\end{align*}
Sečtením dostáváme
$$a_s-a_r = (s-1)d-(r-1)d,$$
takže
$$a_s=a_r+(s-r)d,$$
což jsme chtěli dokázat.\qedhere
\end{enumerate}
\end{proof}

\begin{veta}
Nechť $\left \{ a_n \right \}_{n=1}^\infty $ je aritmetická posloupnost s diferencí
$d$. Nechť
$$S_n=a_1+a_2+\dots+a_n$$
je součet prvních $n$ členů nekonečné aritmetické posloupnosti nebo součet
$n$-členné aritmetické poslouppnosti. Pak platí:
$$\forall n \in \mathbb N:S_n = \frac{1}{2}n(a_1+a_n).$$
\end{veta}

\begin{proof}
\begin{align*}
    S_n &= a_1+a_2+a_3+\dots+a_{n-1}+a_n \\
    &= a_1 + (a_1+d)+ (a_1+2d)+ \dots + [a_1+(n-2)d]+[a_1+(n-1)d]\\
    &= [a_1+(n-1)d]+[a_1+(n-2)d]+\dots+(a_1+2d)+(a_1+d)+a_1\\
    2S_n &= 2a_1 + (n-1)d + 2a_1 + (n-1)d + \dots + 2a_1 + (n-1)d + 2a_1 + (n-1)d\\
    2S_n &= n [a_1 + a_1 + (n-1)d], \textrm{ takže }\\
    S_n &= \frac{n(a_1+a_n)}{2},
\end{align*}
což jsme chtěli dokázat. \qedhere
\end{proof}

\begin{definition}
Posloupnost $\left \{ a_n \right \}_{n=1}^\infty $ je \textbf{geometrická} právě
tehdy, když
$$\exists q \in \mathbb R: \forall n \in \mathbb N: a_n=a_{n-1}\cdot q$$
a číslo $q$ se nazývá \textbf{kvocient}.
\end{definition}

\begin{veta}
    Nechť $\left \{ a_n \right \}_{n=1}^\infty $ je geometrická posloupnost s kvocientem
    $q$. Pak platí:
    \begin{enumerate}[$i.$]
    \item $\forall n\in \mathbb N: a_n = a_1 \cdot q^{n-1},$
   	\item $\forall r,s \in \mathbb N: a_s = a_r \cdot q^{s-r}.$
    \end{enumerate}
\end{veta}

\begin{proof}
\begin{enumerate}[$i.$]
\item Matematickou indukcí:
\begin{enumerate}[1.]
\item $n=1: a_1 = a_1\cdot q^0$ platí
\item $a_n = a_1\cdot q^{n-1}\implies a_{n+1}=a_1\cdot q^n,$ takže $a_{n+1}=a_na_n\cdot q=a_1\cdot q^{n-1}\cdot q=a_1\cdot q^n$
\end{enumerate}
\item Užitím již dokázeného vztahu:
\begin{align*}
    a_s &= a_1\cdot q^{s-1},\\
    a_r &= a_1\cdot q^{r-1}.
\end{align*}
Vydělením dostáváme
$$\frac{a_s}{a_r} = \frac{a_1\cdot q^{x-1}}{a_1\cdot q^{r-1}}=q^{s-r},$$
takže
$$a_s=a_r\cdot q^{s-r},$$
což jsme chtěli dokázat.\qedhere
\end{enumerate}
\end{proof}

\begin{veta}
Nechť $\left \{ a_n \right \}_{n=1}^\infty $ je geometrická posloupnost s kvocientem
$q$. Nechť
$$S_n=a_1+a_2+\dots+a_n$$
je součet prvních $n$ členů nekonečné geometrické posloupnosti nebo součet
$n$-členné geometrické poslouppnosti. Pak platí $\forall n \in \mathbb N$:
\begin{enumerate}[$i.$]
\item $q=1: S_n=na_1,$
\item $q\ne 1: S_n = a_1\cdot \frac{q^n-1}{q-1}.$
\end{enumerate}
\end{veta}

\begin{proof}
\begin{align*}
    S_n &= a_1+a_2+a_3+\dots+a_{n-1}+a_n \\
    &= a_1 + a_1q+ a_1q^2+ \dots + a_1q^{n-1}\\
    qS_n &= a_1q + a_1q^2 +  \dots + a_1q^{n-1} + a_1q\\
    qS_n-S_n &= -a_1+a_1q^n\\
    S_n(q-1) &= a_1(q^n-1), \textrm{ jestli } q\ne 0, \textrm{ pak}\\
    S_n &= \frac{a_1(q^n-1)}{q-1}
\end{align*}
což jsme chtěli dokázat. \qedhere
\end{proof}

\begin{pozn}
    Rozlišujeme dva typy zadání posloupnosti, a to:
    \begin{enumerate}[$i.$]
    \item \textbf{rekurentní}: pomocí jednoho nebo několika předchozích členů
    \begin{align*}
        a_n=a_{n-1}+d, & & a_{n-1}\cdot q;
    \end{align*}
   	\item \textbf{explicitní}: $n$-tý člen je vyjádřen pomocí $n$
    \begin{align*}
        a_n=a_1+(n-1)d, & & a_n = a_1\cdot q^{n-1}.
    \end{align*}
    \end{enumerate}
\end{pozn}
