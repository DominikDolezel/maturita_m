\section{Polynomy, kořeny polynomů}
\begin{definition}
  \textbf{Polynomem} nazýváme každý výraz $P(x)$ tvaru
  \[
    P(x)=a_nx^n + a_{n-1}x^{n-1}+\dots + a_2x^2+a_1x+a_0,
  \]
  kde $a_i\in \mathbb R, i\in \{ 0,1,\dots , n \},n\in \mathbb N$. Čísla $a_i$ se nazývají \textbf{koeficienty polynomu}, sčítance $a_ix^i$ \textbf{členy polynomu}. Je-li $a_n\ne 0$, číslo $n$ se nazývá \textbf{stupeň polynomu} a označuje se $n= \text{st}(P(x))$. Je-li $a_i=0$ pro všechna $i \in \{ 0,1,\dots ,n\}$, pak klademe $\text{st}(P(x))=-\infty$ a $P(x)$ se nazývá \textbf{nulovým polynomem}. Označujeme jej $O(x).$ Je-li $a_n=1$, nazývá se $P(x)$ \textbf{normovaným polynomem}.
\end{definition}

\begin{veta}[O dělení polynomů se zbytkem]
  Nechť $A(x), B(x) \in \mathbb R[x],B(x) \ne O(x).$ Pak existuje právě jedna dvojice polynomů $Q(x),R(x)\in \mathbb R[x]$ tak, že
  \[
    A(x)=B(x)\cdot Q(x)+R(x),
  \]
  kde $R(x)=O(x)$ nebo $\text{st}(R(x))<\text{st}(B(x)).$
\end{veta}

\begin{definition}
  Nechť $A(x), B(x) \in \mathbb R[x]$. Řekneme, že polynom $B(x)$ \textbf{dělí} polynom $A(x)$ právě tehdy, když existuje takový polynom $C(x) \in \mathbb R[x]$ tak, že
  \[
    A(x) = B(x)\cdot C(x).
  \]
  Zapisujeme $B(x) \, | \, A(x).$
\end{definition}

\begin{definition}
  Nechť $P(x)\in \mathbb R[x], P(x)=a_nx^n+a_{n-1}x^{n-1}+\dots +a_1x+a_0.$ Nechť $c\in \mathbb R$ je libovolné číslo. \textbf{Hodnotou polynomu} $P(x)$ \textbf{v čísle} (v bodě) $c$ nazýváme reálné číslo $P(c)$
  \[
    P(c) = a_nc^n+a_{n-1}c^{n-1}+\dots+a_1c+a_0.
  \]
  Číslo $c\in \mathbb R$ nazveme \textbf{kořenem polynomu} $P(x) \iff P(c) = 0$.
\end{definition}


\begin{definition}
  Nechť $P(x) \in \mathbb R [x]$ a $c\in \mathbb R$ je jeho kořen. Lineární polynom $x-c$ nazveme \textbf{kořenovým činitelem}.
\end{definition}

\begin{definition}
  Nechť $P(x) \in \mathbb R [x], c \in \mathbb R$ je jeho kořen a $k \in \mathbb N$. Číslo $C \in \mathbb R$ nazýváme \textbf{$k$-násobným kořenem} polynomu $P(x)$ právě tehdy, když platí
  \[
    (x-c)^k \, | \, P(x) \land (x-c)^{k+1} \nmid  P(x).
  \]
\end{definition}

\begin{veta}[Vi\`{e}tovy vztahy]
  Nechť $P(x)=a_nx^n+a_{n-1}x^{n-1}+\dots + a_1x+a_0$ je polynom stupně $n$, který má v množině $\mathbb R$ právě $n$ kořenů $x_1,x_2\dots,x_n$ (každý počítáme tolikrát, jaká je jeho násobnost). Pak platí:
  \begin{align*}
    \sum_{i=1}^n x_i = & -\frac{a_{n-1}}{a_n} \\
    \sum_{i,j=1; i<j}^{n}x_ix_j= & \frac{a_{n-2}}{a_n} \\
    \vdots & \\
    \prod_{i=1}^nx_i=&(-1)^n\frac{a_0}{a_n}
  \end{align*}
\end{veta}

\begin{pozn}
  \textbf{Hornerovo schéma} je numerická metoda pro vyhodnocení funkční hodnoty polynomu $P(x) \in \mathbb R [x]$ v bodě $ x_0 \in \mathbb R$. Příklad:
  Vyhodnoťte $f_{1}(x)=2x^{3}-6x^{2}+2x-1\,$ v bodě $x=3\;$.
  Opakovaným vytknutím $x$, může být $f_{1}$ zapsáno jako $x(x(2x-6)+2)-1\;$. Pro větší přehlednost užijeme k zápisu průběhu výpočtu tzv. syntetický diagram.
  \begin{tabular}{ c|c c c c }
    $x_{0}$ & $x^{3}$ & $x^{2}$ & $x^{1}$ & $x^{0}$\\
    \hline
    $3$ & $2$ & $0$ & $2$ & $5$
  \end{tabular}

  Do prvního místa opíšeme $a_n$. Čísla v řádku jsou součty koeficientu $a_k$ součinu hodnoty $x$, v níž polynom vyhodnocujeme (v tomto příkladě tedy $3$) s číslem v řádku o jeden sloupec vlevo (tedy pod $a_{k+1}$). Výsledek vyhodnocování je vpravo dole – v našem případě tedy $5$.

  Důsledkem věty o dělení polynomu polynomem je, že zbytek po vydělení f1 polynomem (x-3) je 5 a výsledkem tohoto dělení je polynom stupně 2 s koeficienty danými zbylými třemi čísly ve třetím řádku. Díky tomuto pozorování lze Hornerovo schéma použít i jako efektivní algoritmus k dělení polynomů.
\end{pozn}

\begin{veta}[Hledání racionálních kořenů polynomu s racionálními koeficienty]
  Mějme polynom $P(x) \in \mathbb Q [x]$. Potom najdeme jeho kořeny $\frac{r}{s} \in \mathbb Q$ takto:
  \begin{enumerate}[1.]
    \item Nalezneme všechny celočíselné dělitele $r$ absolutního členu polynomu $a_0$.
    \item Nalezneme všechny přirozené dělitele $s$ vedoucího členu $a_n$.
    \item Utvoříme všechny zlomky tvaru $\frac{r}{s}, (r,s) = 1$.
    \item Hornerovým schématem určíme $P(1)$, případně $P(-1)$ ($1$ a $-1$ také mohou být kořeny).
    \item Vyškrtáme ty zlomky $\frac{r}{s}$, které nesplňují podmínky $(r-s) \mid P(1) \land (r+s) \mid P(-1)$.
    \item U ostatních zlomků vyzkoušíme Hornerovým schématem, zda jsou kořeny daného polynomu.
  \end{enumerate}
\end{veta}

\begin{veta}[Rozklad polynomu v reálném a komplexním oboru]
  Nechť $P(x) \in \mathbb R [x], P(x) = a_n x^n + a_{n-1} x^{n-1} + ... + a_1 x + a_0$, kde $st(P(x)) \geq 1$. Pak $P(x)$ lze vyjádřit jako součin polynomů 1. a 2. stupně a koeficientu $a_n$:
\end{veta}
