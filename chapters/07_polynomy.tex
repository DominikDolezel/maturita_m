\section{Polynomy, kořeny polynomů}
\begin{definition}
  \textbf{Polynomem} nazýváme každý výraz $P(x)$ tvaru
  \[
    P(x)=a_nx^n + a_{n-1}x^{n-1}+\dots + a_2x^2+a_1x+a_0,
  \]
  kde $a_i\in \mathbb R, i\in \{ 0,1,\dots , n \},n\in \mathbb N$. Čísla $a_i$ se nazývají \textbf{koeficienty polynomu}, sčítance $a_ix^i$ \textbf{členy polynomu}. Je-li $a_n\ne 0$, číslo $n$ se nazývá \textbf{stupeň polynomu} a označuje se $n= \text{st}(P(x))$. Je-li $a_i=0$ pro všechna $i \in \{ 0,1,\dots ,n\}$, pak klademe $\text{st}(P(x))=-\infty$ a $P(x)$ se nazývá \textbf{nulovým polynomem}. Označujeme jej $O(x).$ Je-li $a_n=1$, nazývá se $P(x)$ \textbf{normovaným polynomem}.
\end{definition}

\begin{veta}[O dělení polynomů se zbytkem]
  Nechť $A(x), B(x) \in \mathbb R[x],B(x) \ne O(x).$ Pak existuje právě jedna dvojice polynomů $Q(x),R(x)\in \mathbb R[x]$ tak, že
  \[
    A(x)=B(x)\cdot Q(x)+R(x),
  \]
  kde $R(x)=O(x)$ nebo $\text{st}(R(x))<\text{st}(B(x)).$
\end{veta}

\begin{definition}
  Nechť $A(x), B(x) \in \mathbb R[x]$. Řekneme, že polynom $B(x)$ \textbf{dělí} polynom $A(x)$ právě tehdy, když existuje takový polynom $C(x) \in \mathbb R[x]$ tak, že
  \[
    A(x) = B(x)\cdot C(x).
  \]
  Zapisujeme $B(x) \, | \, A(x).$
\end{definition}

\begin{definition}
  Nechť $P(x)\in \mathbb R[x], P(x)=a_nx^n+a_{n-1}x^{n-1}+\dots +a_1x+a_0.$ Nechť $c\in \mathbb R$ je libovolné číslo. \textbf{Hodnotou polynomu} $P(x)$ \textbf{v čísle} (v bodě) $c$ nazýváme reálné číslo $P(c)$
  \[
    P(c) = a_nc^n+a_{n-1}c^{n-1}+\dots+a_1c+a_0.
  \]
  Číslo $c\in \mathbb R$ nazveme \textbf{kořenem polynomu} $P(x) \iff P(c) = 0$.
\end{definition}


\begin{definition}
  Nechť $P(x) \in \mathbb R [x]$ a $c\in \mathbb R$ je jeho kořen. Lineární polynom $x-c$ nazveme \textbf{kořenovým činitelem}.
\end{definition}

\begin{definition}
  Nechť $P(x) \in \mathbb R [x], c \in \mathbb R$ je jeho kořen a $k \in \mathbb N$. Číslo $C \in \mathbb R$ nazýváme \textbf{$k$-násobným kořenem} polynomu $P(x)$ právě tehdy, když platí
  \[
    (x-c)^k \, | \, P(x) \land (x-c)^{k+1} \nmid  P(x).
  \]
\end{definition}

\begin{veta}[Vi\`{e}tovy vztahy]
  Nechť $P(x)=a_nx^n+a_{n-1}x^{n-1}+\dots + a_1x+a_0$ je polynom stupně $n$, který má v množině $\mathbb R$ právě $n$ kořenů $x_1,x_2\dots,x_n$ (každý počítáme tolikrát, jaká je jeho násobnost). Pak platí:
  \begin{align*}
    \sum_{i=1}^n x_i = & -\frac{a_{n-1}}{a_n} \\
    \sum_{i,j=1; i<j}^{n}x_ix_j= & \frac{a_{n-2}}{a_n} \\
    \vdots & \\
    \prod_{i=1}^nx_i=&(-1)^n\frac{a_0}{a_n}
  \end{align*}
\end{veta}

\begin{pozn}
  \textbf{Hornerovo schéma} je numerická metoda pro vyhodnocení funkční hodnoty polynomu $P(x) \in \mathbb R [x]$ v bodě $ x_0 \in \mathbb R$. Příklad:
  Vyhodnoťte $f_{1}(x)=2x^{3}-6x^{2}+2x-1$ v bodě $x=3$.
  Opakovaným vytknutím $x$, může být $f_{1}$ zapsáno jako $x(x(2x-6)+2)-1$. Pro větší přehlednost užijeme k zápisu průběhu výpočtu tzv. syntetický diagram.
  \begin{center}
    \begin{tabular}{ c|c c c c }
        $x_{0}$ & $x^{3}$ & $x^{2}$ & $x^{1}$ & $x^{0}$\\
        \hline
        $3$ & $2$ & $0$ & $2$ & $5$
    \end{tabular}
  \end{center}
  Do prvního místa opíšeme $a_n$. Čísla v řádku jsou součty koeficientu $a_k$ součinu hodnoty $x$, v níž polynom vyhodnocujeme (v tomto příkladě tedy $3$) s číslem v řádku o jeden sloupec vlevo (tedy pod $a_{k+1}$). Výsledek vyhodnocování je vpravo dole – v našem případě tedy $5$.

  Důsledkem věty o dělení polynomu polynomem je, že zbytek po vydělení f1 polynomem (x-3) je 5 a výsledkem tohoto dělení je polynom stupně 2 s koeficienty danými zbylými třemi čísly ve třetím řádku. Díky tomuto pozorování lze Hornerovo schéma použít i jako efektivní algoritmus k dělení polynomů.
\end{pozn}

\begin{veta}
    Nechť je dána algebraická rovnice
    \begin{equation}\label{alg_rce}
        a_nx^n + a_{n-1}x^{n-1}+\dots + a_1x+a_0=0,
    \end{equation}
    $a_n \ne 0$ s celočíselnými koeficienty. Nechť $\frac{r}{s}, r\in \mathbb Z, s \in \mathbb N, D(r,s)=1$ je kořenem této rovnice.
    Pak platí: $r \, | \, a_0 \land s \, |\, a_n.$
\end{veta}

\begin{veta}\label{odecitanivpolynomu}
    Nechť $\frac{r}{s}$ je racionální kořen algebraické rovncie \ref{alg_rce} s celočíselnými koeficienty.
    Nechť $m$ je pevné celé číslo. Pak platí:
    $$(r-ms) \, |\, a(m),$$
    kde $a(m)$ je hodnota polynomu $a(x)$ pro $x=m.$ Dále $r \, | \, a_0$ a $s\, |\, a(-1).$
\end{veta}

\begin{comment}
\begin{veta}[Hledání racionálních kořenů polynomu s racionálními koeficienty]
  Mějme polynom $P(x) \in \mathbb Q [x]$. Potom najdeme jeho kořeny $\frac{r}{s} \in \mathbb Q$ takto:
  \begin{enumerate}[1.]
    \item Nalezneme všechny celočíselné dělitele $r$ absolutního členu polynomu $a_0$.
    \item Nalezneme všechny přirozené dělitele $s$ vedoucího členu $a_n$.
    \item Utvoříme všechny zlomky tvaru $\frac{r}{s}, (r,s) = 1$.
    \item Hornerovým schématem určíme $P(1)$, případně $P(-1)$ ($1$ a $-1$ také mohou být kořeny).
    \item Vyškrtáme ty zlomky $\frac{r}{s}$, které nesplňují podmínky $(r-s) \mid P(1) \land (r+s) \mid P(-1)$.
    \item U ostatních zlomků vyzkoušíme Hornerovým schématem, zda jsou kořeny daného polynomu.
  \end{enumerate}
\end{veta}
\end{comment}

\begin{priklad}
Najděte racionální kořeny polynomu $P(x)=\frac{3}{5}x^4+x^3+\frac{1}{5}x^2+x-\frac{2}{5}.$
\end{priklad}

\begin{reseni}
Vynásobme polynom 5, abychom měli všechny koeficienty celé.
$$P(x) = 3x^4+5x^3+x^2+5x-2.$$
Pak $r \, | \, a_0 \implies r\, | \, -2 \implies r \in \left \{ \pm 2, \pm 1 \right \} $
a $s \, | \, a_n\implies s \, | \, 3 \implies s \in \left \{ 1, 3 \right \} $.
Sestavíme všechny možnosti a vyzkoušíme. S výhodou taky můžeme využít věty
\ref{odecitanivpolynomu}.
\end{reseni}


\begin{veta}[Rozklad polynomu v reálném a komplexním oboru]
  Nechť $P(x) \in \mathbb R [x]$,
  $$P(x) = a_n x^n + a_{n-1} x^{n-1} + \dots + a_1 x + a_0,$$
  kde $\textrm{st}(P(x)) \geq 1$. Pak $P(x)$ lze v $\mathbb R$ vyjádřit jako součin polynomů 1. a 2. stupně a koeficientu $a_n$ následovně:
  \begin{align*}
    P(x) & = a_n(x-c_1)^{k_1}(x-c_2)^{k_2} \dots (x-c_k)^{k_k}\\
    & \cdot (x^2+p_1 x + q_1)^{r_1}(x^2+p_2 x + q_2)^{r_2} \dots (x^2+p_r x + q_r)^{r_r},
  \end{align*}


  kde $c_1,c_2, \dots, c_k$ jsou všechny jeho reálné různé kořeny s násobnostmi $l_1, k_2, \dots, k_k \in \mathbb N$;
  $p_1, p_2, \dots, p_r$ a $q_1, q_2, \dots, q_r$ jsou reálná čísla, $r_1, r_2, \dots, r_n \in \mathbb N$.
  Polynomy
  $$(x^2+p_1 x + q_1)^{r_1}(x^2+p_2 x + q_2)^{r_2} \dots (x^2+p_r x + q_r)^{r_r}$$
  jsou kvadratické polynomy se záporným diskriminantem. Uvedený rozklad je až na pořadí činitelů
  jednoznačný a platí
  $$\textrm{st}(P(x)) = k_1 + k_2 + \dots + k_k + 2(r_1 + r_2 + \dots + r_r).$$

  Jestliže $a_1 \pm ib_1, a_2 \pm ib_2, \dots, a_s \pm ib_s$ jsou všechny navzájem různé
  dvojice komplexně sdružených kořenů s násobností $r_1, r_2, \dots, r_s$, $P(x)$ můžeme v $\mathbb C$ psát ve tvaru:
  \begin{align*}
    P(x) & = a_n(x-c_1)^{k_1}(x-c_2)^{k_2} \dots (x-c_k)^{k_k}\\
    & \cdot  \left [(x-a_1)^2+b_1^2\right ]^{r_1}\left [(x-a_2)^2+b_2^2\right]^{r_2} \dots \left [(x-a_s)^2+b_s^2\right ]^{r_s}
  \end{align*}
\end{veta}

\begin{priklad}
    Určete, pro která $x \in \mathbb R$ platí:
    $$P(x)=(x-1)^2(x+5)^3x^4(x^2-5x+6)(x^2+1)(x-5)^5 \geq 0.$$
\end{priklad}

\begin{reseni}
Rozložíme v reálném oboru. Členy se záporným diskriminantem nemají reálný kořen,
neuvažujeme je tedy. V sudých mocninách se nemění znaménko.
\end{reseni}

\begin{definition}
  Nechť $A(x), B(x) \in \mathbb R [x]$. Polynom $C(x) \in \mathbb R [x]$ se nazývá \textbf{společný dělitel polynomů} $A(x), B(x)$
  právě tehdy, když platí: $C(x) \mid A(x) \land C(x) \mid B(x)$.
  Polynom $D(x) \in \mathbb R [x]$ se nazývá \textbf{největší společný dělitel polynomů} $A(x), B(x)$, který označujeme $D(A(x), B(x))$ nebo $(A(x), B(x))$, právě tehdy, když platí:
  \begin{enumerate}[i.]
    \item $D(x) \mid A(x) \land D(x) \mid B(x)$ a
    \item $\forall C(x) \in \mathbb R [x]: C(x) \mid A(x) \land C(x) \mid B(x) \implies C(x) \mid D(x).$
  \end{enumerate}
\end{definition}

\begin{pozn}
  Největší společný dělitel hledáme \textbf{Euklidovým algoritmem}: Nechť $A(x), B(x) \in \mathbb R [x]$.
  \begin{enumerate}[a.]
    \item $A(x) = O(x) \implies D(A(x), B(x)) = B(x)$
    \item $\text{st}(A(x)) \geq \text{st}(B(x)) \geq 0$
  \end{enumerate}
  Proveďme následující posloupnost dělení se zbytkem. Toto dělení ukončíme, až bude zbytek nulový polynom.
  Vzhledem k nerovnosti na pravé straně tento nulový zbytek existuje.

  \begin{minipage}{0.6\textwidth}
  \begin{align*}
    A(x) & = B(x) \cdot Q_1(x) + R_1(x),  \\
    B(x) & = R_1(x) \cdot Q_2(x) + R_2(x),  \\
    R_1(x) & = R_2(x) \cdot Q_3(x) + R_3(x), \\
    \dots & = \dots \\
    R_{n-2}(x) & = R_{n-1}(x) \cdot Q_n(x) + R_n(x),  \\
    R_{n-1}(x) & = R_{n}(x) \cdot Q_{n+1}(x),
  \end{align*}
  \end{minipage}
  \hfill
  \begin{minipage}{0.38\textwidth}
  \begin{align*}
   \text{st}(R_1(x)) & < \text{st}(B(x)) \\
   \text{st}(R_2(x)) & < \text{st}(R_1(x)) \\
 \text{st}(R_3(x)) & < \text{st}(R_2(x)) \\
    \dots & < \dots \\
 \text{st}(R_n(x)) & < \text{st}(R_{n-1}(x)) \\
   R_{n}(x) & < O(x).
  \end{align*}
  \end{minipage}

  Potom $(A(x), B(x)) = R_n(x)$, či libovolný násobek $R_n(x)$. \textbf{Normovaný největší společný dělitel polynomů}
  ale existuje právě jeden, ten se označuje $(A(x), B(x))$.
\end{pozn}

\begin{priklad}
    Určete největšího společného dělitele polynomů
    \begin{equation*}
        x^3+8x^2+17x+10,\\
        x^3+5x^2+2x+10.
    \end{equation*}
\end{priklad}

\begin{reseni}
Euklidovým algoritmem. Dělíme tak dlouho, dokud nevyjde nulový zbytek.
Mezi jednotlivými děleními můžeme polynomy vynásobit libovolnou konstantou (protože
to nezmění kořeny polynomu).
\end{reseni}

\begin{definition}
  Nechť polynomy $A(x), B(x)$. Pokud $(A(x), B(x)) = 1$, řekneme, že polynomy $A(x), B(x)$ jsou \textbf{nesoudělné}. V opačném případě jsou \textbf{soudělné}.
\end{definition}

\begin{definition}
    Nechť $P(x) \in \mathbb R[x], P(x)=a_nx^n+a_{n-1}x^{n-1}+\dots+a_1x+a_0.$ \textbf{Derivací polynomu} $P(x)$ rozumíme polynom $P^\prime(x)$ definovaný takto:
    $$
        P^\prime(x)=\begin{cases}
        0, &\text{ je-li st} (P(x)) \leq 0,\\
        a_n n x^{n-1} + a_{n-1}(n-1)x^{n-2} + \dots + a_1, & \text{ je-li st} (P(x)) \geq 1.
        \end{cases}
    $$
\end{definition}


\begin{veta}
    Nechť $P(x) \in R[x], c \in \mathbb R$ je jeho $k$-násobný kořen, $k\in \mathbb N.$ Pak platí: $c$ je
    $k$-násobný kořen $P(x) \iff P(c)=P^\prime(c)=P^{\prime \prime}(c)=\dots= P^{k-1}(c)=0 \land P^{k}(c)\ne 0$.
\end{veta}


\begin{priklad}
Je dán polynom $x^3+3x-2$, který má dvojnásobný kořen.
Určete všechny jeho kořeny.
\end{priklad}

\begin{reseni}
Určíme první derivaci. Dvojnásobný kořen je kořenem první derivace, ale ne naopak.
Vyzkoušíme tedy v Hornerovém schématu.
\end{reseni}
