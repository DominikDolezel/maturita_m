\section{Komplexní čísla}
\begin{definition}\label{kompl_c_def}
Množinu označenou $\mathbb C = \left \{ (a,b)\ a,b, \in \mathbb R \right \} $
spolu s operacemi $+$ a $\cdot$ definovanými následovně:
\begin{enumerate}[$i.$]
\item $(a,b) + (c,d) = (a+c, b+d),$
\item $(a,b) \cdot (c,d) = (ac-bd, ad+bc)$,
\end{enumerate}
nazýváme \textbf{komplexními čísly}.
\end{definition}

\begin{veta}
    Pro všechna $x=(a,b),y=(c,d) \in \mathbb C$ platí
    \begin{enumerate}[$i.$]
    \item $x-y=(a-c, b-d)$,
   	\item $y\ne (0,0): \frac{x}{y}=\left ( \frac{ac+bd}{b^2+d^2}, \frac{bc-ad}{b^2+d^2} \right ) .$
    \end{enumerate}
\end{veta}

\begin{veta}
    Množina $\mathbb C$ spolu s operacemi $+,\cdot$ tvoří komutativní těleso.
\end{veta}

\begin{pozn}
    Množina $M$ spolu s operacemi $+,\cdot$ tvoří (komutativní) těleso, pokud
    \begin{enumerate}[$i.$]
    \item $(M,+)$ je komutativní grupa,
   	\item $(M-\left \{ 0 \right \} ,\cdot)$ je komutativní grupa,
   	\item platí distributivní zákony.
    \end{enumerate}
    Tělesem je třeba $\mathbb R, \mathbb Q;$ $\mathbb Z$ však tělesem není (neexistují
    inverzní prvky vzhledem k násobení -- 5 krát co je jedna?).
\end{pozn}

\begin{veta}
    Množiny $\mathbb R$ a $(x,0)\in \mathbb C$ jsou izomorfní.
\end{veta}

\begin{proof}
    Pro všechna $x,y \in \mathbb R$ platí
    \begin{enumerate}[$i.$]
    \item $\varphi(x)+\varphi(y) = \varphi(x+y),$
   	\item $\varphi(x)\cdot\varphi(y) = \varphi(xy),$
    \end{enumerate}
    kde $\varphi$ je zobrazení z $\mathbb R$ do $\mathbb C$: $\varphi(a)=(a,0).$
\end{proof}

\begin{pozn}
    Dále budeme zapisovat komplexní číslo $(a,b)$ jako $a+bi.$
\end{pozn}

\begin{definition}
    Tvar komplexního čísla $a+bi$ nazýváme \textbf{algebraický}. Číslo $a$ je jeho
   \textbf{reálná část}, $b$ \textbf{imaginární část}.\\
  Číslo $(0,1)$ označme $i$ a nazývejme \textbf{imaginární jednotka}. \\
  Čísla $a+bi,$ kde $b\ne 0$ nazývejme \textbf{imaginární}. Pokud taky $a=0$,
  nazýváme je \textbf{ryze imaginární}.
\end{definition}

\begin{pozn}
    Komplexní číslo $x=(a,b)$ lze chápat jako bod v rovině se souřadnicemi $[a,b].$
\end{pozn}

\begin{definition}
    Nechť je dáno komplexní číslo $x= a+bi.$ Číslo $\overline{x}=a-bi$ nazýváme
   \textbf{komplexně sdruženým} k číslu $x$. Číslo $|x|=\sqrt{a^2+b^2}$ je
   \textbf{absolutní hodnota} čísla $x$. Jestliže $|x|=1,$ číslu $x$ říkáme \textbf{komplexní jednotka}.
\end{definition}

\begin{definition}
\textbf{Argumentem} komplexního čísla nazýváme orientovaný úhel $\varphi$, který svírá kladná
poloosa reálné osy s polohovým vektorem daného bodu představujícím dané komplexní číslo.
\end{definition}

\begin{veta}
    Každé komplexní číslo $x=a+bi, x\ne 0,$ lze zapsat ve tvaru
    $$x=|x|(\cos \varphi + i \sin \varphi),$$
    kde $\varphi$ je argumentem čísla $x$. Dále platí:
    \begin{align*}
        \cos \varphi = \frac{a}{|x|}, && \sin \varphi = \frac{b}{|x|}.
    \end{align*}
\end{veta}

\begin{definition}
    Tvar komplexního čísla $x= |x|(\cos \varphi + i\sin\varphi)$ nazýváme \textbf{goniometrický}.
\end{definition}

\begin{pozn}
    Každé komplexní číslo lze zapsat uspořádanou dvojicí $x=(|x|, \varphi).$
    Tato čísla jsou \textbf{polárními souřadnicemi} komplexního čísla $x$.
\end{pozn}

\begin{veta}[Moivreova věta]
    Nechť je dáno komplexní číslo $x=|x|(\cos \varphi + i \sin \varphi)$ a $n\in \mathbb N$
    je nenulové číslo. Pak platí
    $$x^n = |x|^n (\cos n\varphi + i \sin n \varphi).$$
\end{veta}

\begin{pozn}
    Gaussova rovina je množina bodů $[a,b]; a,b \in \mathbb R$.
    Každému prvku této množiny odpovídá komplexní číslo $(a,b).$ Osa $x$ odpovídá
    reálné části tohoto komplexního čísla, osa $y$ imaginární části.
\end{pozn}

\begin{priklad}
Jsou dána komplexní čísla $a=2+i,b=1-2i$. Zobrazte v Gaussově rovině jejich součin a podíl.
\end{priklad}

\begin{reseni}
Součin získáme vynásobením velikostí (stejnolehlostí) a sečtením úhlů. Podíl obdobně.
\end{reseni}

\begin{definition}
(\textbf{Algebraická}) \textbf{rovnice v komplexní neznámé} $n$-tého stupně s jednou neznámou $x\in \mathbb C$ je každá
rovnice tvaru $P(x)=0,$ kde $P(x)$ je polynom $n$-tého stupně s reálnými (resp.
komplexními) koeficienty.
\end{definition}

\begin{veta}[Základní věta algebry]
    Každý polynom $n$-tého stupně s komplexními koeficienty má v množině $\mathbb C$
    právě $n$ kořenů, počítáme-li každý kořen tolikrát, kolik je jeho násobnost.
\end{veta}

\begin{veta}
    Má-li algebraická rovnice s reálnými koeficienty imaginární kořen
    $x = a+bi$, má taky kořen $\overline{x} = a-bi.$
\end{veta}

\begin{definition}
\textbf{Binomickou rovnicí} s neznámou $x\in \mathbb C$ nazýváme každou rovnici
tvaru $x^n=a$, kde $a\in \mathbb C,n\in \mathbb N, n\geq 2.$ Každý komplexní
kořen binomické rovnice nazýváme \textbf{komplexní $n$-tou odmocninou} z čísla $a.$
\end{definition}

\begin{priklad}
Vypočtěte $z=\sqrt{1+\sqrt{3}i }. $
\end{priklad}

\begin{reseni}
Umocněním získáme $z^2=1+\sqrt{3}i. $ Platí $|z^2|=2,$ takže $z^2=2(\cos \frac{\pi}{3}+i\sin \frac{\pi}{3})=|z|^2(\cos 2x+i\sin 2x)$.
Dále řešíme jako goniometrickou rovnici.
\end{reseni}

\begin{priklad}
Řešte v $\mathbb C$: $z^3=i.$
\end{priklad}

\begin{reseni}
Vyjádříme v goniometrickém tvaru a dále řešíme pomocí Moivreovy věty.
\end{reseni}

\begin{definition}
\textbf{Kvadratickou rovnicí s} (komplexní) \textbf{neznámou} $x\in \mathbb C$ a
\textbf{reálnými} (resp. \textbf{komplexními}) \textbf{koeficienty} $a,b,c$
nazýváme každou rovnici tvaru
$ax^2+bx+c=0,a,b,c\in \mathbb R,a\ne0$ (resp. $a,b,c \in \mathbb C,a\ne 0$).
\end{definition}

\begin{priklad}
V $\mathbb C$ řešte $z^2+3z+10i=0.$
\end{priklad}

\begin{reseni}
\begin{enumerate}[1.]
\item způsob: Doplněním na čtverec dostáváme $\left ( z+\frac{3}{2} \right )^2=\frac{9}{4}-10i. $
Substituujeme a dále řešíme jako binomickou rovnici.
\item způsob: Pomocí diskriminantu.
\item způsob: Vyjádřením neznámé v algebraickém tvaru ($z=a+bi$) a roznásobením.
\end{enumerate}
\end{reseni}

\begin{definition}
\textbf{Trinomická rovnice} s neznámou $x\in \mathbb C$ a reálnými koeficienty $a,b,c \in \mathbb R$
nazýváme rovnici tvaru $ax^p + bx^q + c = 0,$ kde $p,q\in \mathbb N, p>q, a,b\ne 0.$
\end{definition}

\begin{priklad}
Užitím Moivreovy věty odvoďte vzorce pro $\sin 3x, \cos 3x.$
\end{priklad}

\begin{reseni}
Platí $(\cos x + i\sin x)^3=\cos 3x+i\sin 3x.$ Z binomické věty máme
$(\cos x+i\sin x)^3=\cos^3 x + 3i\cos^2 x\sin x - 3\cos x\sin^2 x-i\sin^3 x.$
Metodou porovnání koeficientů u $i$ vzorce dostáváme.
\end{reseni}
