\section{Analytická geometrie lineárních útvarů -- polohové vlastnosti}
\begin{definition}
    Nechť je dána přímka $p$ a vektor $\vec u\ne \vec o$
    takový, že existuje orientovaná úsečka $\overrightarrow{AB}$ taková, že
     $A\in p, B\in p.$ Pak $\vec u$
    je \textbf{směrový vektor} přímky $p.$
\end{definition}

\begin{definition}
    Rovnici $X = A+t\vec u, t\in \mathbb R, \vec u \ne  \vec o,$ nazveme
    \textbf{parametrickou rovnicí přímky} v $\mathbb E_2,$ resp. $\mathbb E_3.$
\end{definition}

\begin{priklad}
Napište rovnici přímky $p(A,\vec u),$ je-li $A[1,-1,2], \vec u = (0,1,2).$
\end{priklad}

\begin{veta}[Určení vzájemné polohy dvou přímek v $\mathbb E_2$]
    Jsou dány přímky $p(A,\vec u), q(B,\vec v).$ Pak
    \begin{enumerate}[$i.$]
    \item Jestliže má soustava rovnic
    \begin{align*}
        a_1+tu_1 &= b_1 + rv_1 \\
        a_2+tu_2 &= b_2 + rv_2
    \end{align*}
    \begin{enumerate}[$a.$]
    \item 0 řešení, jsou přímky $p,q$ rovnoběžné různé.
   	\item 1 řešení, jsou přímky $p,q$ různoběžné.
   	\item nekonečně mnoho řešení, přímky $p,q$ splývají.
    \end{enumerate}
   	\item Jestliže
    \begin{enumerate}[$a.$]
    \item jsou vektory $\vec u, \vec v$ lineárně nezávislé, jsou přímky $p,q$ různoběžné.
   	\item jsou vektory $\vec u, \vec v$ lineárně závislé a
   	\begin{itemize}
    \item $B\in p,$ přímky $p,q$ splývají.
   	\item $B\notin p,$ přímky $p,q$ jsou rovnoběžné různé.
    \end{itemize}
    \end{enumerate}
    \end{enumerate}
\end{veta}
\begin{priklad}
Určete vzájemnou polohu dvou přímek $AB, CD$, je-li $A[3,2], B[4,-1],$\linebreak $C[-4,5],D[-1,-2]$.
\end{priklad}

\begin{reseni}
Zjišťujeme, zda jsou směrové vektory přímek lineárně závislé.
\end{reseni}

\begin{veta}[Určení vzájemné polohy dvou přímek v $\mathbb E_3$]
    Jsou dány přímky $p(A,\vec u), q(B,\vec v).$ Pak
    \begin{enumerate}[$i.$]
    \item Jestliže má soustava rovnic
    \begin{align*}
        a_1+tu_1 &= b_1 + rv_1 \\
        a_2+tu_2 &= b_2 + rv_2 \\
        a_3+tu_3 &= b_3 + rv_3
    \end{align*}
    \begin{enumerate}[$a.$]
    \item 0 řešení a vektory $\vec u, \vec v$ jsou
    \begin{itemize}
    \item lineárně nezávislé, jsou přímky $p,q$ mimoběžné.
   	\item lineárně závislé, jsou přímky $p,q$ rovnoběžné různé.
    \end{itemize}
   	\item 1 řešení, jsou přímky $p,q$ různoběžné.
   	\item nekonečně mnoho řešení, přímky $p,q$ splývají.
    \end{enumerate}
   	\item Jestliže
    \begin{enumerate}[$a.$]
    \item jsou vektory $\vec u, \vec v$ lineárně nezávislé a
    \begin{itemize}
    \item $p\cap q \ne \emptyset,$ jsou přímky $p,q$ různoběžné.
   	\item $p\cap q = \emptyset,$ jsou přímky $p,q$ mimoběžné.
    \end{itemize}
   	\item jsou vektory $\vec u, \vec v$ lineárně závislé a
   	\begin{itemize}
    \item $B\in p,$ přímky $p,q$ splývají.
   	\item $B\notin p,$ přímky $p,q$ jsou rovnoběžné různé.
    \end{itemize}
    \end{enumerate}
    \end{enumerate}
\end{veta}

\begin{priklad}
Rozhodněte o vzájemné poloze přímek $p=\left \{ [1-4t,2+4t,3+t], t \in \mathbb R \right \},q=\left \{ [-4,8r,-5-8r,-2r],r \in \mathbb R \right \} . $
\end{priklad}

\begin{reseni}
Zjišťujeme, zda jsou směrové vektory přímek lineárně závislé.
\end{reseni}

\begin{veta}[Vzájemná poloha dvou přímek]
    Nechť $p(A,\vec u), q(B, \vec v)$ jsou dvě přímky. Pak platí
    \begin{enumerate}[$i.$]
    \item $p\parallel q \land p=q \iff \dim(\vec u, \vec v, \overrightarrow{AB})=1,$
   	\item $p\parallel q \land p\ne q \iff \dim(\vec u, \vec v, \overrightarrow{AB})=2 \land \dim ( \vec u, \vec v) = 1$,
    \item $p, q$ jsou různoběžné $ \iff \dim(\vec u, \vec v, \overrightarrow{AB})=2 \land \dim ( \vec u, \vec v) = 2$,
    \item $p, q$ jsou mimoběžné $\iff \dim(\vec u, \vec v, \overrightarrow{AB})=3.$
    \end{enumerate}
\end{veta}

\begin{priklad}
Rozhodněte o vzájemné poloze dvou přímek $p=\left \{ [t,-1+2t,-2+2t], t \in \mathbb R \right \}, q=\left \{ [1+t,1,r], r \in \mathbb R \right \}.  $
\end{priklad}

\begin{reseni}
Je $\vec u = (1,2,2), \vec v = (1,0,1), \overrightarrow{AB}=(1,2,2).$ Hledáme tedy
dimenzi prostorů $\left < \vec u, \vec v \right >, \left < \vec u, \vec v, \overrightarrow{AB} \right >.  $
\end{reseni}

\begin{definition}
    $ax+by+c=0, (a,b)\ne(0,0)$ je \textbf{obecná rovnice přímky} v $\mathbb E_2$.
\end{definition}

\begin{definition}
    Vektor kolmý ke směrovému vektoru přímky se nazývá \textbf{normálový}.
\end{definition}

\begin{veta}
    Normálový vektor přímky o rovnici $ax+by+c=0$ je $\vec n=(a,b)$.
\end{veta}

\begin{priklad}
Napište obecnou rovnici přímky $p=\left \{ [1+t,2-t], t \in \mathbb R \right \} .$
\end{priklad}

\begin{reseni}
Buď vyloučením parametru: je-li $x=1+t, y= 2-t,$ pak $x+y=3,$ takže $x+y-3=0$, nebo
přes normálový vektor: $\vec n(1,1),$ takže dosadíme do rovnice a dopočteme $c$ tak,
aby jí lib. bod přímky $p$ vyhovoval.
\end{reseni}

\begin{priklad}
Napište parametrickou rovnici přímky $x-2y+1=0.$
\end{priklad}

\begin{reseni}
Buď substitucí, např. $y=t$, pak $x-2t+1=0,$ takže $x=-1+2t$, tedy $p=\left \{ [-1+2t,t], t \in \mathbb R \right \}$,
nebo přes normálový vektor.
\end{reseni}

\begin{definition}
    Rovnice $y=kx+q$ se nazývá \textbf{směrnicový tvar rovnice přímky} v $\mathbb E_2$,
    $k$~je \textbf{směrnice} přímky.
\end{definition}

\begin{definition}
Rovnice $\frac{x}{q}+\frac{y}{p}=1$ se nazývá \textbf{úsekový tvar rovnice přímky} v $\mathbb E_2$.
\end{definition}

\begin{definition}
Nechť $\rho$ je rovina, $\vec u, \vec v$ lineárně nezávislé vektory a $\overrightarrow{AB},
\overrightarrow{AC}, A,B,C\in \rho$ jejich umístění. Pak $\vec u, \vec v$ je
\textbf{zaměření} roviny $\rho.$
\end{definition}

\begin{definition}
    Rovnice $X=A+r\vec u+s\vec v$ se nazývá \textbf{parametrická rovnice roviny}.
\end{definition}

\begin{definition}
    $ax+by+cz+d=0, (a,b,c)\ne(0,0,0)$ se nazývá \textbf{obecná rovnice roviny}.
\end{definition}

\begin{definition}
    Vektor kolmý k oběma směrovým vektorům roviny se nazývá \textbf{normálový}.
\end{definition}

\begin{priklad}
Napište obecnou rovnici roviny $\rho = \overleftrightarrow{ABC}, A[1,1,1], B[2,0,-1], C[1,0,0].$
\end{priklad}

\begin{reseni}
Opět buď vyloučením parametrů, nebo přes normálový vektor: spočítáme dva libovolné vektory
mezi body $A,B,C$ a $\vec n = \vec u \times \vec v$, posun dopočteme.
\end{reseni}

\begin{priklad}
Napište parametrickou rovnici roviny $\rho: x-y+z-1=0.$
\end{priklad}

\begin{reseni}
Pokud $y=r, z=s,$ pak $x-r+s-1=0,$ takže $x=1+r-s$, tedy $\rho = \left \{ [1+r-s,r,s],r,s\in \mathbb R \right \}. $
\end{reseni}

\begin{veta}[Vzájemná poloha dvou rovin daných obecnými rovnicemi]
    Nechť $\rho: ax+by+cz\linebreak+ d=0, \sigma: ex+fy+gz+h=0$ jsou roviny. Pak
    \begin{enumerate}[$i.$]
    \item $\rho = \sigma \iff (a,b,c,d) = k(e,f,g,h), k\in \mathbb R,$
   	\item $\rho \parallel \sigma \land \rho \ne \sigma \iff (a,b,c) = k(e,f,g), k\in \mathbb R
    \land d\ne kh,$
   	\item $\rho \nparallel \sigma \iff \forall k \in \mathbb R: (a,b,c) \ne k(e,f,g)$.
    \end{enumerate}
\end{veta}

\begin{priklad}
Určete vzájemnou polohu roviny $\rho:2x+3y+4z+5=0$, $\sigma: x-y-z+1=0$.
\end{priklad}

\begin{reseni}
Normálové vektory jsou lineárně nezávislé. Průsečnice je řešení soustavy rovnic
\begin{align*}
    2x+3y+4z+5& =0,\\
    x-y-z+1&=0.
\end{align*}
\end{reseni}

\begin{veta}[Vzájemná poloha dvou rovin daných parametrickými rovnicemi]
    Nechť $\rho(A,\vec u, \vec v), \linebreak\sigma(B,\vec k, \vec l)$ jsou roviny. Pak
    \begin{enumerate}[$i.$]
    \item $\rho = \sigma \iff \dim (\vec u, \vec v, \vec k, \vec l) = 2 \land \dim (\vec u, \vec v, \vec k, \vec l, \overrightarrow{AB})=2,$
   	\item $\rho \parallel \sigma \land \rho \ne \sigma \iff \dim (\vec u, \vec v, \vec k, \vec l) = 2 \land \dim (\vec u, \vec v, \vec k, \vec l, \overrightarrow{AB})=3,$
   	\item $\rho \nparallel \sigma \iff \dim (\vec u, \vec v, \vec k, \vec l) = 3$.
    \end{enumerate}
\end{veta}

\begin{priklad}
Určete vzájemnou polohu rovin $\rho = \{ [4+t_1+2t_2,5+2t_1,3+2t_1+2t_2],$ $t_1,t_2\in \mathbb R \}$,
$\sigma = \left \{ [1+2r_1+r_2,-2-2r_1-2r_2,1+r_1],r_1,r_2\in \mathbb R \right \}. $
\end{priklad}

\begin{reseni}
Hledáme dimenzi prostorů $\left < \vec u, \vec v, \vec k, \vec l \right >, \left < \vec u, \vec v, \vec k, \vec l, \overrightarrow{AB} \right > .$
\end{reseni}

\begin{veta}[Vzájemná poloha přímky a roviny dané parametrickou rovnicí]
    Nechť $p(A,\vec u),\linebreak \rho(B,\vec v, \vec w)$ jsou přímka a rovina. Pak
    \begin{enumerate}[$i.$]
    \item $p\subseteq \rho \iff \dim (\vec v, \vec w, \vec u) = 2 \land \dim (\vec v, \vec w, \vec u, \vec l, \overrightarrow{AB})=2,$
   	\item $p \parallel \rho \land p \not\subset \rho \iff \dim (\vec v, \vec w, \vec u) = 2 \land \dim (\vec v, \vec w, \vec u, \vec l, \overrightarrow{AB})=3,$
   	\item $p \nparallel \rho \iff \dim (\vec v, \vec w, \vec u) = 3$.
    \end{enumerate}
\end{veta}

\begin{priklad}
Určete vzájemnou polohu přímky $p=\left \{ [3+t,1+2t,2-t],t \in \mathbb R \right \} $
a roviny $\rho:\left \{ [1-3r+s,2r-s,1+4r-s],r,s\in \mathbb R \right \} .$
\end{priklad}

\begin{reseni}
Hledáme dimenzi prostorů $\left < \vec u, \vec v, \vec w \right >, \left < \vec u, \vec v, \vec w,  \overrightarrow{AB} \right > .$
\end{reseni}

\begin{veta}[Vzájemná poloha přímky a roviny dané obecnou rovnicí]
    Nechť $p(A,\vec u), \rho: ax+by+cz+d=0$ jsou přímka a rovina. Pak
    \begin{enumerate}[$i.$]
    \item $p\subseteq \rho \iff \vec u \cdot \vec n = 0\land A\in p,$
   	\item $p \parallel \rho \land p \not\subset \rho \iff \vec u\cdot \vec n = 0\land A\notin p,$
   	\item $p \nparallel \rho \iff \vec u\cdot \vec n\ne 0$.
    \end{enumerate}
\end{veta}

\begin{priklad}
Určete vzájemnou polohu přímky $p=\left \{ [1-t,1+3t,-3],t \in \mathbb R \right \} $ a
roviny $\rho:3x+y+5z+7=0.$
\end{priklad}

\begin{reseni}
Počítáme skalární součin směrového vektoru přímky a normálového vektoru roviny.
\end{reseni}

\begin{definition}
Nechť $p,q$ jsou dvě mimoběžné přímky. Přímka $r$, která je různoběžná s~přímkami
$p,q$ se nazývá \textbf{příčkou mimoběžek} $p,q$. Pokud je navíc $r$ kolmá na $p,q$,
nazývá se \textbf{osou mimoběžek} $p,q$.
\end{definition}

\begin{priklad}\label{prmimob}
Jsou dány mimoběžky  $p(A,\vec u)$ $q(B,\vec v)$ a dále vektor $\vec w$, přičemž
$A[1,-2,5],$ $B[-1,1,-5],$ $\vec u = (1,3,-1),$ $\vec v=(1,1,2),$ $\vec w = (1,1,4).$
Nalezněte příčku $p,q$, která je rovnoběžná s vektorem $\vec w.$
\end{priklad}

\begin{reseni}
Nechť $P=A+k\vec u, Q=B+l\vec v.$ Pak $\overrightarrow{PQ}=Q-P=B+l\vec v - A - k\vec u=x\vec w,$
úpravou dostaneme $\overrightarrow{AB}=-l\vec v+k\vec u+x\vec w.$ Máme tedy
sosutavu tří rovnic o třech neznámých.
\end{reseni}

\begin{priklad}
Jsou dány mimoběžky $p(A,\vec u), q(B,\vec v)$ a bod $M$, přičemž
$A[1,5,2],B[0,-1,1],$ $\vec u=(1,2,1),$ $\vec v = (3,1,0),$ $M[0,1,-5].$  Nalezněte
příčku $p,q$, která prochází bodem $M$.
\end{priklad}

\begin{reseni}
Obdobně jako v příkladu \ref{prmimob} dostaneme $\overrightarrow{MA}=x\overrightarrow{MB}+m\vec v-k\vec u$,
tedy $A-M=x(B-M)+m\vec v - k\vec u$. Máme tedy
sosutavu tří rovnic o třech neznámých.
\end{reseni}
