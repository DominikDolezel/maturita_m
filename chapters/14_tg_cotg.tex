\section{Funkce tangens, kotangens, arkustangens, arkuskotangens}
\begin{definition}[Tangens a kotangens]
  Funkcí \textbf{tangens} (resp. \textbf{kotangens}) nazýváme funkci danou vztahem:
  \begin{align*}
    \tg x = \frac{\sin x}{\cos x}, & & \cotg x = \frac{\cos x}{\sin x}.
  \end{align*}
\end{definition}

\begin{veta}
    Vlastnosti funkce tangens:\\
    Nechť $k\in \mathbb Z.$
    \begin{enumerate}[$i.$]
        \item $D(f)= \mathbb R-\left \{ (2k+1)\frac{\pi}{2}, k\in \mathbb Z \right \} $, $H(f)= \mathbb R $.
       	\item Je lichá.
        \item Je rostoucí v každém z intervalů $\left ( -\frac{\pi}{2}+k\pi, \frac{\pi}{2}+k\pi \right ) $.
        \item Není omezená.
        \item Nemá extrémy.
        \item Je periodická s nejmenší periodou $\pi.$
    \end{enumerate}
    Vlastnoti funkce kotangens:
    \begin{enumerate}[$i.$]
        \item $D(f)= \mathbb R-\left \{ k\pi, k\in \mathbb Z \right \} $, $H(f)= \mathbb R $.
       	\item Je lichá.
        \item Je klesající v každém z intervalů $\left ( k\pi, (k+1)\pi \right ) $.
        \item Není omezená.
        \item Nemá extrémy.
        \item Je periodická s nejmenší periodou $\pi.$
    \end{enumerate}
\end{veta}

\begin{priklad}
Vypočtěte $\tg \left ( -\frac{19}{6}\pi \right ) $.
\end{priklad}

\begin{definition}[Arkustangens]
  Funkce \textbf{arkustangens}, označená $f^{-1}: y=\arctg x$, se nazývá funkce inverzní k funkci $f: y=\tg x$, kde $D(f)=\left ( -\frac{\pi}{2}, \frac{\pi}{2} \right )$.
\end{definition}

\begin{pozn}
Funkce arkustangens má následující vlastnosti:
\begin{enumerate}
\item $D(f)^{-1} = \mathbb{R}$, $H(f)^{-1} = \left ( -\frac{\pi}{2}, \frac{\pi}{2} \right )$,
\item $f^{-1}$ je rostoucí v celém $D(f^{-1})$.
\end{enumerate}
\end{pozn}

\begin{definition}[Arkuskotangens]
  Funkce \textbf{arkuskotangens}, označená $g^{-1}: y=\arccotg x$, se nazývá funkce inverzní k funkci $g: y=\cotg x$, kde $D(g)=\left ( 0, \pi \right )$.
\end{definition}

\begin{pozn}
Funkce arkuskotangens má následující vlastnosti:
\begin{enumerate}
\item $D(g)^{-1} = \mathbb{R}$, $H(g)^{-1} = \left ( 0, \pi \right )$,
\item $g^{-1}$ je klesající v celém $D(g^{-1})$.
\end{enumerate}
\end{pozn}

\begin{veta}
    V pravoúhlém trojúhelníku $ABC$ s přeponou $AB$ platí:
    \begin{align*}
        \tg \alpha = \frac{ |BC| }{ |AC| }, & & \cotg \alpha = \frac{|AB|}{|BC|}.
    \end{align*}
\end{veta}

\begin{pozn}
    Grafy jednotlivých funkcí a tabulka se základními hodnotami jsou k nalezení
    v příloze \ref{appa}.
\end{pozn}
