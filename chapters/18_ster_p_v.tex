\section{Stereometrie -- polohové vlastnosti}

\begin{definition}
    Body ležící na jedné přímce (resp. v jedné rovině) se nazývají \textbf{kolineární}
    (resp. \textbf{komplanární}).
\end{definition}

\begin{definition}
    Přímky $p,q \in \mathscr P$ nazveme:
    \begin{enumerate}[$i.$]
        \item \textbf{různé rovnoběžky}, jestliže $p\cap q = \emptyset \land
            p,q$ jsou komplanární;
        \item \textbf{mimoběžky}, jestliže $p\cap q = \emptyset \land p,q$ nejsou
            komplanární;
        \item \textbf{různoběžky} s \textbf{průsečíkem} $P$, jestliže $p\cap q =
            \left \{ P \right \} $;
       	\item \textbf{splývající rovnoběžky}, jestliže $p\cap q = p$.
    \end{enumerate}
\end{definition}

\begin{veta}[Axiom rovnoběžnosti]
    Každým bodem v $\mathbb E_2$ lze vést ke každé přímce právě jednu rovnoběžku.
\end{veta}

\begin{definition}
    Roviny $\alpha,\beta \subseteq \mathbb E_3$ nazveme:
    \begin{enumerate}[$i.$]
        \item \textbf{rovnoběžné splývající}, jestliže $\alpha = \beta$;
        \item \textbf{rovnoběžné různé}, jestliže $\alpha \ne \beta \land \alpha \cap
        \beta = \emptyset$;
        \item \textbf{různé roviny} s \textbf{průsečnicí} $p$, jestliže $\alpha
        \ne \beta \land \alpha \cap \beta = p$.
    \end{enumerate}
\end{definition}

\begin{veta}[Kritérium rovnoběžnosti dvou rovin]
    Nechť $\alpha, \beta$ jsou roviny. Jestliže rovina $\alpha$ obsahuje dvě různoběžky
    $a,b$ takové, že $a\cap \beta = \emptyset \land b \cap \beta = \emptyset,$
    pak $\alpha \parallel \beta.$
\end{veta}

\begin{proof}
    Je-li $\alpha = \beta,$ předpoklad tvrzení neplatí, takže implikace platí
    triviálně. \\
    Dále sporem: Nechť $\alpha \nparallel \beta \implies \alpha \cap \beta \ne
    \emptyset \implies \exists c = \alpha \cap \beta.$ Protože $a,b$ jsou různoběžky,
    alespoň jedna z těchto přímek protíná přímku $c.$ Nechť je to např. $a.$ Pak
    $a \cap c = \left \{ B \right \} .$ Protože $c\subseteq \beta,$ průsečík
    $B\in \beta \implies \alpha \cap \beta \ne \emptyset,$ což je spor s předpokladem
    $\alpha \cap \beta = \emptyset.$
\end{proof}

\begin{definition}
    Přímku $a \subseteq \mathbb E_3$ nazveme s rovinou $\alpha \subseteq \mathbb E_3:$
    \begin{enumerate}[$i.$]
        \item \textbf{rovnoběžnou}, jestliže $a\cap \alpha = \emptyset$;
        \item \textbf{různoběžnou}, jestliže $a\cap \alpha = \left \{ P \right \} $;
        \item $a$ \textbf{leží v rovině} $\alpha$, jestliže $a \cap \alpha = a$.
    \end{enumerate}
\end{definition}

\begin{veta}[Kritérium rovnoběžnosti přímky a roviny]
    $\forall p \in \mathscr P, \forall \rho \subseteq \mathbb E_3: p \parallel \rho
    \iff \exists q\subseteq \rho: p\parallel q.$
\end{veta}

\begin{proof}
    Pokud $p \subseteq \rho \implies q=p$ a tvrzení platí.\\
    Nechť $p\not \subseteq \rho:$
    \begin{enumerate}[$i.$]
        \item \uv{$\implies$}: Nechť $p\parallel\rho\implies p\land \rho=\emptyset.$
        Nechť $A\in\rho$ je lib. bod. Potom bodem $A$ a přímkou $p$ je jednoznačně
        určena rovina $\sigma.$ $A\in \rho\cap\sigma \implies \rho\cap\sigma=q,$
        $p,q$ jsou komplanární a mají prázdný průnik $\implies p\parallel q.$
        \item \uv{$\impliedby$}: Sporem: Předpokládejme, že $p\parallel q \land
        p\nparallel \rho.$ $p\parallel q \implies p\cap q=\emptyset\land q\subseteq \rho,
        p,q$ leží v téže rovině a $p\ne q\implies p\cap q = \emptyset$, což je spor.
    \end{enumerate}
\end{proof}


\begin{priklad}
Je dána krychle $ABCDEFGH$, na jejích hranách body $R,S,T$ podle obrázku. Určete
průsečík přímky $DF$ a $RST$.
\end{priklad}

\begin{definition}\label{kolmeprimky}
    Přímky $p,q \subseteq \mathbb E_3$ se nazývají navzájem \textbf{kolmé}, právě když
    existují $p^\prime, q^\prime \subseteq \mathbb E_3$ takové, že
    \begin{enumerate}[$i.$]
    \item   $p^\prime \parallel p \land q^\prime \parallel q,$
   	\item $p^\prime, q^\prime$ jsou komplanární,
   	\item $p^\prime \perp q^\prime.$
    \end{enumerate}
\end{definition}

\begin{pozn}
    Definice \ref{kolmeprimky} je i kritériem.
\end{pozn}

\begin{veta}\label{kolmostprimekposun}
    Nechť $p,q \subseteq \mathbb E_3, p\perp q.$ Pak $\forall p^\prime, q^\prime
    \subseteq \mathbb E_3: p^\prime \parallel p, q^\prime \parallel q \implies
    p^\prime \perp q^\prime.$
\end{veta}

\begin{priklad}
Je dána krychle $ABCDEFGH$. Bod $K$ je středem $EA$, $L$ je středem $FG$. Rozhodněte,
zda následující dvojice přímek jsou kolmé:
\begin{enumerate}[$a.$]
\item $DH$ a $BC$,
\item $CL$ a $KH$.
\end{enumerate}
\end{priklad}

\begin{reseni}
Využijeme věty \ref{kolmostprimekposun}.
\begin{enumerate}[$a.$]
\item Triviálně.
\item Jednu z přímek posuneme do roviny té druhé. Pak plyne triviálně.
\end{enumerate}
\end{reseni}

\begin{definition}
    Nechť $p\subseteq \mathbb E_3$ je přímka a $\alpha \subseteq \mathbb E_3$ je rovina.
    Řekneme, že \textbf{přímka} $p$ je \textbf{kolmá k rovině} $\alpha$ právě tehdy,
    když $\forall q \subseteq \alpha: q \perp p.$
\end{definition}

\begin{veta}[Kritérium kolmosti přímky a roviny]
    Nechť $p\subseteq \mathbb E_3$ je přímka a $\alpha \subseteq \mathbb E_3$ je rovina.
    Pak  $p\perp \alpha \iff \exists q,r \subseteq \alpha: q \nparallel r: q\perp p\land
    r\perp p.$
\end{veta}

\begin{priklad}
Nechť je dán pravidelný čtyřstěn $ABCD$. Dokažte, že $AB$ je kolmá k $CD$.
\end{priklad}

\begin{reseni}
Jednou přímkou proložíme rovinu a dokážeme, že tato rovina je kolmá k druhé přímce.
Konkrétně: přímkou $CD$ proložíme rovinu $\overleftrightarrow{SCD}$ ($S$ je střed
$AB$). Pak plyne triviálně.
\end{reseni}

\begin{definition}\label{kolmeroviny}
    Řekneme, že \textbf{rovina} $\alpha$ \textbf{je kolmá k rovině} $\beta$, jestliže
    $\exists a\subseteq\alpha: \alpha \perp\beta.$
\end{definition}

\begin{pozn}
    Definice \ref{kolmeroviny} je i kritériem.
\end{pozn}

\begin{pozn}
    Musíme umět sestrojit průsečík přímky a roviny, průsečnici dvou rovin a řez
    tělesa rovinou.
\end{pozn}

\begin{definition}
\textbf{Shodným} (resp. \textbf{podobným}) \textbf{zobrazením} v prostoru (shodností,
resp. podobností) nazýváme zobrazení $\mathscr Z:
\mathbb E_3 \to \mathbb E_3$, jestliže platí
\begin{align*}
    \forall X, Y \in \mathbb E_3: |\mathscr Z(X)\mathscr Z(Y)| =|XY|, & & \textrm{resp. } |\mathscr Z(X)\mathscr Z(Y)|=k|XY|, \,\,\, k \in \mathbb R^+
\end{align*}
a číslo $k$ \textbf{koeficientem podobnosti}.
\end{definition}

\begin{pozn}
    Shodné zobrazení je podobné zobrazení s koeficientem podobnosti 1.
\end{pozn}

\begin{definition}
    Nechť $S\in \mathbb E_3$ je bod. Pak zobrazení $\mathscr S_S: \mathbb E_3 \to
    \mathbb E_3$ nazýváme \textbf{středovou souměrností} se středem $S$, jestliže
    \begin{enumerate}[$i.$]
    \item $X=S\implies X=\mathscr S_S(X),$
   	\item $X\ne S \implies S$ je střed $X\mathscr S_S(X).$
    \end{enumerate}
\end{definition}

\begin{definition}
    $\forall \mathscr U \subseteq \mathbb E_3: \mathscr U\ne \emptyset \,\,\, \mathscr
    U$ je \textbf{útvar}.
\end{definition}

\begin{definition}
    $S\in \mathbb E_3$ je \textbf{střed souměrnosti} útvaru $\mathscr U,$ jestliže
    $\exists \mathscr U = \mathscr S_S(\mathscr U).$ Útvar je středově souměrný,
    pokud $\exists S:\exists\mathscr U = \mathscr S_S(\mathscr U).$
\end{definition}

\begin{definition}
    Nechť $\alpha$ je rovina. Pak zobrazení $\mathscr S_\alpha:\mathbb E_3 \to
    \mathbb E_3$ je \textbf{rovinová}  souměrnost, jestliže
    \begin{enumerate}[$i.$]
    \item $X\in\alpha\implies \mathscr S_\alpha(X)=X,$
   	\item $X\in\alpha\implies X\mathscr S_\alpha(X) \perp\alpha\,\land$ střed úsečky
    $X\mathscr S_\alpha(X)\in \alpha.$
    \end{enumerate}
\end{definition}

\begin{definition}
    Nechť $A,B$ jsou dva různé body, $\alpha \subseteq \mathbb E_3$ rovina, která
    prochází středem $AB$ a $AB\perp \alpha$. Pak rovinu $\alpha$ nazýváme
    \textbf{rovinou souměrnosti} bodů $A,B.$
\end{definition}

\begin{definition}
\begin{enumerate}[$i.$]
\item $\alpha$ je \textbf{rovina souměrnosti} útvaru $\mathscr U\subseteq \mathbb E_3,$
pokud $\mathscr U = \mathscr S_\alpha(\mathscr U).$
\item \textbf{Útvar} $\mathscr U \subseteq \mathbb E_3$ je \textbf{rovinově souměrný},
jestliže existuje jeho rovina souměrnosti.
\end{enumerate}
\end{definition}

\begin{veta}
    Nechť $\mathscr S_\beta \circ \mathscr S_\alpha: \mathbb E_3 \to \mathbb E_3$ je
    složené zobrazení dvou rovin souměrných s rovinami $\alpha, \beta\subseteq \mathbb
    E_3$. Nechť $\alpha \parallel \beta$, pak $\forall X\in \mathbb E_3: \mathscr
    S_\beta \cap \gamma,$ kde $\gamma$ je taková rovina, že $\alpha \perp \gamma \land
    \beta \perp \gamma, X \in \gamma.$
\end{veta}

\begin{definition}
    Složením dvou rovinových souměrností s rovnoběžnými rovinami souměrnosti vznikne
    zobrazení, které nazýváme \textbf{posunutím} v $\mathbb E_3.$ Směr kolmý k těmto
    rovinám nazýváme \textbf{směr postunutí}.
\end{definition}

\begin{definition}
    Složením dvou rovinových souměrností s různoběžnými rovinami souměrnosti vznikne
    zobrazení, které nazýváme \textbf{otočením} v $\mathbb E_3.$ Průsečnici těchto
    rovin nazýváme \textbf{osu otočení}.
\end{definition}

\begin{definition}
    Zobrazením $\mathscr S_\beta \circ \mathscr S_\alpha = \mathscr O_p,$ kde
    $\alpha \perp \beta, p=\alpha \cap \beta$, nazýváme \textbf{osovou souměrností}
    a $p$ \textbf{osou} osové souměrnosti.
\end{definition}

\begin{definition}
\begin{enumerate}[$i.$]
\item $p$ je \textbf{osa souměrnosti} útvaru $\mathscr U\subseteq \mathbb E_3,$
pokud $\mathscr U = \mathscr O_p(\mathscr U).$
\item \textbf{Útvar} $\mathscr U \subseteq \mathbb E_3$ je \textbf{osově souměrný},
jestliže existuje jeho osa souměrnosti.
\end{enumerate}
\end{definition}

\begin{definition}
    Nechť je dán bod $S\in \mathbb E_3$ a číslo $\lambda \in \mathbb R - \left \{ 0
    \right \}. $ Pak zobrazení $\mathscr H_{S,\lambda}: \mathbb E_3 \to \mathbb E_3$
   nazýváme \textbf{stejnolehlost}, jestliže $\forall X \in \mathbb E_3$ platí:
   \begin{enumerate}[$i.$]
   \item $X=S\implies X^\prime =X=S$,
  	\item $X\ne S\implies |SX^\prime| = |\lambda|\cdot |SX|,$ přičemž
   \begin{enumerate}[$a.$]
   \item $\lambda > 0: X\in \overrightarrow{SX},$
  	\item $\lambda < 0: X$ leží na opačné polopřímce k $\overrightarrow{SX}.$
   \end{enumerate}
   \end{enumerate}
\end{definition}

\begin{priklad}
Sestrojte řez krychle rovinou $\rho=\overleftrightarrow{VWU}$, kde $V$ je střed
úsečky $AE$, $W$ je střed úsečky $AB$ a $V$ je bod hrany $CG$ takový, že $|CU|:|UG|=2:1$.
\end{priklad}
