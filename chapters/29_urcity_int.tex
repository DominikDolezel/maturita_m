\section{Určitý integrál}
\begin{pozn}
    Supremum a infimum množiny jsme již zavedli v definici \ref{supinf}.
\end{pozn}

\begin{definition}
\textbf{Dělením} $D$ uzavřeného \textbf{intervalu} $\left < a,b \right > $
rozumíme každou konečnou množinu čísel $x_0,\dots,x_n \in \mathbb R$ takových, že
$a=x_0<x_1<\dots<x_n=b.$ Intervaly $\left < x_0, x_1 \right > ,
\left < x_1, x_2 \right >, \dots, \left < x_{n-1}, x_n \right >  $ nazýváme
\textbf{dělící intervaly}, které označíme $D=\left \{ x_0,x_1,\dots,x_n \right \} .$
Body $x_0,\dots,x_n$ nazýváme \textbf{dělící body} dělení $D$. Číslo
$\max \left ( x_i-x_{i-1} \right ), i=1,\dots,n $ nazýváme \textbf{normou dělení}
$D$ a označujeme $\nu(D).$ Množinu všech dělení na intervalu $\left < a,b \right > $
označme $\mathscr D(a,b).$
\end{definition}

\begin{definition}
Nechť $D_1, D_2 \in \mathscr D(a,b)$. Řekneme, že dělení $D_2$ je \textbf{zjemněním
dělení} $D_1,$ jestliže $D_1 \subset D_2.$
\end{definition}

\begin{definition}
Nechť $f(x)$ je omezená na $\left < a,b \right > , D\in \mathscr D(a,b),
D=\left \{ x_0,\dots,x_n \right \}. $ Pak množina $\left \{ f(x),
x \in \left < x_{i-1},x_i \right >  \right \} $ je pro všechna $i \in \left \{
1,\dots,n\right \} $ neprázdná a omezení a tedy má supremum a infimum.
Označme
\begin{align*}
    m_i &= \inf \left \{ f(x), x \in \left < x_{i-1},x_i \right >  \right \} \textrm{ pro } D=\left \{ x_0,\dots,x_n \right \},\\
    M_i &=   \sup \left \{ f(x), x \in \left < x_{i-1},x_i \right >  \right \} \textrm{ pro } D=\left \{ x_0,\dots,x_n \right \}.
\end{align*}
Číslo $S(D,f)=\sum_{i=1}^n M_i\cdot (x_i-x_{i-1})$
(resp. $s(D,f)=\sum_{i=1}^n m_i\cdot (x_i-x_{i-1})$) je \textbf{horní} (resp. \textbf{dolní}) \textbf{součet funkce}
$f(x)$ příslušný dělení $D$.
\end{definition}

\begin{pozn}
    Platí $s(D,f)\leq S(D,f).$
\end{pozn}

\begin{veta}
Nechť $f(x)$ je omezená na intervalu $\left < a,b \right >. $ Pak
$\left \{ s(D,f); D\in \mathscr D(a,b) \right \} $ je shora omezená a množina
$\left \{ S(D,f); D\in \mathscr D(a,b) \right \} $ je zdola omezená.
\end{veta}

\begin{pozn}
     Množina dolních součtů má supremum a množina horních součtů má infimum.
\end{pozn}

\begin{definition}
Nechť $f(x)$ je omezená na intervalu $\left < a,b \right > .$ Označme
\begin{align*}
    \int_{\underline{a}} ^b f(x) \, dx & = \sup \left \{ s(D,f); D \in \mathscr D(a,b) \right \}, \\
    \int_{a} ^{\underline{b}} f(x) \, dx & = \inf \left \{ S(D,f); D \in \mathscr D(a,b) \right \}.
\end{align*}
Číslo $\int_{\underline{a}} ^b f(x)\, dx$ (resp. $\int_{a} ^{\underline{b}} f(x) \, dx$)
nazýváme \textbf{dolní} (resp. \textbf{horní}) \textbf{integrál} funkce $f(x)$ od
$a$ do $b$.
\end{definition}

\begin{veta}
Nechť $f(x)$ je omezená na intervalu $\left < a,b \right >. $ Pak
$$\int_{\underline{a}} ^b f(x)\, dx \leq \int_{a} ^{\underline{b}} f(x) \, dx.$$
\end{veta}

\begin{definition}
Nechť $f(x)$ je omezená na intervalu $\left < a,b \right > .$ Pak funkce
$f(x)$ je na intervalu $\left < a,b \right > $ (Riemmanovsky) \textbf{integrovatelná}
(integrace schopna), jestliže
$$\int_{\underline{a}} ^b f(x)\, dx = \int_{a} ^{\underline{b}} f(x) \, dx.$$
Je-li $f(x)$ na intervalu $\left < a,b \right > $ integrovatelná, klademe
$$\int_{\underline{a}} ^b f(x)\, dx = \int_{a} ^{\underline{b}} f(x) \, dx = \int_{a} ^b f(x) \, dx.$$
Číslo $\int_{a} ^b f(x)\, dx$ nazýváme \textbf{Riamennův integrál} funkce $f(x)$ od
$a$ do $b$. Pokud $\int_{\underline{a}} ^b f(x)\, dx < \int_{a} ^{\underline{b}} f(x) \, dx$,
pak $f(x)$ není na intervalu $\left < a,b \right > $ integrovatelná a
$\int_{a} ^b f(x)\, dx$ nedefinujeme.
\end{definition}

\begin{veta}[Newton-Leibnitzova věta]
Nechť $f(x)$ je integrovatelná na intervalu $\left < a,b \right > , F(x)$ je spojitá
na intervalu $\left < a,b \right > $. Buď $F(x)$ primitivní funkce k funkci $f(x)$
na intervalu $\left ( a,b \right ) $. Pak platí:
$$\int_{a} ^b f(x)\, dx=F(b)-F(a).$$
\end{veta}

\begin{veta}
Nechť $f(x)$ je spojitá na intervalu $\left < a,b \right > $. Pak $f(x)$ je na
intervalu $\left < a,b \right > $ integrovatelná.
\end{veta}

\begin{veta}
Nechť $f(x), f_1(x),\dots,f_n(x), g(x)$ jsou integrovatelné funkce na intervalu $\left < a,b \right > $
a $c; c_1,\dots,c\in \mathbb R.$ Pak
\begin{enumerate}[$i.$]
\item funkce $f(x)+g(x)$ je integrovatelná na int. $\left < a,b \right > $ a platí
$$\int _a^b \left [ f(x)+g(x) \right ] \, dx = \int_a ^b f(x)\, dx + \int_a ^b g(x) \, dx,$$
\item funkce $c\cdot f(x)$ je integrovatelná na int. $\left < a,b \right > $ a platí
$$\int _a ^b c\cdot f(x) \, dx = c \int _a ^b f(x)\, dx,$$
\item funkce $c_1\cdot f_1(x) + \dots + c_n \cdot f_n(x)$ je integrovatelná na int. $\left < a,b \right > $ a platí
$$\int_a ^b \left [ c_1\cdot f_1(x) + \dots + c_n \cdot f_n (x) \right ]\, dx = c_1 \int _a ^b f_1(x)\, dx + \dots + c_n \int _a ^b f_n(x)\, dx. $$
\end{enumerate}
\end{veta}
