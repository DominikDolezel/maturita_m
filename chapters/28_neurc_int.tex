\section{Neurčitý integrál}
\begin{definition}\label{integral}
Nechť $f(x)$ je definována na intervalu $I$. Funkce $F(x)$ se nazývá \textbf{primitivní}
k funkci $f(x)$ na $I$, jestliže platí $F^\prime(x)=f(x)$ pro každé $x\in I.$
Množina všech primitivních funkcí k funkci $f(x)$ na $I$ se nazývá \textbf{neurčitý
integrál} z funkce $f(x)$ a značí se $\int f(x)\, dx.$ Tedy
$$\int f(x)\, dx= \left \{ F(x):F(x) \textrm{ je primitivní k } f(x) \textrm{ na } I \right \}. $$
\end{definition}

\begin{veta}
    Je-li $F(x)$ primitivní funkce k $f(x)$ na intervalu $I$, je taky
    $F(x)+c$ primitivní funkce k $f(x)$ na $I$. Má-li funkce $f(x)$ aspoň
    jednu primitivní funkci, má jich nekonečně mnoho.
\end{veta}

\begin{veta}
Je-li funkce $f$ spojitá na intervalu $I$, pak na tomto intervalu existuje
primitivní funkce k funkci $f$.
\end{veta}

\begin{veta}[Pravidla pro integrování]
Nechť na intervalu $I$ existují integrály $\int f(x) \, dx$ a~$\int g(x)\, dx.$
Pak na $I$ existují také integrály $\int(f(x)\pm g(x))\, dx$ a $\int \alpha f(x), \alpha \in \mathbb R,$ přičemž platí:
\begin{align*}
\int (f(x)\pm g(x)) \, dx &= \int f(x)\, dx \pm \int g(x) \, dx,\\
\int \alpha f(x)\, dx &= \alpha\int f(x)\, dx.
\end{align*}
\end{veta}

\begin{pozn}[Tabulkové integrály]
\begin{align*}
    &\int 0\, dx = c & & \, \\
    &\int dx = x+c & & \, \\
    &\int x^n \, dx = \frac{x^{n+1}}{n+1}+c, n\ne -1 & & \, \\
    &\int \frac{1}{x} \, dx = \ln |x| + c & & \int \frac{1}{x+a}\, dx = \ln |x+a|+c \\
    &\int e^x\, dx = e^x+c & & \int e^{ax}\, dx = \frac{1}{a}e^{ax}+c \\
    &\int a^x \, dx = \frac{a^x}{\ln a} + c, a >0 & & \int a^{bx} \, dx = \frac{1}{b}\cdot \frac{a^{bx}}{\ln a}+c, a>0\\
    &\int \sin x \, dx = -\cos x + c & & \int \sin ax \, dx = -\frac{1}{a}\cos ax+c \\
    &\int \cos x \, dx = \sin x +c & & \int \cos ax\, dx = \frac{1}{a}\sin ax + c \\
    &\int \frac{1}{\cos^2 x}\, dx = \tg x + c & & \int \frac{1}{\cos^2 ax}\, dx=\frac{1}{a}\tg ax+c\\
    &\int \frac{1}{\sin^2 x} \, dx = -\cotg x+c & & \int \frac{1}{\sin^2 ax}\, dx = \frac{1}{a}\cotg ax+c \\
    &\int \frac{1}{\sqrt{1-x^2} }\, dx = \arcsin x + c & & \int \frac{1}{\sqrt{1-a^2x^2} }\, dx = \frac{1}{a}\arcsin ax+c \\
    &\int \frac{1}{x^2+1}\, dx = \arctg x + c & & \int \frac{1}{a^2x^2 + 1}\, dx = \frac{1}{a}\arctg ax + c
\end{align*}
\end{pozn}

\begin{priklad}
Řešte integrál $\int x \, dx.$
\end{priklad}

\begin{reseni}
Je $\int x \, dx=\frac{x^2}{2}+c.$
\end{reseni}

\begin{veta}[Integrační metoda \textit{per partes}]
Nechť funkce $u(x) $ a $v(x)$ mají derivaci na intervalu $I$. Pak platí
$$\int u(x)v^\prime (x) \, dx = u(x) v(x) - \int u^\prime (x) v(x) \, dx,$$
pokud alespoň jeden z integrálů existuje.
\end{veta}

\begin{priklad}
Řešte $\int x \sin x \, dx $.
\end{priklad}

\begin{reseni}
Řešme \textit{per partes}:
\begin{align*}
\int x \sin x \, dx &=\left |\begin{array}{l l}
    u=x & u^\prime= 1\\
    v^\prime = \sin x & v=-\cos x
\end{array} \right | = x(-\cos x)-\int 1\cdot(-\cos x)\, dx \\
&= -x\cos x+\int \cos x \, dx=-x \cos x +\sin x+c
\end{align*}
\end{reseni}

\begin{priklad}
Řešte $\int (x^2+1)e^{-x}\, dx.$
\end{priklad}

\begin{reseni}
Řešme \textit{per partes}:
\begin{align*}
    \int \left ( x^2+1 \right )e^{-x}\, dx &= \left |\begin{array}{l l}
        u=x^2+1 & u^\prime= 2x\\
        v^\prime = e^{-x} & v=-e^{-x}
    \end{array} \right | \\
    &= (x^2+1)(-e^{-x})-\int 2x(-e^{-x})\, dx = \dots
\end{align*}
\end{reseni}

\begin{pozn}[Substituční metoda]
    Označme si substituci $\varphi(x)=u.$ Tuto rovnost diferencujeme.
    $$\int f(\varphi(x))\varphi^\prime(x)\, dx=\begin{vmatrix}
        \varphi(x)=u \\
        \varphi^\prime(x) \, dx=du
    \end{vmatrix}=\int f(u)\, du.$$
\end{pozn}

\begin{priklad}
Řešte $\int \frac{(1+\ln x)^4}{x}\, dx.$
\end{priklad}

\begin{reseni}
Metodou substituční:
\begin{align*}
    \int \frac{(1+\ln x)^4}{x}\, dx &= \begin{vmatrix}
        1+\ln x=u \\
        \frac{1}{x}\, dx=du
    \end{vmatrix}=\int u^4\, du \\
    &=\frac{u^5}{5}+c=\frac{(1+\ln x)^5}{5}+c
\end{align*}
\end{reseni}

\begin{priklad}
Řešte $\int e^{\sqrt{x} }\, dx$.
\end{priklad}

\begin{reseni}
Substitucí $\sqrt{x}=t $.
\end{reseni}

\begin{priklad}
Řešte $\sqrt{1-x^2}\, dx.$
\end{priklad}

\begin{reseni}
Substitucí $x=\sin t$.
\end{reseni}

\begin{pozn}[Integrály typu $\frac{f^\prime(x)}{f(x)}$]
    Platí
    $$\int \frac{f^\prime(x)}{f(x)}\, dx=\begin{vmatrix}
        f(x)=t \\
        f^\prime(x) \, dx = dt
    \end{vmatrix}=\int \frac{dt}{t}=\ln |t|+c=\ln |f(x)|+c.$$
\end{pozn}

\begin{priklad}
Řešte $\int \tg x\, dx.$
\end{priklad}

\begin{reseni}
Platí
$$\int \tg x\, dx=\int \frac{\sin x}{\cos x}\, dx.$$
Řešíme substitucí $\cos x=t.$
\end{reseni}

\subsection*{Integrace racionálně lomených funkcí}
\begin{pozn}
    Každou racionální lomenou funkci $\frac{P(x)}{Q(x)}$ lze vyjádřit ve tvaru
    $$\frac{P(x)}{Q(x)}=S(x)+R_1(x)+\dots+R_n(x),$$
    kde $S(x)$ je polynom a $R_1(x),\dots,R_n(x)$ jsou parciální zlomky.
    Na libovolném intervalu, který neobsahuje kořeny $Q(x)$, jsou tyto funkce spojité,
    takže k nim existuje primitivní funkce a platí
    $$\int \frac{P(x)}{Q(x)} \, dx=\int S(x)\, dx + \int R_1(x)\, dx + \dots + \int R_n(x)\, dx.$$
\end{pozn}

\begin{pozn}[Integrace parciálních zlomků s reálnými kořeny ve jmenovateli]
    Parciální zlomky $\frac{A}{(x-\alpha)^k}$ integrujeme následovně:
    \begin{enumerate}[$i.$]
    \item je-li $k=1,$ pak triviálně
    $$\int \frac{A}{x-\alpha} \, dx = A\ln |x-\alpha|+c.$$
   	\item je-li $k\geq 2,$ použijeme substituci:
    \begin{align*}
    \int \frac{A}{(x-\alpha)^k}\, dx &=\begin{vmatrix}
      x-\alpha=t \\
      dx = dt
  \end{vmatrix}=A\int \frac{dt}{t^k}=A\int t^{-k}\, dt \\
 & = A\cdot \frac{t^{-k+1}}{-k+1}+c=\frac{A}{(1-k)t^{k-1}}+c=\frac{A}{(1-k)(x-\alpha)^{k-1}}+c.
    \end{align*}
    \end{enumerate}
\end{pozn}

\begin{priklad}
Řešte $\int \frac{3x+16}{x^2-x-6}\, dx$.
\end{priklad}

\begin{reseni}
Rozkladem na parciální zlomky. Vede na $\ln$.
\end{reseni}

\begin{pozn}[Integrace parciálních zlomků s komplexními kořeny ve jmenovateli]
Parciální zlomky $\frac{Mx+N}{(x^2+px+q)^n}$ integrujeme následovně:
\begin{enumerate}[1.]
    \item je-li $p=0$, pak $q>0,$ položíme $q=a^2,a>0$. Je tedy
    $$\frac{Mx+N}{(x^2+a^2)^n}.$$
    Pak
    $$\int \frac{Mx+N}{(x^2+a^2)^n}\, dx = M\underbrace{\int \frac{x}{(x^2+a^2)^n}\, dx}_{I.} + \underbrace{N\int \frac{1}{(x^2+a^2)^n}\, dx}_{II.}.$$
    \begin{enumerate}[$I.$]
    \item Zlomky tvaru $\frac{x}{(a^2+a^2)^n}$ integrujeme následovně:
    \begin{enumerate}[$i.$]
    \item je-li $n=1$, pak
    \begin{align*}
        \int \frac{x}{x^2+a^2}\, dx &= \frac{1}{2}\int \frac{2x}{x^2+a^2}\, dx = \begin{vmatrix}
            x^2+a^2=t \\
            2x\, dx = dt
        \end{vmatrix} \\
        &=\frac{1}{2} \int \frac{dt}{t}=\frac{1}{2}\ln |t|+c=\frac{1}{2}\ln(x^2+a^2)+c.
    \end{align*}
   	\item je-li $n \geq 2,$ pak
    \begin{align*}
        \int \frac{x}{(x^2+a^2)^n}\, dx &= \begin{vmatrix}
            x^2+a^2=u \\
            2x\, dx = du \\
            x\, dx = \frac{1}{2}\, du
        \end{vmatrix}=\int \frac{1}{u^n}\cdot \frac{du}{2} \\
        &= \frac{1}{2}\int u^{-n}\, du = \frac{1}{2}\cdot \frac{u^{-n+1}}{-n+1}+c =\frac{1}{2(1-n)}\cdot \frac{1}{u^{n-1}}+c \\
        & = \frac{1}{2(1-n) }\cdot\frac{1}{(x^2+a^2)^{n-1}}+c.
    \end{align*}
    \end{enumerate}
   	\item Označme
    $$J_n(x,a)=\int \frac{1}{(x^2+a^2)^n}\, dx.$$ Pak
    $$
        J_1(x)= \int \frac{1}{x^2+a^2} \, dx = \frac{1}{a^2}\int \frac{1}{\left ( \frac{x}{a} \right )^2+1 }\, dx =\frac{1}{a^2}\cdot \frac{1}{\frac{1}{a}}\cdot \arctg \frac{x}{a} + c = \frac{1}{a}\cdot \arctg \frac{x}{a}+c.
    $$
    $J_2$ určíme rekurentně tak, že vyjádříme metodou \textit{per partes} $J_1$ ($u=J_1$, $v^\prime =1$),
    $J_3$ tak, že vyjádříme metodou \textit{per partes} $J_2$, \dots
    \end{enumerate}
   	\item je-li $p\ne 0,$ upravíme zlomek tak, aby v čitateli byla derivace jmenovatele, tj.
    $$\frac{Mx+N}{(x^2+px+q)^n}=r\frac{2x+p}{(x^2+px+q)^n}+s \frac{1}{(x^2+px+q)^n}.$$
    Bude tedy
    $$\int \frac{Mx+N}{(x^2+px+q)^n}\, dx =r\underbrace{\int \frac{2x+p}{(x^2+px+q)^n}\, dx}_{I.}+s \underbrace{\int \frac{1}{(x^2+px+q)^n}\, dx}_{II.}.$$
    \begin{enumerate}[$I.$]
    \item Integrál $\int \frac{2x+p}{(x^2+px+q)^n}\, dx$ spočítáme obdobně jako první integrál v 1.
    \begin{enumerate}[$i.$]
    \item pro $n=1$ je
    $$\int \frac{2x+p}{x^2+px+q}\, dx = \ln(x^2+px+q)+c.$$
   	\item pro $n \geq 2$ použijeme substituci
    \begin{align*}
    \int \frac{2x+p}{(x^2+px+q)^n}\, dx &= \begin{vmatrix}
x^2+px+q=u \\
(2x+p)\, dx = du
\end{vmatrix}=\int \frac{du}{u^n}\\
& = \frac{1}{1-n}\cdot \frac{1}{u^{n-1}}+c = \frac{1}{1-n}\cdot \frac{1}{(x^2+px+q)^{n-1}}+c.
    \end{align*}
    \end{enumerate}
   	\item Integrál $\int \frac{1}{(x^2+px+q)^n}\, dx$ převedeme substitucí na výpočet
    integrálu $J_n$.
    \begin{align*}
        \int \frac{1}{(x^2+px+q)^n}\, dx &= \int \frac{1}{\left [ \left ( x+\frac{p}{2} \right )^2 +a^2 \right ]^n }\, dx = \begin{vmatrix}
            x+\frac{p}{2}=u \\
            dx = du
        \end{vmatrix} \\
        &= \int \frac{1}{(u^2+a^2)^n}\, du  = J_n(u,a).
    \end{align*}
    \end{enumerate}
\end{enumerate}
\end{pozn}

\begin{priklad}
Řešte $\int \frac{dx}{(x^2+2)^2}\, dx = J_2.$
\end{priklad}

\begin{reseni}
Vyjádříme si \textit{per partes} $J_1$ tak, že derivujeme celý výraz $J_1$ a integrujeme
1. Dostaneme rovnici v proměnné integrál. $J_1$ spočítáme snadno (vede na $\arctg$).
\end{reseni}

\begin{priklad}
Řešte $\int \frac{5x+6}{x^2+28x+300}\, dx.$
\end{priklad}

\begin{reseni}
Do čitatele dostaneme derivaci jmenovatele. Protože nevychází konstatní člen
v~čitateli, \uv{něco si půjčíme a vrátíme}. Zlomek se rozdělí na více sčítanců.
Poslední z~nich integrujeme tak, že ve jmenovateli doplníme na čtverec a vhodně vynásobíme
tak, abychom dostali něco tvaru $x^2+1$.
\end{reseni}


\subsection*{Integrace goniometrických funkcí}
\begin{pozn}
    Budeme vyšetřovat integrály typu
    $$\int R(\sin x, \cos x) \, dx,$$
    kde $R(\sin x, \cos x)$ je funkce v proměnných $\sin x, \cos x.$ Substitucí
    $\tg \frac{x}{2}=t$ lze tento integrál vždy převést na integrál z racionální
    lomené funkce v proměnné $t$. \\
    Je-li $\tg \frac{x}{2}=t,$ pak ze vztahů v pravoúhlém trojúhelníku dostáváme
    \begin{align*}
        \sin \frac{x}{2}=\frac{t}{\sqrt{1+t^2} }, & & \cos \frac{x}{2}=\frac{1}{\sqrt{1+t^2} }.
    \end{align*}
    Dále
    \begin{align*}
        \sin x = 2\sin \frac{x}{2}\cos \frac{x}{2}=\frac{2t}{1+t^2}, & &
        \cos x = \cos^2 \frac{x}{2}-\sin^2 \frac{x}{2}=\frac{1-t^2}{1+t^2}.
    \end{align*}
    Pokud $\tg \frac{x}{2}=t,$ pak $\frac{x}{2}=\arctg t,$ a tedy
    $$dx = \frac{2}{t^2+1}\, dt.$$
\end{pozn}

\begin{priklad}
Řešte $\int \frac{dx}{1-\cos x+\sin x}$.
\end{priklad}

\begin{reseni}
Řešíme substitucí $\tg \frac{x}{2}=t$. Ostatní funkce popř. odvodíme ze vztahů
v pravoúhlém trojúhelníku.
\end{reseni}

\begin{veta}
    Je-li funkce $R(\sin x, \cos x)$ lichá vzhledem k
    proměnné $\sin x$, používáme substituci $\cos x=t$.
\end{veta}

\begin{veta}
    Je-li funkce $R(\sin x, \cos x)$ lichá vzhledem k
    proměnné $\cos x$, používáme substituci $\sin x=t$.
\end{veta}

\begin{veta}
    Je-li funkce $R(\sin x, \cos x)$ sudá vzhledem k
    oběma proměnným, používáme substituci $\tg x=t$.
\end{veta}

\begin{priklad}
Řešte $\int \frac{2\cos x - \sin x}{\cos x - 2\sin x}\, dx$.
\end{priklad}

\begin{reseni}
Řešíme substitucí $\tg x = t \implies x=\arctg t$. Ostatní funkce popř.
odvodíme ze vztahů v pravoúhlém trojúhelníku.
\end{reseni}

\begin{pozn}
    Integrály $\int \sin^2 x \, dx, \int \cos^2 x \, dx$ neřešíme substitucí
    $\tg x  = t,$ nýbrž ze vzorce pro poloviční argument:
    \begin{align*}
        \int \sin^2 x \, dx &= \int \frac{1-\cos 2x}{2}\, dx = \frac{1}{2}x - \frac{1}{4} \sin 2x + c, \\
        \int \cos^2 x \, dx &= \int \frac{1+\cos 2x}{2}\, dx = \frac{1}{2}x + \frac{1}{4} \sin 2x + c.
    \end{align*}
\end{pozn}

\begin{priklad}
Řešte $\int \cos^2 x \, dx$.
\end{priklad}

\begin{reseni}
Vyjádříme si ze vzorce pro dvojnásobný argument.
\end{reseni}

\subsection*{Integrace některých iracionálních funkcí}
\begin{veta}
    Integrály tvaru
    $$\int R(x,\sqrt[s]{x}) \, dx, \,\,\, \textrm{kde } s \in \mathbb N, s \geq 2$$
    řešíme substitucí $x=t^s$.
\end{veta}

\begin{veta}
    Integrály tvaru
    $$\int S(x, \sqrt[s_1]{x}, \dots, \sqrt[s_k]{x}  ) \, dx ,\,\,\, \textrm{kde } k \in \mathbb N, s_i \geq 2$$
    řešíme substitucí $x=t^s,$ kde $s$ je nejmenší společný násobek $s_1,\dots, s_k.$
\end{veta}

\begin{veta}
    Integrály tvaru
    $$\int R(x, \sqrt[s]{ax+b} )\, dx, \,\,\, \textrm{kde } s \in \mathbb N, s \geq 2, a,b \in \mathbb R$$
    řešíme substitucí $t^s=ax+b$.
\end{veta}

\begin{veta}
    Integrály tvaru
    $$\int R\left (x, \sqrt[s]{\frac{ax+b}{cx+d}} \right)\, dx, \,\,\, \textrm{kde } s \in \mathbb N, s \geq 2, a,b,c,d \in \mathbb R, ad-bc \ne 0$$
    řešíme substitucí $t^s=\frac{ax+b}{cx+d}.$
\end{veta}

\begin{veta}[Eulerovy substituce]
    Integrály tvaru
    $$\int R(x,\sqrt{ax^2+bx+c}) \, dx, \,\,\, \textrm{kde } a,b,c\in \mathbb R, a \ne 0$$
    řešíme substitucemi
    \begin{align*}
    \sqrt{ax^2+bx+c} &= \sqrt{|a|}\cdot t(x-\alpha) & & \textrm{je-li }D >0,  \\
    \sqrt{ax^2+bx+c} &= \pm\sqrt{|a|}\cdot x\pm t & & \textrm{je-li }D <0, a>0, \\
    \sqrt{ax^2+bx+c} &= \pm xt \pm \sqrt{c}  & & \textrm{je-li }c >0.
    \end{align*}
    Pokud $b=0$, řešíme integrály
    \begin{align*}
        &\int R(x, \sqrt{k^2-x^2} )\, dx & & \textrm{substitucí } x=k\sin t, \\
        &\int R(x, \sqrt{x^2+k^2})\, dx & & \textrm{substitucí } x=k\tg t, \\
        &\int R(x, \sqrt{x^2-k^2} )\, dx & & \textrm{substitucí } x=\frac{k}{\sin t}.
    \end{align*}
\end{veta}

\begin{priklad}
Řešte $\int \sqrt{a^2-x^2}\, dx. $
\end{priklad}

\begin{priklad}
Řešte $\int \frac{dx}{x \sqrt{x^2+2} }.$
\end{priklad}

\begin{reseni}
Řešíme substitucí $x=\sqrt{2}\tg t. $
\end{reseni}

\begin{priklad}
Řešte $\int \frac{1}{x}\sqrt{x^2-1}\, dx, x \in (1,\infty). $
\end{priklad}

\begin{reseni}
Řešíme substitucí $x=\frac{1}{\sin t}.$
\end{reseni}

\begin{pozn}
    Přehled všech základních tabulkových integrálů a substitucí, které musíme umět, je
    k nalezení v příloze \ref{appint}.
\end{pozn}
