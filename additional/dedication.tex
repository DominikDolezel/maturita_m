%!TEX root = ../main.tex

\clearpage
\phantomsection
\addcontentsline{toc}{section}{Předmluva}
\begin{adjustwidth}{0.1\textwidth}{0.1\textwidth}
\begingroup
\null\vspace{0.2\textheight}
\begin{center}
{\bfseries\Large Předmluva}\par\vspace{2em}
\end{center}

\textit{Na počátku všeho řekl Bůh:}
\begin{center}
\uv{\textit{Nechť je dán polynom.}}
\end{center}
\textit{Byl spokojen se svým výtvorem, ale přišlo mu, že všechno vědění je vlastně
bezúčelné, nemá-li ho kdo studovat. A~tak stvořil člověka\dots}

\par Vážené čtenářstvo, v~rukou držíte soubor všech základních definic, vět
a~příkladů středoškolské matematiky tak, jak byla probírána na matematickém gymnáziu
v~první polovině dvacátých let jednadvacátého století. Seznam je sestaven podle
požadavků k maturitní zkoušce.
\par Všichni, již se na psaní a~úpravě podíleli, přejí příjemné čtení. A~nezapomeňte,
všechno to znáte ze základní školy!


\hfill Věnováno RNDr. Pavlu Boucníkovi
\endgroup
\end{adjustwidth}
\clearpage
